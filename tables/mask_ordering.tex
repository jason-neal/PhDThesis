%!TEX root = ../thesis.tex

\begin{table}
    \centering
    \caption{Relative {RV} precision difference for Condition~\#2 due to spectral splitting and order of applying the pixel mask.
        The ratio are the difference between \textit{Split} and \textit{Masked} implementations with the same gradient calculation.
        The last column is the ratio between the \textit{Masked} versions using the {FFD} and numpy gradient methods and are consistent with \cref{tab:numerical_gradients}.
        Results a for an M0 spectral type, with $\vsini=1.0$ and R=100\,000.}
    \begin{tabular}{c|ccc|ccc|c}
        \toprule
        & Split & Masked & \(\Delta\)Ratio & Split & Masked & \(\Delta\)Ratio & Masked \\
        Gradient & \multicolumn{3}{c|}{FFD} & \multicolumn{3}{c|}{Numpy} & \(\Delta\)Ratio\\
        Band & \mps{} & \mps{} &  \%  & \mps{} & \mps{} &   \% & \% \\
        \midrule
        Z &  7.42 &  7.38 & -0.66 &  7.76 &  7.77 & 0.13 & 5.3\\
        Y &  4.75 &  4.74 & -0.22 &  5.06 &  5.06 & 0.06 & 6.8\\
        J & 18.58 & 18.53 & -0.29 & 19.57 & 19.57 & 0.01 & 5.6\\
        H &  6.08 &  6.05 & -0.53 &  6.25 &  6.26 & 0.08 & 3.5\\
        K & 32.21 & 32.14 & -0.22 & 33.48 & 33.49 & 0.05 & 4.2\\
        \bottomrule
    \end{tabular}\label{tab:mask_ordering}
\end{table}

{\rd{} The results from these spectra for conditions \#1 and \#3 are consistent with~\citept{figueira_raial_2016}. \todo{put this some where}}
