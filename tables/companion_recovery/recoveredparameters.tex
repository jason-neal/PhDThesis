%!TEX root = ../thesis.tex

\begin{table*}
      \centering
 \begin{threeparttable}
    \caption{Input and recovered parameters on simulations and an observation when applying a single (\(\rm C^1\)) and binary (\(\rm C^2\)) models. The logg and metallicity were fixed at \(\rm logg_1 = 4.50\), \(\rm logg_2=5.0\) and \feh{}=0.0 equally for both components. Gaussian noise was added to both simulations with a {SNR} of 150. Here \(m\) and \(n\) are the number of data points and parameters used in each model.}

    \begin{tabular}{c | *3c | *3c | *3c}
        \toprule
        & \multicolumn{3}{c|}{Simulation 1} & \multicolumn{3}{c|}{Simulation 2} & \multicolumn{3}{c}{Observed {HD 211847}} \\
        \midrule
        & Input & \multicolumn{2}{c|}{Recovered} & Input & \multicolumn{2}{c|}{Recovered} & Expected & \multicolumn{2}{c}{Recovered} \\
        & & \(C^1\) & \(C^2\) & & \(C^1\) & \(C^2\) & & \(C^1\)  & \(C^2\) \\
        \midrule
        \(\teff{}_1\) & 5\,800 & 5\,800 & 5\,800 & 5\,700 & 5\,800 & 5\,700 & \(5\,715 \pm 24\) & 5\,900 & 5\,800\\
        \(\teff{}_2\) & 4\,000 & -- & 3\,800 & 3\,200 & -- & 3\,100 & \(\sim\)3\,200 & -- & >3\,800\tnote{a}\\
        \({rv}_1\) & 0 & 0.1 & 0 & 6.6 & 6.6 & 6.6 & \(6.6 \pm 0.3\) & 7& 7.6 \\
        \({rv}_2\) &  10 & -- & 9.8 & 0.5 & -- &  -1& \(0.5 \pm 2\) & -- &-12.6\\
        \midrule
        \(R_1/R_2\)& 2.57 & -- & 2.71& 3.16 & - & 3.27 & 3.16 & -- & <2.71\tnote{a}\\
        \(\rm F_2/F_1\)& 0.084 & -- & 0.066 & 0.030 & -- & 0.026 & 0.030 & -- & >0.066\tnote{a}\\
        \(m\) & - & 3\,072 & 3\,072 & -- & 3\,072 & 3\,072 & -- & 2\,612 & 2\,612\\
        \(n\) & - & 2 & 4 & -- & 2 & 4 & -- & 2 & 4\\
        \(\chi^2\)& -- & 4\,978 & 3\,792 & -- & 3\,746 & 3\,630  & -- & 37\,688 & 33\,860\\
        \(\chi^2_{red}\) & -- & 1.62 & 1.24 & -- & 1.22 & 1.18 & -- & 21.3 & 19.2\\
        {BIC} & -- & -20\,145 & -22\,315 & -- & -21\,477 & -21\,377& -- & 18\,281 & 14\,468\\
        \bottomrule
    \end{tabular}

\label{tab:example_params}
    \begin{tablenotes}
       \item [a] {At the arbitrary upper limit for companion temperature grid (3\,800\K{}).}
    \end{tablenotes}
\end{threeparttable}
\end{table*}
