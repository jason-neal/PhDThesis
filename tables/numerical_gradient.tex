%!TEX root = ../thesis.tex

\begin{table}
    \centering
    \caption[Effect of numerial gradient.]{The effect of the numerical gradient function on {RV} precision.
        The band label \(\rm VIS\) and \(\rm NIR\) indicate the full visible and \nir{} bands while  \(\rm CARM_{VIS}\)  and \(\rm CARM_{NIR}\) indicate the two wavelength bands of the {CARMENES} spectrograph.
        \(\Delta\lambda\) is a wavelength shift applied to move the FFD values between pixels for this comparison only.
        The {M0} spectra used here had no rotation, or instrument broadening performed and was normalized to a maximum of 1 in each band.
        The precision values given here are for accessing the relative precision change due to the different gradient methods.}
    \begin{tabular}{ccccccc}
        \toprule
        %% Band & \(\lambda\) range\_min & wl\_max &  dy/dx   & gradient & Q(dy/dx) & Q(grad) & Q(frac) & {RV}(dy/dx) & {RV}\_adj &    {RV}(grad)    &    {RV}(frac\_grad)    & {RV}(frac\_adj)          \\
        Gradient method &  &  A &   B & C & (B-A)/A & (C-A)/A \\
        &   \(\lambda\) range & FFD & FFD+\(\Delta\lambda\) &  Numpy & \(\Delta\delta V\) ratio& \(\Delta\delta V\) ratio\\

        Band  & (\si{\micro\meter})  & \multicolumn{3}{c}{\(\delta V_{rms}\) (\mps{})}  & (\%) & (\%) \\
        \midrule
        VIS & 0.38 -- 0.78 & 16.1 & 16.2 & 16.9  & 0.6 & 4.9\\
        \(\rm CARM_{VIS}\) & 0.52 --  0.96 & 20.9 & 21.0 & 22.0 & 0.3 & 5.2 \\
        Z & 0.83 -- 0.93 & 76.9 & 77.0 & 78.8  & 0.1 & 2.5\\
        Y & 1.00 -- 1.10 & 78.3 & 78.5 & 83.8 & 0.2 & 7.0 \\
        J & 1.17 -- 1.33 & 149.3 & 149.4 & 156.4 & 0.1 & 4.7 \\
        H & 1.50 -- 1.75 & 119.4 & 119.5 & 122.3 & 0.1 & 2.5 \\
        K & 2.07 -- 2.35 & 153.4 & 153.7 & 157.7  & 0.2 & 2.8\\
        \(\rm CARM_{NIR}\) & 0.96 -- 1.71 & 46.1 & 46.2 & 48.0 & 0.1 & 4.2 \\
        NIR & 0.83 -- 2.35 & 36.9 & 36.9 & 38.2 & 0.1 & 3.6  \\
        \bottomrule
    \end{tabular}\label{tab:numerical_gradients}
\end{table}
