% Stuff cut out of paper

\subsection{Targets}
\label{subsec:targets}
\subsubsection{HD 30501}
\label{subsubsec:HD30501}
\object{HD 30501} is a large BD companion with a mass of \(\sim \rm 90 M_{Jup} \) determined by \citet{sahlmann_search_2011}. Of our sample it was seen to be favorable as it had the most observations and the largest time separation and \(\rm \Delta RV \) between observations. Due to the long orbital period of \emph{XXXX} days this is still not sufficient to achieve meaningful results from the differential subtraction method.

The estimated flux ratio of 1\% probably puts it close to the detection limit with a SNR of >100.


\subsubsection{HD 211847}
\label{subsubsec:HD211847}
\object{HD211847B} is a binary companion that was observed by Sahlmann but not well constrained. \citet{moutou_eccentricity_2017} used photometry and the BT-Settl models (Allard 2014) to report its mass to be \(\rm 155 M_{Jup} \) star, assuming an age of 3~Gyr. This places it as a low mass star, in a nearly face on with an \(i=7\circ \).
The high flux ratio (26\%?) of this companion makes it an optimal choice for testing the companion recovery with the mixed spectra models.

Our method of using the Baraffe evolutionary models with this companion mass returns a consistent K-band magnitude, and flux ratio, compared to the photometry value.

From first looks the temperature I get is higher then the published value by \(\sim800\)~K  (I was at the end of my model run)

\object{HD 211847} also has a second smaller companion...

If the mass of the B companion is found can we detect the c?

What is the expected flux ratio of c' would need a triple spectrum model? Do we see an indication of a second local minimum somewhere?
% IRDIS K1 and K2 magnitudes for \object{HD 211847}
% Table 5.
% Photometric values for \object{HD 211847} B using IRDIS and IFS and corresponding values in mass.
%                     Y       J       H        K1      K2
% Absolute magnitude  9.87    9.29    8.93    8.43    8.24
% Mass (\(\rm M_{Jup} \))         150.4   155.5   142.4   162.7   164.0
%
% They used btsettle models with age of 3 years.

\subsubsection{HD 202206}
\label{subsubsec:HD202206}
The companions \object{HD 202206} were initially revealed in..., Recent work by \citet{benedict_hd_2017} revealed that this is a binary system orbited by a BD, giving the mass of the companions as \object{HD 202206}B and  \object{HD 202206}c as \(\rm 93.6 M_{Jup} \) and \(\rm 17.6 M_{Jup} \) respectively.
We estimate the flux ratio of the binary companion to be 13\% \emph{so we are likely to detect it via companion recovery.}, while the circumbinary BD companion to be X. \textit{relative to the host}.


\subsubsection{HD166765}
\label{subsubsec:HD166765}
The current \(\textrm{M}\sin{i} \) of HD166765b is \(\rm 50.3 M_{Jup} \) \citep{patel_fourteen_2007}.


Our estimated flux ratio is \(2\times10^{-3} - 2\times10^{-5} \) due to the uncertainty in the age.


\subsubsection{HD 4747}
\label{subsubsec:HD4747}
The single observation of \object{HD 4747} ruled out the possibility of applying the differential technique but it would not have made a difference due to the extremely long period of \(\sim37.88 \) years.
\citet{crepp_trends_2016} recently placed \object{HD 4747} in the BD desert, determined the companion dynamical mass of \object{HD 4747} to be m=\(60.2\pm3.3\textrm{M}_{J} \) using astrometry and RV measurements and estimated a model-dependent (Baraffe 2003) mass of \(\rm M = 72^{+3}_{-13} M_{Jup} \) from relative photometry.

With only one observation performed this was unsuitable for the differential method. And a estimated flux ratio of \(1\times10^{-4} \) is below the SNR level of our observations.

The analysis on spectral recovery achieved an upper mass limit of XXXX (which the limit of our analysis)?
\textbf{below our SNR so I am not sure if we will detect with companion recovery technique}


\subsection{HD 168443}
\label{subsubsec:HD168443}
\object{HD 168443} is known to host two companions: an inner giant planet and an outer BD with \(\textrm{M}_{2}\sin{i} \) values of \(7.7 \rm M_{Jup} \) and \(\rm 17.2 M_{Jup} \) respectively \citep{pilyavsky_search_2011}.
\object{HD 168443}b is a popular target for planet atmospheres...

\subsection{HD 162020}
\label{subsubsec:HD162020}


\begin{figure}
	\begin{tabular}{c}
	    \includegraphics[width=0.5\linewidth]{images/tmp-images/HD211847-1_coadd_gamma_teff_2_red_chi2_3_contour__auto_more_temps.png} \\
	    \includegraphics[width=0.5\linewidth]{images/tmp-images/HD211847-1_coadd_rv_teff_2_red_chi2_3_contour__auto_more_temps.png} \\
	    %  \includegraphics[width=0.5\linewidth]{images/tmp-images/chi2_alpha_p005test_res50000_snr100.png} &
	    %  \includegraphics[width=0.5\linewidth]{images/tmp-images/chi2_alphatest_res50000_snr50.png}   \\
	    %  \includegraphics[width=0.5\linewidth]{images/tmp-images/chi2_alphatest_res50000_snr100.png} &
	    %  \includegraphics[width=0.5\linewidth]{images/tmp-images/chi2_alphatest_res50000_snr1000.png}
	\end{tabular}
	
	\caption{\textbf{An example of some \(\chi^2\) maps around a minimum?} The is currently an issue with these not finding good values. Should I remove the grid?}
\end{figure}

\begin{figure}
	\centering
	\includegraphics[width=0.4\linewidth]{images/tmp-images/temp_increment}
	\includegraphics[width=0.4\linewidth]{images/tmp-images/comp_increment}
	\includegraphics[width=0.4\linewidth]{images/tmp-images/comp_logg_increment}
	\includegraphics[width=0.4\linewidth]{images/tmp-images/comp_logg_incremnt}
	\includegraphics[width=0.4\linewidth]{images/tmp-images/host_feh_increment}
	\includegraphics[width=0.4\linewidth]{images/tmp-images/host_logg_increment}
	
	\caption{Statistical differences in binary models between incremental grid step steps of one component. Each plot shows the median difference between two successive grid steps , all other parameters fixed. The error bars indicate the standard deviation in the difference better indicating the magnitude of change. The increments of the model \(\rm Teff, logg\) and \([Fe/H]\) are shown in the top, middle and bottom plots respectively. The first column shows the changes due to the host star changing, while the second column show results for varying the companion parameters. Each is performed for 3 values of the same parameter in the other component.
	The parameters used for the non-varied parameters in each instance are [6000, 4.5, 0.0] [XXX, 4.5, 0.0]. Note the magnitude of the differences for each parameter. 
	\textbf{These plots are not correct. legends are wrong etc. and some are duplicates. need to remake nicer.}}
	\label{fig:comp_increment}
\end{figure}



\begin{table*}
	\caption{Semi-amplitude and velocity measurements of Equation~\ref{eqn:q_relation}.}
	\label{tab:q_relation}
	\begin{tabular}{lcccccc}
		\toprule
		Object      & \(\textrm{K}_{1} \) & \(v_{1}\)\tablefootmark{a} & \(\textrm{M}_{2}\sin{i} \) & q & \(\textrm{K}_{2}\)\tablefootmark{b} & \(v_{2}\)\tablefootmark{a}\tablefootmark{b}    \\
		&  m/s & m/s & m/s &   & m/s & m/s \\
		\midrule
		\object{HD 4747}     &  703.3   & v1 & 39.6  &   &   &      \\
		\object{HD 4747}     &  755.3   & v1 & 39.6  &   &   &      \\
		\object{HD 162020}   & 1813.    & v1 & 14.4  &   &   &     \\
		\object{HD 167665}   &  609.5   & v1 & 50.3  &   &   &    \\
		\object{HD 168443}b  &  475.133 & v1 & 17.2  &   &   &   \\  %
		\object{HD 168443}c  &  297.70  & v1 &       &   &   &   \\
		\object{HD 202206}b  &  564.75  & v1 & 17.4  &   &   &   \\
		\object{HD 202206}c  &   42.01  & v1 &       &   & k2 & v2   \\
		\object{HD 211847}   &  291.4   & v1 & 19.20 &  &   &     \\
		\object{HD 30501}    & 1703.1   & v1 & 62.3  &   &   &      \\
		\bottomrule
	\end{tabular}
	\tablefoot{
		\tablefoottext{a}{Velocity only due to orbit, not including constant drift velocity.}
		\tablefoottext{b}{Maximum \(\textrm{K}_{2}\) based on companion minimum mass \(\textrm{M}_{2}\sin{i} \).}
	}
\end{table*}
