%!TEX root = ../../thesis.tex

\section{Measuring the RV}
\todo{something about measuring the RV and CCF}
The motion of the star towards and away from the Earth shifts the lines of its spectra through he Doppler Effect. In the non-relativistic limit this can be written as:
\begin{equation}
\frac{\Delta\lambda}{\lambda} = \frac{v}{c},
\end{equation}
where $\Delta\lambda$ is the wavelength shift of wavelength $\lambda$ with a velocity relative to the observe of \(v\). The \emph{speed of light} is the constant $c$. 
To measure the RV the relative positions of the stellar lines need to be measured over time. Typically this is done via the cross-correlation (CCF) of the observed spectrum with a template mask\footnote{binary mask~\citep{baranne 1996} or weighted~\citep{pepe 2002} mask} suitable for the spectral type of the observed star \citep[e.g.][]{Baranne 1996, pepe 2002}. The {CCF} stacks together the spectral lines crating and ``average'' line, reducing the random noise on the individual $N$ spectral lines by a factor of $\sqrt{N}$. The {CCF} collapses the RV from all the lines into one number with higher precision than individual lines.


\section{RV precision}
\label{section:rv_precision}
One important factor in RV measurements is the level of precision attainable from the stellar spectra in the presence of noise. The fundamental source of noise is photon noise, which follows a Poisson distribution. That is an observable with an average value of $N$ has a standard deviation $\sqrt{N}$.

A very general formula for the RV precision of a spectrum, in terms of general spectral parameters is given by~\citet{hatzes_spectrograph_1992} as:

\begin{equation}
\sigma \propto \frac{1}{\sqrt{F} \sqrt{\delta \lambda} R^{1.5}}.
\end{equation}
$\sqrt{F}$ represents the S/N of the spectrum in the poison-dominated noise regime, while $\Delta \lambda$ is the bandwidth of the observed spectrum. {red{} This assumes that the spectrum contains a homogeneous distribution of uniform lines, per unit wavelength. \todo{Finish this relation, R 1.5}}


Pedros school section other precision source \({r}^{1.5}\)\todo{}


An alternate derivation comes from \citet{bouchy_fundamental_2001} in which the optimal weight for each pixel is calculated for a spectrum $A_0$ via:
\begin{equation}
    W(i) = \frac{\lambda^{2}(i) {(\partial A_0(i)/\partial\lambda)}^{2}}{A_0(i) + \sigma_D} \label{eqn:pixel_weigth}
\end{equation}
with the RV precision calculated over all pixels as
\begin{equation}
    \delta v_{RMS} = \frac{c}{\sqrt{\Sigma_i W(i)}}.
\end{equation}
where $\lambda$ the wavelength and $\sigma_D$ the detector noise. An important thing to note from \cref{eqn:pixel_weigth} is that the RV information comes from the gradient of the spectral lines, and therefore the edges or wings of the spectral lines.

These formula have recently been used to assess the theoretical RV precision of synthetic spectra for the development of instrument designs of new \nir{} spectrographs~\citep[e.g.][]{figueira_radial_2016} as well as compare precision of real and synthetic spectra~\citep[e.g.][]{artigau_optical_2018}.
 
The RV precision will be analysed further in \cref{cha:nir_content} in which a detailed derivation of the \citet{bouchy_fundamental_2001} method is given and implemented to analysis the precision of \nir{} spectra.
