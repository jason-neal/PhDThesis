%!TEX root = ../../thesis.tex

\section{Measuring the RV}
The motion of the star towards and away from the Earth shifts the lines of its spectra through he Doppler Effect.
In the non-relativistic limit this can be written as:
\begin{equation}
\frac{\Delta\lambda}{\lambda} = \frac{v}{c},
\end{equation}
where $\Delta\lambda$ is the wavelength shift of wavelength $\lambda$ with a velocity relative to the observe of \(v\).
The \emph{speed of light} is the constant $c$.
To measure the RV the relative positions of the stellar lines need to be measured over time.
Typically this is done via the cross-correlation ({CCF}) of the observed spectrum with a template mask (e.g. a binary mask~\citep{baranne_elodie_1996} or weighted mask~\citep{pepe_coralie_2002}) suitable for the spectral type of the observed star.
The {CCF} stacks together the spectral lines creating an ``average'' line, reducing the random noise on the individual $N$ spectral lines by a factor of $\sqrt{N}$.
The {CCF} collapses the RV from all the lines into one number with higher precision than individual lines.


\section{RV precision}
\label{section:rv_precision}
To achieve a high-level RV precision, two sources of noise must be controlled: statistical error from the RV measurement and systematic errors induced by the spectrograph.
This requires a stabilized spectrograph, precisely calibrated in wavelength, and the combination of thousands of lines~\citep[e.g.][]{pepe_instrumentation_2014}.
The fundamental source of noise is photon noise, which follows a {Poisson} distribution.
That is, an observable with an average value of $N$ has a standard deviation $\sqrt{N}$.

A very general formula for the RV precision achievable of a given  spectrum, in terms of general spectral parameters is given by~\citet{hatzes_spectrograph_1992} as:
\begin{equation}
\sigma \propto \frac{1}{\sqrt{F} \sqrt{\delta \lambda} R^{1.5}}.
\end{equation}
$\sqrt{F}$ represents the signal-to-noise ratio (\snr{}) of the spectrum in the Poisson-dominated noise regime, while $\Delta \lambda$ is the bandwidth of the observed spectrum. This assumes that the spectrum contains a homogeneous distribution of uniform lines, per unit wavelength.
The precision depends more steeply on the spectral resolution with $R^{-1.5}$ occurring due  to the bandwidth and the number of lines visible on  a fixed detector size decreases with increased resolution.
In high resolution RVs this tends to $R^-{1}$ indicating the higher precision at the expense of wavelength coverage is desired~\citet{hatzes_spectrograph_1992}.
As a star rotates its lines become broadened by an area preserving rotation kernel (see \cref{subsec:rotational_convolution}).
In a similar way to the resolution the RV precision is dependant on rotational velocity of the star (\Vsini) to the power 1.5.

Assuming a spectral line is comprised of Gaussian-shaped absorption lines it, the precision of a line can be shown to be:
\begin{equation}
    \sigma\sim\frac{\sqrt{\fwhm{}}}{C \cdot \snr}
\end{equation}
where \fwhm{} is the full width at half maximum of the line, $C$ is the line contrast (depth relative to continuum) and \snr{} the signal-to-noise of the continuum.
An alternate derivation comes from \citet{bouchy_fundamental_2001} in which the optimal weight for each pixel is calculated for a spectrum $A_0$ via:
\begin{equation}
    W(i) = \frac{\lambda^{2}(i) {(\partial A_0(i)/\partial\lambda)}^{2}}{A_0(i) + \sigma_D} \label{eqn:pixel_weigth}
\end{equation}
with the RV precision calculated over all pixels as
\begin{equation}
    \delta v_{RMS} = \frac{c}{\sqrt{\Sigma_i W(i)}}.
\end{equation}
where $\lambda$ is the wavelength and $\sigma_D$ the detector noise.

In all these formulations there are some key properties to achieve high RV precision: a high stellar flux $F$ to achieve a high \snr{}, observed at high resolution to have sharp spectral lines.
The deeper (high C) and narrower (small \fwhm{}) a spectral line\footnote{With steeper gradients $\partial A_0(i)/\partial\lambda$.}, the better defined the position of the line will be, allowing for a higher precision measurement of the RV.
The flux $F$ or \snr{} of an observation increases with the number of photons collected which depends on the stellar brightness as well as distance\footnote{Basically the stars apparent magnitude}, coupled with the telescope size and with the exposure time.
This limits the highest precision RV measurements to relatively nearby stars.
Cross-dispersed echelle spectrographs mounted on the largest telescopes are capable of delivering the high-resolution, high \snr{} and high bandwidth requirements necessary to achieve high precision RV measurements.

These formula have recently been used to assess the theoretical RV precision of synthetic spectra for the development of instrument designs of new \nir{} spectrographs~\citep[e.g.][]{figueira_radial_2016} as well as compare precision of real and synthetic spectra~\citep[e.g.][]{artigau_optical_2018}.
The RV precision will be analysed further in \cref{cha:nir_content} with a detailed derivation of the \citet{bouchy_fundamental_2001} method given and implemented to analyse the precision of \nir{} spectra.
