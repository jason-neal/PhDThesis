%!TEX root = ../thesis.tex
% Tools for spectral analysis

\chapter{Spectroscopic reduction} % Main chapter title
\label{cha:reduction}

The work of this thesis relies on the use of \nir{} spectra obtained by the {CRIRES} instrument.
This chapter contains an overview of the steps undertaken to extract astronomical spectra, focusing on the {CRIRES} instrument specifically.
A comparison between two different reduction pipelines for {CRIRES} is performed.
Details of the issues encountered with the reduction output are presented.
The post reduction steps performed, specifically wavelength calibration and telluric correction, are also included in this chapter.
The reduced spectra produced in this chapter will be analysed in \cref{cha:direct_recovery} and \cref{cha:model_comparison}.
We wrap up this chapter by highlighting other techniques in high resolution \nir{} spectra reduction and calibration which may have improved some of the results.

\section{Summary of dataset}
\todo{}{}
The work performed relies on high resolution \nir{} spectra, obtained around 2.1\um{} with {CRIRES}.
In brief, we have 17 individual observations of 7 target stars obtained in 2012.
Each observation contains 8 individual observations performed in an {ABBAABBA} nod cycle pattern.
Here we detail the steps performed to extract the stellar information from the raw observations.
Further details of the observations and the motivation for them are given in \cref{cha:direct_recovery}.


\subsection{General reduction Concepts}
\label{subsec:nirreduction}

Focus on our observations\todo{}

There are 3 main effects that need to be accounted for in \nir{} spectroscopy which influence the observations and calibrations taken.
We briefly detail them here and how they are corrected...

%\begin{figure}[h]
%\centering
%%\includegraphics[width=0.4\textwidth]{figures/reduction/Master_Darks.png}
%\includegraphics[width=0.9\textwidth]{figures/reduction/master_darks_1.pdf}
%\caption{Master dark frames for exposure times of 3 and 180 seconds. Each master is created by averaging 3 images in which the detector received no incident light. Both frames are on the same scale and show dark current grows with exposure time. The colour has been inverted so that black is the recorded measurement.}
%\label{fig:darkcurrent}
%\end{figure}
\begin{figure}[h]
    \centering
    %\includegraphics[width=0.4\textwidth]{figures/reduction/Master_Darks.png}
    \includegraphics[width=0.45\textwidth]{figures/reduction/MasterDarkFlat_1.png}
    \includegraphics[width=0.45\textwidth]{figures/reduction/MasterDarkSpec_1.png}
    \caption{Master dark frames for exposure times of 3 and 180 seconds.
Each master is created by averaging 3 images in which the detector received no incident light.
Both frames are on the same scale and show dark current grows with exposure time.
The colour has been inverted so that black is the recorded measurement.}
    \label{fig:darkcurrent_colour}
\end{figure}

\subsubsection{Dark Current}
\label{subsubsec:darkcurrent}
The dark current is a form of instrumental noise, in which the detector measures photons while not being illuminated.
It is the detection of thermal electrons moving inside the detector, creating spurious photon counts.
Calibration and removal of the dark current is performed by taking exposures in which the detector is not illuminated.
The detectors in the {CRIRES} instrument are cryogenically cooled to a temperature of \(\sim 27\)\K{} (within 0.1\K{}) to significantly reduce the dark current, and to maintain consistency.
The electrical components of the {CRIRES} detectors create thermal energy while operational which impacts the dark current.
A strong glow is observed in the bottom corners of the {CRIRES} detector in \cref{fig:darkcurrent_colour} due to nearby amplifiers.
As per the {CRIRES} calibration plan, ``dark frames'' need to be taken for each exposure time used. \Cref{fig:darkcurrent_colour} shows the master dark frame created from averaging three dark frames for exposure times of 3 seconds (for the flats) and 180 seconds (for the science), both on the same amplitude scale.
For the {CRIRES} detectors the dark current per pixel is around 0.2-0.4\,(\(e^{-}\)\si{\per\second}), while the glow at the two corners of the 180 second exposure shown here is around \(\sim9000 / 180\approx50\)\,\(e^{-}\)\si{\per\second}.

%
%\begin{figure}[h]
%    \centering
%    \includegraphics[width=0.9\textwidth]{figures/reduction/master_flats_1.pdf}
%    \caption{A flat-field image for detector \#1 before (left) and after (right) the non-linearity corrections are performed.
%    A perfect detector would have all pixels in the flat-field equal to 1. \bf{\red{} Remove flat and flatR titles.}}
%    \label{fig:masterflats}
%\end{figure}


\begin{figure}[h]
    \centering
  %  \begin{tabular}[ll]
        \includegraphics[width=0.45\textwidth]{figures/reduction/Flat_2.png} %
        \includegraphics[width=0.45\textwidth]{figures/reduction/FlatR_2.png} %\\
    %\end{tabular}
    \caption{A flat-field image for detector \#2 before (left) and after (right) the non-linearity corrections are performed.}
    \label{fig:masterflats_colour}
\end{figure}


\subsubsection{Flat-field}
\label{subsubsec:flat-field}
No detector is truly perfect, with all pixels performing equally well.
To correct for pixel-to-pixel variations in photon sensitivity across the detector and for any distortions in the optical path, a flat-field correction is needed.
Exposures of a uniform\footnote{Ideally uniform intensity and spectral distribution} light source are taken, allowing the individual pixel-to-pixel sensitivity to be measured and corrected.
The flat-field frames are corrected for dark current by subtraction of the master dark frame with the appropriate exposure time.

The {CRIRES} detector suffers from nonlinearities in sensitivity across the detector.
This can be seen in the flat-field image on the left of \cref{fig:masterflats_colour} where there is an intensity gradient from dark to light across the detector.
A set of coefficients for each pixel is provided by {ESO}\footnote{Available at \href{https://www.eso.org/sci/facilities/paranal/instruments/crires/tools.html}{https://www.eso.org/sci/facilities/paranal/instruments/crires/tools.html}} to apply the correction for the nonlinearity of the detectors.
This also corrects for a slight difference in sensitivity between the pixels from the odd and even columns in the {CRIRES} detectors, commonly called the ``odd-even effect''.
The frame on the right of \cref{fig:masterflats_colour} has been corrected for the nonlinearities.

The flat-field can also reveal defects in the detector e.g.\ dead pixels, dust or scratches; a large scratch is show clearly visible in \cref{fig:masterflats_colour}.

\subsubsection{Nodding and Jitter}
\label{subsec:nod-jitter}
The technique of \emph{nodding} is used to remove sky emission, detector dark current, and glow.
First, an observation is divide into multiple separate exposures.
Between each exposure the telescope is moved to change the vertical position of the target in the slit.
The light from the star travels through a slightly different optical path and is recorded on a different part of the detector.
The frames from the two nod positions (A, B) are then subtracted (A-B) to remove the background measurement from each spectra.

A visual example of the nodding is shown in \cref{fig:nodimages}.
On the left are slices of 150 pixel columns from successive nod positions A and B as well as the difference A-B.
On the right is a single pixel column from each image on the left.
The background signal at the level of 20-30 counts in the image is almost cancelled out by the opposite nod.
This efficiently removes the background signal/noise from the observed spectra target.

Observations of faint targets, that need long exposure times, are also broken up into multiple images so that the instrument glow from \cref{fig:darkcurrent_colour} does not saturate the detector.

\todo{FIXup this table}
\newcolumntype{Y}{>{\centering\arraybackslash}X}  % Tablularx centered column for eqaul columns (nod cycle image)
%\newcolumntype{Y}{>{\Centering}X}  % Tablularx centered column for eqaul columns (nod cycle image)

\begin{figure}
    \centering
    % Using new column type Y
    %\begin{tabularx}{0.7\textwidth}{|*6{>{\Centering\arraybackslash}X}|}    % 3 Y columns (equaly spaced) 1 extra column

      \begin{tabularx}{0.75\textwidth}{|*6{Y|}|}    % 3 Y columns (equaly spaced) 1 extra column
      A & B & A-B& & & \\
        % \includegraphics[width=0.4\textwidth]{figures/reduction/nod_image_sample.pdf} &
        \multicolumn{3}{|c|}{\includegraphics[width=0.4\textwidth]{figures/reduction/Nods_AB_A-B.png}} &
        \multicolumn{3}{|c|}{\includegraphics[width=0.37\textwidth]{figures/reduction/nod_slice_example.pdf}}\\
    \end{tabularx}
    \caption{Illustration of the nodding technique.
Left: Sample slice of successive images at nod positions A and B, and their difference A-B for detector \#1.
Right: A vertical slice along the slit at column 512 (middle of detector).
The background level observed in A and B is effectively removed by the subtraction.}
    \label{fig:nodimages}
\end{figure}
\todo{fix \cref{fig:nodimages}}

A small random vertical offset is applied to each observation which ensures that all spectra at the same nod position do not consistently land on the same pixels each time.
This is known as the \emph{jitter} and allows for the correction of bad pixels and decreases the effect from any systematics of the detector.

\change{move}{As an example, the {CRIRES} observations analysed for the work performed in \cref{cha:direct_recovery,cha:model_comparison} were performed with an {ABBAABBA} nod cycle pattern with an exposure time of 180\si{\second} each, or 24 minutes all up.}


\subsection{\thar{} lamp calibration}
\label{subsec:th-ar}
As part of the {CRIRES} calibration procedure, spectra are taken of \thar{} lamps.
The \thar{} spectra are placed into the instrument using 6 optical fibres, creating 6 uniformly spaced spectra across the detector.
\cref{fig:caliblamps} contains the \thar{} spectra for all four detectors.
There are \(\sim50\) \thar{} lines that fall across the four detectors for the wavelength setting here, although most of them are faint.

The purpose of the \thar{} fibres is to perform cross-correlation against a \thar{} template to obtain a wavelength solution for each detector.
Optical defects (e.g.\ scratches on the detector) interfere with the performance of the correlation between \thar{} lines and the spectral template in {ESO} pipeline.
This can be resolved by first applying a pixel mask (although we did not attempt it).
At the top and bottom there are also two meteorological fibres that can not be used for wavelength calibration as they pass through a different optical pathway.
The brightest one at the bottom has strong features that seem to overwhelm or washout many columns in detector \#2--4 (vertical stripes). \unfinished{Check if it is mentioned how these get corrected/ if the impact the wavelength solution.}
This information is included here for completeness, as the \thar{} were not used in the chosen data reduction method, see \cref{subsec:wavecalib}.

%\begin{figure}
%\includegraphics[width=\hsize]{./figures/reduction/lamp_plots_cbar_each.pdf}
%\caption{A \thar{} calibration lamp frame for each detector, corrected from the dark current.}
%  \label{fig:caliblamps}
%\end{figure}

\begin{figure}
    \begin{tabular}{cc}
         \includegraphics[width=.45\hsize]{./figures/reduction/Thar_1.png} & \includegraphics[width=.45\hsize]{./figures/reduction/Thar_2.png} \\
         \includegraphics[width=.45\hsize]{./figures/reduction/Thar_3.png} & \includegraphics[width=.45\hsize]{./figures/reduction/Thar_4.png} \\
    \end{tabular}

    \caption{Example \thar{} calibration lamp frames for each detector.
        These are the raw frames in which dark current correction has not been performed (i.e the dark current is still visible on the bottom corners).}
    \label{fig:caliblamps}
\end{figure}

\todo{Note: Useful information about {MIR} reduction in the
\href{https://www.eso.org/sci/facilities/paranal/instruments/visir/doc/VLT-MAN-{ESO}-14300-3514_2018-02-01.pdf}{VISIR manual}}

\subsection{Extraction}
\label{subsec:extraction}
The process of extraction is, in brief, turning the two-dimensional image of the spectrum into a one-dimensional representation.
This is done by summing the photon counts along the spatial direction, for each column in the dispersion direction in a small window around the spectrum.
To do this one needs to identify the position of the spectrum across the detector, referred to as {order tracing}.
In our case there is only one spectral order present but in a cross-dispersed configuration there can and will be many orders.
A rectangular box, with a specified aperture width, is centred along the traced spectrum.

There are two types of extraction commonly used.
The \emph{rectangular} extraction performs a simple aperture sum in the spatial direction, counting all photon counts that fall within the aperture.
An \emph{optimal} extraction~\citep{horne_optimal_1986} also includes variance weighting to reduce the impact of the noise and deviant pixels on the spectral extraction.
A spatial profile is fitted to the spectra and the pixels weighted such that the those furthest away from the centre have less impact on the sum.
Optimal extraction can increase the effective exposure time by up to 1.69 times at \(3 \sigma\)~\citep{horne_optimal_1986}.
Both rectangular and optimal extraction methods are available from both the {ESO} {CRIRES} and {DRACS} pipeline, compared below.

\section{Pipeline Comparison}
\label{sec:pipelines}
\change{Reword}{To be able to extract the target spectra from the calibration and observation images, a number of steps (some outlined above in \cref{subsec:nirreduction}) have to be performed in sequence.} The series of steps, performed by various software tools, is referred to as a \emph{pipeline}.
Each stage in the pipeline performs a specific task, for example creating the master dark frame, or performing the nod subtraction.
The result of one stage is passed to the next (either automatically or manually).
Two different pipelines were available to reduce the {CRIRES} observations used in this work.
The first is the standard {CRIRES} pipeline\footnote{\href{https://www.eso.org/sci/software/pipelines/}{https://www.eso.org/sci/software/pipelines/}}, available from {ESO}.
The second is an in-house pipeline previously used in~\citet{figueira_radial_2010} called {DRACS} (Data Reduction Algorithm for {CRIRES} Spectra) \change{Pedro it is my understanding that the {ESO} pipeline did not exist/was not publicly available when you created {DRACS}, correct?}

In these next sections we document our experience using both pipelines, comparing the extracted spectra and user experience.


\subsection{{ESO} {CRIRES} pipeline}
\label{subsec:eso-crires}
The {ESO} {CRIRES} pipeline was used to reduce {CRIRES} nodding spectra following direction from the {CRIRES} pipeline user manual\footnote{\href{ftp://ftp.eso.org/pub/dfs/pipelines/crires/crire-pipeline-manual-1.13.pdf}{ftp://ftp.eso.org/pub/dfs/pipelines/crires/crire-pipeline-manual-1.13.pdf}} and the {CRIRES} reduction cookbook\footnote{\href{https://www.eso.org/sci/facilities/paranal/instruments/crires/doc/VLT-MAN-{ESO}-14200-4032\_v91.pdf}{https://www.eso.org/sci/facilities/paranal/instruments/crires/doc/VLT-MAN-{ESO}-14200-4032\_v91.pdf}}

The GASGANO\footnote{\href{https://www.eso.org/sci/software/gasgano.html}{https://www.eso.org/sci/software/gasgano.html}} graphical user interface (GUI) was used to interact with the pipeline with guidance from the GASGANO manual\footnote{\href{https://www.eso.org/sci/software/gasgano/VLT-PRO-{ESO}-19000-1932-V4.pdf}{https://www.eso.org/sci/software/gasgano/VLT-PRO-{ESO}-19000-1932-V4.pdf}}.
This pipeline provides a number of \emph{recipes} which perform the required extraction steps.
From the GUI each recipe is manually selected, then the correct calibration and observation files need to be selected to use with each recipe.
The final output from the {ESO} pipeline is a table in a fits format with the combined extracted spectra (both \emph{rectangular} and \emph{optimal} extractions), pixel errors and a wavelength solution.

For a novice of spectral extraction this pipeline and the available documentation was very helpful to get started and perform the extraction.
However, to reduce many spectra it soon became a long tedious process.

Constant revision of the documentation was necessary to ensure all the correct image and calibration files were added to each recipe.
The {ESO} pipeline makes all the recipe parameters easily accessible to modify via a window of the recipe interface.
This is great for adjusting the parameters and for identifying which parameters are relevant to each recipe.
When trying to experiment with the recipe parameters to achieve high quality spectra extraction it was repetitive to change the same parameters for each observation.
It was also difficult to keep track of all the changes while assessing their effect on the final output.
The recipe parameter defaults get restored for each observation, making it difficult to reduce all of the observations in a consistent manner.

The parameters for the wavelength calibration were the most tedious.
To try and improve the wavelength calibration, the {y-positions} of the 6 \thar{} fibres were manually found from the images for each detector and every observation and entered as input parameters for the calibration recipe.
This helped the wavelength calibration recipe to correctly identify/fit more of the \thar{} spectra in most cases, but this took time.
Because of this, it was chosen to use the other pipeline and these calibration were not used.

{ESO} has a new reduction ``workflow'' called {ESO} Reflex\citep{freudling_automated_2013}\footnote{\href{https://www.eso.org/sci/software/esoreflex/}{https://www.eso.org/sci/software/esoreflex/}}.
This enables automated reduction with the ability to chain together the extraction recipes in the specific order desired, repeat steps to optimize the reduction, and automatically handle the data organization (no need to manually select the files for each recipe).
This would likely have enabled a quicker and more consistent reduction of the spectra.
Unfortunately, it is not available for the {CRIRES} pipeline.

\subsection{{DRACS}}
\label{subsec:dracs}
{DRACS} (Data Reduction Algorithm for {CRIRES} Spectra) is a custom reduction pipeline~\citep{figueira_radial_2010} written in {IRAF}'s CL\footnote{{IRAF} is distributed by the National Optical Astronomy Observatories, which are operated by the Association of Universities for Research in Astronomy, {Inc.}, under cooperative agreement with the National Science Foundation.}~\citep{tody_iraf_1993}.
It provides for automated dark and nonlinearity corrections (using the nonlinearity coefficients provided by {ESO}), as well as the flagging and replacement of bad pixels.
Sensitivity variations are corrected by dividing by a flat-field which was corrected from the blaze function effect.
The nodding pairs are mutually subtracted and the order tracing is accomplished by fitting cubic splines.
Order tracing allows the extraction algorithm to follow the shape for the dispersion across the detector \unfinished{is footnote correct ``instruments optics''}\footnote{The spectra do not fall perfectly horizontally on the detector and usually contain some curvature due to the optics of the instruments.}.
By default the pipeline returns the \emph{optimal} extraction~\citep{horne_optimal_1986}, but the \emph{rectangular} extraction can also be obtained.
The extracted spectra from each nod is continuum normalized by dividing by a polynomial fitted to the continuum, with the polynomial degree selected individually for each spectrum and detector.
Finally, the normalized nod-cycle spectra are averaged together to give a single reduced spectrum, normalized to 1.

The {DRACS} pipeline was originally developed to work with \emph{H}-band spectra.
Because of this, some of the parameters were adjusted in an attempt to achieve a better reduction on the \emph{K}-band spectra analysed here.
The parameters adjusted were the tracing and normalization polynomial functions and their specific order on the detector.
Also, there was initially no reduction parameters present for detector \#3 (as this had been not used with in previous works using this pipeline).
Therefore, the pipeline was extended to be able to reduce detector \#3 of CRIRES, by adjusting the pipeline code and finding suitable parameters for detector \#3.

When using the {DRACS} pipeline on a new target, the tedious part is initially creating the lists of files identifying which files are the dark, flat, and the science frames\footnote{The GASGANO GUI was helpful to identify the distinction of each fits file for a beginner.}.
After these lists are created, the reduction pipeline can be run and re-run easily, making the effort worth it.

The first time through is interactive with a number of manual checks decisions to be made (e.g.\ confirming the order tracing position, and fit is good).
{DRACS} remembers choices in a database, allowing for a reduced level of user interaction a second or third time through the pipeline.
This was useful to experiment with and iterate the reduction parameters of the pipeline.
When changing the parameters which affect the order tracing, the database with the order tracing results needed to be deleted so that it would not influence the new fits.
These parameters were therefore slower to iterate on.

This semi-autonomous nature of the {DRACS} pipeline means that all the spectra can be reduced in a consistent way, relatively quickly, and does not require manual spectra selection for each individual recipe as was done with the {ESO} pipeline.

There are a couple of drawbacks with using the {DRACS} pipeline.
The first is the lack of documentation on how to use it.
This required looking through the source code to find which functions and scripts do which part of the extraction.
{DRACS} makes use of many {IRAF} packages which have documentation available online\footnote{Such as at the \href{Space Telescope Science Institute}{http://stsdas.stsci.edu/gethelp/pkgindex\_noao.html}.}.
Searching through this documentation was difficult, but required, to understand how to use and modify {DRACS}.

The second drawback is that {DRACS} does not perform wavelength calibration, which is included with the {ESO} pipeline.
This means an external wavelength calibration was the only option (see \cref{subsec:wavecalib}).
The last drawback is the discovery of artefacts present in the nod spectra, which we discuss in detail in \cref{subsubsec:reductionartefacts}.

\subsection{Pipeline comparison and selection}
\label{subsec:pipeline-selection}
\begin{figure}
    \begin{tabular}{cc}
        \includegraphics[width=0.5\linewidth]{figures/reduction/pipeline_compare/pipeline_compare_HD30501-1_chip_1} & \includegraphics[width=0.5\linewidth]{figures/reduction/pipeline_compare/pipeline_compare_HD202206-1_chip_1}\\
        \includegraphics[width=0.5\linewidth]{figures/reduction/pipeline_compare/pipeline_compare_HD30501-1_chip_2} & \includegraphics[width=0.5\linewidth]{figures/reduction/pipeline_compare/pipeline_compare_HD202206-1_chip_2}\\
        \includegraphics[width=0.5\linewidth]{figures/reduction/pipeline_compare/pipeline_compare_HD30501-1_chip_3} & \includegraphics[width=0.5\linewidth]{figures/reduction/pipeline_compare/pipeline_compare_HD202206-1_chip_3}\\
        \includegraphics[width=0.5\linewidth]{figures/reduction/pipeline_compare/pipeline_compare_HD30501-1_chip_4} & \includegraphics[width=0.5\linewidth]{figures/reduction/pipeline_compare/pipeline_compare_HD202206-1_chip_4}\\
    \end{tabular}
    \caption{Comparison between the {ESO} pipeline and {DRACS} pipeline for two observations, {HD30501-1} and {HD202206-1}.
The blue lines are the extracted spectra from the {ESO} pipeline, the orange dashed lines are the combined optimal extraction from the {DRACS} pipeline, and the green dash-dotted line is the modified {DRACS} extraction (after removing artefacts addressed in \cref{subsubsec:reductionartefacts}).
The wavelength information applied to both spectra is from the {ESO} pipeline.}
    \label{fig:reduction-comparison}
\end{figure}

Both reduction methods were applied to the same {CRIRES} spectra to check the quality and consistency of both methods.
Two examples of the extraction for HD30501-1 (left) and HD202206-1 (right) from both pipelines are provided in \cref{fig:reduction-comparison}.
The blue lines are the extracted spectra from the {ESO} pipeline, the orange dashed lines are the optimal extraction from the {DRACS} pipeline, while the green dash-dotted line is the {DRACS} extraction after dealing with artefacts in the optimal extraction addressed in \cref{subsubsec:reductionartefacts}.

One of the important things checked was the line depths of each spectra to ensure that the pipelines were consistent.
The {ESO} pipeline has noticeable issues, with many large spikes still present in the spectra, likely caused by bad pixels or cosmic rays that are not correctly removed.
There also appears to be problem with edges of the {ESO} reduced spectra, with very large spikes at either end.

At this stage the {DRACS} pipeline was chosen, as it was considered that the {DRACS} pipeline produced better extracted spectra than the {ESO} pipeline.
This decision was based on the quality of the reduced spectra, as well as the relative ease of use of the pipeline (being semi-automated once set up).
Another deciding factor was the need to create software to remove the bad pixels observed in the {ESO} reduced spectra.
However, this was unavoidable as it was eventually required for the {DRACS} reduced spectra\footnote{see \cref{subsubsec:reductionartefacts}!}.
Though the {ESO} pipeline provided a wavelength solution for the spectra, this did not factor into the decision as it was considered too unreliable due to known issues with {CRIRES} wavelength calibration, necessitating a new wavelength calibration anyway.

The spectra for the pipeline comparison in \cref{fig:reduction-comparison} are the combination of the 8 nod spectra.
Later, it was discovered that individual nod spectra from the {DRACS} pipeline had issues (see \cref{subsubsec:reductionartefacts}).
These are difficult to notice on this scale as they each constitute one eighth of the information in the combined spectrum.
An example of this is seen in detector \#2 of {HD202206-1} in \cref{fig:reduction-comparison}.
In \cref{fig:resizednods} an artefact from a single nod spectra, is barely visible in the pipeline comparison of \cref{fig:reduction-comparison}, shown as a slight depression of the orange dashed line between 2\,132 and 2\,134\nm{}.
The identification of these artefacts and how they are removed (green lines) are explained in the following section.

\subsubsection{Reduction issues}
\label{subsubsec:reductionartefacts}
\textbf{\todo{After X thing}{At a later date}} some large artefacts in the {DRACS} extracted spectra were identified.
An investigation into the cause of these artefacts was undertaken to identify their source and remove them from the spectra.
Separating out the individual nod spectra side-by-side revealed that the artefacts were occurring in only a few of the individual nod spectra from the optimal extraction, and that they were not present in the rectangular extracted nods. \textbf{Four} examples are shown in \textbf{\cref{fig:resizednods} and blah}.
In each panel the top sub-plot contains the optimal extracted spectra, while the middle sub-plot contains the rectangularly extracted spectra.


The occurrence of artefacts in the observations did not appear to have a pattern with nod position or detector. \textbf{They do occur with large spikes in the rectangular extractions.
There are many other spikes in the rectangular extractions that do not create the artefacts observed.
}\todo{this doesn't quite flow}
It is clear that the artefacts in the optimally extracted nods correspond to large pixel spikes present in the rectangular extracted nod.
As mentioned in \cref{subsec:extraction} the \emph{optimal} extraction includes variance weighting across the spatial direction.
It appears that the presence of cosmic rays or bad pixels heavily affected the variance weighting procedure during the \emph{optimal} extraction.
It is also observed that the artefacts created are not localized to the region around the bad pixel region but affect an extended spectral range (100's of pixels in some cases).

Numerous parameters in the {DRACS} pipeline were experimented with to try and remove the observed artefacts with limited success.
For instance, no complete removal of the artefacts was found by manually changing the \(\sigma\) rejection limits (between \(1-5 \sigma\)) and increasing the tracing width parameter in {IRAF}s DOSLIT\footnote{Documentation for DOSLIT can be found here \href{http://stsdas.stsci.edu/cgi-bin/gethelp.cgi?doslit}{http://stsdas.stsci.edu/cgi-bin/gethelp.cgi?doslit}} recipe, although it did slightly affect the shape of the artefacts.
During the creation of this document it was discovered that enabling the automated aperture resizing\footnote{Using \href{apresize}{http://stsdas.stsci.edu/cgi-bin/gethelp.cgi?apresize.hlp}} by setting the ``resize'' parameter to yes manages to remove one of the large artefacts in the observation of {HD202206-1}.
We show the effect of this in \cref{fig:resizednods} with the optimal and rectangular extraction from the {DRACS} pipeline with only the ``resize'' parameter changed.
The artefact in the \nth{6} nod is removed by enabling the automatic ``resize'' but the artefact in the \nth{7} nod is still present.

\begin{figure}
    \centering
    \begin{tabular}{cc}
    \includegraphics[width=0.5\linewidth]{figures/reduction/bp_plots/non_resized_nods_HD202206-1_chip_2} & \includegraphics[width=0.5\linewidth]{figures/reduction/bp_plots/resized_nods_HD202206-1_chip_2}\\
    \end{tabular}
    \caption{{DRACS} reduction for detector \#2 of HD202206-1 with a single pipeline parameter changed.
Left: Reduction with the DRACS pipeline with the ``doslit.resize'' set to ``no''.
Right: The reduction from the same pipeline but with ``doslit.resize'' set to ``yes''.
During the order tracing the aperture size is automatically adjusted while the rest of the parameters remain identical.
With this one parameter changed one large artefact is removed but a second one is not removed.}
    \label{fig:resizednods}
\end{figure}

This discovery came too late to implement for all of the observations as it would require all of the analysis to be repeated re-doing all of the analysis.
However, it does indicate that some improvements may be obtained with extensive time and patience tweaking the parameters of the {DRACS} pipeline.

The artefacts in the individual nod spectra were observed to create flux deviations in the fully combined optimally extracted spectra of around \(0.5\%\).
The purpose of these spectra was to detect faint companion spectra with expected flux ratios \(\rm {F_2}/{F_1} < 1\%\).
Therefore, measures are needed to remove these artefacts from the combined spectra.

The solution that was chosen was to replace the optimally extracted nods that contained artefacts with their rectangular counterparts, creating a combined spectra that had a mix of optimally reduced and rectangularly reduced nod spectra.
To replace the spikes in the rectangular extraction that created the artefacts and others an iterative 4-\(\sigma\)\footnote{There is no scientific justification why 4-\(\sigma\) was chosen over the commonly used 3-\(\sigma\).} rejection algorithm\footnote{Found at \url{https://github.com/jason-neal/nod_combination}} was applied to the rectangular extractions.
The \(\sigma\) for each pixel was calculated as the standard deviation of the nearest 2 pixels on either side of it, across all 8 nod spectra.
Any rejected pixels were replaced using linear interpolation along the spectra.


Combined spectra were finally constructed by averaging the eight nod-cycle spectra together, where some of the optimally extracted spectra were replaced using the above method. \textbf{The third panel of \cref{fig:badpixelreplacement} shows the difference between the combined spectra using only optimally extracted spectra and the mixed combination with replacements.} The actual spectra from the two different methods can be seen in \cref{fig:reduction-comparison} where orange is the combination of optimally reduced spectra only and green uses the replacement method outlined here.


\todo{This needs to be fixe up with a different plot, possibly even removeda as there are examples in appendix}
\textbf{There is a clear large extended artefact on the \nth{7} nod of the top panel is created by the large spike.
For the first panel a single large spike in the \nth{7} nod (pink) near pixel 230 creates a wide and noticeable artefact in the optimal extraction.
This is the same spectra as the \nth{2} down the right of \cref{fig:reduction-comparison}.
}

\textbf{
EXAMPLE with the new-old \cref{fig:badpixelreplacement}}.
\todo{CHANGE the figure here}
\begin{figure}
    \centering
    \includegraphics[width=\hsize/2]{figures/reduction/bp_plots/Bad_pixel_replacement}
    \caption{Example of an artefact in the optimally extracted spectra from detector \#2 of {\red{} HDXXXXXX}.
The top panel contains the 8 normalized nod spectra obtained using optimal extraction.
The middle panel shows the same nod spectra reduced using only rectangular extraction.
The bottom panel shows the difference between a combined spectrum using optimal nods only and a combined spectrum in which the identified nods are replaced with their rectangular counterparts as per \cref{subsubsec:reductionartefacts}.
A vertical offset is included between each spectra for clarity.
The nod spectra are in observation order from top to bottom.}
    \label{fig:badpixelreplacement}
\end{figure}

The percentage of individual nod spectra from all detectors and observations affected by artefacts was 14\%.
These are detailed in \cref{tab:nod_replacement}.
More examples of observations that contain artefacts are given in \cref{appendix:artefacts}, selected to show a variation in appearance.


The continuum normalization is performed in {IRAF} while the mixed combination is carried out in \emph{Python} along with the post reduction procedures detailed below.
This pipeline was chosen over the {ESO} {CRIRES} pipeline because it seemed relatively simpler to use, being semi-automated, and appeared to have less bad pixel/cosmic ray artefacts in the resulting spectra.
In hindsight the artefacts that appeared in the {DRACS} reduction took longer to investigate and create a solution for.
The solution involved a script to remove the bad pixels that created the artefacts.
The avoidance of bad pixels and the creation of a removal code, was one of the original choices for using the {DRACS} pipeline instead of the {ESO} pipeline.


One hypothesis for these artefacts is detector glow, a heating of the detector by the nearby amplifiers in the chip (see \cref{subsubsec:darkcurrent}\todo{check correct section}).
\unfinished{check this statement}{The artefacts in the \emph{K}-band spectra were not observed in previous works in the \emph{H}-band using this pipeline~\citep[e.g.][]{figueira_radial_2010}, and as such may have a wavelength dependent effect, like detector glow.
However, this no longer seems plausible as the location of the artefacts do not seem to indicate a pattern corresponding to the shape of the glow shown in \cref{fig:darkcurrent_colour}.
It could also be that there were artefacts in intermediate steps of the previous works that were missed due to the semi-automated pipeline, with the resultant combined spectrum being only slightly affected.}
Anther possibility is that the artefacts are bad pixels not removed correctly.
Regardless, they do not have a significant impact on the results found in this work, though it is expected that that mixing optimally reduced spectra with rectangularly reduced spectra will have a slight negative impact on the noise or SNR of the combined spectrum.

\textbf{Note for HD\,202206 the 1st two nods have a much higher extracted flux.
I am still unsure why this is\ldots{}}


Much time and effort went into trying to extract the spectra as best as possible, without artefacts, to have the best chance at obtaining high quality scientific results of the faint features sought after in this work.
Due to time constraints the replacement of nods with artefacts (indicated in \cref{tab:nod_replacement}) with their rectangular counterparts was performed.

\subsection{Reduction experience:}
\label{subsec:experience}
The experience gained in reducing {CRIRES} spectra enabled participation in collaboration with other science cases.
The {DRACS} pipeline was used to extract the spectra of two other targets.
A target and a very brief aim of each science case is given below.
\begin{itemize}
\item Barnard's Star\footnote{Programme {{ID}}: 085.D-0161(A)}: The carefully reduced \nir{} spectra of Barnard's Star was meant to extend the work of~\citet{andreasen_nearinfrared_2016} in deriving the spectroscopic parameters of cool M-stars in the \nir{}.
Unfortunately the work did not advance enough to analyse M-star's and a spectrum of {Arcturus} (K0) and a fully reduced spectrum of {10Leo} (K1) from the {CRIRES}-POP library~\cite{nicholls_crirespop_2017} were analysed instead in~{Andreasen et al. (in prep.)}.
\item \(\eta\) Tel\footnote{Programme {{ID}}: 083.C-0759(A)}: The spectra of a telluric standard star (HIP100090) and {HR\,7329-B} (\(\eta\) Tel-B) a rapidly rotating Brown Dwarf, were reduced to accurately determine the BD rotation rate by measuring the line broadening.
The results from this data will be published in an upcoming paper Hagelberg et al. (in prep.).
\item The same spectra of {HR\,7329-B} were also used in~\citet{ulmer-moll_telluric_2018} to compare different telluric correction methods in the \nir{}.
An example from~\citet[][(B.3)]{ulmer-moll_telluric_2018} of the reduced target spectrum (black), the telluric model () and the telluric corrected spectrum (green) is provided in \cref{fig:ulmermol2018tellcorrcrires48}.
\end{itemize}

\begin{figure}
    \centering
    \includegraphics[width=0.7\linewidth]{figures/reduction/ulmermol2018_tell_corr_CRIRES_48}
    \caption{Example of telluric corrected spectrum of {HR\,7329-B} using {TAPAS}.
Credit~\citet[][]{ulmer-moll_telluric_2018}.
The yellow shaded region is the region of stellar lines that was avoided while applying the correction methods.}
    \label{fig:ulmermol2018tellcorrcrires48}
\end{figure}

For these works only the spectral extraction outlined above was performed.
The post extraction and reduction steps detailed in the following sections were not.



\input{chapters/reduction/post_reduction}