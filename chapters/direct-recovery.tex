%3-Direct-Spectral-Recovery
% Shift and subtract Method
% Calculation of planet RV 
% Similation of spectral recovery
% RV difference effect on amplitude of signal simulation
% Chisquared method of model plus stellar at different RVs? technique
% Limitations? 
% Future Prospects 
% CRIRES+, JWST
% Calculation of exposure time required etc?


\chapter{Direct-Spectral-Recovery}  % Main chapter title

\label{Chapter3} 

%----------------------------------------------------------------------------------------
%	SECTION 1
%----------------------------------------------------------------------------------------

\section{Main Section 1}



%-----------------------------------
%	SUBSECTION 1
%-----------------------------------
\subsection{Subsection 1}


%-----------------------------------
%	SUBSECTION 2
%-----------------------------------

\subsection{Subsection 2}

%----------------------------------------------------------------------------------------
%	SECTION 2
%----------------------------------------------------------------------------------------

\section{Main Section 2}



# from draft of paper 18/7/2017
\subsection{Companion K}
\label{sec:companion_RV}
\emph{These sections might be unnecessary}\\

To calculate the RV of the companion at the time of each observation. For a two-body system the RV semi-amplitude of the companion $\textrm{K}_{2}$ can be determined from the orbital host-companion mass ratio $q = \textrm{M}_{1}/\textrm{M}_{2} = \textrm{K}_{2}/\textrm{K}_{1}$.
Note, that for the targets in which only the minimum mass (msini) is known, this will give the maximum RV semi-amplitude of the companion.
This relation was used to estimate $\textrm{K}_2$ for the companion from the minimum mass or mass we have for each companion. These values along with the other orbital parameters of the system were use to calculate the RV of the companions for each observation. These values are provided with the observations in Table~\ref{tab:observations}.


\subsection{Companion-host Flux ratio}
The companion-host flux or contrast ratio of the systems are calculated using $ \frac{F_{2}}{F_{1}} \approx 2.512^{m_1-m_2} $, where $m_1$ and $m_2$ are the magnitude of the host and companion respectively.\ $m_1$ is the observed magnitude of the host while $m_2$ is obtained from stellar evolution models of \citet{baraffe_evolutionary_2003, baraffe_new_2015}, using the companion mass or minimum mass and the host age. The models, interpolated to the companion mass, also provide a first estimate of companion properties such as $T_{eff}$, $R/R_{\odot}$ and the magnitude in many wavelength bands.
We use the K band magnitude specifically for this work to estimate the flux ratio between the host and companion using and the host K band magnitude obtained from SIMBAD \citep{wenger_simbad_2000}. Our flux ratio estimates are presented in table~\ref{tab:flux_table}.

If only the companion minimum mass is known then this will correspond to the flux ratio lower limit, (or worst case scenario).

\begin{table}
    \caption{Estimating the host/companion flux ratios using the \citet{baraffe_evolutionary_2003} brown dwarf tables. $F_1/F_2$ is the estimated host/companion flux ratio calculated from the magnitude difference. Ages is taken from the table~\ref{tab:starparams}}.
    \begin{tabular}{lccccc}
        \hline\hline
        Object & HostK & Companion Mass & T$_{eff}$ & $\textrm{M}_{k}$ & $F_1/F_2$\\
        & (MJup) & (K) &   &  & \\
        \hline
        \object{HD 4747} &   &   &   &  & \\
        \object{HD 4747} & 5.30 & 72.0 & 1473 & 13.66 & 2192.9 \\    % https://arxiv.org/pdf/1604.00398.pdf new results
        \object{HD 162020} &   &   &   & \\
        \object{HD 167665} &   &   &   & \\
        \object{HD 168443} &   &   &   & \\
        \object{HD 211847} &   &   &   & 50\\
        \object{HD 162020} &   &   &   & \\
        \object{HD 202206} &   &   &   & \\
        \object{HD 30501} & 90 & 2494 & 10.3 & 85.8 \\
        \object{HD 30501} & 5.53 & 90.0 & 2498 & 10.36 & 85.8 \\
    \end{tabular}
    \label{tab:flux_table}
\end{table}

\begin{table}
    \tiny
    \begin{tabular}{l c c c c c c c}
        Host       & SpT  & $\textrm{M}_2\sin i$ & $\textrm{M}_2$     & Flux ratio  &  $\delta$ RV   & Period & Phase coverage \\
        &      & $\textrm{M}_{Jup}$   & $\textrm{M}_{Jup}$ & K band      &  m/s           & Days   & (\%) \\
        \hline
        \object{HD 211847}   & G5V  & 19.20 & 155$^\dagger$   & 0.26        & 13   & 7929.4  & 0.09  \\  %B % 2017
        \object{HD 202206}*  & G6V  & 17.4  & 93.6$^\dagger$  & 0.133       & 427  & 256     & 0.78  \\  %(B)   % May2017
        \object{HD 30501}    & K2V  & 62.3  & 89.6            & 0.011       & 1226 & 2073.6  & 5.8   \\
        \object{HD 4747}     & K0V  & 39.6  & 60.2$^\dagger$  & $1e^{-4}$   & N/A  & 13826.2 &  N/A  \\  % 2017
        \object{HD 167665}   & F9V  & 50.3  &                 & $2e^{-3}$   & 59   & 4385    & 0.18  \\  %  -- $2e^{-5}$  best case based on age rage.
        \object{HD 168443}*  & G6V  & 17.2  &                 & $2e^{-10}$  & 0    & 1749.8  & 0.001 \\  %(c)
        \object{HD 162020}   & K3V  & 14.4  &                 & $7e^{-6}$   & 363  & 8.4     & 0.24  \\  %
    \end{tabular}\\
    \vspace{1em}
    * only the largest companion.
    $\dagger$ Masses recently published
\end{table}


A simple tool to do this calculation was created with python. Given a host name, companion mass and a stellar age it can give a flux contrast ratio of the system. A script to work backwards from a flux-ratio to a companion mass was also created as a check for the results from \S~\ref{subsec:companion_recovery}. These scripts can be found at \url{https://github.com/jason-neal/baraffe_tables}.

\begin{figure}
    \includegraphics[width=0.4\textwidth]{images/tmp-images/HD30501-spectum.png}\\
    \includegraphics[width=0.4\textwidth]{images/tmp-images/HD30501-telluric-correction.png}\\
    \includegraphics[width=0.4\textwidth]{images/tmp-images/HD30501-difference.png}\\
    \caption{Example spectrum, telluric correction, and difference}
    \label{fig:spectral_example}
\end{figure}
