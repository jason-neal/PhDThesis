%!TEX root = ../thesis.tex
%3-Direct-Spectral-Recovery
% Shift and subtract Method
% Calculation of planet RV 
% Similation of spectral recovery
% RV difference effect on amplitude of signal simulation
% Chisquared method of model plus stellar at different RVs? technique
% Limitations? 
% Future Prospects 
% CRIRES+, JWST
% Calculation of exposure time required etc?


\chapter{Direct-Spectral-Recovery}  % Main chapter title

\label{cha:Chapter3} 

%----------------------------------------------------------------------------------------
%	SECTION 1
%----------------------------------------------------------------------------------------

\section{Main Section 1}



%-----------------------------------
%	SUBSECTION 1
%-----------------------------------
\subsection{Subsection 1}


%-----------------------------------
%	SUBSECTION 2
%-----------------------------------

\subsection{Subsection 2}

%----------------------------------------------------------------------------------------
%	SECTION 2
%----------------------------------------------------------------------------------------

\section{Main Section 2}



% from draft of paper 18/7/2017
\subsection{Companion K}
\label{sec:companion_RV}
\emph{These sections might be unnecessary}\\

To calculate the RV of the companion at the time of each observation. For a two-body system the RV semi-amplitude of the companion $\textrm{K}_{2}$ can be determined from the orbital host-companion mass ratio $q = \textrm{M}_{1}/\textrm{M}_{2} = \textrm{K}_{2}/\textrm{K}_{1}$.
Note, that for the targets in which only the minimum mass (msini) is known, this will give the maximum RV semi-amplitude of the companion.
This relation was used to estimate $\textrm{K}_2$ for the companion from the minimum mass or mass we have for each companion. These values along with the other orbital parameters of the system were use to calculate the RV of the companions for each observation. These values are provided with the observations in Table~\ref{tab:observations}.


\subsection{Companion-host Flux ratio}
The companion-host flux or contrast ratio of the systems are calculated using $ \frac{F_{2}}{F_{1}} \approx 2.512^{m_1-m_2} $, where $m_1$ and $m_2$ are the magnitude of the host and companion respectively.\ $m_1$ is the observed magnitude of the host while $m_2$ is obtained from stellar evolution models of \citet{baraffe_evolutionary_2003, baraffe_new_2015}, using the companion mass or minimum mass and the host age. The models, interpolated to the companion mass, also provide a first estimate of companion properties such as $T_{eff}$, $R/R_{\odot}$ and the magnitude in many wavelength bands.
We use the K band magnitude specifically for this work to estimate the flux ratio between the host and companion using and the host K band magnitude obtained from SIMBAD \citep{wenger_simbad_2000}. Our flux ratio estimates are presented in table~\ref{tab:flux_table}.

If only the companion minimum mass is known then this will correspond to the flux ratio lower limit, (or worst case scenario).


A simple tool to do this calculation was created with python. Given a host name, companion mass and a stellar age it can give a flux contrast ratio of the system. A script to work backwards from a flux-ratio to a companion mass was also created as a check for the results from \S~\ref{subsec:companion_recovery}. These scripts can be found at \url{https://github.com/jason-neal/baraffe_tables}.

\subsection{CRIRES data}
\label{subsec:CRIRES}
Observations were performed with the CRIRES instrument~\citep{kaeufl_crires_2004} configured to observe a narrow wavelength domain of the K-band between \(\rm2120-2160~nm \) using the Ks and the Hx5e-2 filters. The slit width of \(0.4\sec \) resulted in an instrumental resolving power of R\(\sim50000 \), with no adaptive optics to ensure that the slit was entirely covered by each target.

The observations were performed in service mode during period 89 with run ID.~089.C-0977(A) between April and August 2012. An observation is composed of 8 individual spectra with an integration time of 180 seconds, observed in the ABBAABBA nod cycle pattern to obtain a high (>100) signal-to-noise when combined. The targets and observation times taken by CRIRES are provided in Table~\ref{tab:observations}.




TAPAS XML REQUESTER

Although TAPAS provides a convenient web-interface to retrieve telluric spectra, it is tedious to use when requesting many models. A script\footnote{Can be found at \url{github.com/separate repo}} was used to automatically fill-out the XML template provided by TAPAS directly from values in the header information of each observation. Although created for these CRIRES observations it could be adjusted to work with other instruments if many TAPAS spectra are required. \textbf{It can be found at \ldots. Make separate Github repo for TAPAS requests.}


%!TEX root = ../thesis.tex
\begin{table*}
	\centering
	\small
	\caption{Stellar parameters of the target companion's host stars.}
		\begin{tabular}{l c c r@{$~\pm~$}l r@{$~\pm~$}l r@{$~\pm~$}l r@{$~\pm~$}l r@{$~\pm~$}l c}
		\toprule
		Star & SpT & V &  \multicolumn{2}{c}{\(T_{\textrm{eff}}\) (K)} &  \multicolumn{2}{c}{logg (cm s\(^{-2} \))} & \multicolumn{2}{c}{[Fe/H]} &  \multicolumn{2}{c}{\(M_1\) (M\(_{\sun} \))} & \multicolumn{2}{c}{Age (Gyr)} & Reference\\
		\midrule
        \object{HD 4747}     & K0V & 7.15 & 5316 & 50 & 4.48 & 0.10  & $-$0.21 & 0.05 & 0.81 & 0.02  & 3.3   & 2.3 & 1, 2, 3\\ 
		\object{HD 162020} & K3V & 9.12 & 4723 & 71 & 4.31 & 0.18  & $-$0.10 & 0.03 & 0.74 & 0.07  & 3.1   & 2.7 & 4, 5    \\  
		\object{HD 167665} & F9V & 6.48 & 6224 & 50 & 4.44 & 0.10  & 0.05       & 0.06 & 1.14 & 0.03  & 0.7   & 3.6 & 1        \\
		\object{HD 168443} & G6V & 6.92 & 5617 & 35 & 4.22 & 0.05 & 0.06       & 0.05 & 1.01 & 0.07  & 10.0 & 0.3 & 5, 6    \\ 
		\object{HD 202206} & G6V & 8.07 & 5757 & 25 & 4.47 & 0.03 & 0.29       & 0.02 & 1.04 & 0.07  & 2.9   & 1.0 & 5, 7    \\ 
		\object{HD 211847} & G5V & 8.62 & 5715 & 24 & 4.49 & 0.05  & $-$0.08 & 0.02 & 0.92 & 0.07  & 0.1   & 6.0 & 2, 4    \\ 
		\object{HD 30501}   & K2V & 7.59  & 5223 & 50 & 4.56 & 0.10 & 0.06       & 0.06 & 0.81 & 0.02  & 0.8   & 7.0 & 1, 4    \\ 
		\bottomrule
	\end{tabular} \\
	\tablebib{
	          (1) \citet{sahlmann_search_2011}; (2) \citet{santos_spectroscopic_2005}; (3) \citet{crepp_trends_2016}; (4) \citet{tsantaki_deriving_2013}; (5) \cite{bonfanti_age_2016}; (6) \citet{santos_spectroscopic_2004};
	          (7) \citet{sousa_spectroscopic_2008};
	          }
	\label{tab:starparams}
\end{table*}
%!TEX root = ../thesis.tex
% Table of orbit parameters

\todo{Rotate orbital parameter table.}
\begin{table*}
		\centering
		\tiny
		\caption{Orbital parameters for the BD companions obtained from the literature.}
		\begin{tabular}{l c r@{$ \,\pm\, $}l r@{$ \,\pm\, $}l r@{$ \,\pm\, $}l r@{$ \,\pm\, $}l r@{$ \,\pm\, $}l cc c c}
		\toprule
		Object  & \(\gamma \)  	& \multicolumn{2}{c}{Period}   & \multicolumn{2}{c}{\(e \) } &  \multicolumn{2}{c}{\(\textrm{K}_{1} \) } &  \multicolumn{2}{c}{\(T_{0} \)}  &  \multicolumn{2}{c}{ \(\omega \) } & \(M_2\sin{i}\) & \(M_2\) & Ref.\\
		 & (km\(^{-1} \)) 	& \multicolumn{2}{c}{(day)}   	&    \multicolumn{2}{c}{}    &  \multicolumn{2}{c}{(ms\(^{-1} \))}   & \multicolumn{2}{c}{ (JD-2,450,000) } &  \multicolumn{2}{c}{(deg) } &   \(\rm {M}_{Jup} \)  &   \(\rm {M}_{Jup} \)   & \\
		\midrule
		\object{HD 4747}       & $0.215 \pm 11 $        	 &  13826.2  &  314.1            &  0.740 & 0.002  	& 755.3   &  12      & 463.1       &  7.3    & 269.1      &  0.6   &  39.6    & 60.2 			  & 1 \\
		\object{HD 162020}   & $-27.328\pm0.002$  	    &  8.42819  &  $6e^{-5}$   &  0.277 & 0.002   & 1813    &  4        & 1990.68   &  0.01  & 28.4        &  0.2   & 14.4     &     -   			  & 2 \\
		\object{HD 167665}   & $8.003 \pm 0.008$    	 & 4451.8 & 27.6   				   & 0.340 & 0.005 	  & 609.5   &  3.3     & 6987.6     &  29     & $-$134.3 & 0.9     & 50.3    &     -    			& 3 \\
		\object{HD 168443}b  & $-0.047\pm0.552$ 		& 58.1124 & $4e^{-4}$  		& 0.529 & 0.001   & 475.13 & 0.9      & 5626.20  &  0.02   & 172.9      & 0.1     & 7.7      &     -    			& 4 \\ 
		\object{HD 168443}c  & $-0.047\pm0.552$ 		 & 1749.83 & 0.57  			     & 0.211 & 0.002  	 & 297.7  & 0.6      & 5521.3     &  2.2     & 64.9       & 0.5     & 17.1    &     -    			  & 4 \\ 
		\object{HD 202206}B & 14.721     						& 256.33  &  0.02    		     & 0.432 & 0.001  	  & 567     &  1       & 2176.14    &  0.12   & 161.9     & 0.2  	& 17.4    & $93.2\pm7.3$   & 5, 6\\  
		\object{HD 202206}c & 14.721    						 & 1260 &  11			        	& 0.22 & 0.03 		  & 41    	& 1          & 3103 		& 452    & 280 		   & 4  	  & 2.3      & $17.9\pm2.9$  & 5, 6\\ 
		\object{HD 211847} 	 & 6.689\tablefootmark{a} & 7929.4 & 2500  		    	 & 0.685 & 0.068   	  & 291.4   & 12.2   & 12030.1    & 2500   & 159.2 		& 2.0     & 19.2  & 155  				& 3, 7\\
		\object{HD 30501}  	  & $23.710\pm0.028$         & 2073.6 & 3.0 				& 0.741 & 0.004   	   & 1703.1 & 26.0   & 3851.5 		& 3.0     & 70.4 		& 0.7     & 62.3   & 89.6      			& 3  \\
		\bottomrule
	\end{tabular}
	\tablebib{
	          (1) \citet{crepp_trends_2016}; (2) \citet{udry_coralie_2002}; (3) \citet{sahlmann_search_2011}; 
	          (4) \citet{pilyavsky_search_2011}; (5) \citet{correia_coralie_2005};  (6) \citet{benedict_hd_2017}; (7) \citet{moutou_eccentricity_2017}
	}
	\tablefoot{
	\tablefoottext{a}{fixed}
	}
	\label{tab:orbitparams}
\end{table*}

%!TEX root = ../thesis.tex
\begin{table*}
    %\tiny
    \small
    \centering
    \caption{Estimated flux ratios and semi-amplitude of the companion given the companion \(\textrm{M}_{2}/\textrm{M}_{2} \sin{i}\) from Table~\ref{tab:orbitparams}. The flux ratio \(F_{2}/F_{1} \) is calculated using the K-band magnitude difference of the host star to the Baraffe evolutionary model magnitude for the companion mass. The model ages used are those closest to host age value in Table~\ref{tab:starparams}.     The noise ratio is  calculated via \(N_{2}/N_{1} = \sqrt{2} \times\sqrt{F_{1}/F_{2}}\). The orbital properties are calculated using the orbital parameters given above along with the times of observations in Table~\ref{tab:observations}.} 
    \begin{tabular}{l c c c c c c c c}
        \toprule
        &  Estimated  & Estimated &  Estimated & Estimated &  &    \\  % 2017
        Host           & \(\rm F_{2}/F_{1} \)   & \(\rm N_{2}/N_{1} \) (noise ratio) & \(\rm K_2\) &   \(\Delta RV\) & Phase coverage \\
        & K-band     & & (kms\(^{-1}\)) & (ms\(^{-1}\)) & (\%) \\
        \midrule
        \object{HD 4747}        & \(3\times10^{-4} \)   & 76 &  -10.65 & -  &  -  \\  % 2017
        \object{HD 162020}   & \(7\times10^{-7} \)   & 1615  &  -98.92\tablefootmark{a} &  2344.24     & 0.28~~  \\  %
        \object{HD 167665}    & \(2\times10^{-4} \)   &  105    &  -14.47\tablefootmark{a}   &   138.45     & 0.18~~  \\  %  -- \(2\times10^{-5} \)  best case based on age rage.
        \object{HD 168443b} & \(1\times10^{-16} \)  &    \(1\times10^{8} \)   &  -64.65\tablefootmark{a} &   257.16   & 0.035 \\ 
        \object{HD 168443c} &  \(1\times10^{-11} \)  &   \(4\times10^{5} \)     &  -18.05\tablefootmark{a}  &   0.95   &  0.001 \\  %(c)
        \object{HD 202206}B  & \(8\times10^{-7} \)  &   1586 &  -6.79 & 145.17   & 0.74~  \\  %(B)   % May2017
        \object{HD 202206}c  &  \(5\times10^{-15}\)   &     \(2\times10^{7} \) &   -2.50     &   0.67     &  0.15~  \\  %(B)   % May2017
        \object{HD 211847}B  &  0.01 &  14   & $-$1.85 & 3.88   & 0.09~  \\  %B % 2017
        \object{HD 30501}      &  0.002  &  27  &  -16.12    &  1346.46      & 5.8~~  \\
        \bottomrule
    \end{tabular}\\
    \tablefoot{
        \tablefoottext{a}{Maxium \(K_2\) only given \(M_2 \sin{i}\)}
    }
    \label{tab:flux_table}
\end{table*}
%!TEX root = ../thesis.tex
\begin{table*}
    \centering    
    \caption{Details about the CRIRES observations times and settings. The estimated RV values for both component in each observation are provided. These are calculated using the best known orbital parameters and the companion mass \((M_{2}\sin{i}/M_{2})\). For hosts with multiple companions the RV value is for the largest companion only i.e. \object{HD 202206}B and \object{HD 168443}c. The RV difference between the host and the companion (\(RV_2 - RV_1\)) corresponds to the \(rv_2\) parameter in the binary model from Sect.~\textbf{ref{subsubsec:binary-model}}.}
    \begin{tabular}{l c c c c r@{.}l  r@{.}l  r@{.}l}
        \toprule
        Object & Obs. \# & Start date  & Filter & Airmass  & \multicolumn{2}{c}{\(RV_1\)} & \multicolumn{2}{c}{\(RV_2\)} & \multicolumn{2}{c}{\(rv_2\)}  \\  % & \(Date \)
        &   & (yyyy-mm-dd hh:mm:ss)  &  & (at start) & \multicolumn{2}{c}{kms\(^{-1}\)} & \multicolumn{2}{c}{kms\(^{-1}\)} & \multicolumn{2}{c}{kms\(^{-1}\)}\\ % data ref    % & (JD\(^{\star} \))
        \midrule
        \object{HD 4747}   & 1 & 2012-07-06 07:36:06 & Ks     	      & 1.25  	  & $-$0    & 219 & $-$0  & 154 & 0&065 \\ %-1      & 2456114.81674
        \object{HD 162020} & 1 & 2012-07-04 06:23:22 & Ks     		& 1.30 		& $-$28  & 760 & 50 & 785\tablefootmark{a} & 79&545\tablefootmark{a} \\ %-1      & 2456112.76624
        \object{HD 162020} & 2 & 2012-07-04 06:57:48 & Ks     		& 1.44  	& $-$28  & 717 & 48 & 440\tablefootmark{a} & 77&157\tablefootmark{a} \\ %-2      & 2456112.79015
        \object{HD 167665} & 1 & 2012-07-28 05:00:53 & Hx5e-2 	& 1.24 		& 7         & 581 & 18 & 024\tablefootmark{a} & 10&443\tablefootmark{a} \\ %-1a     & 2456136.70895
        \object{HD 167665} & 2 & 2012-07-28 05:37:27 & Hx5e-2 	& 1.39  	& 7         & 581 & 18 & 025\tablefootmark{a}  & 10&444\tablefootmark{a} \\ %-1b     & 2456136.73434
        \object{HD 167665} & 3 & 2012-08-05 02:54:03 & Hx5e-2 	& 1.04  	& 7         & 575 & 18 & 163\tablefootmark{a} & 10&588\tablefootmark{a} \\ %-2      & 2456144.62087
        \object{HD 168443} & 1 & 2012-08-05 04:29:32 & Ks     		& 1.31 		& $-$0   & 121 & 50 & 932\tablefootmark{a, b}  & 51&053\tablefootmark{a,b} \\ %-1      & 2456144.68718
        \object{HD 168443} & 2 & 2012-08-05 04:58:50 & Ks     		& 1.47 		& $-$0   & 121 & 51 & 189 \tablefootmark{a, b} & 51&310\tablefootmark{a,b} \\ %-2      & 2456144.70753
        \object{HD 202206} & 1 & 2012-07-12 06:54:44 & Ks     		& 1.01 		& 14      & 843 & 12 & 992\tablefootmark{b}  & -1&851 \\ %-1      & 2456120.78801
        \object{HD 202206} & 2 & 2012-07-13 05:41:40 & J       		  & 1.01 	  & 14      & 837 & 13 & 065\tablefootmark{b}  & -1&772 \\ %-2      & 2456121.73727
        \object{HD 202206} & 3 & 2012-07-11 08:29:55 & Ks     		& 1.15		& 14      & 849 & 12 & 920\tablefootmark{b}  & -1&929 \\ %-3      & 2456119.85411
        \object{HD 211847} & 1 & 2012-07-06 07:02:57 & Ks     		& 1.07 		& 6        & 613 & 7   & 171 & 0& 558\\ %-1      & 2456114.79372
        \object{HD 211847} & 2 & 2012-07-13 06:54:37 & Ks     		& 1.05 		& 6        & 614 & 7   & 167 & 0&553 \\ %-2      & 2456121.78793
        \object{HD 30501}  & 1 & 2012-04-07 00:08:29 & Hx5e-2 	 & 1.60 	 & 22      &  372 & 36 & 377 & 14&005 \\ %-1      & 2456024.50590
        \object{HD 30501}  & 2 & 2012-08-01 09:17:30 & Hx5e-2    & 1.42		 & 22      & 505 & 35  & 120 & 12&615 \\ %-2a     & 2456140.88716
        \object{HD 30501}  & 3 & 2012-08-02 08:47:30 & Hx5e-2 	 & 1.53 	 & 22      & 507 &  35 & 102 & 12&595 \\ %-3      & 2456141.86633
        \object{HD 30501}  & 4 & 2012-08-06 09:42:07 & Ks     		 & 1.28 	 & 22      & 514 & 35 & 031 & 12&517 \\ %-2b     & 2456145.90426
        \bottomrule
        
    \end{tabular}
    \tablefoot{
        \tablefoottext{a}{Maximum RV given \(\textrm{M}_2\sin{i}\) only.}
    	\tablefoottext{b}{Largest mass companion only.}
    }
    \label{tab:observations}
\end{table*}






CHECK out LOCKWOOD 2014 - maximum likelyhood with todcor   1e-4 flux ratio double lined spectra


\section{Direct recovery in the MIR}
Early on in 2015 we began investigating extending this technique to the MIR. In particular looked at applying for observing time with a MIR instrument to perform similar work.

We investigating applying for time with VISIR.

VISIR is a MIR spectroscopy with resolution ... and

Looking at the best candidates XXXX was chosen a a likely target. I calculated flux ratios and performance considerations we determined that observations with VISIR to achieve a SNR of 100 were unfeasible, requiring 1000's of hours observing time.

The goal would be to observe with High-resolution and high through-put spectrographs. Ideally with CRIRES+, which at the times was scheduled for 2017. Currently\footnote{June 2018} first light is scheduled for Q1 2019.




