%!TEX root = ../thesis.tex
%3-Direct-Spectral-Recovery
% Shift and subtract Method
% Calculation of planet {RV}
% Simulation of spectral recovery
% {RV} difference effect on amplitude of signal simulation
% Chi squared method of model plus stellar at different RVs? technique
% Limitations?
% Future Prospects
% {CRIRES+}, JWST
% Calculation of exposure time required etc?


\chapter{Direct Spectral Recovery} % Main chapter title
\label{cha:direct_recovery}

There many different techniques to disentangle the spectra of binary objects. In this chapter we focus on applying a direct subtraction method to near-infrared (\nir{}) spectra of {FGK} stars with Brown Dwarf (BD) companions. The data used was obtained with the {CRIRES} instrument in 2012 with the purpose to apply this technique specifically. A level of trust was placed in the quality of the observations, which was misplaced. We will begin this chapter with the motivation for these specific targets and detail the observations obtained. We will then present the direct subtraction technique and explore in detail how the observations are insufficient to apply this method, and explore this technique at low RV separations by comparison to simulated results.

\section{Motivation}

Motivation section

\textbf{Why this technique and not others.}
- separation of components
- direct imaging (cant separate)


Find direct imaging limitations ---


\section{Motivation and target selection}
\label{sec:motivation}
The work of~\citet{sahlmann_search_2011} identified several candidate brown dwarf companions of FGK stars with \(\rm M_2 \sin{i}\) values > 10~\Mjup{}. Seven candidates from~\citet{sahlmann_search_2011}, which were visible in Period 89 (2012), were selected for further observation in order to identify their stellar nature. That is, to refine the mass of the companions to distinguish if they companion is a large giant planet (M$\apprle13$\Mjup), a Brown dwarf (13$\apprle $M$\apprle80$\Mjup) or a low-mass star (M$\apprge80$\Mjup).

The spectral differential approach was chosen with the goal to constrain the companion masses while minimizing the observational time required to observe. It was deemed that it should be possible to obtain a detection of a companion at 1\% contrast ratio with an exposure time of around 20 minutes. Two observations at ``clearly separate RVs'' were requested to constrain each target. Observations were preformed without telluric standard star observations to avoid the wasted observing time, choosing to instead rely on synthetic model correction (see \ref{subsec:telluric_correction}). The \textit{K}-band was chosen to achieve a high contrast relative to the host star, detected in the extreme V-K colour indexes (>7.8).

The list target host stars are presented in Table~\ref{tab:starparams} along with their stellar parameters, while the companion orbital parameters are provided in Table~\ref{tab:orbitparams}.

We note that the orbital parameters of some targets have been refined in the literature since the observations took place. For example three candidates have had their masses refined in recent works. The companion to {HD\,211847} was determined to be a low mass star with \(\rm M_2=155~\)\Mjup{}~\citep{moutou_eccentricity_2017}, while the companion to {HD\,4747} was found to have a mass of \(\rm M_2=60~\)\Mjup{}~\citep{crepp_trends_2016}. The two companions of {HD\,202206} (B and c) were found to have masses of \(\rm M_B =93.6\)~\Mjup{} and \(\rm M_c = 17.9\)~\Mjup{}, respectively, classifying {HD\,202206}c as a ``circumbinary brown dwarf''~\citep{benedict_hd_2017}. These three target with recently refined masses (along with {HD\,30501}) create a good set benchmarks to compare any results from the techniques developed here, and show that the masses of these targets do span the BD -- low mass star range. All target companions except {HD\,162020} (P=8.4 days) are in (very) long period orbits (P=0.7--38 years) with masses (or \mtwosini{}) greater than 10~\Mjup{}.

%!TEX root = ../thesis.tex
\begin{table*}
	\centering
	\small
	\caption{Stellar parameters of the target companion's host stars.}
		\begin{tabular}{l c c r@{$~\pm~$}l r@{$~\pm~$}l r@{$~\pm~$}l r@{$~\pm~$}l r@{$~\pm~$}l c}
		\toprule
		Star & SpT & V &  \multicolumn{2}{c}{\(T_{\textrm{eff}}\) (K)} &  \multicolumn{2}{c}{logg (cm s\(^{-2} \))} & \multicolumn{2}{c}{[Fe/H]} &  \multicolumn{2}{c}{\(M_1\) (M\(_{\sun} \))} & \multicolumn{2}{c}{Age (Gyr)} & Reference\\
		\midrule
        \object{HD 4747}     & K0V & 7.15 & 5316 & 50 & 4.48 & 0.10  & $-$0.21 & 0.05 & 0.81 & 0.02  & 3.3   & 2.3 & 1, 2, 3\\ 
		\object{HD 162020} & K3V & 9.12 & 4723 & 71 & 4.31 & 0.18  & $-$0.10 & 0.03 & 0.74 & 0.07  & 3.1   & 2.7 & 4, 5    \\  
		\object{HD 167665} & F9V & 6.48 & 6224 & 50 & 4.44 & 0.10  & 0.05       & 0.06 & 1.14 & 0.03  & 0.7   & 3.6 & 1        \\
		\object{HD 168443} & G6V & 6.92 & 5617 & 35 & 4.22 & 0.05 & 0.06       & 0.05 & 1.01 & 0.07  & 10.0 & 0.3 & 5, 6    \\ 
		\object{HD 202206} & G6V & 8.07 & 5757 & 25 & 4.47 & 0.03 & 0.29       & 0.02 & 1.04 & 0.07  & 2.9   & 1.0 & 5, 7    \\ 
		\object{HD 211847} & G5V & 8.62 & 5715 & 24 & 4.49 & 0.05  & $-$0.08 & 0.02 & 0.92 & 0.07  & 0.1   & 6.0 & 2, 4    \\ 
		\object{HD 30501}   & K2V & 7.59  & 5223 & 50 & 4.56 & 0.10 & 0.06       & 0.06 & 0.81 & 0.02  & 0.8   & 7.0 & 1, 4    \\ 
		\bottomrule
	\end{tabular} \\
	\tablebib{
	         (1)~\citet{sahlmann_search_2011}; (2)~\citet{santos_spectroscopic_2005}; (3)~\citet{crepp_trends_2016}; (4)~\citet{tsantaki_deriving_2013}; (5)~\cite{bonfanti_age_2016}; (6)~\citet{santos_spectroscopic_2004};
	         (7)~\citet{sousa_spectroscopic_2008};
	          }
	\label{tab:starparams}
\end{table*}
\input{./tables/direct_recovery/orbital_params}

\subsection{The Data}

\subsection{CRIRES data}
\label{subsec:CRIRES}
Observations were performed with the {CRIRES} instrument~\citep{kaeufl_crires_2004} configured to observe a narrow wavelength domain of the \emph{K}-band between 2\,120--2\,160\nm{}. The slit width of \(0.4^{\prime\prime}\) resulted in an instrumental resolving power of \(\R=50\,000\)\footnote{The rule of thumb resolution for {CRIRES} is \(100\,000\frac{0.2^{\prime\prime}}{slit width}\)}. No adaptive optics were used to ensure that the entrance slit was entirely covered by each target. This is to prevent strong slit illumination variations that could change the shape of spectral lines.

The observations were performed in service mode during Period 89 with run {ID.~089.C-0977(A)} between April and August 2012. An single observation is composed of eight individual spectra with an integration time of 180\si{\second} each, observed in the {ABBAABBA} nod cycle pattern to obtain a high signal-to-noise (>100) when combined. The list of observations obtained with {CRIRES} are provided in \tref{tab:observations}.

There is a slight inconsistency with the some of the observations, taken in service mode. For instance {HD\,202206} has two observations taken with the {Ks} filter, while one is taken with the {J} filter. There is also the last observation of {HD\,30501} taken with a different filter to to rest. We are unable to access the phase two proposal to determine if this was requested as such or it was an observational mistake. \todo{I still assume this is a mistake}

There is also a inconsistency with the ordering of th observations again with {HD\,202206}. The observation that was performed first is labelled as the third in the fits header file.

The target {HD\,4747} only has one observation. There could be two possible reasons for this, these observations were preformed in service mode as filler and there was time to complete it, or only one observation was requested due to the very long orbital period of the target.

\textbf{Is there a reason for the two different filters?}

%!TEX root = ../../nir_companions.tex
\todo{rotate table?}
% Table of observations
\begin{table*}
    \small
    \centering
    \begin{threeparttable}[b]

        \caption{Details about the each {CRIRES} observation. The number of artefacts removed in \sref{subsubsec:reductionartefacts} as well as the \snr{} of the combined spectra is provided. The last three columns are the calculated {RV} of both host and largest companion, from the orbital solution, as well as the {RV} difference between the two components.}
        %\begin{tabular}{l c c c c cl cl r@{.}l r@{.}l r@{.}l}
        \begin{tabular}{l c c c c c c | r@{.}l r@{.}l r@{.}l}
            \toprule
            Object & Obs.& Start date  & Filter & Airmass  & Artefacts & \snr{} & \multicolumn{2}{c}{\(RV_1\)} & \multicolumn{2}{c}{\(RV_2\)} & \multicolumn{2}{c}{\(rv_2\)}  \\  % & \(Date \)
            &  \#   & (yyyy-mm-dd hh:mm:ss)  &  & (at start) & {/ 32} & & \multicolumn{2}{c}{\kmps{}} & \multicolumn{2}{c}{\kmps{}} & \multicolumn{2}{c}{\kmps{}}\\ % data ref    % & (JD\(^{\star} \))
            \midrule
            {HD 4747}   & 1 & 2012-07-06 07:36:06 & Ks            & 1.25     & 7 & 340 & $-$0 & 219 & $-$0  & 154 & 0&065 \\ %-1   & 2456\,114.81674
            {HD 162020} & 1 & 2012-07-04 06:23:22 & Ks      & 1.30  & 2 & 127 & $-$28  & 760 & 50 & 785\tnote{a}  & 79&545\tnote{a} \\ %-1   & 2456\,112.76624
            {HD 162020} & 2 & 2012-07-04 06:57:48 & Ks      & 1.44   & 2 & 128 & $-$28  & 717 & 48 & 440\tnote{a} & 77&157\tnote{a} \\ %-2   & 2456\,112.79015
            {HD 167665} & 1 & 2012-07-28 05:00:53 & Hx5e-2  & 1.24  & 7 & 371 & 7      & 581 & 18 & 024\tnote{a} & 10&443\tnote{a} \\ %-1a  & 2456\,136.70895
            {HD 167665} & 2 & 2012-07-28 05:37:27 & Hx5e-2  & 1.39   & 4 & 374 & 7      & 581 & 18 & 025\tnote{a}  & 10&444\tnote{a} \\ %-1b  & 2456\,136.73434
            {HD 167665} & 3 & 2012-08-05 02:54:03 & Hx5e-2  & 1.04   & 4 & 358 & 7      & 575 & 18 & 163\tnote{a} & 10&588\tnote{a} \\ %-2   & 2456\,144.62087
            {HD 168443} & 1 & 2012-08-05 04:29:32 & Ks      & 1.31  & 2& 192 & $-$0   & 121 & 50 & 932\tnote{a,b}  & 51&053\tnote{a,b} \\ %-1   & 2456\,144.68718
            {HD 168443} & 2 & 2012-08-05 04:58:50 & Ks      & 1.47  & 4 & 190 & $-$0   & 121 & 51 & 189 \tnote{a,b} & 51&310\tnote{a,b} \\ %-2   & 2456\,144.70753
            {HD 202206} & 1 & 2012-07-12 06:54:44 & Ks      & 1.01  & 3& 189 & 14   & 843 & 12 & 992\tnote{b}  & -1&851 \\ %-1   & 2456\,120.78801
            {HD 202206} & 2 & 2012-07-13 05:41:40 & J          & 1.01    & 3 & 209 & 14   & 837 & 13 & 065\tnote{b}  & -1&772 \\ %-2   & 2456\,121.73727
            {HD 202206} & 3 & 2012-07-11 08:29:55 & Ks      & 1.15 & 4& 180 & 14   & 849 & 12 & 920\tnote{b}  & -1&929 \\ %-3   & 2456\,119.85411
            {HD 211847} & 1 & 2012-07-06 07:02:57 & Ks      & 1.07  & 4& 272 & 6     & 613 & 7   & 171 & 0& 558\\ %-1   & 2456\,114.79372
            {HD 211847} & 2 & 2012-07-13 06:54:37 & Ks      & 1.05  & 5& 283 & 6     & 614 & 7   & 167 & 0&553 \\ %-2   & 2456\,121.78793
            {HD 30501}  & 1 & 2012-04-07 00:08:29 & Hx5e-2   & 1.60   & 3& 217 & 22   &  372 & 36 & 377 & 14&005 \\ %-1   & 2456\,024.50590
            {HD 30501}  & 2 & 2012-08-01 09:17:30 & Hx5e-2 & 1.42  & 10& 212 & 22   & 505 & 35  & 120 & 12&615 \\ %-2a  & 2456\,140.88716
            {HD 30501}  & 3 & 2012-08-02 08:47:30 & Hx5e-2   & 1.53   & 8& 237 & 22   & 507 &  35 & 102 & 12&595 \\ %-3   & 2456\,141.86633
            {HD 30501}  & 4 & 2012-08-06 09:42:07 & Ks       & 1.28   & 7& 235& 22   & 514 & 35 & 031 & 12&517 \\ %-2b  & 2456\,145.90426
            \bottomrule
            & & & &
        \end{tabular}\label{tab:observations}
        \begin{tablenotes}
            \item  [a]{Maximum {RV} given \mtwosini{} only.}
            \item  [b]{Largest mass companion only.}
        \end{tablenotes}
    \end{threeparttable}
\end{table*}





\textbf{The filters ... using the {Ks} and the {Hx5e-2} filters. Why use the different filters???}

Before we go into detail about the differential subtraction we will first calculate for the observations taken.


\subsection{Estimating parameters of observations}
To estimate the differential amplitude signal we estimated the expected parameters of the impact of the

%!TEX root = ../../thesis.tex
\todo{Fix caption position here}
\begin{table*}
    \small
    \centering
    \begin{threeparttable}[b]
        \caption{Estimated orbital semi-amplitude and {RV} separation of the companion, given the companion mass (\(\mtwo\) or \mtwosini{}) from \tref{tab:orbitparams} and observation times from \tref{tab:observations}.}
        \begin{tabular}{l c c c c c c}%[hb]
            \toprule
             & Estimated & Estimated & & \\  % 2017
             Companion & \(\rm K_2\) & |\(\Delta {RV}\)| & Phase coverage\\
             & (\kmps{}) & (\mps{}) & (\%)\\
             \midrule
             {HD 4747} & -10.65 & -- & --\\  % 2017
             {HD 162020} & -98.92\tnote{a} & 2\,388 & 0.28\\  %
             {HD 167665} & -14.47\tnote{a} & 145 & 0.18\\  %  -- \(2\times10^{-5} \)  best case based on age rage.
             {HD 168443b} & -64.65\tnote{a} & 258 & 0.035\\
             {HD 168443c} & -18.05\tnote{a} & <1 & 0.001\\  %(c)
             {HD 202206}B & -6.79 & 78 & 0.74\\  %(B)   % May2017
             {HD 202206}c & -2.50 & <1 & 0.15\\  %(B)   % May2017
             {HD 211847}B & -1.85 & 5 & 0.09\\  %B % 2017
             {HD 30501} & -16.12 & 1\,410 & 5.8\\
             \bottomrule
         \end{tabular}\label{tab:estimated_rv}
         \begin{tablenotes}
            \item[a] {Maximum \(K_2\) only given \(M_2 \sin{i}\)}
         \end{tablenotes}
    \end{threeparttable}
\end{table*}
\todo{recalculate values that are less than 1}
%%!TEX root = ../thesis.tex
\begin{table*}
    %\tiny
    \small
    \centering
    \caption{Estimated flux ratios and semi-amplitude of the companion given the companion \(\textrm{M}_{2}/\textrm{M}_{2} \sin{i}\) from Table~\ref{tab:orbitparams}. The flux ratio \(F_{2}/F_{1} \) is calculated using the K-band magnitude difference of the host star to the Baraffe evolutionary model magnitude for the companion mass. The model ages used are those closest to host age value in Table~\ref{tab:starparams}.     The noise ratio is calculated via \(N_{2}/N_{1} = \sqrt{2} \times\sqrt{F_{1}/F_{2}}\). The orbital properties are calculated using the orbital parameters given above along with the times of observations in Table~\ref{tab:observations}.} 
    \begin{tabular}{l c c c c c c c c}
        \toprule
        &  Estimated  & Estimated &  Estimated & Estimated &  &    \\  % 2017
        Host           & \(\rm F_{2}/F_{1} \)   & \(\rm N_{2}/N_{1} \) (noise ratio) & \(\rm K_2\) &   \(\Delta RV\) & Phase coverage \\
        & K-band     & & (kms\(^{-1}\)) & (ms\(^{-1}\)) & (\%) \\
        \midrule
        \object{HD 4747}        & \(3\times10^{-4} \)   & 76 &  -10.65 & -  &  -  \\  % 2017
        \object{HD 162020}   & \(7\times10^{-7} \)   & 1615  &  -98.92\tablefootmark{a} &  2344.24     & 0.28~~  \\  %
        \object{HD 167665}    & \(2\times10^{-4} \)   &  105    &  -14.47\tablefootmark{a}   &   138.45     & 0.18~~  \\  %  -- \(2\times10^{-5} \)  best case based on age rage.
        \object{HD 168443b} & \(1\times10^{-16} \)  &    \(1\times10^{8} \)   &  -64.65\tablefootmark{a} &   257.16   & 0.035 \\ 
        \object{HD 168443c} &  \(1\times10^{-11} \)  &   \(4\times10^{5} \)     &  -18.05\tablefootmark{a}  &   0.95   &  0.001 \\  %(c)
        \object{HD 202206}B  & \(8\times10^{-7} \)  &   1586 &  -6.79 & 145.17   & 0.74~  \\  %(B)   % May2017
        \object{HD 202206}c  &  \(5\times10^{-15}\)   &     \(2\times10^{7} \) &   -2.50     &   0.67     &  0.15~  \\  %(B)   % May2017
        \object{HD 211847}B  &  0.01 &  14   & $-$1.85 & 3.88   & 0.09~  \\  %B % 2017
        \object{HD 30501}      &  0.002  &  27  &  -16.12    &  1346.46      & 5.8~~  \\
        \bottomrule
    \end{tabular}\\
    \tablefoot{
        \tablefoottext{a}{Maxium \(K_2\) only given \(M_2 \sin{i}\)}
    }
    \label{tab:flux_table}
\end{table*} % This is the old table.



\section{Orbital variations}

The insufficient spacing becomes clear when the RV variation during the orbits are visualized. Figures~\ref{fig:hd4747p89}--\ref{fig:hd30501p89} show the RV curves for each target.
Stars with two companions are shown twice, with each companion treated as a single Keplerian (ignoring the presence of the other companion).
For each panel on the left hand plot shows the RV variation across a orbit of the companion, while the panel on the right shows the RV variation for the 6 months observation window of Period 89.
The solid black line indicates the RV of the host star (with scale on the left), while the blue dashed line shows the RV of the companion (with scale on the right axis).
The orange crosses and red stars indicate the times at which observations were obtained for each target, for the host and companion respectively.

they show the phase placements, etc.

the companions are calculated by ... eq ...

Orbital parameters are the values provided in \tref{tab:orbitparams}.

Some code to create plots like this is available on \textit{Github} under the iastro-pt/Observationtools \footnote{\href{https://github.com/iastro-pt/ObservationTools}{https://github.com/iastro-pt/ObservationTools}} repository with documentation on \href{Read the Docs}{https://ia-observationtools.readthedocs.io/en/latest/rv.html}.



The sampling of points in the RV curve reveal that the choice of points was not favourable...

\todo{change RV scale for hd4747 orbit}

\begin{figure}
    \centering
    \begin{tabular}{cc}
        \includegraphics[width=0.45\linewidth]{figures/direct-recovery/orbital-plots/HD4747_orbital_phase.pdf}& \includegraphics[width=0.45\linewidth]{figures/direct-recovery/orbital-plots/HD4747_p89.pdf}\\
    \end{tabular}
    \caption{RV orbital single companion Keplerian for the {HD\,4747}. The left hand shows the RV for a whole orbit of the target while the right hand panel shows the RV curve over 6 months (Period 89). The solid black line indicates the RV of the host star (with scale on the left), while the blue dashed line indicates the RV of the companion (with scale on the right axis). The orange crosses and red stars indicate the times at which observations were obtained for the target, for the host and companion respectively.}
    \label{fig:hd4747p89}
\end{figure}

\begin{figure}
    \centering
    \begin{tabular}{cc}
        \includegraphics[width=0.45\linewidth]{figures/direct-recovery/orbital-plots/HD162020_orbital_phase.pdf}&
        \includegraphics[width=0.45\linewidth]{figures/direct-recovery/orbital-plots/HD162020_p89.pdf}\\
    \end{tabular}
    \caption{Same as \fref{fig:hd4747p89} but for {HD\,162020}.}
    \label{fig:hd162020p89}
\end{figure}

\begin{figure}
    \centering
    \begin{tabular}{cc}
        \includegraphics[width=0.45\linewidth]{figures/direct-recovery/orbital-plots/HD167665_orbital_phase.pdf}&
        \includegraphics[width=0.45\linewidth]{figures/direct-recovery/orbital-plots/HD167665_p89.pdf}\\
    \end{tabular}
    \caption{Same as \fref{fig:hd4747p89} but for {HD\,167665}.}
    \label{fig:hd167665p89}
\end{figure}

\begin{figure}
    \centering
    \begin{tabular}{cc}
        \includegraphics[width=0.45\linewidth]{figures/direct-recovery/orbital-plots/HD168443b_orbital_phase.pdf}&
        \includegraphics[width=0.45\linewidth]{figures/direct-recovery/orbital-plots/HD168443b_p89.pdf}\\
    \end{tabular}
    \caption{Same as \fref{fig:hd4747p89} but for {HD\,168443}b. Analysed as if this was a single companion.}
    \label{fig:hd168443bp89}
\end{figure}


\begin{figure}
    \centering
    \begin{tabular}{cc}
        \includegraphics[width=0.45\linewidth]{figures/direct-recovery/orbital-plots/HD168443c_orbital_phase.pdf}&
        \includegraphics[width=0.45\linewidth]{figures/direct-recovery/orbital-plots/HD168443c_p89.pdf}\\
    \end{tabular}
    \caption{Same as \fref{fig:hd4747p89} but for {HD\,168443}c. Analysed as if this was a single companion.}
    \label{fig:hd168443cp89}
\end{figure}

\begin{figure}
    \centering
    \begin{tabular}{cc}
        \includegraphics[width=0.45\linewidth]{figures/direct-recovery/orbital-plots/HD202206B_orbital_phase.pdf}&
        \includegraphics[width=0.45\linewidth]{figures/direct-recovery/orbital-plots/HD202206B_p89.pdf}\\
    \end{tabular}
    \caption{Same as \fref{fig:hd4747p89} but for {HD\,202206}B. Analysed as if this was a single companion.}
    \label{fig:hd202206bp89}
\end{figure}

\begin{figure}
    \centering
    \begin{tabular}{cc}
        \includegraphics[width=0.45\linewidth]{figures/direct-recovery/orbital-plots/HD202206c_orbital_phase.pdf}&
        \includegraphics[width=0.45\linewidth]{figures/direct-recovery/orbital-plots/HD202206c_p89.pdf}\\
    \end{tabular}
    \caption{Same as \fref{fig:hd4747p89} but for {HD\,202206}c. Analysed as if this was a single companion. }
    \label{fig:hd202206cp89}
\end{figure}

\begin{figure}
    \centering
    \begin{tabular}{cc}
        \includegraphics[width=0.45\linewidth]{figures/direct-recovery/orbital-plots/HD211847_orbital_phase.pdf}&
        \includegraphics[width=0.45\linewidth]{figures/direct-recovery/orbital-plots/HD211847_p89.pdf}\\
    \end{tabular}
    \caption{Same as \fref{fig:hd4747p89} but for {HD\,211847}.}
    \label{fig:hd211847p89}
\end{figure}

\begin{figure}
    \centering
    \begin{tabular}{cc}
        \includegraphics[width=0.45\linewidth]{figures/direct-recovery/orbital-plots/HD30501_orbital_phase.pdf}&
        \includegraphics[width=0.45\linewidth]{figures/direct-recovery/orbital-plots/HD30501_p89.pdf}\\
    \end{tabular}
    \caption{Same as \fref{fig:hd4747p89} but for {HD\,30501}.}
    \label{fig:hd30501p89}
\end{figure}
\todo{Check M2sini labels - can we get M2 only for those that we know}




\section{Direct Subtraction Method}
\label{sec:direct-subtraction}
Here we present the direct subtraction method used, which is similar to previous works~\citep{ferluga_separating_1997,kostogryz_spectral_2013}. Assuming that the instrumental profile and atmospheric absorption are dealt with appropriately the spectra of the observed targets are assumed to composed of two components. A bright host star blended with the faint companion. The spectrum received from the host-companion pair is given by the superposition of two spectral components (\(J_{1}\), \(J_{2}\)):
\begin{equation}
\textrm{I}(\lambda) = \textrm{J}_{1}(\lambda - v_{1}) + \textrm{J}_{2}(\lambda - v_{2})
\end{equation}
where the subscripts 1 and 2 indicate the spectrum of the host and companion respectively, and \(\lambda\) represents the wavelength of the spectra and \(\lambda-v\) represents the Doppler shift \(\lambda(1-v/c)\) by a velocity \(v\).

This can be shifted into the rest frame of the host star by applying the shift \(v_1\):
\begin{equation}
\textrm{I}(\lambda + v_{1}) = \textrm{J}_{1}(\lambda) + \textrm{J}_{2}(\lambda - v_{2} + v_{1})
\end{equation}

To analyse \(J_2\), the spectral component of interest, the component from the host needs to be carefully removed. If two observations of the same target are observed, denoted with subscripts \(a\) and \(b\), there will be relative motion between the components due to the orbit. Assuming that the stellar spectra do not change over time (\(J_{1a} = J_{2a}\)) and each spectrum can be individually Doppler shifted to the rest frame of the host star \(J_1{\lambda}\), then the spectrum of the host star (\(\textrm{J}_{1}\)) can be removed though subtraction of the two observations. Mutually cancelling the host component while leaving two components of the companion, with a relative Doppler shift between them.

\begin{align}
S(\lambda) &= \textrm{I}_{a}(\lambda + v_{1a}) - \textrm{I}_{b}(\lambda + v_{1b}) \nonumber \\
&= (\textrm{J}_{1a} + \textrm{J}_{2a}(\lambda - v_{2a} + v_{1a})) - (\textrm{J}_{2b} +\textrm{J}_{2b}(\lambda - v_{2b} + v_{1b})) \nonumber \\
&= \textrm{J}_{2a}(\lambda - v_{2a} + v_{1a}) - \textrm{J}_{2b}(\lambda - v_{2b} + v_{1b}) \nonumber \\
% &= \textrm{J}_{2}(\lambda - v_{2a}) - \textrm{J}_{2}(\lambda - v_{2b} - v_{1a} + v_{1b}) \nonumber \\
S(\lambda + v_{2a}-v_{1a}) &= \textrm{J}_{2a}(\lambda) - \textrm{J}_{2b}(\lambda - v_{2b} - v_{1a} + v_{1b} + v_{2a})\\
S(\lambda') &= \textrm{J}_{2a}(\lambda) - \textrm{J}_{2b}(\lambda - \Delta {RV}_2) \label{eqn:sprofile}
\end{align}

where,
\begin{equation}
\Delta {RV}_2 = v_{1a} - v_{1b} - v_{2a} + v_{2b} \label{eqn:companion_difference}
\end{equation}

is the {RV} difference between the two companion spectral components when the host components are mutually subtracted.
and \(\lambda' = \lambda + v_{2a}-v_{1a}\)

The resulting differential spectra \(S{\lambda'}\), dubbed \emph{s-profile} by~\citet{ferluga_separating_1997}, is composed of just the companion spectra, shifted and subtracted from itself.

\cite{ferluga_separating_1997} even provide a analytical form for the \emph{s-profile} given a single Gaussian line of the form
$J(\lambda) = 1- D \cdot\exp^{-\pi (\lambda - \lambda_0)^2 /W^2}$:
 ....\begin{equation}
 S(\lambda) = 2 D\cdot\exp^{-\pi D^2[(\lambda - \lambda_0)^2 +(k/2)^2]/W^2}\cdot \sinh{\frac{\pi D^2(\lambda-\lambda_0)k}{W^2}},\label{eqn:sprofile_gaussain}
 \end{equation}
where $\lambda_0$, $D$, and $W$ are the central wavelength, depth and equivalent width of the Gaussian line, and $k=\Delta {RV}_2 $ is the shift between the two companion spectra.


From binary dynamics~\citep[e.g.,][]{murray_keplerian_2010} the {RV} amplitudes of the host and companion are related through the mass ratio, \(q\), while having an opposite sign, due to their phase:
\begin{align}
v_{2} &= -q * v_{1} \label{eqn:q_relation}
\end{align}

We can simplify \eref{eqn:companion_difference} by expressing it in terms of the mass ratio and host {RV} only:

\begin{align}
\Delta RV_2 &= q v_{1a} - q v_{1b} + v_{1a} - v_{1b} \nonumber \\
&= (1 + q)(v_{1a} - v_{1b}). \label{eqn:companion_difference_simplified}
\end{align}

If the \(\Delta {RV}_2\) between the companion spectra is able to be derived from the s-profile~\citep[see][]{ferluga_separating_1997} then we can determine the mass ratio of the system, \(q\), thereby constraining the mass of the companion.

The values \(v_{1a}\) and \(v_{1b}\) are radial velocity of the host components. The values we use calculated using the RV equation \ref{eq:rv_equation} using the orbital parameters from the literature and provided in \ref{tab:orbitparams} and are used to shift each spectra to the hosts rest frame. These components can also be determined directly from the spectrum by cross-correlating the observed spectrum with a stellar template of the host and were in reasonable agreement. The same values already used to shift and mutually cancel the host spectrum.

There should be checks for consistency between \(v_{1a}\), \(v_{1b}\) and how well the host component is removed in the s-profile.

\textbf{If v1a and v1b are incorrect what will happen to q?}

\todo{put elsewhere}{\eref{eqn:q_relation} was used to calculate the expected companion {RV} for each observation in \tref{tab:observations}.}


Example simulations?
\missingfigure{simulated example of combined binary spectra, and the subtraction, with a clear companion visible }

\textbf{kostov.. 2013 use a very similar method, choosing two observations from the extrema. They also preform a number of simulations regarding observing CRIRES spectra with different brown dwarfs. Figure our the flux ratio limit they get too???}

\subsection{Results of spectral differential analysis}
\label{subsec:differential_results}

We applied the spectral differential procedure outlined above on the wavelength-calibrated and telluric-corrected {CRIRES} observations. The spectra were corrected for Earth's barycentric {RV} using the \emph{helcor} PyAstronomy\footnote{https://pyastronomy.readthedocs.io} function ported from the REDUCE IDL package (see~\citet[][]{piskunov_new_2002}) and Doppler shifted to the rest frame velocity of the system. The spectra were then subtracted from each other and analysed as described above.

It is necessary to have a consistent instrumental setup~\citet{ferluga_separating_1997}, to avoid introducing extra instrumental effects (e.g.\ slit-width and/or filters) into the spectral differentials and to always observe the same wavelength range and maximize the information to be extracted. In our case, the second observation of {HD\,202206} and fourth of {HD\,30501} were taken with different filters compared to the other observations. Therefore, these two observations could not be used for this differential analysis. As noted in~\citep{hadrava_disentangling_2009}, any spectral differences in the filters would add extra unknown signal/noise making it harder to disentangle the faint spectral differences.


We performed the differential analysis for all targets but only show our most favourable case here, {HD\,30501}, because it is the second largest companion in our sample at 90~\Mjup{} and also has the second largest {RV} separation between observations. The differential spectra recovered for {HD\,30501} is shown at the bottom panel of \fref{fig:spectral_example}. The presence of deep (\(>4\%\)) stellar and telluric lines in the original spectrum is shaded by the blue and green regions respectively. This indicates that the features of the differential spectrum near these shaded regions are likely due to imperfect telluric correction and host mutual cancellation.

The mutual cancellation of the stellar host works well for the \(\sim40\%\) deep line near 2\,117\nm{}, being completely removed, but it does not do so well for the smaller \(\sim10\%\) deep line around 2\,121.5\nm{}. The residual for the large \(\sim40\%\) deep telluric line near 2\,118.5\nm{} is quite prominent. There is also a wider residual due to three neighbouring lines \(\sim10\%\) deep around 2\,120\nm{} which cause features in the differential spectrum. One possible explanation is that the continuum normalization near 2\,120\nm{} was influenced by this grouping of lines.

To understand the observed differential signal we simulated a differential spectrum of {HD\,30501} using a synthetic {PHOENIX-ACES} spectra with parameters \(\teff{}=2\,500\)\K{}, \logg{}=5.0, and \feh{}=0.0, with a {RV} offset estimated from the observation times. These parameters represent an estimated companion \(\teff{}\) with the metallicity and \logg{} similar to the host (closest grid model). The model spectra were convolved to \(\R=50\,000\), continuum normalized and scaled by the estimated flux ratio of the companion. We do not include any synthetic host or telluric spectra and as such simulate the differential result of a ``perfect'' host cancellation with no telluric contamination present. This is the ideal-case scenario, and we stress that it is impossible to simulate the effect of improper telluric correction in a meaningful way. When comparing the simulated and observed differential in the bottom panel of \fref{fig:spectral_example}, there is a striking amplitude difference. The orange-dashed line of the simulated differential spectrum amplitude is of a much smaller scale than the observed differential. This demonstrates that the amplitude of the differential signal we are trying to detect is much smaller than the residuals created by this differential technique.

The amplitude of the differential signal is lower than we expected due to the very low \(\Delta {RV}\) between the observation pairs. The maximum \(\Delta {RV}\) between observation pairs, for the observations investigated in this work, are provided in \tref{tab:estimated_rv}. {\red{} There is no \(\Delta {RV}\) for {HD\,4747} as there was only a single observation. We also provide the phase coverage for our targets. We calculate this as the ratio of time between the observed pairs and the orbital period, and show that the fraction of the orbit covered is very small, all except one covering less than 1 percent of the orbit.}

In our best case, {HD\,30501}, the \(\Delta {RV}\) of the companion between observations is 1.34\kmps{}. For comparison, a single Gaussian absorption line, to be shifted by \(\Delta\lambda = {\fwhm}\) would need a \(\Delta {RV}\) of \(v_{\fwhm} = c/\R =~\sim6\)\kmps{}. Since the \(\Delta {RV}\) are shifted by a smaller value than the {\fwhm}, the spectral lines of the reconstruction mutually cancel themselves, diminishing the amplitude of the differential signal significantly. As the companion spectra are already faint (with a flux ratio at the percent level) the differential signal is not detectable within these observations and noise level.

When the \(\Delta {RV}\) of the companion is smaller than the {\fwhm} of a line there is a mutual subtraction of the companion spectra, diminishing the detected amplitude of the differential signal, and removing the ability to detect the companions using this method. Observations need to be spaced further apart in time/phase to achieve a larger \(\Delta {RV}\) separation and increase the amplitude of the differential. Of course once there is a separation there will be complex interactions between neighbouring lines that need to be accounted for.

\begin{figure}
    \centering
    \includegraphics[width=0.8\hsize]{figures/direct-recovery/differential.pdf}\\
    \caption{(Top) A reduced {CRIRES} observation of {HD\,30501} (blue) for detector 1 between 2\,112--2\,124\nm{} along with the tapas telluric absorption model ({orange} dashed) used for the wavelength calibration and telluric correction.
        (Middle) The telluric corrected spectra.
        (Bottom) ({blue}) Differential spectra for {HD\,30501} between observations 1 and 3. ({orange} dashed) Simulated ``perfect'' differential using {PHOENIX-ACES} spectra with parameters \(\teff{}=2\,500\)\K{}, \logg{}=5.0, and \feh{}=0.0, with the same \(\Delta {RV}\) as the observations.
        The shaded regions indicate where the telluric {green} and host star {blue} spectra are \(> 4\%\) deep.}
    \label{fig:spectral_example}
\end{figure}

Since the amplitude is so small, and the we did not attempt to pursue this further..
We did not attempt the binary reconstruction from the combination...

\citet{ferluga_separating_1997} present a reconstruction method to recover the secondary from the s-profile. Unfortunate this was not appropriate due to insufficient separation.


\subsection{Relative differential amplitude}
To investigate the differential subtraction under tiny \(\Delta RV\) further the change in differential amplitude against variation in {RV} was explored. 
Simulations were performed creating a differential spectra for a range of \(\Delta {RV}\)s between \(\pm10\)\kmps{} using the same {PHOENIX-ACES} spectra for the companion of {HD\,30501} (\(\teff{}=2\,500\)\K{}, \logg{}=5.0, \feh{}=0.0) convolved to \(\R=50\,000\). These simulations were focused between the wavelength range 2\,110--2\,123\nm{}, corresponding detector 1 of the {CRIRES} observations. The differential spectra was created for each by taking the synthetic spectrum for the companion, Doppler shifting a copy the spectrum and subtracting it from the original. At each {RV} step the maximum absolute differential amplitude (peak to peak) of the simulated differential spectrum observed was recorded. Again these simulations are performed assuming perfect telluric correction and removal of the host star by only considering the spectrum of the companion alone. 

The result of this simulation is given in  \fref{fig:diff_amp}. As this absolute amplitude is specific to the lines present in the analysed wavelength range, the values were normalized by the median amplitude value outside of the line {\fwhm} (dashed vertical lines), between \(\pm(7-10)\)\kmps{}, to give a relative differential amplitude, independent from the depth of a specific line. Differential subtraction simulations we also performed using a spectrum made up of a single Gaussian line and a single Lorentzian, these are shown in \fref{fig:diff_amp} as the orange dashed and green dash-dotted lines respectively. The spectral profile shape of differential for the Gaussian line was also checked for consistency with the analytical form of the differential spectra~\citet[][Equation~A.1]{ferluga_separating_1997} (included above as \eref{eqn:sprofile_gaussain}).

\fref{fig:diff_amp} that a \(\Delta {RV}\) of zero between companion spectra, the spectral lines of the companion completely cancel each other out, resulting in zero amplitude. As the {RV} separation increases in either direction, the individual lines begin separating, and stop cancelling themselves out. A maximum differential amplitude is achieved when the individual lines are fully separated. We did not consider the shape/width of the differential spectral lobes as done in \citet[][eqn.~A.1]{ferluga_separating_1997}, but this could also have been measured.

In the simulations of the synthetic spectrum (and of course real spectra) neighbouring spectral lines begin to strongly interfere, leading to a variable measured relative amplitude beyond 10 km/s.  The shape of the relative amplitude becomes complicated due to the interaction but since the \(\Delta {RV}\) for all the observations fall well short of this region it was not investigated further. It is suspected that the interaction of neighbouring lines is one possible cause for the difference in the relative differential amplitude between the single theoretical line profiles and synthetic spectrum between 2 and 6\kmps{}. 

The vertical dotted lines indicate the line \(\rm {\fwhm} = \lambda /\R=v /c \)velocity of 6\kmps{} at  2\um{} with \R=50\,000, showing that the amplitude is almost maximum when the lines are separated beyond their line width. The two solid vertical lines in \fref{fig:diff_amp} indicate the  \(\Delta \textrm{RV}\)=1.34\kmps{} separation calculated for our best target, {HD\,30501} from \tref{tab:estimated_rv}, given known orbital parameters and the observation times. This shows that our differentials have severely reduced amplitude, \(<20\%\) relative to well separated individual lines. As the companion spectra are already faint and in combination with a host star at >1\% flux ratio the >80\% extra reduction in signal amplitude makes this detection impossible with these observations.

\begin{figure}
    \centering
    \includegraphics[width=0.8\textwidth]{figures/direct-recovery/rv_diff_final.pdf}
    \caption{Simulated relative amplitude of differential spectra at different companion \(\Delta RV\) separations revealing the diminished amplitude at very small orbital separations. 
        The solid blue line shows the maximum relative amplitude of the differential signal (from a shifted copy of itself) of a {PHOENIX-ACES} spectrum with \(\teff{}=2500\)\K{}, \logg{}=5.0, \feh{}=0.0 in the wavelength region 2\,110--2\,123\nm{}. 
        The maximum difference is normalized by the median amplitude between \(\pm7\)--10\kmps{}, representing a complete line separation. 
        The orange (dashed) and green (dot-dashed) lines represent the relative amplitude of a differential spectrum of a spectrum containing a single Gaussian and single Lorentzian absorption line respectively, each with a unitary amplitude and a \(\rm {\fwhm} = \lambda / R\). 
        The solid vertical lines indicate the estimated companion \(\Delta {RV}\) in these observations while the dashed vertical lines indicate the {RV} corresponding to the {\fwhm} at this wavelength and resolution. .}
    \label{fig:diff_amp}
\end{figure}
\unfinished{Increase size of diff amp plots to two}








\subsubsection{Differential scheduling challenges {copy from paper}}
\label{subsubsec:differential-schedualing}
This work has revealed that more care needs to be taken in planning the observations for the spectral differential analysis of faint companions in the future.
Paying attention in particular to the {\fwhm} of the lines in the region (governed by resolution and wavelength); the estimated companion \(\Delta {RV}\); the previous observations from different observing periods; and keeping consistent detector settings.

The original goal for the observations was to obtain two different and ``clearly separated radial-velocities'' for the secondary companion.
However, the program was assigned a low-priority (C, in ESO grading) and, possibly due to operational reasons, the original time requirements necessary to secure well separated RVs for the companion spectra could not be met.
This meant that all observations were insufficiently separated to extract a differential spectra for the companion.

The long orbital periods of these targets is also a contributing factor to the insufficient separations.
Most of the targets observed here have orbital periods much longer than an observing semester (183 days).
An optimal pair of observations (achieved at the extrema) would need to have been obtained from separate observing periods (between 2 months and 19 years apart).
In some cases, even observations taken at the beginning and end of a single observing semester would not be sufficient to achieve companion separation (depending on the phase).
Requiring separate observing periods to even achieve the minimum \(\Delta \rm {RV}\) larger than the line {\fwhm}.
At the time it was impossible to ask for time over several semesters in a regular proposal.

Our study demonstrates the importance of proposals for projects that need to be extended over several semesters or years. In the ESO context, this corresponds to ``Monitoring proposals''~\citep[e.g.,][pg. 18]{eso_eso_2017}.
Observations of the targets explored here, with long orbital periods in particular, would benefit from the ability for multi-period proposals and newer scheduling systems which allow for tighter scheduling constraints, such as a companion {RV} separation.

For future observations we suggest that the known orbital solution of the companion be used to estimate the companions' {RV} curve during the observing period, with the companion \mtwosini{} providing an {RV} upper-limit.
Knowing the instrumental wavelength and resolution, a constraint can then be set to avoid taking observations when the companion spectra are insufficiently separated, or \(\Delta {RV}\) < {\fwhm}.\@This constraint can be set using the absolute and relative \emph{time-link} constraints available in {ESO}'s {Phase 2 Proposal Preparation} (P2PP) tool.
Additionally, analysing the known orbital solution before-hand, to determine {RV} constraints will also help identify the best time to observe, if observations from separate periods will be required or, if an optimally separated companion differential is even feasible.


It was unable to be determined if the time differential was asked for in the 2nd proposal round....




\textbf{
CHECK out LOCKWOOD 2014 - maximum likelihood with todcor 1e-4 flux ratio double lined spectra}




\section{Direct recovery in the \mir{}}
It was investigated if this differential technique could be extended into the mid-infrared {\mir{}} domain.
There were two reasons for this, to develop experience with the {\mir{}} domain where the contrast ratios are higher and due to the lack of high-resolution \nir{} spectrographs available at the time, as {CRIRES} was being upgraded to {CRIRES+}.

{VISIR} is a \mir{} spectrograph on the {VLT}, offering diffraction-limited imaging at high sensitivity in three mid-infrared (MIR) atmospheric windows: the \emph{M}-band at 5\um, the \emph{N}-band between 8--13\um and the \emph{Q}-band between 17 -- 20\um, respectively.
We explored the use of {VISIR} to detect a the spectra of Brown Dwarf companions in the {\mir{}}. The candidate selected as the best target to investigate was {HD\,219828} which has a hot-Neptune (\mtwosini{}=0.066\Mjup) \citep{melo 2007} and a recently discovered super Jupiter (\mtwosini{}=15.1\Mjup) on a long period (13 yr) eccentric orbit (e=0.81)\citep{santos 2016}.

Based on the spectra of a cool brown dwarfs in the \mir{}, and the detector configuration available at the time the chosen detector settings were the low resolution mode covering the wavelength region 8--13 \um. This wavelength region would have encompassed the \ce{NH4} signature at 10.5\um and the edge of  a \ce{CH4} band at 7.7\um, both large features in the BD \mir{} spectrum.

After performing flux ratio calculations between the host and companion using the \citet{barrafe_evolution_2003} models and considering the performance of the {VISIR} instrument and the exposure time calculator it was determined that observations with {VISIR} to achieve a {SNR} of 100 were infeasible, requiring 1\,000's of hours observing time to achieve the necessary signal-to-noise level to separate the companion from a blended spectra. As such the direct separation approach was not continued further in the \mir{}.



...

This method is very similar to~\citet{kostogryz_spectral_2013} except that they focus on M-dwarfs host stars with the observations taken at the extrema, in which the companion lines are well separated. \todo{They get a result???}

\section{Summary}
This Chapter  presented the observations that were gathered having in mind the application of a differential subtraction method~\citep[e.g.,][]{ferluga_separating_1997, kostogryz_spectral_2013} to recover the spectra of the faint {BD} companions.  Due to the poorly separated observation times relative to the long orbital periods, the differential subtraction method presented in \sref{sec:direct-subtraction} was revealed to be inappropriate for these observations as the {RV} separation of the companion spectra between observations is significantly smaller than the width of individual spectral lines. The small separation of the companion causes the lines of the companion to also mutually cancel, severely reducing the residual signal to well below the available noise level. The requirement of well separated RVs for the companion spectra was clearly stated in the original proposal but was not satisfied by the observations. {\red{} The very large orbital periods of some of the targets would not produce a sufficient {RV} signal during one semester. This was a possible oversight during the proposal stage.} The largest estimated companion \(\Delta {RV}\) separation between the observations of each target is provided in the seventh column of \tref{tab:estimated_rv}. \todo{move}{Radial velocity constraints are also valid for other studies such as the detection of reflected light from exoplanets~\cite{martins_evidence_2015}.}
