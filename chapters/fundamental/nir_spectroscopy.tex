%!TEX root = ../../thesis.tex

NIR spectroscopy...

Spectrograph design stuff


\section{NIR spectroscopy}
To perform spectroscopy requires an instrument called a spectrograph. The purpose of which is to take the incoming star light and disperse it so that the different wavelength components separate. A simple example of this is the splitting of white light into a rainbow when passed through a prism.

All spectrographs are comprised of a few key components.
The first part is the front end optics responsible for collecting the light and getting it to the spectrograph, such as a telescope, entrance slit or fibre.
The second part component dispersive elements inside the spectrograph, such as a prism or grating used to separate the wavelength components.
The final part is the detector or camera, used to record the resulting spectrum.

The design of spectrographs is highly dependant on the wavelength range observed. For instance {CCD} cameras (based on silicon) are readily used in the optical but silicon is a poor detector at \nir{} wavelengths.
Instead \nir{} spectrographs need to use CMOS detectors which are composed of different materials that are sensitive to different wavelengths.
{CRIRES} detectors specifically are made of \ce{InSb} (Indium antimonide) to perform spectroscopy between 1--5\um{}~\cite{dorn_crires_2004}.
Other considerations are needed such as the composition of the optical components as well as any anti-reflection coatings used, all of which have different wavelength characteristics that need to be matched.

\todo{I don't know where I am going with this section.}
