%!TEX root = ../thesis.tex

\chapter{Synthetic companion recovery}  % Main chapter title
\label{cha:model_comparison}

Following on from \cref{cha:direct_recovery}, in which the differential subtraction technique was unsuccessful due to the insufficient separation of the observations, a second method is attempted to detect the presence of the faint {BD} companions in the blended spectrum.
In this chapter a \textchisquared{} fitting approach is used to fit the observed spectra with a binary model comprised of two synthetic spectral components.
An overview of the method is presented followed by the preparation of the models.
A simulation with a larger M-dwarf companion to a G2V star is presented first, followed by a simulation and observations of {HD~211847}.
Injection-recovery simulations are performed to understand the limitations of this method and the results obtained.
A discussion of the results and a comparison to other similar techniques is given at the end.

%!TEX root = ../../thesis.tex


\section{Binary \texorpdfstring{\textchisquared}\ \ spectral recovery}
\label{subsec:companion_recovery}
The approach developed here is to fit the observed CRIRES spectra, consisting of a {FGK} star with a {BD} companion, with a model comprised of a combination of two synthetic spectra, one for each component.
This will be done using the \textchisquared{} approach which has been extensively used in the literature~\citep[e.g.][to list a few]{astudillo-defru_harps_2015, passegger_fundamental_2016, passegger_carmenes_2018, zechmeister_spectrum_2018, nemravova_xtauri_2016, kolbl_detection_2015, rajpurohit_exploring_2018}.
The recoverable information from the fitting will be the parameters of synthetic spectra fitted to the companion which will provide some indication of the companion temperature and spectral type, but it will not produce a direct mass constraint that was the original aim of the observations.

\subsection{The \texorpdfstring{\textchisquared}\ \ method}
\label{subsec:chi2}

The well known \textchisquared{} technique measures the weighted sum (for all data points \(i\)) of the squared deviation between the observation (\({O}\)) and the computed models (\({C}\)), with the minimum \textchisquared{} value representing the best-fit parameters:
\begin{equation}
\chisquared{} = {\sum}_i \frac{{(O_{i} - {C}_{i})}^2}{{\sigma}_{i}^{2}},
\end{equation}
where \({\sigma}_{i}\) is the error on each measurement.

The inverse survival function of the \textchisquared{} distribution is used to determine the confidence levels of the resulting parameters from the minimum \textchisquared{}.
The inverse survival function returns a \(\Delta\chisquared\) value from the minimum \textchisquared{} value for a given $\sigma$ level and degree of freedom.
This can be achieved in \Python{} with the \scipy{} package as the single line of code \pythoninline{scipy.stats.chi2{(dof)}.isf{(1-p)}}, where \textit{dof} is the degree of freedom and \(p\) the probability; for example \(p = 0.68\) for 1-\(\sigma\).
For example, the \(\Delta \chisquared\) for a single degree of freedom required for the 1-, 2-, and 3-\(\sigma\) confidence levels is 1, 4, and 9 respectively~\citep{bevington_data_2003}.
This method assumes that the measured flux is observed with a \snr{} sufficiently high so that the noise on the spectrum is approximately Gaussian, and the \textchisquared{} method appropriate.

For a given observation, the \textchisquaredreduced{} is computed as \(\chisquaredreduced{}=\chisquared{} / \nu\) where \(\nu = n - m\), the number of observed pixels, \(n\), minus the number of parameters of interest, \(m\)\footnote{\(m=2\) or \(m=4\) in the examples explored below.}.
In the cases explored below the \textchisquaredreduced{} is only calculated after the summation across the detectors is performed.

For each observation \(O\), the \(\sigma\) is estimated using the \(\beta\sigma\) method~\citep{czesla_posteriori_2018}, using the {MAD} (median absolute deviation about the median) robust estimator.
The \(\beta\sigma\) method estimates the spectral noise of the spectra using a series of numerical derivatives\footnote{Applying a Taylor expansion to the spectra.}.
We followed the procedure outlined in~\citet{czesla_posteriori_2018} to analyse the results from successive parameter combinations to settle on an order of approximation (derivative level) of \(N=5\), and a jump parameter (pixels skipped to avoid correlations) of \(j=2\).
The same \(\sigma\) value determined is applied to all points in the spectrum so that \({\sigma}_{i} = \sigma\).
The \(\beta\sigma\) method provides \(\sigma\) estimates for the target spectra which correspond inversely to \snr{} values between 100--400.
These \snr{} values are similar to the values given in \cref{tab:observations} which were calculated only using the standard deviation of the continuum of detector \#2.

The computed models \(C\) are described in \cref{models} and result in a multidimensional grid of \textchisquared{} values for each combination of model parameters, namely the spectral parameters  (e.g.\ \Teff{}), and the {RV} of the host and companion {RV} for each: detector, observation, and target.

The global minimum of the multidimensional \textchisquared-space is used to represent the best fitting model combination to the observed spectra.
The multidimensional \textchisquared{} grid is summed across multiple detectors to also determine a global minimum \textchisquared{} for the whole observation \(\chisquared_{obs} = \Sigma^{N}_{n=1} \chisquared_n\), where \(N\) is the number of detectors used.
However, the separate observations are not combined as the {RV} parameters of the host and companion will vary between each observation\footnote{Since the current observations are insufficiently separated, it may be possible to combine the separate observations; but in general this would not be the case.}.
To incorporate the separate observations a model that incorporates the phase information will be required and is beyond the scope of this work due to the limited number of observations (1--4).
A promising method to incorporate the phase information for the detection of exoplanetary spectra is given by~\citet{lockwood_nearir_2014,piskorz_evidence_2016}.
They detect evidence of an exoplanet spectrum with an contrast of order \({10}^{-4}\) using \nir{} (\textit{L}- and \textit{K}-band) spectra with a \snr{}\(\approx2000\) observed at six epochs over the whole orbit.


%%%%%%%%%%%%%%%%%%%%%%%%%%%%%%%%%%%%%%%%%%%%%%%%%%%%%%%%%%%%%%%%%%%%%%%%%%%%%%%
%%%%%%%%%%%%%%%%%%%%%%%%%%%%%%%%%%%%%%%%%%%%%%%%%%%%%%%%%%%%%%%%%%%%%%%%%%%%%%%

\subsection{Computed model spectra}
\label{models}
In this section the transformation of the synthetic {PHOENIX-ACES} spectra into the computed models (\(C\)) is explained.
These computed models will then be fit to the observed spectra.

Firstly, this assumes that the synthetic spectra are loaded and converted to consistent units.
The loading is easily performed using Starfish's \href{https://iancze.github.io/Starfish/current/grid_tools.html}{GridTools}~\citep{czekala_constructing_2015}, which can load the library spectra with a list of stellar parameters [\Teff{}, \Logg{}, \feh{}, \alphafe{}].
The {PHOENIX-ACES} spectra are converted from the units of the spectral energy distribution (\(\rm erg\,s^{-1}\,cm^{2}\,cm^{-1}\)) (at the stellar surface) into photon flux (\(\rm photon\,s^{-1}\,cm^{2}\)) by dividing by the photon energy\footnote{Photon energy \(E=\frac{hc}{\lambda}\), where \(h\), \(c\) and, \(\lambda\) are Plank's constant, the speed of light, and wavelength, respectively.}.
This can be simply achieved by multiplying the spectrum by the wavelength (multiplicative constants ignored), as done in~\citet{figueira_radial_2016}.
The spectra are also convolved with a Gaussian kernel to the instrumental resolution of the observations, in this case R=50\,000.
Due to the distributive property\footnote{\(I(\lambda) \ast (f(\lambda) + g(\lambda)) = I(\lambda) \ast f(\lambda) + I(\lambda) \ast g(\lambda) \)} of convolution the instrumental broadening is performed on each individual library spectrum first.
This is to avoid performing convolution for each combination of model parameters in the binary model after the spectra have been combined.
It would be more computationally expensive to perform the convolution on every model combination, \(C\).

These synthetic spectra are used individually for the single component model (\cref{subsubsec:single-model}) and also combined together into a binary model (\cref{subsubsec:binary-model}).
The results of these models are then interpolated to the wavelength grid of the observed spectra and the \textchisquared{} calculated by comparing the model and observation at each wavelength point, \(i\).

\subsubsection{Single component model}
\label{subsubsec:single-model}
The single component model \(C^{1}_{i}\) comprises of a single synthetic spectrum, \(J\), (with model parameters \Teff{}, \Logg{}, \feh{}, \alphafe{}) \alphafe{} that can be Doppler shifted by a {RV} value \Rvone{}.
\begin{equation}
C^{1}_{i}(\lambda) = \J{}(\lambda_{0}(1-\frac{\rvone}{c}))
\end{equation}
where \(\lambda\) is the shifted wavelength, \(\lambda_{0}\), the model rest wavelength and, \(c\), the speed of light in a vacuum.
The model's flux in the wavelength region of the observations is continuum normalized to unity to match the observed spectra, and then interpolated to the wavelength grid of the observation.

This single component model analysis is similar to the~\citet{passegger_fundamental_2016} \textchisquared{} fitting, using {PHOENIX-ACES} spectra to fit and determine the parameters of {M-dwarf} spectra.
A similar re-normalization (see \cref{subsec:renorm}) as~\citet{passegger_fundamental_2016} is used to account for slight differences in the continuum level and possible linear trends between the normalized observation and models.
However, unlike~\citet{passegger_fundamental_2016}, no dynamical masking was applied to sensitive lines to make the \textchisquared{} minima more distinct nor a linear interpolation of the stellar parameters between the grid models to obtain higher precision stellar parameters.
This was because, at this stage, only the presence of the secondary is trying to be detected, not a determination off precise stellar parameters.
These processes and others could however be included in the future to improve the detectability and precision of the best-fit model.
Instead, a radial velocity component is included in the \textchisquared{} fitting, which is not included in~\citet{passegger_fundamental_2016}.

{\red Rotational broadening of the host was not included in these models as an extra variable fitting  parameter.
In \citet{passegger_carmenes_2018} rotational broadening is only included at the fine grid search stage, using only a fixed value for each target to determine, determined in a separate work, to reduce the number of parameters.
A fine search is not attempted in this work.}

\subsubsection{Binary model}
\label{subsubsec:binary-model}
In the binary model case the model is considered to be the superposition of two synthetic spectral components, one each for the host and companion respectively.
Both components are Doppler shifted by \Rvone{} which represents the {RV} motion of the host star relative to Earth's barycentre, while the companion spectra is Doppler shifted a second time by \Rvtwo{} which represents the {RV} difference between of spectrum the host and companion.
This choice is arbitrary\footnote{Having \Rvtwo{} as the companion's {RV} offset relative to the barycentre is also a valid choice.}, but in this way the mean motion of the system relative to Earth is captured only in \Rvone{}, along with the orbital motion of the host.
The two spectra are scaled by their radius squared (see \cref{subsection-radius}) then added together, thus setting the relative amplitude of the two spectral components.
Given two spectral components \Jone{} and \Jtwo{} with radii \Rone{}, \Rtwo{} this equates to
\begin{align}
C^{2}_{i}(\lambda) = & \jone{}(\lambda_{0}(1 - \frac{\rvone{}}{c}))\times {\rone}^{2} +\nonumber \\
& \jtwo{}(\lambda_{0}(1 - \frac{\rvone{}}{c})(1 - \frac{\rvtwo{}}{c}))\times {\rtwo{}}^{2}
\end{align}

The combined binary model is continuum normalized by dividing it by an exponential fitted to the continuum of the combined spectrum.
Here at 2100\nm{} it assumed that the Rayleigh-Jeans regime is appropriate.
This assumption is wavelength dependent and other continuum normalization techniques may also be valid.
In the case of a {BD} companion around an {FGK} star investigated here, the continuum is dominated by the contribution from the host star, contributing the majority of the spectrum with flux ratios below \(\sim\)1\%, in the wavelength range 2110--2165\nm{}.

Models are combined in this way to represent the correct absolute flux ratio of the spectral components.
A further method could allow a variable flux ratio to be included as an extra parameter and be fitted as well.
However, it would add an extra dimension to the \textchisquared{} grid and potentially add more degeneracy between models of the companion.
The median flux ratio between the two components is calculated for the wavelength range used here as an indication of the flux ratio level.
This is given as \FtwoFone{} in \cref{tab:example_params}.

This binary model should provide meaningful information about the likely companion parameters (e.g.\ \Teff{}) and a possible estimate of the flux ratio of the system.
These can be combined with the~\citet{baraffe_evolutionary_2003} models to constrain the mass of the companion.
However, care is needed with the binary model as the inclusion of extra spectral components and associated parameters could also provide a ``better'' fit to observations which have faint or even no companions, by fitting components of the noise.

The full list of grid parameters for the binary model are \(\teffsub{1}\), \(\logg{}_{1}\), \feh{}\(_{1}\), \alphafe{}\(_{1}\), \Rvone{}, \(\teffsub{2}\), \(\logg{}_{2}\), \feh{}\(_{2}\), \alphafe{}\(_{2}\), \Rvtwo{} where the subscripts 1 and 2 indicate the host and companion models respectively.


\subsubsection{Effective radius}
\label{subsection-radius}

The {PHOENIX-ACES} spectra are calculated at the stellar surface.
To combine the two synthetic spectra with the correct absolute flux ratio the stellar size needs to be accounted for.
The emitted flux needs to be integrated over the effective surface area of each emitting body respectively.
Ignoring the common multiplicative constants that will not affect the ratio between spectra and disappear with normalization the two synthetic spectra of the binary model are individually scaled by the square of their respective radii, \(R_{1}\) and \(R_{2}\) \Rone{}.

In this work the radius used to scale the spectra is the effective radius of each component from the {PHOENIX} model header; the {PHXREFF} keyword.
This radius is utilized in the modelling of the {PHOENIX-ACES} stellar atmospheres.
This is used as it is directly tied to each model spectrum, and already calculated and available.
In this way it does not incorporate an extra assumption or model relating the library model parameters to a stellar radius.
The ratio of the radii from the two synthetic spectra in the binary model examples presented are provided in \cref{tab:example_params} as \(\rone/\rtwo\).

Using the radii in this way for the companions has its limitations because as stated previously, there is a degeneracy in {BD} mass, age, and luminosity of the companion, and in particular a combination of radius-mass and radius-age relationships~\citep{sorahana_radii_2013}.
Using the {PHOENIX-ACES} model effective radii does not allow for any independent age constraints to be incorporated into the stellar radius, or allow for any variability in the radii to account for uncertainties.

The targets analysed here do not transit, but in cases that did transit the radius ratio can be independently determined from the photometric transit method~\citep{deeg_photometric_1998}.
This independent radius ratio could be used as a constraint when combing the binary model spectra.

\todo{Radii of different parameters}

\subsection{Re-normalization}
\label{subsec:renorm}
Slight trends in the continuum level between the observed spectra and computed models were removed using the re-normalization following~\citet{passegger_fundamental_2016}:
\begin{equation}
F^{obs}_{re-norm} = F^{obs} \cdot \frac{\textrm{continuum fit}_{model}}{\textrm{continuum fit}_{observations}}.
\end{equation}
The polynomial continuum fits to the normalized observations and models are used to re-normalize the observed spectrum to the continuum of the models.
For detectors \#1--3 a polynomial of first degree was used, while for detector \#4 a polynomial of second degree was needed to fit the edge of a strong Hydrogen line (Brackett-\(\gamma\)) at 2166\nm{}, which lies just off of the edge of detector \#4.
This broad line is only observed in the synthetic spectra but not in the reduced observations.
It is assumed that the edge of this line was normalized out during the reduction process.

For each model the continuum level is further allowed to be slightly varied by \(\pm0.05\) as a free parameter taking the model that fits with the smallest \textchisquared{} value as the choice for this combination of binary parameters.


\subsection{Reducing dimensionality}
\label{subsec:reduce-params}
The high dimensionality of the binary model makes it computationally challenging and difficult to analyse the \textchisquared{} space.
This section discusses the choices made to reduce the dimensionality and the parameter space, to reduce the computation time.
For reference, the parameter space of the models is multiplicative.
That is, each new dimension or parameter added multiplies the number of possible model combinations.
When increasing from the single model to the binary model, the number of parameters doubles, but the number of possible models is actually squared, assuming that each spectral component can explore the full parameter space.
The binary model is therefore more computationally expensive.
In general the number of possible parameter combinations for \(k\) spectral components each with a grid of \(l\) models increases to \({k}^{l}\).
If the full set of {PHOENIX-ACES} library spectra (66456) is explored with a binary fit then this naively balloons to over 4.4 billion possible combinations.
This is the worst case scenario as half of these combinations are not unique as the host and companion components will be swapped.
A number of assumptions are implemented to vastly reduce the parameter-space enabling faster computation.

The first assumption is to restrict the Alpha element abundance (\alphafe{}) of the models to zero.
This is likely a very good approximation as all the targets have solar metallicity and are thus very likely to belong to the thin disk of the Galaxy, where \alphafe{} values are close to zero, i.e.\ solar~\citet[e.g.][]{adibekyan_chemical_2012}.
The second assumption is that the search space can be significantly reduced by using literature values of the host star given in \cref{tab:star_params}.
The metallicity of both model components are fixed to the closest grid to the literature value of the host star, usually \feh{}=0.00.
The \Logg{} of the host star is also fixed to the closest grid literature value.
The uncertainties on the literature measurements for \Logg{} (\(\sim\)0.1) and metallicity (\(\sim\)0.05) are both smaller than the grid steps of 0.5 for these parameters.
The \Logg{} of the companion is obtained from the~\citet{baraffe_evolutionary_2003,baraffe_new_2015} evolutionary models for the given companion's mass (\Mtwo{} or\Mtwosini{}) and host's age.

A starting point for \(\teffsub{2}\), the estimated companion temperature from the Baraffe evolutionary models is used, given the companion mass and stellar age.
The temperature grid is extended about this value in each direction, within the temperature limits of the synthetic model limits.
For example the companion temperature grid spans \(-600\) to \(+400\)\K{} in \cref{fig:Mdwarf_contours} and \(\pm400\)\K{} in \cref{fig:HD211847_simulated_contours,fig:HD211847_result_contours} about the estimated companion temperature.

The large number of possible combinations stated above also do not include the {RV} grid for each component, which is user defined and can have a variable resolution, and amplitude.
For example, the number of models to compute increases  with a decrease in the grid step size (a finer {RV} resolution) for a fixed {RV} range.
The {RV} grid space can be reduced significantly by tailoring it to the target being examined.
For each target and observation, the estimated {RV} values from \cref{tab:observations} are used as a centre starting point for the \Rvone{} and \Rvtwo{} values and then incremented within a few {\fwhm} around those values, or out to the targets \Kone{} or \Ktwo{} values.

An iterative process could potentially be implemented to refine the {RV} grids, starting with a larger grid with lower {RV} resolution and then performing a higher resolution grid about the minimum \textchisquared{} {RV} values.
This was attempted manually during testing but it could be easily automated in the future, at the cost of recalculating the \textchisquared{} at different {RV} resolutions.
One could expect that a good starting {RV} grid step be governed by the spectral resolution, e.g.\ comparable to the {\fwhm} velocity.

Also for the targets with a fully resolved orbit of the known {RV} of the host star, \Rvone{} could also be held fixed, however this was not performed in this work.


%!TEX root = ../thesis.tex

\todo{Example with a large mass companion? and/or \snr{}.}

\section{Simulation and results}
\label{sec:chi2_results}

%!TEX root = ../../thesis.tex

\begin{table*}
      \centering
      \begin{threeparttable}
          \caption[\textchisquared{} simulation results.]{Input and recovered parameters on simulations and an observation when applying the single (\(\rm C^1\)) and binary (\(\rm C^2\)) models.
              The \Logg{} and metallicity were fixed at values of \(\logg{}_{1} = 4.50\), \(\logg{}_{2}=5.0\) and \feh{}\(_1\) = \feh{}\(_{2}\)=0.0.
              Gaussian noise with a \snr{} of 150 was added to both simulations.
              The number of data points and parameters used in each model are \(m\) and \(n\) respectively.}
          \begin{tabular}{c | *3c | *3c | *3c}
              \toprule
              & \multicolumn{3}{c|}{Simulation 1} & \multicolumn{3}{c|}{Simulation 2} & \multicolumn{3}{c}{Observed {HD 211847}} \\
              \midrule
          & Input & \multicolumn{2}{c|}{Recovered} & Input & \multicolumn{2}{c|}{Recovered} & Expected & \multicolumn{2}{c}{Recovered} \\
          & & \(C^1\) & \(C^2\) & & \(C^1\) & \(C^2\) & & \(C^1\)  & \(C^2\) \\
          \midrule
          \(\teffsub{1}\) & 5800 & 5800 & 5800 & 5700 & 5800 & 5700 & \(5715 \pm 24\) & 5900 & 5800 \\
          \(\teffsub{2}\) & 4000 & -- & 3800 & 3200 & -- & 3100 & \(\sim\)3200 & -- & >3800\tnote{a} \\
          \Rvone{} & 0 & 0.1 & 0 & 6.6 & 6.6 & 6.6 & \(6.6 \pm 0.3\) & 7 & 7.6 \\
          \Rvtwo{} &  10 & -- & 9.8 & 0.5 & -- & -1 & \(0.5 \pm 2\) & -- & -12.6 \\
          \midrule
          \(R_1/R_2\) & 2.57 & -- & 2.71 & 3.16 & -- & 3.27 & 3.16 & -- & <2.71\tnote{a} \\
          \FtwoFone{} & 0.084 & -- & 0.066 & 0.030 & -- & 0.026 & 0.030 & -- & >0.066\tnote{a} \\
          \(m\) & -- & 3072 & 3072 & -- & 3072 & 3072 & -- & 2612 & 2612 \\
          \(n\) & -- & 2 & 4 & -- & 2 & 4 & -- & 2 & 4 \\
          \textchisquared{} & -- & 4978 & 3792 & -- & 3746 & 3630  & -- & 37\,688 & 33\,860 \\
          \textchisquaredreduced{} & -- & 1.62 & 1.24 & -- & 1.22 & 1.18 & -- & 21.3 & 19.2 \\
          {BIC} & -- & -20\,145 & -22\,315 & -- & -21\,477 & -21\,377 & -- & 18\,281 & 14\,468 \\
          \bottomrule
        \end{tabular}\label{tab:example_params}
        \begin{tablenotes}
            \item [a] {At the arbitrary upper limit for companion temperature grid (3800\K{}).}
        \end{tablenotes}
    \end{threeparttable}
\end{table*}


\begin{figure*}
    \centering
    \includegraphics[width=0.7\linewidth]{figures/companion_recovery/Mdwarf_pcolors}
    \caption[\textchisquared{} contour for companion recovery of a simulated Sun - M-dwarf binary.]{\textchisquared{} results for companion recovery of a simulated binary observation of a Sun-like star (\(\teffsub{1}=5800\)\K{}) with an M-dwarf companion (\(\teffsub{2}=4000\)\K{}).
        The top right plot shows the application of a single component model (\(C^1\)) while the other three are using a binary model (\(C^2\)).
        Both left hand panels show the distribution of host temperature and host {RV}.
        The top right panel shows the distribution for host and companion temperature, and the bottom right the companion temperature and radial velocity.
        The red circle and yellow star indicate the location of the simulation input and recovered parameters respectively.
        The white line shows a 3-\(\sigma\) confidence level about the minimum \textchisquared{} solution grid point.
        Each box is centred on the parameter values and shows the grid resolution.}
    \label{fig:Mdwarf_contours}
\end{figure*}

\begin{figure*}
    \centering
    \includegraphics[width=0.7\linewidth]{figures/companion_recovery/HD211847_example_pcolors}
    \caption[\textchisquared{} contour for companion recovery of a simulated observation of {HD 211847}.]{ \textchisquared{} results for companion recovery of a simulated binary observation similar to {HD 211847}, (\(\teffsub{1} = 5800\)\K{}, \(\teffsub{2} = 3200\)\K{}), similar to \cref{fig:Mdwarf_contours}.}
    \label{fig:HD211847_simulated_contours}
\end{figure*}

In this section some results from applying the models for companion recovery model to simulated observations and to the observation with the best estimated contrast are presented.

\subsection{Simulated binaries}
\label{subsec:simulated_binaries}
To test the companion recovery method we create simulated binary observations using {PHOENIX-ACES} spectra.
White noise was added with a standard deviation \(\rm \sigma = 1/{\snr{}}\), for a given signal-to-noise (\snr{}) level.
We then applied the grid-matching recovery technique detailed above and compared the resulting parameters to the inputs.

The results of two example binary simulations are displayed in \cref{fig:Mdwarf_contours,fig:HD211847_simulated_contours}, both simulated with a \snr{} of 150.
The input and recovered parameters for the binary components are indicated by the red circles and yellow stars respectively, and are given in \cref{tab:example_params}.
The 3-\(\sigma\) contour is shown in white on the plots to indicate the shape of the confidence level only.
The 1-\(\sigma\) contours are not shown here as they are much smaller than the temperature grid step and are not easy to visualize at this scale as they are often smaller than the marker shown at the minimum location.
Each coloured rectangle is centred on the grid point, with its shape indicating the resolution of the grid space searched.

The first simulation shown in \cref{fig:Mdwarf_contours} is for a Sun-like star with a M-dwarf companion, with a \(\teffsub{2} =4000\)\K{}.
The top-left panel shows the recovered host parameters when the single model is applied to the simulated binary.
The top-right and both bottom panels are the parameters recovered when using the binary model.
Both left-hand panels display the parameters for the host component to easily compare between models.
With both models the host temperature \(\teffsub{1}\) is correctly recovered.
The host {RV}, \Rvone{}, is 0.1\kmps{} (two grid spaces) different from the simulated value for the single component model and is correctly recovered with the binary model.

The minimum \textchisquared{} location for the companion temperature is 200\K{} below the simulated value, and the {RV} of the companion recovered is 0.2\kmps{} below the input value.
The input values for the companion are just outside of the 3-\(\sigma\) contours shown.
The flux ratio for the input is 0.08 while the flux ratio recovered is 0.066.

The second simulation shown in \cref{fig:HD211847_simulated_contours} is performed with parameters to mimic the observation of our target with highest flux ratio, {HD 211847}.
In this simulation the single component model recovers a host with the correct {RV} but a temperature 100\K{} higher than the input value.
Again, adding the companion with the binary model recovers the correct host temperature.
The companion temperature recovered is 100\K{} lower than the input temperature and the {RV} is different by 2\kmps{} which is around one third the {\fwhm}.

In this case with a companion {RV} offset, \Rvtwo{}, near 0\kmps{} the host and companion lines are blended.
The same spectral lines from both components are trying to match to the same features of the spectra, making it more difficult to recover the companion parameters.
In the bottom right panel there appears to be multiple minima for different \Rvtwo{} and \(\teffsub{2}\) combinations, which we assume is partially due to the small \Rvtwo{}.

In both simulations the reduced \textchisquaredreduced{} for the binary model is closer to 1.
This is not surprising as the binary model contains extra parameters.
As mentioned above, we need to be careful, as the extra components from the binary may just happen to fit components of the noise when a binary is not present, or in our case has a low flux ratio.
{\red{} the significance between the two models is analysed using the ``Bayesian Information Criterion'' ({BIC})~\citep{schwarz_estimating_1978}; }
\begin{equation}
{BIC} = n\ln{(m)} - 2\ln{(\hat{L})}.
\end{equation}
{\red{} Here \(n\) and \(m\) are the number of parameters and data points respectively and \(\hat{L}\) is the maximum of the Gaussian likely-hood function}
\begin{equation}
\hat{L} = {\left(\frac{1}{\sigma \sqrt{2\pi}}\right)}^{m} \exp{\left(-\frac{\chisquared}{2}\right)},
\end{equation}
{\red{} written in terms of \textchisquared{} and a fixed \(\sigma\) for all data points.
    The maximum likely-hood of a Gaussian distribution is equivalent to minimizing the \textchisquared.
    In both simulations \(\Delta {BIC} >10\) so the preference of the binary model, with the lower {BIC} value, over the single component model is considered \emph{significant}.}

\subsection{HD211847 observation}
\label{subsec:results-hd211847}
{HD 211847} is the best candidate for detection as it has a \(\rm 155~M_J\) low-mass star companion~\citet{moutou_eccentricity_2017}.
The angular separation of the two bodies is 222\mas{} (or 11.3\AU).
Even though it is not a {BD} it has the highest estimated flux ratio in our sample, of 0.03 based on the~\citet{baraffe_new_2015} evolution models and the known companion mass (see \cref{tab:estimated_flux_ratios}).
The angular separation of HD\,211847B is 222\mas{} with a projected distance of \todo{find projected distance}{XXX}.
The result of applying \textchisquared{} fitting to the second observation of {HD 211847} is shown in \cref{fig:HD211847_result_contours}.

For this target the metallicity of both components was fixed to 0.0 and the \logg{} for the host was fixed at 4.5.
The \logg{} for the companion is fixed to 5.0, based on the~\citet{baraffe_new_2015} evolutionary models for the given companion mass and system age.
The orbital solution was used to refine the {RV} search space of both components.
The span {RV} for the companion was extended until a value inside the {RV} bounds was found.

Again the top left panel of \cref{fig:HD211847_result_contours} shows the recovery with a single component model with the other three for the binary model.
The single component model finds a temperature of 5900\K{} for the host with a \Rvone{} of 7\kmps{}.
This is 200\K{} and 0.4\kmps{} different above the expected parameters.
The binary model finds a host temperature of 5800\K{}, which is the second closest model to the literature value, >100\K{} different.
The host {RV} value recovered with the binary model is 7.6\kmps{}, which is 1\kmps{} higher than expected.
For the single component model there is a barely noticeable secondary minima near this 7.6\kmps{} {RV} value recovered by the binary model.
Again these {RV} differences are smaller than the {\fwhm} of the lines.
The 3-\(\sigma\) contour is small, just visible on the right hand side of the star in the bottom left panel, and hidden behind the markers in the other panels.


For the companion in the binary model, on the right side of \cref{fig:HD211847_result_contours}, the minimum \textchisquared{} for the companion temperature is at the upper temperature limit of the grid shown.
If we extend the grid of companion temperature towards higher temperatures the best fit location continues to increase in temperature, continually hitting the upper limit until it is close to the host temperature, >2000\K{} above the expected companion temperature.
When the companion temperature becomes this high it also affects the recovered parameters for the host star to offset the features of the brighter companion.

The \textchisquaredreduced{} values for the single and binary models are 21 and 19 respectively, far from 1, indicating that both models are a poor fit to the observations.
{\red{} The $\Delta {BIC} = 3812 >10$ indicating that binary model is still preferred.} We plot the binary model for the best fit solution alongside the observed spectra in \cref{fig:visualinspection-hd2118471}.
We see that there is a large spectral mismatch between the synthetic models and the observation.
Extra wavelength masking was applied to many of the largest mismatched synthetic lines to remove their influence.
The grey areas mark regions which have been masked out, either from the centres of deep telluric lines (the thin masks matching spectral gaps), or the more prominent mismatched lines in the synthetic spectrum excluded from the \textchisquared{} analysis.
One clear example of a mismatched line is a synthetic line at 2132.5\nm{} that is clearly not observed in detector \#2 (top right).
Even with the majority of the mismatched lines removed the detection of the companion was still unsuccessful.

For detectors \#1 and 2 it appears that the synthetic spectra contain many more deeper lines than observed.
For detector \#3 the red half of the detector was masked out as there appears to be an offset between the observed lines.
With 3--4 lines that appear to be consistently offset from the observation it could be a wavelength calibration issue, although the telluric lines appear to be sufficiently corrected in this region, attesting for the quality of the wavelength calibration, and making it incompatible with the offset.
For detector \#4 the observed lines do not agree at all with the models.
With many observed lines not in the model and only one line with some agreement in wavelength, detector \#4 is masked out completely and not used in the \textchisquared{} fit.
Individual inspection of the \textchisquared{} results for each detector also revealed that there was a large discrepancy between the \nth{4} detector and the other three, with a different {RV} value for the host star and a \textchisquared{} values an order of magnitude higher.
The edge of a deep Hydrogen line (Brackett-\(\gamma\)) off the edge of the detector \#4 is also clearly seen in the continuum of the model >2162\nm{}.

We applied this same method to the remaining targets, with similar results.
In brief, we conclude that the companion spectra cannot be correctly detected in our data using this method.

\begin{figure*}
    \centering
    \includegraphics[width=0.7\linewidth]{figures/companion_recovery/HD211847_result_pcolors}
    \caption[\textchisquared{} contour for an observation {HD 211847}.]{\textchisquared{} result grid for the second observation of {HD 211847}, similar to \cref{fig:Mdwarf_contours,fig:HD211847_simulated_contours}.
        The white line shows a 3-\(\sigma\) confidence level about the minimum \textchisquared{} solution grid point, not always visible here due to the large \textchisquared{} values.
        The error bar on \(\teffsub{1}\) is from the literature while the error bars on \Rvone{} and \Rvtwo{} are calculated by propagating the orbital parameter uncertainties though the radial velocity equation (\cref{eqn:rv_equation_intro}).}
    \label{fig:HD211847_result_contours}
\end{figure*}

\todo{Put error paragraph in a good location}
The error on estimated {RV} values, shown in \cref{fig:HD211847_result_contours}, is calculated by applying the general error propagation formula~\citep{ku_notes_1966} to the {RV} equation (\cref{eqn:rv2_equation}) and using the errors on the published orbital parameters.
For a function, \(f\), with errors on the inputs \(\delta x\), \(\delta y\) etc., it follows:
\begin{align}
f &= f(x, y, z, \ldots)\\
\delta f &= \sqrt{{\left( \frac{\partial f}{\partial x} \delta x\right)}^2 + {\left(\frac{\partial f}{\partial y} \delta y\right)}^2 + {\left(\frac{\partial f}{\partial z} \delta z\right)}^2 + \ldots}.
\end{align}


\begin{figure*}
    \centering
    \includegraphics[width=0.7\linewidth]{figures/companion_recovery/visualize_result_residuals}
    \caption[Comparison between observation of {HD 211847} and the best fit synthetic binary model.]{Comparison between the observed {HD 211847} spectrum (blue) and the best fit synthetic binary model (orange dashed) for each detector.
        The bottom section of each panel shows the residuals between the parts of the observation used in the \textchisquared{} fit and recovered binary model (\(\rm O-C^2\)) in purple.
        The red dashed line shows the difference between the recovered binary model and the binary model with the exact same parameters except for the estimated companion temperature of 3200\K{} (\(\rm C^2[3200\K{}]- C^2\)).
        The grey shading indicated the wavelength regions where masking has been applied.
        The thinner masked regions that match with cuts in the observed spectra are where the centres of deep (>5\%) telluric lines that have been masked out are.}
    \label{fig:visualinspection-hd2118471}
    %\label{fig:visualizeresultresiduals}
\end{figure*}



%!TEX root = ../../thesis.tex

\subsection{Companion injection-recovery}
\label{subsec:injection-recovery}
To determine the detection limits for this method an injection-recovery approach was used to simulate spectra with a range of companions.
This is done by using the observed spectra and injecting onto them a synthetic companion, at the absolute flux ratio to which it would have been added to a synthetic host with the same parameters.
The {RV} of injected companion is set to 100\kmps{} so that the companion lines are well separated from the lines of the host.
This separation chosen is slightly larger than the largest host-companion separation ({HD~162020}) in the observed targets, given in \cref{tab:observations} \Rvtwo{}.

The search space for the injection-recovery is restricted by fixing the host parameters \(\teffsub{1}\) and \(\logg{}_{1}\) to those recovered fitting the non-injected spectra by a single component model.
This leaves only the companion \(\teffsub{2}\) and \Rvtwo{} parameters free, to recover the injected companion.
The wavelength masking is used to reduce the level of mismatch between synthetic and the observed spectra.
The spectra were injected with companions between 2500--5000\K{} and the companion recovery attempted on each.

The injection recovery was also performed on a synthetic host spectra representing each target as a comparison.
For the synthetic host injection-recovery the wavelength range of the synthetic spectra used is three sections interpolated to 1024 values in the wavelength span of detectors \#1, \#2, and \#3.
For each section, Gaussian noise is added at the level measured in the corresponding detector in the observation of the target being represented.

In \cref{fig:injectrecoveryhd30501} the results of the injection-recovery on {HD 30501} show the injected companion temperature verses the companion temperature recovered.
The blue dots represent the injection into the real observations, while the orange triangles represent injection into the synthetic host.
Error bars of \(\pm100\)\K{} are included to indicate the temperature grid size only, and do not come from the recovery itself.
The black dashed diagonal is the temperature 1:1 relation, where a correctly recovered companion should reside.

The grey shaded region indicates the \(\pm 1000\)\K{} temperature range explored for the injection-recovery of the companion.
This shows how the bounds of the grid can be recovered at low companion temperatures and that the recovered temperature deviates from the injected companion temperature around 3800\k{}.

For {HD 30501} the injection onto synthetic and observed spectra produce similar results.
At temperatures above 3800\K{}, in both the real and synthetic spectra, the injected companion is recovered within 100\K{}.
For injected companion temperatures below 3800\K{} the temperature recovered is systematically higher than the injected value.
This indicates that the companion is not correctly recovered and is affected by the added noise.
The temperature of deviation is deemed to be the upper temperature limit for the recovery by this method.
For the other stars, an upper limit from the injected observations could not be reliably determined, mainly due to spectral mismatch issues.
In these cases the results from the synthetic injection are used to derive a temperature recovery cut-off for each target, each simulated with the closest synthetic spectrum to the host star.

Using the temperature cut-off values, an upper mass limit is derived for the companions around our stars using the~\citet{baraffe_new_2015} evolutionary models, finding the closest point matching the spectral temperature cut-off and \(\logg{}=5.0\).
These values are given in \cref{tab:mass_limits} and are between 560--618~\Mjup{}.
The flux ratio between the cut-off companion spectra and the host star are also calculated, being between 5--15\% in this wavelength span.

%!TEX root = ../thesis.tex
\begin{table}
       \centering
  \begin{threeparttable}
       \caption[Upper mass limits from \textchisquared{} simulations.]{Upper mass limits of target companions assuming a companion \logg{}=5.0.
           Masses are derived from~\citet{baraffe_new_2015} evolutionary models using \Teff{} and \logg{}.
           The flux ratio \(\rm F_2/F_1\) is the absolute flux ratio between the cut-off temperature and the target host star.}

        \begin{tabular}{l c c c}
            \toprule
            Target & \Teff{} cut-off (K) & \(\rm F_2/F_1\) & Mass limit (\Mjup{})\\
            \midrule
            {HD 4747}     &  3\,900 & 0.084 & 598 \\
            {HD 162020} & 3\,900 & 0.147 & 598 \\
            {HD 167665} & 3\,800 & 0.054 & 560 \\
            {HD 168443} & 4\,000 & 0.094 & 618 \\
            {HD 202206} & 3\,900 & 0.075 & 598 \\
            {HD 211847} & 3\,900 & 0.079 & 598 \\
            {HD 30501}   & 3\,800\tnote{a} & 0.106 & 560 \\
            \bottomrule
        \end{tabular}
        \label{tab:mass_limits}
        \begin{tablenotes}[flushleft]
            \small
                \item [a] {From observed spectra }
        \end{tablenotes}
  \end{threeparttable}

\end{table}


\begin{figure}
    \centering
    \includegraphics[width=0.8\linewidth]{figures/companion_recovery/inject_recovery_hd30501.pdf}
    \caption[Result of simulated injection-recovery of synthetic companions on {HD~30501}.]{Result of simulated injection-recovery of synthetic companions on {HD~30501}.
        The blue dots and orange triangles indicate the recovered companion temperature for the observed and synthetic spectra respectively.
        The \(\pm100\)\K{} error bars are the grid step of the synthetic models.
        The black dashed diagonal shows the 1:1 temperature relation.
        The grey shaded region indicates the \(\pm1000\)\K{} temperature range explored.
        Gaussian noise added to the synthetic spectra was derived from the observed spectra.}
    \label{fig:injectrecoveryhd30501}
\end{figure}


%!TEX root = ../../thesis.tex

\section{Discussion}
\label{sec:chisquared_discussion}




\section{Discussion}
\label{sec:discussion}
The spectral differential and the synthetic recovery methods attempted here were both unsuccessful in a detection of the {BD} companion spectra.
The upper mass limits of \(600^{+20}_{-40}\) we set for these companions is very high, roughly six times higher then the {BD} mass limit \(\sim 80-90\)\Mjup{}.
We discuss potential reasons and solutions for these poor results below, list the lessons learned in this exploratory study, and provide some guidance for any future attempts with these methods.


\subsection{Synthetic recovery limitations}
\label{subsec:limitations}
In this section we discuss some of the limitations from this synthetic recovery method and some options to overcome some of these.

\subsubsection{Mismatch in synthetic models}
\label{subsubsec:mismatch}
We believe the mismatch between the observation and synthetic spectra is the main cause of the unsuccessful companion detection, impacting the recovery in two ways.
The mismatch causes the \textchisquared{} values to be large in general, but also causes the companion temperature to be pushed to higher temperatures.

In our examples the \logg{} and metallicity of the synthetic models are held fixed, leaving only temperature to vary.
The temperature impacts the synthetic spectral models in two main ways: the flux level of the continuum; and the number and strength of the absorption lines.
In the binary model the contributions from the individual components is scaled by the flux ratio.If the temperature of the companion increases then the flux and radius of the companion increases.
Its contribution of the companion to the binary model increases and the flux ratio \FoneFtwo{} decreases.
This effectively makes the lines of the spectrum of the host component relatively smaller in the normalized binary model spectrum.
Due to the initial mismatch of synthetic spectral lines, a decrease in relative strength of the host lines decreases the \textchisquared{} value, and is a better "match" to the observation.
This causes the recovered temperature of the companion to be higher than expected, >2000\K{} higher if allowed by the exploration grid.
The \textchisquared{} approach is dominated by reducing the mismatch in the spectrum of the host rather than detecting the spectra of the companion.
This spectral mismatch is not observed in the simulations in \cref{subsec:simulated_binaries} since they are created using the synthetic spectra themselves, and hence they do recover the correct host spectra.


\subsubsection{Line contribution of faint companions}
\label{subsubsec:line_contributions}
We calculate the line depths of the synthetic companion spectra to determine the \snr{} levels required to detect the lines of the binary companions.
One thing easy to overlook when attempting to detect the binary companion at low flux ratios is the actual contribution of the spectral lines of the companion.
The flux ratio of the continuum for our most promising target is \FtwoFone{}\(\sim\)3\% with the other targets having an expected flux ratio around 1\%, and some well below.
The spectral lines of the individual components which are the features we are trying to detect with the binary model, have depths on average around 10--20\% of their respective continua,  at-least between 2110--2160\nm{}.
In effect, the companion line features have a depth \(\ll 1\%\) relative to the continuum of the combined spectra.

In \cref{tab:line_contributions} we calculate some properties of the spectral lines in the {PHOENIX-ACES} library between 2110--2160\nm{}.
We count the number of spectral lines (\emph{no. lines}) deeper than 5\%, and take the average depth (\emph{avg. depth}) of these lines.
The contribution \emph{cont. depth} of the companion lines to a combined spectrum accounts for the flux ratio between the two components.
Here we use a Sun-like host with \(\teffsub{1}\)=5800\K{}.
This simplified combination neglects the continuum shapes of both spectral components and uses the average flux ratios for this wavelength range.
The {PHOENIX-ACES} spectra in the temperature range of 2500--5000\K{} shown in \cref{fig:comp_spectra} can be used to get a visual indication of the line density and depth measured here.

There are more lines >5\% deep for the lower temperature spectra, with 360--460 lines in this wavelength range, to be compared with the 31 deep absorption lines found in a Sun-like spectrum.
The average line depth of these lines is also larger than the Sun-like spectrum, around twice as deep.
However, when combined, the contribution of the companion lines is 1--2 orders of magnitude smaller than the hosts lines due to the low continuum flux ratios.

For example, with the synthetic model for the companion of {HD~211847}, the average contributions of lines >5\% become only 0.3\% deep in a binary with the Sun-like spectrum.
For a companion with a temperature of 2300\K{} (the lower {PHOENIX-ACES} temperature limit) the deepest lines contribute lines around 0.1\%.

%{\bl We use the contributed line depth values to calculate the \snr{} level required to have Gaussian noise of the same height and the observed \snr{} required to achieve equivalent contribution from all N lines of the spectrum, \(\rm \textrm{SNR}_N = \textrm{SNR} /\sqrt{N}\).
%This is for the synthetic spectra which have many more lines than the observed spectra in this wavelength range.}

The \snr{} of the observed spectra is between 100--300, which is below the \snr{} of 323 needed for the detection of the low-mass star companion of {HD~211847} with temperature of 3200\K{} and \logg{} 5.0.
For our other targets with {BD} companions at and below the {PHOENIX-ACES} temperature range, we would need observed \snr{} >800 to detect the individual spectral lines of the companion.
With the \snr{} increasing with \(\sqrt{N}\) this would require the observational time for each target to be increased by a factor of \(\sim\)~10--64.

Our non-detection of binary companions with low flux ratios is consistent with other works.
For example~\citet{nemravova_xtauri_2016} performed extensive spectral analysis of a quadruple-star system \object{$\xi$ Tauri} using 227 spectra in 3 different wavelength bands.
Of the four stars in the system they were unable to detect the spectral component of the one which had a luminosity ratio below 1\%.

\begin{table}
    \small
    \caption{Contribution of synthetic lines within 2110--2160\nm{} of synthetic {PHOENIX-ACES} spectra to a binary model.
        \FtwoFone{} is the continuum flux ratio between a spectrum with the given \Teff{} and \logg{} and a Sun-like spectrum with \Teff{}=5800, \logg{} = 4.5 (right most column).
        \emph{No. lines} is the number of spectral lines deeper than 5\% from the continuum of the individual spectra while \emph{avg. depth} is the mean depth of those lines.
        \emph{Cont. depth} is the average contribution, or depth, of these lines in the combined spectrum of a binary with a Sun-like spectrum.
        The \snr{} is signal-to-noise level required to have Gaussian noise \(\sigma\) =1/\snr{} equal to the \emph{cont. depth} level in the binary model.
        All synthetic spectra used here have \feh{}=0.0}
    \begin{tabular}{*7c}
        \toprule
        \Teff{} (K)  & \multicolumn{2}{c}{2300} & \multicolumn{2}{c}{3200} & 5800 (\Fone{})\\
        \logg{} & 5.0 & 4.5  & 5.0 & 4.5 & 4.5 \\
        \midrule
        \FtwoFone{} & 0.006 & 0.019 & 0.029  & 0.091 & 1.000 \\  
        % {>2\%}  & no. lines & 470 &  463 & 414  & 444 & 111 \\
        % & avg. depth & 0.20 & 0.23 & 0.10  & 0.12 & 0.04 \\
        % & cont. depth \tablefootmark{a} & 0.0012 &  0.0043 & 0.0028 &  0.0100 &  0.0333\tablefootmark{b} \\ 
        % \midrule
        no. lines & 464 & 463 & 365  & 413 & 31 \\
        avg. depth & 0.2  & 0.23 & 0.11 & 0.12 & 0.10 \\
        cont. depth \tablefootmark{a} & 0.0012 & 0.0043 & 0.0031 & 0.0100 &  0.0833\tablefootmark{b} \\
        \snr{}  & 833 & 232 & 323  & 100 & 12 \\
        %  \snr{}\(\rm _N\)  & 39 & 11 & 17  & 5 & 2 \\
        \bottomrule
    \end{tabular}
    \tablefoot{
        \tablefoottext{a}{avg. depth \(\times~\ftwo{} / (\fone{} + \ftwo{})\), where \Fone{} is the component in the far right column.}
        \tablefoottext{b}{avg. depth \(\times~\fone{} / (\fone{} + \ftwo{})\), where \Ftwo{} is for the companion with \Teff{}=3200, \logg{}=4.5.}
    }
    \label{tab:line_contributions}
\end{table}

\subsubsection{\(\chi^2\) asymmetry} 
\label{subsubsec:chi2_assymetry}
In \cref{fig:injection_shape} we showed that the shape of the recovered \textchisquared{} becomes asymmetric when dealing with companion temperatures below around 3800\K{}.
A visual inspection of the spectra reveals the likely cause.
In \cref{fig:comp_spectra} we show the corresponding spectra between 2111--2165\nm{}.
As the temperature decreases the strongest lines become less prominent, disappearing progressively among the other many small lines that appear at lower temperatures.
Hence there are no strong companion lines to easily distinguish one temperature from another.
In the flatter part of the \textchisquared{} curves several low temperature companions are equally well fitted to the simulation/observation.

\cref{fig:injection-recovery,fig:injection_shape} show different recovered temperatures but both agree above 3800\K{}.
A higher companion temperature is recovered between 2800 and 3800\K{}, where as in \cref{fig:injection_shape} a lower temperature is recovered.
This is probably due to a combination of the noise added, and the asymmetries of the \textchisquared{} lines.
\cref{fig:injection-recovery} uses the noise level from the observed spectrum while \cref{fig:injection_shape} has a \snr{} of 300.
This large asymmetry can also explain the jump observed in the synthetic recovery temperature around 2700\K{} in \cref{fig:injection-recovery}.

The asymmetry also causes an asymmetry in the \textchisquared{} error bars which can be seen in the bottom panel of \cref{fig:injection_shape}.
For instance the recovered value and 1-\(\sigma\) error bars on the 3000\K{} injected companion is \(2800 ^{+20}_{-100}\), with an asymmetric error bar skewed towards lower temperatures.

The bump observed at 5100\K{} in the \textchisquared{} curves is due to a discontinuity in the {PHOENIX-ACES} modelling.
The "reference wavelength defining the mean optical depth grid" is changed at 5000\K{}~\citep[][Sect. 2.3]{husser_new_2013}.
Care needs to be taken if trying to detect a companion near this temperature.

\begin{figure}
    \centering
    %\includegraphics[width=\hsize]{images/final/companion_spectra.pdf}
    \caption{{PHOENIX-ACES} spectra for temperatures between 2500 and 5000\K{}, corresponding to to the same lines in \cref{fig:injection_shape}.
The flux units are the native units of the {PHOENIX-ACES} spectrum, (\(\rm erg\,s^{-1}\,cm^{2}\,cm^{-1}\)), and have not been scaled by the stellar radii.
All spectra have a \(\rm \logg{}=5.0\) and \(\rm \feh{}=0.0\).
The vertical dotted lines indicate the edges of the CRIRES detectors.}
    \label{fig:comp_spectra}
\end{figure}

\subsubsection{Component {RV} separation}
\label{subsubsec:rv_seperation}
Another factor which could contribute to an unsuccessful detection is the {RV} separation between the host and companion, \Rvtwo{}.
Estimates for our observations are given in the last column of \cref{tab:observations}.
If \Rvtwo{} is small compared to the line width, then all the same lines of both components will be blended.
This is indeed the case for {HD~4747}, {HD~211847}, and {HD~202206} with expected \(|\rvtwo{}| < 2\)\kmps{}.
This may have contributed to the lack of recovery with both components of the binary model trying to fit to the same features.
This may even cause some correlation between the parameters of the two components.
The {RV} separation of the two components changes with orbital phase.
Having multiple spectra of the same target distributed in phase may allow the {RV} of the spectral components to be better recovered~\citep [e.g. ][]{czekala_disentangling_2017, sablowski_spectral_2016}.


\subsubsection {Wavelength range}
\label{subsubsec:wavelenght_range_limitation}
The wavelength choice for the spectra analysed here, observed with the intention to apply the spectral differential technique, was selected due to the location of the K-band telluric absorption window.
This wavelength range, with a narrow wavelength range \(\sim50\)\nm{} set by the CRIRES instrument.
This wavelength range is likely not the best choice for the proposed study.

Changing the wavelength coverage to regions with lines sensitive to stellar parameters for both stars and {BD}s, as well as using a larger wavelength range that will be achieved by CRIRES+, may help to improve the recovery results of the companion recovery technique presented here.
We note that if the wavelength range is increased by taking separate observations at different wavelengths, not covered by a single exposure, then changes in the {RV} of both components between the different wavelength observations may need to be accounted for.


\subsubsection{The {BT-Settl} models}
\label{subsubsec:bt-settl}
We note that the {PHOENIX-ACES} models are not the only spectral libraries available with the other notable library considered for this work is the {BT-Settl} models, ~\citep{allard_model_2010,allard_btsettl_2013,baraffe_new_2015}.
The included modelling of dust and cloud formation, as well as hydro-dynamical modelling atmospheric mixing/settling for atmospheres with \Teff{} below \(\sim2600\)\K{}, make the {BT-Settl} models valid across the regime from stars to {BD}s as cool as 400\K{}.
As the {BT-Settl} models are suitable to model the atmospheres of the brown dwarfs they would be useful for the companion recovery technique developed here.
However, as shown in \cref{subsection:results-hd211847, subsection:injection-recovery}, we were unable to successfully recover the 155\Mjup{} (\Teff{}\(\sim3200\)\K{}) low mass star companion of {HD~211847} and derived a temperature upper limit for our methodology of around 3800\K{}.
These are both well above the 2300\K{} cutoff of the {PHOENIX-ACES} models and for the onset of dust- and cloud-formation phenomena, at 2600\K{}.

\cref{fig:hd211847-models} shows again the minimum \textchisquared{} solution for detector 1 of the second {HD~211847} observation, this time including the {BT-Settl} solution with the same parameters.
Although the {PHOENIX-ACES} and {BT-Settl} models differ slightly they both have a large spectral mismatch to the observations.
As such, we did not use the {BT-Settl} models for the simulation and results above as we did not see any special advantage in using them.

The ease of access to find, download, and use {PHOENIX-ACES} spectral library, available in the fits file format, compared to older {BT-Settl} libraries is another reason for the current favoured use of the {PHOENIX-ACES} library.

Although the newer generations synthetic spectral models are improving and match the overall spectral energy distribution reasonably well there are still regions in the \emph{H}- and \emph{K}-band where there is room for improvement~\citet{rajpurohit_spectral_2016}.
The spectral mismatch in the region studied here is still too large for spectral recovery of companion brown dwarfs.
In the \nir{} we have compounding problem: the model input physics of sub-stellar temperatures and chemistry combined with the general difficulty of the \nir{}.

\begin{figure}
    \centering
    %\includegraphics[width=\hsize]{images/final/HD211847_ACES_BTSettl.pdf}
    \caption{Detector 1 spectrum for {HD~211847} (blue) alongside the {PHOENIX-ACES} (orange dash-dot) and {BT-Settl} (green dashed) synthetic spectra for the host star only, with parameters \Teff{}=5700\K{}, \logg{}=4.5 and \feh{}=0.0.
        Both synthetic models have been normalized and convolved to \(\rm R=50\,000\).
        There is a 0.05 off-set between each spectrum}
    \label{fig:hd211847-models}
\end{figure}


\subsubsection{Impact of \logg{}}
\label{subsubsec:logg}
Logg, a measure of surface gravity, is related to evolutionary state and the size of the star with smaller \logg{} values usually indicating larger radii stars.
This parameter has a large impact on the radius and flux ratio of the binary models.
In the {PHOENIX-ACES} models a decrease in \logg{} from 5.0 to 4.5 increases the models effective radius by \(\sim\)1.75 in the temperature range investigated here.
This change in radius alone roughly triples (\(1.75^2\)) the absolute flux of the synthetic spectrum, neglecting any changes to the shape of the actual spectrum.
Therefore, there are large jumps in the model flux ratios if the \logg{} is allowed to vary, with lower \logg{} values for the companion being favoured as the increased flux ratio decreases the mismatch of the host component to the observations.
This large impact of \logg{} on the spectral library absolute flux is one reason for keeping the \logg{} of each component fixed in the \textchisquared{} results presented in \cref{sec:results}.

\subsubsection{Interpolation}
\label{subsubsec:interpolation}
It is common to interpolate between the synthetic spectral grids to fit and derive parameters in between the grid points~\citep[e.g.][]{nemravova_xtauri_2016, passegger_fundamental_2016}.
Instead of interpolation~\cite{czekala_constructing_2015} use a spectral emulator to use Principal Component Analysis to create eigenspectra for the synthetic library and Gaussian processes to derive a probability distribution function of possible interpolation spectra to account for uncertainties in the interpolation required for high signal-to-noise spectra.

However, we did not incorporate any interpolation into the companion recovery at this stage.
This could be something to be added in the future to refine the recovered parameters, and to help the transition between the grid \logg{} values.
Codes are readily available to perform spectral interpolation which could be utilized for this, two of them are \emph{pyterpol}\footnote{https://github.com/chrysante87/pyterpol}~\citet{nemravova_xtauri_2016} and \emph{Starfish}\footnote{https://github.com/iancze/Starfish}~\cite{czekala_constructing_2015}.


\subsection{Future implementation}
\label{subsec:future}
\subsubsection{High resolution instrumentation}
\label{subsubsec:highres}
The future of high resolution near- and mid- IR spectrographs is looking bright, with many new ground- and space-based instruments currently being developed.
Notable examples include CARMENES (550--1710\nm{},  R=82\,000) which is now operational~\citep{quirrenbach_carmenes_2014}, while SPIRou (980--2350\nm{},  R=73\,500)~\cite{artigau_spirou_2014} and NIRPS (970--1810\nm{}, R=100\,000)~\cite{bouchy_nearinfrared_2017} are still being assembled and installed.
The eagerly awaited {JWST} \textbf{cite} will also be launched soon\footnote{Recently pushed to around May 2020} providing observations in both the \nir{} (600--5300\nm{}, R=2700) and mid-IR (4900--28\,800\nm{}, R$\sim$1550--3250) regions without contamination from our atmosphere.

The upgrade of CRIRES to CRIRES+~\citep{dorn_crires_2016} will increase the wavelength coverage of a single shot capture by at least a factor of 3--5.
This larger wavelength span would be extremely beneficial for the \textchisquared{} performance of the spectral recovery method, increasing the number of useful lines and spectral features to be fitted with the models.

On the modelling side, there are continual improvements in atmospheric modelling and their associated synthetic spectral libraries: as seen with the evolution of the {BT-Settl} models~\cite{allard_btsettl_2013}.
With additional physics and improved line lists and solar abundances~\citep [e.g.][]{asplund_chemical_2009,caffau_solar_2011}, the synthetic libraries are reaching a better agreement with \nir{} observations.
An improved agreement between the \nir{} observations and synthetic spectra will be crucial to improve the performance of the spectral recovery technique presented here.

Although not successful with the CRIRES data used here, the instrumental stage is set to attempt these techniques presented here using the next-generation of high resolution spectrographs.
The lessons learned in this analysis need to be taken into account in order to achieve the best chance of a successful detection.

\subsubsection{Differential scheduling challenges}
\label{subsubsec:differential scedualing}
This work has revealed that more care needs to be taken in planning the observations for the spectral differential analysis of faint companions in the future.
Paying attention in particular to the {FWHM} of the lines in the region (governed by resolution and wavelength); the estimated companion \(\Delta RV\); the previous observations from different observing periods; and keeping consistent detector settings.

The original goal for the observations was to obtain two different and ``clearly separated radial-velocities'' for the secondary companion.
However, the program was assigned a low-priority (C, in {ESO} grading) and, possibly due to operational reasons, the original time requirements necessary to secure well separated {RV}s for the companion spectra could not be met.
This meant that all observations were insufficiently separated to extract a differential spectra for the companion.

The long orbital periods of these targets is also a contributing factor to the insufficient separations.
Most of the targets observed here have orbital periods much longer than an observing semester (183 days).
An optimal pair of observations (achieved at the extrema) would need to have been obtained from separate observing periods (between 2 months and 19 years apart).
In some cases, even observations taken at the beginning and end of the semester would not be sufficient to achieve companion separation (depending on the phase).
Requiring separate observing periods to even achieve the minimum \(\Delta \rm RV\) larger than the line {FWHM}.
At the time it was impossible to ask for time over several semesters in a regular proposal.

Our study demonstrates the importance of proposals for projects that need to be extended over several semesters or years.
In the {ESO} context, this corresponds to ``Monitoring proposals''~\citep[e.g.][pg. 18]{eso_eso_2017}.
Observations of the targets explored here, with long orbital periods in particular, would benefit from the ability for multi-period proposals and newer scheduling systems which allow for tighter scheduling constraints, such as a companion {RV} separation.

For future observations we suggest that the known orbital solution of the companion be used to estimate the companions' {RV} curve during the observing period, with the companion \(M_2\sin{i}\) providing an {RV} upper-limit.
Knowing the instrumental wavelength and resolution, a constraint can then be set to avoid taking observations when the companion spectra are insufficiently separated, or \(\Delta RV\) < {FWHM}.
This constraint can be set using the absolute and relative \emph{time-link} constraints available in {ESO}'s Phase 2 Proposal Preparation (P2PP) tool.
Additionally, analysing the known orbital solution before-hand, to determine {RV} constraints will also help identify the best time to observe, if observations from separate periods will be required or, if an optimally separated companion differential is even feasible.

\subsection{Other techniques}
We note that there are many other disentangling techniques to separate mixed spectra of binary systems,~\citep[e.g.][]{hadrava_disentangling_2009}.
These require more than two observations, with  \(n+1\) observations used to set up a system of linear equations to solve for \(n\) spectral components~\citep[e.g.][]{simon_disentangling_1994,czekala_disentangling_2017, sablowski_spectral_2016}.
These methods are ideal for many well spaced observations.
For example the ideal situation for the {SVD} method of~\citet{sablowski_spectral_2016} is homogeneous samples of at least half the period, to identify the moving spectral components.
The few and insufficiently separated observations we analyse here are not suitable to apply any of these advanced techniques and are beyond what we have attempted here.


\section{Conclusions}
\label{sec:conclusions}
This work aimed at pushing down the detection limit of faint companions, using high resolution near-infrared spectra.
Two different methods were explored with many limitations uncovered.

The objective of the observations acquired in this program was to a apply the differential technique.
For the differential technique the observations need to be sufficiently separated, such that the {RV} of the companion is greater than the {FWHM} to avoid spectral cancellation of the companion.
Unfortunately, due to operational reasons, this condition was not met.
As such we employed an alternative method, that we termed the spectral recovery; this method is in principle fully equivalent, but ended up revealing different difficulties.

For the spectral recovery the host-companion {RV} separation should also ideally be greater then the {FWHM} to avoid blended lines.
The spectral mismatch between models and reality in the \nir{} negatively affects the performance of the synthetic recovery technique on the observed spectra.
With all these effects we are unsuccessful in the detection of the \nir{} spectra of {BD} companions,  with the mass upper-limits set at \(\rm 600~M_{Jup}\) from the synthetic recovery technique.

This work highlights many of the difficulties when dealing with the spectral recovery of \nir{} spectra.
The obstacles to overcome are the data reduction of \nir{} CMOS detectors, that are not yet at the level of visible CCDs, along with a precise telluric correction and wavelength calibration (two interrelated aspects, as thoroughly discussed).
Another important aspect is the mismatch between \nir{} high-resolution spectra and the observed spectra.
In spite of the continuous effort of the modelling community, our work, along with several cited contemporary ones, shows that this mismatch is still one of the main factors preventing us from perform spectral recovery in the \nir{}.
This work highlights that this is a compound problem for Brown Dwarfs, for which the spectral models are worse informed due to lack of observations at high-resolution.

Other than the improvement of the spectral models, the observing community can increase their odds of success by paying attention to the scheduling of observations and the wavelength domains to explore.
Our work shows that observing in the areas of lower telluric absorption, as is frequently done, is not a guarantee of success due to the scarcity of deep lines in cold objects.
Moreover, due to the mismatch between models and observations, the ability to obtain a first spectra before settling on a wavelength range, or changing settings on the fly, is extremely useful for the success of these campaigns.

We hope that this work can act as a guide for the planning of future observations of targets with faint {BD} and planetary companions with the upcoming generation of high resolution spectrographs in the near- and mid- infrared such as CRIRES+ and {JWST} observations.














\citet{hoeijmakers_search_2015}
{Context.
    The spectral signature of an exoplanet can be separated from the spectrum of its host star using high-resolution spectroscopy.
    During these observations, the radial component of the planet's orbital velocity changes, resulting in a significant Doppler shift that allows its spectral features to be extracted.\\
    Aims: In this work, we aim to detect\ce{TiO} in the optical transmission spectrum of {HD~209458}b.
    Gaseous\ce{TiO} has been suggested as the cause of the thermal inversion layer invoked to explain the dayside spectrum of this planet.\\
    Methods: We used archival data from the 8.2 m Subaru telescope taken with the High Dispersion Spectrograph of a transit of {HD~209458}b in 2002.
    We created model transmission spectra that include absorption by\ce{TiO}, and cross-correlated them with the residual spectral data after removal of the dominating stellar absorption features.
    We subsequently co-added the correlation signal in time, taking the change in Doppler shift due to the orbit of the planet into account.\\
    Results: We detect no significant cross-correlation signal due to\ce{TiO}, though artificial injection of our template spectra into the data indicates a sensitivity down to a volume-mixing ratio of \textasciitilde{}10\textsuperscript{-10}.
    However, cross-correlating the template spectra with a {HARPS} spectrum of Barnard's star yields only a weak wavelength-dependent correlation, even though Barnard's star is an M4V dwarf that exhibits clear \ce{TiO} absorption.
    We infer that the\ce{TiO} line list poorly matches the real positions of\ce{TiO} lines at spectral resolutions of \textasciitilde{}100 000.
    Similar line lists are also used in the {PHOENIX} and Kurucz stellar atmosphere suites and we show that their synthetic M-dwarf spectra also correlate poorly with the {HARPS} spectra of Barnard's star and five other M dwarfs.
    We conclude that the lack of an accurate\ce{TiO} line list is currently critically hampering this high-resolution retrieval technique.},




\subsection{Junk from paper about BT-settl}

The {PHOENIX-ACES} models also provide dimensions of effective radius of the star, in the header.
This is required for the method presented here as we need to scale the spectra by their respective surface area when combining together.


The effective radius of the modelled star are provided in the fits header and required for the calculation of the flux ratio.


As we do not recover suitable results for HD\,211847 which is supposed to have a temperature around 3400\K{} we suspect that it is not possible for lower temperature either so don't try to extend to the {BT-Settl} models.


\textbf{
    {PHOENIX-ACES} defines grid as \Teff{}, \logg{} and Mass which can then be used to determine Radii.
    See~\citep{husser_new_2013} section 2.3.1 Mass.}




\subsection{Note about a target - discussion of results}
Another example is {HD~162020}, which has the lowest orbital period (8.4 days) of our targets.
It should have been possible to obtain an optimal pair of observations in one semester, but the second observation was taken immediately following the first.
This means that the \(\rm \delta {RV} = 0.363\kmps{}\)between the two observations, is comparable to the \(\Delta {RV}\)that occurs during each individual observation.
With tighter scheduling restrictions this target could have been observed with the optimal {RV} separation at the extrema of \(\Delta {RV}=2 K_{2}\).
Whether the flux ratio of \(7e^{-6}\) for this target and the interaction of different spectral lines would have made it possible to recover companion mass is a separate issue.


\subsubsection{Wavelength range}
The wavelength choice for the spectra analysed here, observed with the intention to apply the spectral differential technique, was selected due to the location of the \emph{K}-band telluric absorption window.
This wavelength range, with a narrow wavelength range \(\sim50\)\nm{} set by the {CRIRES} instrument.
This wavelength range is likely not the best choice for the proposed study.
\todo{finish this line} may contribute to the poor results from the companion recovery technique.

For instance~\citet{passegger_fundamental_2016} used four different spectral regions for the precise parameter determination of M-dwarfs.
Specific lines from the different wavelength regions are affected differently by the model parameters: \Teff{}, \logg{}, and \feh{}; and are used to break degeneracies in the {PHOENIX-ACES} parameter space.

Changing the wavelength coverage to regions with lines sensitive to stellar parameters for both stars and {BD}s, as well as using a larger wavelength range that will be achieved by {CRIRES+}, may help to improve the recovery results of the companion recovery technique presented here.
We note that if the wavelength range is increased by taking separate observations at different wavelengths, not covered by a single exposure, then changes in the {RV} of both components between the different wavelength observations may need to be accounted for.



\subsection{Comparison to other methods}
kolbl 2014, passenger 2016, 2018

Contrast our result to other works.
More phases, longer integration time, higher {\snr{}}.



kobl 2015 also have difficult separating spectra with a {RV} separation below 10 km/s, or when the spectra of the spectra and companion have small separation.\todo{}


\textbf{
    CHECK out LOCKWOOD 2014 - maximum likelihood with todcor 1e-4 flux ratio double lined spectra}



\section{Summary}

\todo{Is another summary needed here? transition to next chapter}
After attempting two different methods to detect the companion BDs these observations were abandoned...
