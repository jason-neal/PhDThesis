%!TEX root = ../thesis.tex
%6-Conclusions

\chapter{Conclusions}  % Main chapter title

\label{cha:conclusions}

%--------------------------------------------------------------------------------------------------------------------------------------------

\section{Conclusions from Paper}
\label{sec:conclusions}

This work explored two methods to try and detect the faint spectra of stellar companions, using high resolution near-infrared spectra. Two different methods were explored with many limitations uncovered.

The objective of the observations acquired in this program was to a apply the differential technique. For the differential technique the observations need to be sufficiently separated, such that the {RV} of the companion is greater than the {FWHM} to avoid spectral cancellation of the companion. Unfortunately, due to operational reasons, this condition was not met. As such we employed an alternative method, that we termed the spectral recovery, which ended up revealing different difficulties.

For the spectral recovery the host-companion {RV} separation should also ideally be greater then the {FWHM} to avoid blended lines. The spectral mismatch between models and reality in the \nir{}negatively affects the performance of the synthetic recovery technique on the observed spectra. With all these effects we are unsuccessful in the detection of the \nir{}spectra of BD companions,  with the mass upper-limits set at 600~\Mjup{} from the synthetic recovery technique.

This work highlights many of the difficulties when dealing with the spectral recovery of \nir{}spectra. The obstacles to overcome are the data reduction of \nir{}CMOS detectors, that are not yet at the level of visible CCDs, along with a precise telluric correction and wavelength calibration (two interrelated aspects, as thoroughly discussed). Another important aspect is the mismatch between \nir{}high-resolution spectra and the observed spectra. In spite of the continuous effort of the modelling community, our work, along with several cited contemporary ones, shows that this mismatch is still one of the main factors preventing us from performing spectral recovery in the \nir{}.\@ This work highlights that this is a compound problem for Brown Dwarfs, for which the spectral models are worse informed due to lack of observations at high-resolution.

Other than the improvement of the spectral models, the observing community can increase their odds of success by paying attention to the scheduling of observations and the wavelength domains to explore. Our work shows that observing in the areas of lower telluric absorption, as is frequently done, is not a guarantee of success due to the scarcity of deep lines in cold objects. Moreover, due to the mismatch between models and observations, the ability to obtain a first spectra before settling on a wavelength range, or changing settings on the fly, is extremely useful for the success of these campaigns.

We hope that this work can act as a guide for the planning of future observations of targets with faint BD and planetary companions with the upcoming generation of high resolution spectrographs in the near- and mid-infrared such as {CRIRES+} and JWST observations.


\todo{NIR analysis work}{Finish off this}



\section{Future Work}

Future of {RV} precision
