%!TEX root = ../thesis.tex
%6-Conclusions

\chapter{Conclusions}  % Main chapter title

\label{cha:conclusions} 

%--------------------------------------------------------------------------------------------------------------------------------------------

\section{Conclusions from Paper}
\label{sec:conclusions}
\
This work aimed at pushing down the detection limit of faint companions, using high resolution \nir spectra. Two different methods were explored with many limitations uncovered. 
\unfinished{Update from paper}

For the differential technique the observations need to be sufficiently separated, such that the RV of the companion is greater than the FWHM to avoid spectral cancellation of the companion.
	
	For the spectral recovery the host-companion RV separation should also ideally be greater then the FWHM to avoid blended lines. The spectral mismatch between models and reality in the \nir negatively affects the performance of the synthetic recovery technique on the observed spectra. With all these effects we are unsuccessful in the detection of the \nir spectra of BD companions,  with the mass upper-limits set at \(\rm 600~M_{Jup}\) from the synthetic recovery technique.

We hope that this work can act as a guide for the planning of future observations of targets with faint BD and planetary companions with the upcoming generation of high resolution spectrographs in the near- and mid- infrared such as CRIRES+ and JWST observations.



\section{Future Work}
