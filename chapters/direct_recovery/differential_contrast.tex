%!TEX root = ../../thesis.tex

\section{Contrast to other works}

As stated previously the differential technique is not new being analytically formulated for binary star separation in~\citet{ferluga_separating_1997}.
This work tried to extend this to lower contrast ratios, to a hosts with smaller companions.

For lack of high-resolution CRIRES data~\citet{kostogryz_spectral_2013} explored simulations of the differential approach to a {BD} companion of a {M-dwarf} star, in which the contrast ratio is around 1/50 between 1/200 and were observed at the {RV} extrema.
Their favourable wavelength choice to achieve a good contrast ratio is \emph{K}-band, as done in work, however they chose the \ce{CO} line region (\(\sim\)2310\nm{}) where there is several narrow spectral lines.
One thing explored in~\citet{kostogryz_spectral_2013} that is not considered here is the effect of rotational broadening on the mass determination, finding that it should be possible to determine the mass of a slowly rotating companion, but a fast rotating companion is more difficult.

The limitations with regards to the {RV} separation between spectral components has also been observed in other works.
Similarly to the companion-companion {RV} separation focused on here,~\citet{kolbl_detection_2015} find a limitation of \(\sim\)10\kmps{} {RV} separation required between the host and the companion.
This is because if lines of the host and companion are blended, it is likely the companion spectra will be fitted, or incorporated into the hosts spectra, making it difficult to accurately detect the companions lines a similar {RV}.
The {RV} difference between the host an companion is given in the last column of \cref{tab:observations} as \Rvtwo{}.
It can be seen {HD~4747}, {HD~202206}, and {HD~218447} do not exceed this 10\kmps{} separation with the obtained observations.

