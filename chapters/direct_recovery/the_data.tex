%!TEX root = ../../thesis.tex

\begin{landscape}
    %!TEX root = ../../nir_companions.tex
\todo{rotate table?}
% Table of observations
\begin{table*}
    \small
    \centering
    \begin{threeparttable}[b]

        \caption{Details about the each {CRIRES} observation. The number of artefacts removed in \sref{subsubsec:reductionartefacts} as well as the {SNR} of the combined spectra is provided. The last three columns are the calculated {RV} of both host and largest companion, from the orbital solution, as well as the {RV} difference between the two components.}
        %\begin{tabular}{l c c c c cl cl r@{.}l r@{.}l r@{.}l}
        \begin{tabular}{l c c c c c c | r@{.}l r@{.}l r@{.}l}
            \toprule
            Object & Obs.& Start date  & Filter & Airmass  & Artefacts & {SNR} & \multicolumn{2}{c}{\(RV_1\)} & \multicolumn{2}{c}{\(RV_2\)} & \multicolumn{2}{c}{\(rv_2\)}  \\  % & \(Date \)
            &  \#   & (yyyy-mm-dd hh:mm:ss)  &  & (at start) & {/ 32} & & \multicolumn{2}{c}{\kmps{}} & \multicolumn{2}{c}{\kmps{}} & \multicolumn{2}{c}{\kmps{}}\\ % data ref    % & (JD\(^{\star} \))
            \midrule
            {HD 4747}   & 1 & 2012-07-06 07:36:06 & Ks            & 1.25     & 7 & 340 & $-$0 & 219 & $-$0  & 154 & 0&065 \\ %-1   & 2456\,114.81674
            {HD 162020} & 1 & 2012-07-04 06:23:22 & Ks      & 1.30  & 2 & 127 & $-$28  & 760 & 50 & 785\tnote{a}  & 79&545\tnote{a} \\ %-1   & 2456\,112.76624
            {HD 162020} & 2 & 2012-07-04 06:57:48 & Ks      & 1.44   & 2 & 128 & $-$28  & 717 & 48 & 440\tnote{a} & 77&157\tnote{a} \\ %-2   & 2456\,112.79015
            {HD 167665} & 1 & 2012-07-28 05:00:53 & Hx5e-2  & 1.24  & 7 & 371 & 7      & 581 & 18 & 024\tnote{a} & 10&443\tnote{a} \\ %-1a  & 2456\,136.70895
            {HD 167665} & 2 & 2012-07-28 05:37:27 & Hx5e-2  & 1.39   & 4 & 374 & 7      & 581 & 18 & 025\tnote{a}  & 10&444\tnote{a} \\ %-1b  & 2456\,136.73434
            {HD 167665} & 3 & 2012-08-05 02:54:03 & Hx5e-2  & 1.04   & 4 & 358 & 7      & 575 & 18 & 163\tnote{a} & 10&588\tnote{a} \\ %-2   & 2456\,144.62087
            {HD 168443} & 1 & 2012-08-05 04:29:32 & Ks      & 1.31  & 2& 192 & $-$0   & 121 & 50 & 932\tnote{a,b}  & 51&053\tnote{a,b} \\ %-1   & 2456\,144.68718
            {HD 168443} & 2 & 2012-08-05 04:58:50 & Ks      & 1.47  & 4 & 190 & $-$0   & 121 & 51 & 189 \tnote{a,b} & 51&310\tnote{a,b} \\ %-2   & 2456\,144.70753
            {HD 202206} & 1 & 2012-07-12 06:54:44 & Ks      & 1.01  & 3& 189 & 14   & 843 & 12 & 992\tnote{b}  & -1&851 \\ %-1   & 2456\,120.78801
            {HD 202206} & 2 & 2012-07-13 05:41:40 & J          & 1.01    & 3 & 209 & 14   & 837 & 13 & 065\tnote{b}  & -1&772 \\ %-2   & 2456\,121.73727
            {HD 202206} & 3 & 2012-07-11 08:29:55 & Ks      & 1.15 & 4& 180 & 14   & 849 & 12 & 920\tnote{b}  & -1&929 \\ %-3   & 2456\,119.85411
            {HD 211847} & 1 & 2012-07-06 07:02:57 & Ks      & 1.07  & 4& 272 & 6     & 613 & 7   & 171 & 0& 558\\ %-1   & 2456\,114.79372
            {HD 211847} & 2 & 2012-07-13 06:54:37 & Ks      & 1.05  & 5& 283 & 6     & 614 & 7   & 167 & 0&553 \\ %-2   & 2456\,121.78793
            {HD 30501}  & 1 & 2012-04-07 00:08:29 & Hx5e-2   & 1.60   & 3& 217 & 22   &  372 & 36 & 377 & 14&005 \\ %-1   & 2456\,024.50590
            {HD 30501}  & 2 & 2012-08-01 09:17:30 & Hx5e-2 & 1.42  & 10& 212 & 22   & 505 & 35  & 120 & 12&615 \\ %-2a  & 2456\,140.88716
            {HD 30501}  & 3 & 2012-08-02 08:47:30 & Hx5e-2   & 1.53   & 8& 237 & 22   & 507 &  35 & 102 & 12&595 \\ %-3   & 2456\,141.86633
            {HD 30501}  & 4 & 2012-08-06 09:42:07 & Ks       & 1.28   & 7& 235& 22   & 514 & 35 & 031 & 12&517 \\ %-2b  & 2456\,145.90426
            \bottomrule
            & & & &
        \end{tabular}\label{tab:observations}
        \begin{tablenotes}
            \item  [a]{Maximum {RV} given \mtwosini{} only.}
            \item  [b]{Largest mass companion only.}
        \end{tablenotes}
    \end{threeparttable}
\end{table*}


\end{landscape}

\subsection{The {CRIRES} data}
\label{subsec:CRIRES}

Observations were performed with the {CRIRES} instrument~\citep{kaeufl_crires_2004} configured to observe a narrow wavelength domain of the \emph{K}-band between 2120--2165\nm{}.
The slit width of \(0.4^{\prime\prime}\) resulted in an instrumental resolving power of \(\R=50\,000\)\footnote{The rule of thumb resolution for {CRIRES} is \(100\,000\times \frac{0.2^{\prime\prime}}{\textrm{slit width}}\) with the slit width in arcseconds}.
No adaptive optics were used to ensure that the entrance slit was entirely covered by each target.
This is to prevent strong slit illumination variations that could change the shape of spectral lines, and introduce and radial velocity variation on the datas.

The observations were performed in service mode during {Period 89} with run {ID.~089.C-0977(A)} between April and August 2012.
A single observation is composed of eight individual spectra with an integration time of 180\si{\second} each, observed in the {ABBAABBA} nod cycle pattern to obtain a high signal-to-noise ratio (>100) when combined.
The list of observations obtained with {CRIRES} are provided in \cref{tab:observations}.

There is a slight inconsistency with some of the observations, taken in service mode.
For instance {HD\,202206} has two observations taken with the \emph{Ks} filter, while one is taken with the \emph{J} filter.
There is also the last observation of {HD\,30501} taken with a different filter compared to the others.
The documents for the phase two observing proposal were unable to be obtained to determine if these `odd' filters were requested or if this was an observational mistake.

There is also an inconsistency with the naming or ordering of the observations again with the target {HD\,202206}.
The observation that was performed first in time is labelled with the observation name of {HD\,202206-3} in the fits header file, while the second and third observations are labelled -1 and -2 respectively.

There could be two possible reasons for the single observation of {HD\,4747}.
The first reason could be that only one observation was requested due to the very long orbital period of the target, although this would not have fulfilled the scientific goal.
The second and more likely reason is that these observations were performed in service mode, as a filler program, and there was no time to observe a second observation of {HD\,4747}.

All observations were reduced using the {DRACS} pipeline with the artefact corrections method applied (see \cref{subsec:pipeline-selection}).
Each observation was then: wavelength calibrated using a synthetic telluric spectrum, corrected for telluric absorption, and then corrected for the barycentric {RV} following  \cref{subsec:wavecalib,subsec:telluric_correction_application,subsec:barycentriccorrection}.

