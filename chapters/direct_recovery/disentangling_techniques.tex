%!TEX root = ../../thesis.tex

\section{Spectral Disentangling techniques}
\label{sec:disentangling_techniques}
\todo{explain some of the different techniques}

Spectral observations of binary systems contain the spectra of both bodies, in proportion to their flux ratio, and Doppler shifted relative to each other due to their orbital motion.
One technique to recover the spectra of the companion is secondary reconstruction through a differential spectrum~\citep{ferluga_separating_1997}.
Spectra from different phases are shifted to the rest frame of the host star and subtracted to mutually cancel out the spectrum of the host star allowing the faint companion spectra to become visible.
Advances in high-resolution and near-infrared (\nir{}) capabilities should enable this technique to be applied to {BD}s and planet companions, in which smaller {RV} shifts can be resolved and the contrast ratio of the smaller companion is improved.


There are many other disentangling techniques to separate mixed spectra of binary systems,~\citep[e.g.][]{hadrava_disentangling_2009}.
These require more than two observations, with  \(n+1\) observations used to set up a system of linear equations to solve for \(n\) spectral components~\citep[e.g.][]{simon_disentangling_1994,czekala_disentangling_2017, sablowski_spectral_2016}.
These methods are ideal for many well spaced observations.
For example the ideal situation for the {SVD} method of~\citet{sablowski_spectral_2016} is homogeneous samples of at least half the period, to identify the moving spectral components.
The few and insufficiently separated observations that were analysed in this work are not suitable to apply these advanced techniques.


Binary stars through brown dwarfs...
to planets.


\subsection{Other techniques}
There are many other disentangling techniques to separate mixed spectra of binary systems,~\citep[e.g.][]{hadrava_disentangling_2009}.
These require more than two observations, with  \(n+1\) observations used to set up a system of linear equations to solve for \(n\) spectral components~\citep[e.g.][]{simon_disentangling_1994,czekala_disentangling_2017, sablowski_spectral_2016}.
These methods are ideal for many well spaced observations.
For example the ideal situation for the {SVD} method of~\citet{sablowski_spectral_2016} is homogeneous samples of at least half the period, to identify the moving spectral components.
The few and insufficiently separated observations that were analysed in this work are not suitable to apply these advanced techniques.

\todo{add extra techniques}


A fairly common technique for matching and separating spectra is to apply cross-correlation against a library of known spectra.
TODCOR~\citep{zucker_study_1994} is a commonly used algorithm for cross-correlation of two spectral components and revealing the faint secondary~\citep[e.g.][]{mazeh_detecting_1997} simulated to be able to detect secondaries with flux ratios down to \(\sim\)1/1000 provided a sufficient \snr{}\citep{mazeh_todcor_1994}.

Mention the other technique here first.
Then move to the one we tried...

Spectral Disentangling techniques
- PSOAP \citet{czekala_disentangling_2017} 
- Differencing Fruluga
- Templates?

TODCOR - 2-d from binary analysis~\citep{zucker_study_1994}
\citep{mazeh_detecting_1997}


cite {LOCKWOOD 2014}
2-D-cross-correlation?~\citet{piskorz_evidence_2016} seems to have promising technique


\textbf{
    See~\citet{kostogryz_spectral_2013} for lots of useful references regarding differential works~\citet{simon_disentangling_1994}}

\citet{rodler_weighing_2012} uses differential of a series of phases to construct a stellar mask for the host to subtract from all measurements.


{\red{}~\citet{kolbl_detection_2015} Use HIRES spectra at 364--799\nm{} (R=60\,000) to determine statistically determine the presence of a faint secondary companion spectra (if any).
    They use \textchisquared{} fitting of the observed spectra against the {SPecMatch} library of observed spectra to find a spectra to best remove the primary star lines,
    accounting for dilution and rotational broadening.
    They also find that a secondary spectrum with a \(\Delta {RV}\) relative to the host would blend and remain undetectable.
    When searching for the secondary they median normalize the spectra into 100\K{} spectra since the difference between stars  <100\K{} different is almost indiscernible in the secondary.
    The residuals are renormalized and then a \textchisquared{} fit is again performed to identify any secondary spectra.
    For a low temperature companion (3500\K{}) at a 1\% flux ratio to a 5000--6000\K{} primary star their recovery technique achieves 80\% injection-recovery, in the optical.}

