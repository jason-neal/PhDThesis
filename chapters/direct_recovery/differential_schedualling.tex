%!TEX root = ../../thesis.tex

\subsubsection{Differential scheduling challenges}
\label{subsubsec:differential-schedualing}
This work has revealed that care needs to be taken in planning the observations for the application of the spectral differential technique of faint companions in the future.
Future attempts need to pay attention in particular to: the {\fwhm} of the lines in the region (governed by resolution and wavelength), the estimated companion separation \(\Delta {RV}_2\); and the previous observations from different observing periods, all while keeping the detector settings consistent.

The original goal for the observations was to obtain two different and ``clearly separated radial-velocities'' for the secondary companion.
However, the program was assigned a low-priority (C, in {ESO} grading) and, possibly due to operational reasons, the original time requirements necessary to secure well separated {RV}s for the companion spectra could not be met.
This meant that all observations were insufficiently separated to extract a differential spectra for the companion.

The long orbital periods of these targets is also a strong contributing factor to the insufficient separations.
Most of the targets observed here have orbital periods much longer than an observing semester (183 days).
An optimal pair of observations (achieved at the extrema) would need to have been obtained from separate observing periods (between 2 months and 19 years apart).
In some cases, even observations taken at the beginning and end of a single observing semester would not be sufficient to achieve a companion separation (depending on the phase and orbital parameters), requiring separate observing periods to even achieve the minimum \(\Delta \rm {RV}\) larger than the line {\fwhm}.
At the time (2012) it was impossible to ask for observation time over several semesters in a regular proposal.

This study demonstrates the importance of proposals for projects that need to be extended over several semesters or years.
In the {ESO} context, this corresponds to ``Monitoring proposals''~\citep[e.g.][pg.~18]{eso_eso_2017}.
Observations of the targets explored here, with long orbital periods in particular, would benefit from the new abilities for multi-period proposals and scheduling systems which allow for tighter scheduling constraints, such as a companion {RV} separation.

For future observations in the context of the differential subtraction technique it is suggested that the best possible orbital solution of the host and companion be used to estimate the companion's {RV} curve during the observing period, with the companion \Mtwosini{} providing an {RV} upper-limit.
Radial velocity constraints are also valid for other studies such as the detection of reflected light from exoplanets~\citep[e.g.]{martins_evidence_2015}.
Knowing the instrumental wavelength and resolution, an observing constraint can be set to avoid taking observations when the companion spectra are insufficiently separated, or the \(\Delta {RV}_2\) < {\fwhm}.
This constraint can be set using the absolute and relative \emph{time-link} constraints available in {ESO}'s {Phase 2 Proposal Preparation} (P2PP) tool.
Additionally, analysing the known orbital solution beforehand to determine {RV} constraints will also help identify the best time to observe, if observations from separate periods will be required or, if an optimally separated companion differential is even feasible.
Again the {P2PP} documentation for this observational proposal could not be obtained to check if these observations had used any of these features, which were available at the time, and set any constraints.
