%!TEX root = ../../thesis.tex

\subsection{Calculating the expected {RV}}
\label{subsec:expected_RV_calc}
In this section calculations are performed to estimate the {RV} of both spectral components in each observation and the likely {RV} separation between the two companions is estimated.
To apply the differential method the observations need to be Doppler shifted so that the host spectra can be subtracted in the same reference frame.
To do this the {RV} of the host in each observation is calculated from the orbital parameters in the literature.
The time of each observation is used as input into \cref{eqn:rv_equation} combined with the orbital parameters from \cref{tab:orbitparams} to calculate the expected {RV} of the host star.
These values calculated are given in \cref{tab:observations} as \({RV}_{1}\).
The companion mass (\Mtwo{} or \Mtwosini{}) is used alongside the stellar mass from \cref{tab:star_params} to also calculate the {RV} of the companion (see \cref{subsec:binary_mass_ratio}).
This is given as \({RV}_{2}\) in \cref{tab:observations}.
The {RV} difference between the host and companion for each observation is also computed and provided as \({rv}_{2} = {RV}_{2}-{RV}_{1}\)\footnote{This will be used in \cref{cha:model_comparison}.}.

For these observations the maximum estimated {RV} separation between the two companion spectra in \(\Delta {RV}\) is calculated following \cref{eqn:companion_difference} below and provided in \cref{tab:estimated_rv}.
This table also contains the estimated semi-major {RV} amplitude for the companion \(K_2\) (from \cref{eqn:q_ratio_K2}) and the phase coverage of the observations.
The phase coverage is the maximum fraction of the orbit covered between the observations for each target.
For {HD\,4747} the \(\Delta {RV}\) and phase coverage values are missing due to the single observation.


%!TEX root = ../../thesis.tex
\begin{table*}
    %\small
    \centering
    \begin{threeparttable}[b]
        \caption[Semi-amplitude and RV separation of companions.]{Estimated orbital semi-amplitude and {RV} separation of the companions, given the companion mass (\Mtwo{} or \Mtwosini{}) from \cref{tab:orbitparams} and observation times from \cref{tab:observations}.}
        \begin{tabular}{l c c c c c c}%[hb]
            \toprule
             & Estimated & Estimated & & \\  % 2017
             Companion & \Ktwo{} & |\(\Delta {RV}\)| & Phase coverage\\
             & (\kmps{}) & (\mps{}) & (\%)\\
             \midrule
             {HD 4747} & -10.65 & -- & --\\  % 2017
             {HD 162020} & -98.92\tnote{a} & 2388 & 0.28\\  %
             {HD 167665} & -14.47\tnote{a} & 145 & 0.18\\  %  -- \(2\times10^{-5} \)  best case based on age rage.
             {HD 168443b} & -64.65\tnote{a} & 258 & 0.035\\
             {HD 168443c} & -18.05\tnote{a} & <1 & 0.001\\  %(c)
             {HD 202206}B & -6.79 & 78 & 0.74\\  %(B)   % May2017
             {HD 202206}c & -2.50 & <1 & 0.15\\  %(B)   % May2017
             {HD 211847}B & -1.85 & 5 & 0.09\\  %B % 2017
             {HD 30501} & -16.12 & 1410 & 5.8\\
             \bottomrule
         \end{tabular}\label{tab:estimated_rv}
         \begin{tablenotes}
            \item[a] {Maximum \Ktwo{} only given \Mtwosini.}
         \end{tablenotes}
    \end{threeparttable}
\end{table*}

%%!TEX root = ../thesis.tex
\todo{check hspace spacing}
\begin{table*}
    %\tiny
    \small
    \centering
    \caption[Estimated flux ratios and semi-amplitude of the companion.]{Estimated flux ratios and semi-amplitude of the companion given the companion \(\textrm{M}_{2}/\textrm{M}_{2} \sin{i}\) from \cref{tab:orbitparams}.
    The flux ratio \(F_{2}/F_{1} \) is calculated using the \emph{K}-band magnitude difference of the host star to the Baraffe evolutionary model magnitude for the companion mass.
    The model ages used are those closest to host age value in \cref{tab:star_params}.
    The noise ratio is calculated via \(N_{2}/N_{1} = \sqrt{2} \times\sqrt{F_{1}/F_{2}}\).
    The orbital properties are calculated using the orbital parameters given above along with the times of observations in \cref{tab:observations}.}
    \begin{tabular}{l c c c c c c c c}
        \toprule
        &  Estimated  & Estimated &  Estimated & Estimated &  &    \\  % 2017
        Host           & \(\rm F_{2}/F_{1} \)   & \(\rm N_{2}/N_{1} \) (noise ratio) & \(\rm K_2\) &   \(\Delta {RV}\) & Phase coverage \\
        & \emph{K}-band     & & (\kmps{}) & (\mps{}) & (\%) \\
        \midrule
        \object{HD 4747}        & \(3\times10^{-4} \)   & 76 &  -10.65 & -  &  -  \\  % 2017
        \object{HD 162020}   & \(7\times10^{-7} \)   & 1\,615  &  -98.92\tablefootmark{a} &  2\,344.24     & 0.28\hspace{4em} \\  %
        \object{HD 167665}    & \(2\times10^{-4} \)   &  105    &  -14.47\tablefootmark{a}   &   138.45     & 0.18\hspace{4em}\\  %  -- \(2\times10^{-5} \)  best case based on age rage.
        \object{HD 168443b} & \(1\times10^{-16} \)  &    \(1\times10^{8} \)   &  -64.65\tablefootmark{a} &   257.16   & 0.035 \\
        \object{HD 168443c} &  \(1\times10^{-11} \)  &   \(4\times10^{5} \)     &  -18.05\tablefootmark{a}  &   0.95   &  0.001 \\  %(c)
        \object{HD 202206}B  & \(8\times10^{-7} \)  &   1\,586 &  -6.79 & 145.17   & 0.74\hspace{3em}   \\  %(B)   % May2017
        \object{HD 202206}c  &  \(5\times10^{-15}\)   &     \(2\times10^{7} \) &   -2.50     &   0.67     &  0.15\hspace{3em} \\  %(B)   % May2017
        \object{HD 211847}B  &  0.01 &  14   & $-$1.85 & 3.88   & 0.09\hspace{3em} \\  %B % 2017
        \object{HD 30501}      &  0.002  &  27  &  -16.12    &  1\,346.46      & 5.8\hspace{4em}\\
        \bottomrule
    \end{tabular}\\
    \tablefoot{
        \tablefoottext{a}{Maximum \(K_2\) only given \(M_2 \sin{i}\)}
    }
    \label{tab:flux_table}
\end{table*}
 % This is the old table.

The full orbital solution for the components along with the times of observations are displayed below in \cref{sec:orbtial_diagrams}.


