%!TEX root = ../../thesis.tex

\section{Motivation and target selection}
\label{sec:target_motivation}

The work of~\citet{sahlmann_search_2011} used the astrometry technique to identify several candidate brown dwarf companions of {FGK} stars with \Mtwosini{} values >10\Mjup{}.
Seven candidates from~\citet{sahlmann_search_2011}, which were visible in {Period 89} (2012), were selected for further observation in order to identify their stellar nature.
That is, to refine the mass of the companions to distinguish if their companion is: a large giant planet (\(M \apprle 13\)\Mjup{}), a Brown dwarf (\(13 \apprle M \apprle 80\)\Mjup{}), or a low-mass star (\(M \apprge 80\)\Mjup).

The list of target host stars that were observed are presented in \cref{tab:star_params} along with their stellar parameters, while \cref{tab:orbitparams} details the orbital parameters of each system from the literature.

It is noted that the orbital parameters of some targets have been refined in the literature since the observations took place.
For example three candidates have had their masses refined in recent works.
The companion to {HD\,211847} was determined to be a low mass star with \(\mtwo=155\)\Mjup{}~\citep{moutou_eccentricity_2017}, while the companion to {HD\,4747} was found to have a mass of \(\mtwo=60\)\Mjup{}~\citep{crepp_trends_2016}.
The two companions of {HD\,202206} (B and c) were found to have masses of \({M}_{B}=93.6\)\Mjup{} and \({M}_{c}=17.9\)\Mjup{}, respectively, classifying {HD\,202206}c as a ``circumbinary brown dwarf''~\citep{benedict_hd_2017}.
These three companions with recently refined masses, along with {HD\,30501}, create a good set of benchmarks to compare any results from the techniques used here, and show that the masses of these targets do span the {BD} to low-mass star range.
All target companions except {HD\,162020} (P=8.4~days) are in (very) long period orbits (P=0.7--38~years) with masses (or \Mtwosini{}) greater than 10\Mjup{}.

The spectral differential approach was chosen with the goal to constrain the companion masses while minimizing the observational time required to observe: in theory only requiring two observations.
The observing proposal determined that it should be possible to obtain a detection of a companion with a 1\% contrast ratio with an exposure time of around 20 minutes.
Two observations at ``clearly separate {RV}s'' were requested to constrain each target.
Observations were performed without telluric standard star observations to avoid the extra observing time overhead, choosing to instead rely on synthetic model correction (see \cref{sec:telluric_correction}).
The \textit{K}-band was chosen to achieve a high contrast relative to the host star, detected in the extreme V-K colour indexes (>7.8), while the specific atmospheric window (2110--2070\nm{}) was chosen in order to reduce the absorption introduced by the atmosphere~\citep{barnes_hd_2008}.

\begin{landscape}
    %!TEX root = ../../thesis.tex

%% Table of stellar parameters
\begin{table*}
    \centering
    \begin{threeparttable}[b]
        \caption{Stellar parameters of the host stars.
V is the apparent magnitude taken from {SIMBAD}~\citep{wenger_simbad_2000}. {Distances were calculated from the GAIA DR2 parallax measurements.}}
        \begin{tabular}{l c c r@{$~\pm~$}l r@{$~\pm~$}l r@{$~\pm~$}l r@{$~\pm~$}l c c c}
            \toprule
            Star & SpT & V & \multicolumn{2}{c}{\(T_{\textrm{eff}}\) (K)} & \multicolumn{2}{c}{logg (cm s\(^{-2} \))} & \multicolumn{2}{c}{[Fe/H]} & \multicolumn{2}{c}{\(M_1\) (\textrm{M}\(_{\odot} \))} & Age (Gyr) & d (pc) & Reference\\
            \midrule
            {HD 4747} & K0V & 7.15 & 5316 & 50 & 4.48 & 0.10 & $-$0.21 & 0.05 & 0.81 & 0.02 & $3.3 \pm 2.3$ & $18.80 \pm 0.04$ & 1, 2, 3, 8 \\
            {HD 162020} & K3V & 9.12 & 4723 & 71 & 4.31 & 0.18 & $-$0.10 & 0.03 & 0.74 & 0.07 & $3.1 \pm 2.7$ & $30.85 \pm 0.06$ & 4, 5, 6, 8 \\
            {HD 167665} & F9V & 6.40 & 6224 & 50 & 4.44 & 0.10 & 0.05 & 0.06 & 1.14 & 0.03 & 0.7 -- 3.6 & $ 31.24 \pm 0.06$ & 1, 8 \\
            {HD 168443} & G6V & 6.92 & 5617 & 35 & 4.22 & 0.05 & 0.06 & 0.05 & 1.01 & 0.07 & $10.0 \pm 0.3$ & $39.67 \pm 0.12$ & 5, 6, 8 \\
            {HD 202206} & G6V & 8.07 & 5757 & 25 & 4.47 & 0.03 & 0.29 & 0.02 & 1.04 & 0.07 & $2.9 \pm 1.0$ & $46.03 \pm 0.14$ & 5, 7, 8 \\
            {HD 211847} & G5V & 8.62 & 5715 & 24 & 4.49 & 0.05 & $-$0.08 & 0.02 & 0.92 & 0.07 & 0.1 -- 6.0 & $48.81 \pm 0.13 $ & 1, 2, 4, 8 \\
            {HD 30501} & K2V & 7.59 & 5223 & 50 & 4.56 & 0.10 & 0.06 & 0.06 & 0.81 & 0.02 & 0.8 -- 7.0 & $20.37 \pm 0.01$ & 1, 4, 8 \\
            \bottomrule
        \end{tabular} \label{tab:starparams}
        \begin{tablenotes}
           \item[] References: (1)~\citet{sahlmann_search_2011}; (2)~\citet{santos_spectroscopic_2005}; (3)~\citet{crepp_trends_2016}; (4)~\citet{tsantaki_deriving_2013}; (5)~\cite{bonfanti_age_2016}; (6)~\citet{santos_spectroscopic_2004}; (7)~\citet{sousa_spectroscopic_2008}; (8)~\citet{collaboration_gaia_2018};
        \end{tablenotes}
    \end{threeparttable}
\end{table*}

    %!TEX root = ../../thesis.tex
% Table of orbit parameters

\todo{Rotate orbital parameter table.}
\begin{table*}
    \centering
    \caption[Orbital parameters of companions.]{Orbital parameters for the BD companions obtained from the literature.}
    \begin{tabular}{l c r@{$ \,\pm\, $}l r@{$ \,\pm\, $}l r@{$ \,\pm\, $}l r@{$ \,\pm\, $}l r@{$ \,\pm\, $}l cc c c}
        \toprule
        Object  & \(\gamma\) & \multicolumn{2}{c}{Period} & \multicolumn{2}{c}{\(e\)} & \multicolumn{2}{c}{\(\kone\)} &  \multicolumn{2}{c}{\(T_{0}\)} & \multicolumn{2}{c}{\(\omega\)} & \Mtwosini{} & \(\mtwo\) & Ref.\\
        & (\kmps{}) & \multicolumn{2}{c}{(day)} & \multicolumn{2}{c}{} & \multicolumn{2}{c}{(\mps{})} & \multicolumn{2}{c}{(JD-2,450,000)} &  \multicolumn{2}{c}{(deg) } & (\Mjup{}) & (\Mjup{}) & \\
        \midrule
        \object{HD 4747}  & $0.215 \pm 11 $    &  13\,826.2  &  314.1   &  0.740 & 0.002 & 755.3   &  12 & 463.1  &  7.3    & 269.1 &  0.6   &  39.6    & 60.2  & 1 \\
        \object{HD 162020}   & $-27.328\pm0.002$ &  8.42819  &  $6e^{-5}$   &  0.277 & 0.002   & 1\,813    &  4   & 1\,990.68   &  0.01  & 28.4   &  0.2   & 14.4     &     -    & 2 \\
        \object{HD 167665}   & $8.003 \pm 0.008$    & 4\,451.8 & 27.6     & 0.340 & 0.005  & 609.5   &  3.3     & 6\,987.6     &  29     & $-$134.3 & 0.9     & 50.3    &     -   & 3 \\
        \object{HD 168443}b  & $-0.047\pm0.552$     & 58.1124 & $4e^{-4}$ & 0.529 & 0.001   & 475.13 & 0.9 & 5\,626.20  &  0.02   & 172.9 & 0.1     & 7.7 &     -   & 4 \\
        \object{HD 168443}c  & $-0.047\pm0.552$ & 1\,749.83 & 0.57 & 0.211 & 0.002  & 297.7  & 0.6 & 5\,521.3     &  2.2     & 64.9  & 0.5     & 17.1    &     -     & 4 \\
        \object{HD 202206}B & 14.721   & 256.33  &  0.02    & 0.432 & 0.001   & 567     &  1  & 2\,176.14    &  0.12   & 161.9     & 0.2 & 17.4    & $93.2\pm7.3$   & 5, 6\\
        \object{HD 202206}c & 14.721   & 1\,260 &  11 & 0.22 & 0.03  & 41   & 1 & 3\,103     & 452    & 280   & 4   & 2.3 & $17.9\pm2.9$  & 5, 6\\
        \object{HD 211847} & 6.689\tablefootmark{a} & 7\,929.4 & 2\,500 & 0.685 & 0.068    & 291.4   & 12.2   & 12\,030.1    & 2\,500   & 159.2     & 2.0     & 19.2  & 155 & 3, 7\\
        \object{HD 30501}   & $23.710\pm0.028$    & 2\,073.6 & 3.0    & 0.741 & 0.004 & 1\,703.1 & 26.0   & 3\,851.5     & 3.0     & 70.4     & 0.7     & 62.3   & 89.6     & 3  \\
        \bottomrule
    \end{tabular}\label{tab:orbitparams}
    \tablebib{
             (1)~\citet{crepp_trends_2016}; (2)~\citet{udry_coralie_2002}; (3)~\citet{sahlmann_search_2011};
             (4)~\citet{pilyavsky_search_2011}; (5)~\citet{correia_coralie_2005}; (6)~\citet{benedict_hd_2017}; (7)~\citet{moutou_eccentricity_2017}
    }
    \tablefoot{
    \tablefoottext{a}{fixed}
    }
\end{table*}
\todo{Needs rotation}
\end{landscape}
