%!TEX root = ../../thesis.tex

\section{Direct recovery in the \mir{}}
\label{sec:mir}
It was investigated if this differential technique could be extended into the mid-infrared {\mir{}} domain.
There were two reasons for this: to develop experience with the {\mir{}} domain where the contrast ratios are higher, and due to the lack of high-resolution \nir{} spectrographs available at the time (see \cref{subsec:new_generation}).

{VISIR} is a \mir{} spectrograph on the {VLT}, offering diffraction-limited imaging at high sensitivity in three mid-infrared (\mir) atmospheric windows: the \emph{M}-band at 5\um{}, the \emph{N}-band between 8--13\um{} and the \emph{Q}-band between 17--20\um{}, respectively.
The use of {VISIR} to detect the spectra of Brown Dwarf companions in the {\mir{}} was briefly explored.
The candidate selected as the best target to investigate was {HD\,219828} which has a hot-Neptune (\Mtwosini{}=21\Mearth)~\citep{melo_new_2007} and a recently discovered super Jupiter (\Mtwosini{}=15.1\,\Mjup) on a long period (13 yr) eccentric orbit (e=0.81)\citep{santos_extreme_2016}.

Based on the spectra of a cool brown dwarfs in the \mir{}, and the detector configuration available at the time, the best option for the observations was the low resolution mode covering the wavelength region 8--13\um{}.
This wavelength region would have encompassed the \ce{NH4} signature at 10.5\um{} and the edge of a \ce{CH4} band at 7.7\um{}, both large features in the {BD} \mir{} spectrum.

After performing flux ratio calculations between the host and companion using the~\citet{baraffe_evolutionary_2003} models (see \cref{subsec:compaion_flux_ratio}) and considering the performance of the {VISIR} instrument and the exposure time calculator it was determined that observations with {VISIR} to achieve a \snr{} of 100 were infeasible, requiring 1000's of hours of observing time to achieve the necessary signal-to-noise level to separate the companion from a blended spectra.
For a different target, {HD\,189733} it was calculated that with an exposure time of 2 hours the \snr{} of the host and companion would be 85 and 4 respectively, using the low resolution spectroscopy mode.
As such the direct separation approach was not explored further in the \mir{}.
