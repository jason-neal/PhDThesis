%!TEX root = ../../thesis.tex

\section{Direct Subtraction Method}
\label{sec:direct-subtraction}
The basic premise of the direct subtraction method is to take two high resolution spectra of a binary system at separate phases, transform them to the rest frame of the star, calculate the difference so as to remove the spectrum of the host star, leaving behind a residual comprised of the difference between the two companion spectra with different Doppler shifts.
Similar techniques were originally developed to separate the spectra of binary stars~\citep[e.g.][]{ferluga_separating_1997} while~\citet{kostogryz_spectral_2013} performed promising simulations of an {M-dwarf} with a low-mass companion, if observed with {CRIRES}.

Assuming that the instrumental profile does not vary between observations and atmospheric absorption are dealt with appropriately, the spectra of the observed targets are assumed to be composed of two spectral components: a bright host star blended with the spectrum of a faint companion.
The spectrum received from the host-companion pair is given by the superposition of two spectral components (\(\textrm{J}_{1}\), \(\textrm{J}_{2}\)):
\begin{equation}
\textrm{I}(\lambda) = \textrm{J}_{1}(\lambda - v_{1}) + \textrm{J}_{2}(\lambda - v_{2}),
\end{equation}
where the subscripts 1 and 2 indicate the spectrum of the host and companion respectively, \(\lambda\) represents the wavelength of the spectra and \(\lambda-v\) represents the Doppler shift \(\lambda(1-v/c)\) by a velocity \(v\).

This can be shifted into the rest frame of the host star by applying the shift \(v_1\):
\begin{equation}
\textrm{I}(\lambda + v_{1}) = \textrm{J}_{1}(\lambda) + \textrm{J}_{2}(\lambda - v_{2} + v_{1}).
\end{equation}

To analyse \(\textrm{J}_2\), the spectral component of interest, the component from the host needs to be carefully removed.
If two observations of the same target are observed, denoted with subscripts \(a\) and \(b\), there will be relative motion between the components due to the orbit.
Assuming that the stellar spectra do not change over time\footnote{\citet{kostogryz_spectral_2013} found the stellar activity residual is smaller than the companion differential flux.} (\(\textrm{J}_{1a} = \textrm{J}_{2a}\)) and each spectrum can be individually Doppler shifted to the rest frame of the host star \(\textrm{J}_{1}(\lambda)\), then the spectrum of the host star can be removed though subtraction of the two observations.
Mutually cancelling the host component leaves two components of the companion subtracted from each other, with a relative Doppler shift between them.
\begin{align}
S(\lambda) &= \textrm{I}_{a}(\lambda + v_{1a}) - \textrm{I}_{b}(\lambda + v_{1b}) \nonumber \\
&= (\textrm{J}_{1a} + \textrm{J}_{2a}(\lambda - v_{2a} + v_{1a})) - (\textrm{J}_{2b} +\textrm{J}_{2b}(\lambda - v_{2b} + v_{1b})) \nonumber \\
&= \textrm{J}_{2a}(\lambda - v_{2a} + v_{1a}) - \textrm{J}_{2b}(\lambda - v_{2b} + v_{1b}) \nonumber \\
% &= \textrm{J}_{2}(\lambda - v_{2a}) - \textrm{J}_{2}(\lambda - v_{2b} - v_{1a} + v_{1b}) \nonumber \\
S(\lambda + v_{2a}-v_{1a}) &= \textrm{J}_{2a}(\lambda) - \textrm{J}_{2b}(\lambda - v_{2b} - v_{1a} + v_{1b} + v_{2a})\\
S(\lambda') &= \textrm{J}_{2a}(\lambda) - \textrm{J}_{2b}(\lambda - \Delta {RV}_2) \label{eqn:sprofile}
\end{align}
where,
\begin{equation}
\Delta {RV}_2 = v_{1a} - v_{1b} - v_{2a} + v_{2b} \label{eqn:companion_difference}
\end{equation}
is the {RV} difference between the two companion spectral components when the host components are mutually subtracted,
and \(\lambda' = \lambda + v_{2a}-v_{1a}\).

The resulting differential spectra \(S({\lambda'})\), dubbed \emph{s-profile} by~\citet{ferluga_separating_1997}, is composed of just the companion spectra, shifted and subtracted from itself.

\citet{ferluga_separating_1997} provide an analytical form for the \emph{s-profile} given a single Gaussian line of the form
$J(\lambda) = 1- D \cdot\exp^{{-\pi {(\lambda - \lambda_0)}^2} / {{W}^{2}}}$:
\begin{equation}
S(\lambda) = 2 D\cdot\exp^{{-\pi {D}^{2} [{(\lambda - \lambda_0)}^{2} +{(k/2)}^{2}]}/{{W}^{2}}} \cdot \sinh{\frac{\pi {D}^{2}(\lambda-\lambda_0)k}{{W}^{2}}},\label{eqn:sprofile_gaussain}
\end{equation}
where $\lambda_0$, \(D\), and \(W\) are the central wavelength, depth and equivalent width of the Gaussian line, and $k=\Delta {RV}_2 $ is the shift between the two companion spectra.

From binary dynamics~\citep[e.g.][]{murray_keplerian_2010} the {RV} amplitudes of the host and companion (ignoring the system velocity $gamma$) are related through the mass ratio, \(q\), while having an opposite sign\footnote{The opposite sign arises from a \(180^\circ\) difference in the angle of periapsis, \(\omega\), for the companion.} (see \cref{subsec:binary_mass_ratio}):
\begin{align}
v_{2} &= -q * v_{1} \label{eqn:q_relation}
\end{align}

\cref{eqn:companion_difference} can be simplified by expressing it in terms of the mass ratio and host {RV} only:
\begin{align}
\Delta {RV}_{2} &= q v_{1a} - q v_{1b} + v_{1a} - v_{1b} \nonumber \\
&= (1 + q)(v_{1a} - v_{1b}).\label{eqn:companion_difference_simplified}
\end{align}

If the \(\Delta {RV}_2\) between the companion spectra is able to be constrained or derived from the s-profile~\citep[see][]{ferluga_separating_1997} then the mass ratio of the system, \(q\), can be determined, thereby constraining the mass of the companion.

The values \(v_{1a}\) and \(v_{1b}\) are radial velocity of the host components.
The host's {RV} are calculated using the \cref{eqn:rv_equation} with the orbital parameters from the literature and provided in \cref{tab:orbitparams}.
These are used to shift each spectrum into the rest frame of the host star to mutually cancel the host's spectrum.
These components can also be determined directly from the spectrum by cross-correlating the observed spectrum with a stellar template of the host and gave results in reasonable agreement.

It is necessary to have a consistent instrumental setup~\citep{ferluga_separating_1997,hadrava_disentangling_2009}, to avoid introducing extra instrumental effects (e.g.\ slit-width and/or filters) into the spectral differentials and to always observe the same wavelength range and maximize the information to be extracted.
For these observations, the second observation of {HD\,202206} and fourth of {HD\,30501} were taken with different filters compared to the other observations.
This is according to the filter settings given in the fits file headers.
Therefore, these two observations could not be used for this differential analysis.
As noted in~\citet{hadrava_disentangling_2009}, any spectral differences in the filters would add extra unknown signal/noise making it harder to disentangle the faint spectral differences.
