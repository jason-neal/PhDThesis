%!TEX root = ../../thesis.tex


\section{Binary \texorpdfstring{\textchisquared}\ \ spectral recovery}
\label{subsec:companion_recovery}
The approach developed here is to fit the observed CRIRES spectra, consisting of a {FGK} star with a {BD} companion, with a model comprised of a combination of two synthetic spectra, one for each component.
This will be done using the \textchisquared{} approach which has been extensively used in the literature~\citep[e.g.][to list a few]{astudillo-defru_harps_2015, passegger_fundamental_2016, passegger_carmenes_2018, zechmeister_spectrum_2018, nemravova_xtauri_2016, kolbl_detection_2015, rajpurohit_exploring_2018}.
The recoverable information from the fitting will be the parameters of synthetic spectra fitted to the companion which will provide some indication of the companion temperature and spectral type, but it will not produce a direct mass constraint that was the original aim of the observations.

\subsection{The \texorpdfstring{\textchisquared}\ \ method}
\label{subsec:chi2}

The well known \textchisquared{} technique measures the weighted sum (for all data points \(i\)) of the squared deviation between the observation (\({O}\)) and the computed models (\({C}\)), with the minimum \textchisquared{} value representing the best-fit parameters:
\begin{equation}
\chisquared{} = {\sum}_i \frac{{(O_{i} - {C}_{i})}^2}{{\sigma}_{i}^{2}},
\end{equation}
where \({\sigma}_{i}\) is the error on each measurement.

The inverse survival function of the \textchisquared{} distribution is used to determine the confidence levels of the resulting parameters from the minimum \textchisquared{}.
The inverse survival function returns a \(\Delta\chisquared\) value from the minimum \textchisquared{} value for a given $\sigma$ level and degree of freedom.
This can be achieved in \emph{Python} with the \emph{scipy} package as the single line of code \texttt{scipy.stats.chi2{(dof)}.isf{(1-p)}}, where \textit{dof} is the degree of freedom and \(p\) the probability; for example \(p = 0.68\) for 1-\(\sigma\).
For example, the \(\Delta \chisquared\) for a single degree of freedom required for the 1-, 2-, and 3-\(\sigma\) confidence levels is 1, 4, and 9 respectively~\citep{bevington_data_2003}.
This method assumes that the measured flux is observed with a \snr{} sufficiently high so that the noise on the spectrum is approximately Gaussian, and the \textchisquared{} method appropriate.

For a given observation, the \textchisquaredreduced{} is computed as \(\chisquaredreduced{}=\chisquared{} / \nu\) where \(\nu = n - m\), the number of observed pixels, \(n\), minus the number of parameters of interest, \(m\)\footnote{\(m=2\) or \(m=4\) in the examples explored below.}.
In the cases explored below the \textchisquaredreduced{} is only calculated after the summation across the detectors is performed.

For each observation \(O\), the \(\sigma\) is estimated using the \(\beta\sigma\) method~\citep{czesla_posteriori_2018}, using the {MAD} (median absolute deviation about the median) robust estimator.
The \(\beta\sigma\) method estimates the spectral noise of the spectra using a series of numerical derivatives\footnote{Applying a Taylor expansion to the spectra.}.
We followed the procedure outlined in~\citet{czesla_posteriori_2018} to analyse the results from successive parameter combinations to settle on an order of approximation (derivative level) of \(N=5\), and a jump parameter (pixels skipped to avoid correlations) of \(j=2\).
The same \(\sigma\) value determined is applied to all points in the spectrum so that \({\sigma}_{i} = \sigma\).
The \(\beta\sigma\) method provides \(\sigma\) estimates for the target spectra which correspond inversely to \snr{} values between 100--400.
These \snr{} values are similar to the values given in \cref{tab:observations} which were calculated only using the standard deviation of the continuum of detector \#2.

The computed models \(C\) are described in \cref{models} and result in a multidimensional grid of \textchisquared{} values for each combination of model parameters, namely the spectral parameters  (e.g.\ \Teff{}), and the {RV} of the host and companion {RV} for each: detector, observation, and target.

The global minimum of the multidimensional \textchisquared-space is used to represent the best fitting model combination to the observed spectra.
The multidimensional \textchisquared{} grid is summed across multiple detectors to also determine a global minimum \textchisquared{} for the whole observation \(\chisquared_{obs} = \Sigma^{N}_{n=1} \chisquared_n\), where \(N\) is the number of detectors used.
However, the separate observations are not combined as the {RV} parameters of the host and companion will vary between each observation\footnote{Since the current observations are insufficiently separated, it may be possible to combine the separate observations; but in general this would not be the case.}.
To incorporate the separate observations a model that incorporates the phase information will be required and is beyond the scope of this work due to the limited number of observations (1--4).
A promising method to incorporate the phase information for the detection of exoplanetary spectra is given by~\citet{lockwood_nearir_2014,piskorz_evidence_2016}.
They detect evidence of an exoplanet spectrum with an contrast of order \({10}^{-4}\) using \nir{} (\textit{L}- and \textit{K}-band) spectra with a \snr{}\(\approx2000\) observed at six epochs over the whole orbit.


%%%%%%%%%%%%%%%%%%%%%%%%%%%%%%%%%%%%%%%%%%%%%%%%%%%%%%%%%%%%%%%%%%%%%%%%%%%%%%%
%%%%%%%%%%%%%%%%%%%%%%%%%%%%%%%%%%%%%%%%%%%%%%%%%%%%%%%%%%%%%%%%%%%%%%%%%%%%%%%

\subsection{Computed model spectra}
\label{models}
In this section the transformation of the synthetic {PHOENIX-ACES} spectra into the computed models (\(C\)) is explained.
These computed models will then be fit to the observed spectra.

Firstly, this assumes that the synthetic spectra are loaded and converted to consistent units.
The loading is easily performed using Starfish's \href{https://iancze.github.io/Starfish/current/grid_tools.html}{GridTools}~\citep{czekala_constructing_2015}, which can load the library spectra with a list of stellar parameters [\Teff{}, \Logg{}, \feh{}, \alphafe{}].
The {PHOENIX-ACES} spectra are converted from the units of the spectral energy distribution (\(\rm erg\,s^{-1}\,cm^{2}\,cm^{-1}\)) (at the stellar surface) into photon flux (\(\rm photon\,s^{-1}\,cm^{2}\)) by dividing by the photon energy\footnote{Photon energy \(E=\frac{hc}{\lambda}\), where \(h\), \(c\) and, \(\lambda\) are Plank's constant, the speed of light, and wavelength, respectively.}.
This can be simply achieved by multiplying the spectrum by the wavelength (multiplicative constants ignored), as done in~\citet{figueira_radial_2016}.
The spectra are also convolved with a Gaussian kernel to the instrumental resolution of the observations, in this case R=50\,000.
Due to the distributive property\footnote{\(I(\lambda) \ast (f(\lambda) + g(\lambda)) = I(\lambda) \ast f(\lambda) + I(\lambda) \ast g(\lambda) \)} of convolution the instrumental broadening is performed on each individual library spectrum first.
This is to avoid performing convolution for each combination of model parameters in the binary model after the spectra have been combined.
It would be more computationally expensive to perform the convolution on every model combination, \(C\).

These synthetic spectra are used individually for the single component model (\cref{subsubsec:single-model}) and also combined together into a binary model (\cref{subsubsec:binary-model}).
The results of these models are then interpolated to the wavelength grid of the observed spectra and the \textchisquared{} calculated by comparing the model and observation at each wavelength point, \(i\).

\subsubsection{Single component model}
\label{subsubsec:single-model}
The single component model \(C^{1}_{i}\) comprises of a single synthetic spectrum, \(J\), (with model parameters \Teff{}, \Logg{}, \feh{}, \alphafe{}) \alphafe{} that can be Doppler shifted by a {RV} value \Rvone{}.
\begin{equation}
C^{1}_{i}(\lambda) = \J{}(\lambda_{0}(1-\frac{\rvone}{c}))
\end{equation}
where \(\lambda\) is the shifted wavelength, \(\lambda_{0}\), the model rest wavelength and, \(c\), the speed of light in a vacuum.
The model's flux in the wavelength region of the observations is continuum normalized to unity to match the observed spectra, and then interpolated to the wavelength grid of the observation.

This single component model analysis is similar to the~\citet{passegger_fundamental_2016} \textchisquared{} fitting, using {PHOENIX-ACES} spectra to fit and determine the parameters of {M-dwarf} spectra.
A similar re-normalization (see \cref{subsec:renorm}) as~\citet{passegger_fundamental_2016} is used to account for slight differences in the continuum level and possible linear trends between the normalized observation and models.
However, unlike~\citet{passegger_fundamental_2016}, no dynamical masking was applied to sensitive lines to make the \textchisquared{} minima more distinct nor a linear interpolation of the stellar parameters between the grid models to obtain higher precision stellar parameters.
This was because, at this stage, only the presence of the secondary is trying to be detected, not a determination off precise stellar parameters.
These processes and others could however be included in the future to improve the detectability and precision of the best-fit model.
Instead, a radial velocity component is included in the \textchisquared{} fitting, which is not included in~\citet{passegger_fundamental_2016}.

{\red Rotational broadening of the host was not included in these models as an extra variable fitting  parameter.
In \citet{passegger_carmenes_2018} rotational broadening is only included at the fine grid search stage, using only a fixed value for each target to determine, determined in a separate work, to reduce the number of parameters.
A fine search is not attempted in this work.}

\subsubsection{Binary model}
\label{subsubsec:binary-model}
In the binary model case the model is considered to be the superposition of two synthetic spectral components, one each for the host and companion respectively.
Both components are Doppler shifted by \Rvone{} which represents the {RV} motion of the host star relative to Earth's barycentre, while the companion spectra is Doppler shifted a second time by \Rvtwo{} which represents the {RV} difference between of spectrum the host and companion.
This choice is arbitrary\footnote{Having \Rvtwo{} as the companion's {RV} offset relative to the barycentre is also a valid choice.}, but in this way the mean motion of the system relative to Earth is captured only in \Rvone{}, along with the orbital motion of the host.
The two spectra are scaled by their radius squared (see \cref{subsection-radius}) then added together, thus setting the relative amplitude of the two spectral components.
Given two spectral components \Jone{} and \Jtwo{} with radii \Rone{}, \Rtwo{} this equates to
\begin{align}
C^{2}_{i}(\lambda) = & \jone{}(\lambda_{0}(1 - \frac{\rvone{}}{c}))\times {\rone}^{2} +\nonumber \\
& \jtwo{}(\lambda_{0}(1 - \frac{\rvone{}}{c})(1 - \frac{\rvtwo{}}{c}))\times {\rtwo{}}^{2}
\end{align}

The combined binary model is continuum normalized by dividing it by an exponential fitted to the continuum of the combined spectrum.
Here at 2100\nm{} it assumed that the Rayleigh-Jeans regime is appropriate.
This assumption is wavelength dependent and other continuum normalization techniques may also be valid.
In the case of a {BD} companion around an {FGK} star investigated here, the continuum is dominated by the contribution from the host star, contributing the majority of the spectrum with flux ratios below \(\sim\)1\%, in the wavelength range 2110--2165\nm{}.

Models are combined in this way to represent the correct absolute flux ratio of the spectral components.
A further method could allow a variable flux ratio to be included as an extra parameter and be fitted as well.
However, it would add an extra dimension to the \textchisquared{} grid and potentially add more degeneracy between models of the companion.
The median flux ratio between the two components is calculated for the wavelength range used here as an indication of the flux ratio level.
This is given as \FtwoFone{} in \cref{tab:example_params}.

This binary model should provide meaningful information about the likely companion parameters (e.g.\ \Teff{}) and a possible estimate of the flux ratio of the system.
These can be combined with the~\citet{baraffe_evolutionary_2003} models to constrain the mass of the companion.
However, care is needed with the binary model as the inclusion of extra spectral components and associated parameters could also provide a ``better'' fit to observations which have faint or even no companions, by fitting components of the noise.

The full list of grid parameters for the binary model are \(\teffsub{1}\), \(\logg{}_{1}\), \feh{}\(_{1}\), \alphafe{}\(_{1}\), \Rvone{}, \(\teffsub{2}\), \(\logg{}_{2}\), \feh{}\(_{2}\), \alphafe{}\(_{2}\), \Rvtwo{} where the subscripts 1 and 2 indicate the host and companion models respectively.


\subsubsection{Effective radius}
\label{subsection-radius}

The {PHOENIX-ACES} spectra are calculated at the stellar surface.
To combine the two synthetic spectra with the correct absolute flux ratio the stellar size needs to be accounted for.
The emitted flux needs to be integrated over the effective surface area of each emitting body respectively.
Ignoring the common multiplicative constants that will not affect the ratio between spectra and disappear with normalization the two synthetic spectra of the binary model are individually scaled by the square of their respective radii, \(R_{1}\) and \(R_{2}\) \Rone{}.

In this work the radius used to scale the spectra is the effective radius of each component from the {PHOENIX} model header; the {PHXREFF} keyword.
This radius is utilized in the modelling of the {PHOENIX-ACES} stellar atmospheres.
This is used as it is directly tied to each model spectrum, and already calculated and available.
In this way it does not incorporate an extra assumption or model relating the library model parameters to a stellar radius.
The ratio of the radii from the two synthetic spectra in the binary model examples presented are provided in \cref{tab:example_params} as \(\rone/\rtwo\).

Using the radii in this way for the companions has its limitations because as stated previously, there is a degeneracy in {BD} mass, age, and luminosity of the companion, and in particular a combination of radius-mass and radius-age relationships~\citep{sorahana_radii_2013}.
Using the {PHOENIX-ACES} model effective radii does not allow for any independent age constraints to be incorporated into the stellar radius, or allow for any variability in the radii to account for uncertainties.

The targets analysed here do not transit, but in cases that did transit the radius ratio can be independently determined from the photometric transit method~\citep{deeg_photometric_1998}.
This independent radius ratio could be used as a constraint when combing the binary model spectra.

\todo{Radii of different parameters}

\subsection{Re-normalization}
\label{subsec:renorm}
Slight trends in the continuum level between the observed spectra and computed models were removed using the re-normalization following~\citet{passegger_fundamental_2016}:
\begin{equation}
F^{obs}_{re-norm} = F^{obs} \cdot \frac{\textrm{continuum fit}_{model}}{\textrm{continuum fit}_{observations}}.
\end{equation}
The polynomial continuum fits to the normalized observations and models are used to re-normalize the observed spectrum to the continuum of the models.
For detectors \#1--3 a polynomial of first degree was used, while for detector \#4 a polynomial of second degree was needed to fit the edge of a strong Hydrogen line (Brackett-\(\gamma\)) at 2166\nm{}, which lies just off of the edge of detector \#4.
This broad line is only observed in the synthetic spectra but not in the reduced observations.
It is assumed that the edge of this line was normalized out during the reduction process.

For each model the continuum level is further allowed to be slightly varied by \(\pm0.05\) as a free parameter taking the model that fits with the smallest \textchisquared{} value as the choice for this combination of binary parameters.


\subsection{Reducing dimensionality}
\label{subsec:reduce-params}
The high dimensionality of the binary model makes it computationally challenging and difficult to analyse the \textchisquared{} space.
This section discusses the choices made to reduce the dimensionality and the parameter space, to reduce the computation time.
For reference, the parameter space of the models is multiplicative.
That is, each new dimension or parameter added multiplies the number of possible model combinations.
When increasing from the single model to the binary model, the number of parameters doubles, but the number of possible models is actually squared, assuming that each spectral component can explore the full parameter space.
The binary model is therefore more computationally expensive.
In general the number of possible parameter combinations for \(k\) spectral components each with a grid of \(l\) models increases to \({k}^{l}\).
If the full set of {PHOENIX-ACES} library spectra (66456) is explored with a binary fit then this naively balloons to over 4.4 billion possible combinations.
This is the worst case scenario as half of these combinations are not unique as the host and companion components will be swapped.
A number of assumptions are implemented to vastly reduce the parameter-space enabling faster computation.

The first assumption is to restrict the Alpha element abundance (\alphafe{}) of the models to zero.
This is likely a very good approximation as all the targets have solar metallicity and are thus very likely to belong to the thin disk of the Galaxy, where \alphafe{} values are close to zero, i.e.\ solar~\citet[e.g.][]{adibekyan_chemical_2012}.
The second assumption is that the search space can be significantly reduced by using literature values of the host star given in \cref{tab:star_params}.
The metallicity of both model components are fixed to the closest grid to the literature value of the host star, usually \feh{}=0.00.
The \Logg{} of the host star is also fixed to the closest grid literature value.
The uncertainties on the literature measurements for \Logg{} (\(\sim\)0.1) and metallicity (\(\sim\)0.05) are both smaller than the grid steps of 0.5 for these parameters.
The \Logg{} of the companion is obtained from the~\citet{baraffe_evolutionary_2003,baraffe_new_2015} evolutionary models for the given companion's mass (\Mtwo{} or\Mtwosini{}) and host's age.

A starting point for \(\teffsub{2}\), the estimated companion temperature from the Baraffe evolutionary models is used, given the companion mass and stellar age.
The temperature grid is extended about this value in each direction, within the temperature limits of the synthetic model limits.
For example the companion temperature grid spans \(-600\) to \(+400\)\K{} in \cref{fig:Mdwarf_contours} and \(\pm400\)\K{} in \cref{fig:HD211847_simulated_contours,fig:HD211847_result_contours} about the estimated companion temperature.

The large number of possible combinations stated above also do not include the {RV} grid for each component, which is user defined and can have a variable resolution, and amplitude.
For example, the number of models to compute increases  with a decrease in the grid step size (a finer {RV} resolution) for a fixed {RV} range.
The {RV} grid space can be reduced significantly by tailoring it to the target being examined.
For each target and observation, the estimated {RV} values from \cref{tab:observations} are used as a centre starting point for the \Rvone{} and \Rvtwo{} values and then incremented within a few {\fwhm} around those values, or out to the targets \Kone{} or \Ktwo{} values.

An iterative process could potentially be implemented to refine the {RV} grids, starting with a larger grid with lower {RV} resolution and then performing a higher resolution grid about the minimum \textchisquared{} {RV} values.
This was attempted manually during testing but it could be easily automated in the future, at the cost of recalculating the \textchisquared{} at different {RV} resolutions.
One could expect that a good starting {RV} grid step be governed by the spectral resolution, e.g.\ comparable to the {\fwhm} velocity.

Also for the targets with a fully resolved orbit of the known {RV} of the host star, \Rvone{} could also be held fixed, however this was not performed in this work.
