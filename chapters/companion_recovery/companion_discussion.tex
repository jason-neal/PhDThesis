%!TEX root = ../../thesis.tex

\section{Discussion}
\label{sec:chisquared_discussion}




\section{Discussion}
\label{sec:discussion}
The spectral differential and the synthetic recovery methods attempted here were both unsuccessful in a detection of the {BD} companion spectra.
The upper mass limits of \(600^{+20}_{-40}\) we set for these companions is very high, roughly six times higher then the {BD} mass limit \(\sim 80-90\)\Mjup{}.
We discuss potential reasons and solutions for these poor results below, list the lessons learned in this exploratory study, and provide some guidance for any future attempts with these methods.


\subsection{Synthetic recovery limitations}
\label{subsec:limitations}
In this section we discuss some of the limitations from this synthetic recovery method and some options to overcome some of these.

\subsubsection{Mismatch in synthetic models}
\label{subsubsec:mismatch}
We believe the mismatch between the observation and synthetic spectra is the main cause of the unsuccessful companion detection, impacting the recovery in two ways.
The mismatch causes the \textchisquared{} values to be large in general, but also causes the companion temperature to be pushed to higher temperatures.

In our examples the \logg{} and metallicity of the synthetic models are held fixed, leaving only temperature to vary.
The temperature impacts the synthetic spectral models in two main ways: the flux level of the continuum; and the number and strength of the absorption lines.
In the binary model the contributions from the individual components is scaled by the flux ratio.If the temperature of the companion increases then the flux and radius of the companion increases.
Its contribution of the companion to the binary model increases and the flux ratio \FoneFtwo{} decreases.
This effectively makes the lines of the spectrum of the host component relatively smaller in the normalized binary model spectrum.
Due to the initial mismatch of synthetic spectral lines, a decrease in relative strength of the host lines decreases the \textchisquared{} value, and is a better "match" to the observation.
This causes the recovered temperature of the companion to be higher than expected, >2000\K{} higher if allowed by the exploration grid.
The \textchisquared{} approach is dominated by reducing the mismatch in the spectrum of the host rather than detecting the spectra of the companion.
This spectral mismatch is not observed in the simulations in \cref{subsec:simulated_binaries} since they are created using the synthetic spectra themselves, and hence they do recover the correct host spectra.


\subsubsection{Line contribution of faint companions}
\label{subsubsec:line_contributions}
We calculate the line depths of the synthetic companion spectra to determine the \snr{} levels required to detect the lines of the binary companions.
One thing easy to overlook when attempting to detect the binary companion at low flux ratios is the actual contribution of the spectral lines of the companion.
The flux ratio of the continuum for our most promising target is \FtwoFone{}\(\sim\)3\% with the other targets having an expected flux ratio around 1\%, and some well below.
The spectral lines of the individual components which are the features we are trying to detect with the binary model, have depths on average around 10--20\% of their respective continua,  at-least between 2110--2160\nm{}.
In effect, the companion line features have a depth \(\ll 1\%\) relative to the continuum of the combined spectra.

In \cref{tab:line_contributions} we calculate some properties of the spectral lines in the {PHOENIX-ACES} library between 2110--2160\nm{}.
We count the number of spectral lines (\emph{no. lines}) deeper than 5\%, and take the average depth (\emph{avg. depth}) of these lines.
The contribution \emph{cont. depth} of the companion lines to a combined spectrum accounts for the flux ratio between the two components.
Here we use a Sun-like host with \(\teffsub{1}\)=5800\K{}.
This simplified combination neglects the continuum shapes of both spectral components and uses the average flux ratios for this wavelength range.
The {PHOENIX-ACES} spectra in the temperature range of 2500--5000\K{} shown in \cref{fig:comp_spectra} can be used to get a visual indication of the line density and depth measured here.

There are more lines >5\% deep for the lower temperature spectra, with 360--460 lines in this wavelength range, to be compared with the 31 deep absorption lines found in a Sun-like spectrum.
The average line depth of these lines is also larger than the Sun-like spectrum, around twice as deep.
However, when combined, the contribution of the companion lines is 1--2 orders of magnitude smaller than the hosts lines due to the low continuum flux ratios.

For example, with the synthetic model for the companion of {HD~211847}, the average contributions of lines >5\% become only 0.3\% deep in a binary with the Sun-like spectrum.
For a companion with a temperature of 2300\K{} (the lower {PHOENIX-ACES} temperature limit) the deepest lines contribute lines around 0.1\%.

%{\bl We use the contributed line depth values to calculate the \snr{} level required to have Gaussian noise of the same height and the observed \snr{} required to achieve equivalent contribution from all N lines of the spectrum, \(\rm \textrm{SNR}_N = \textrm{SNR} /\sqrt{N}\).
%This is for the synthetic spectra which have many more lines than the observed spectra in this wavelength range.}

The \snr{} of the observed spectra is between 100--300, which is below the \snr{} of 323 needed for the detection of the low-mass star companion of {HD~211847} with temperature of 3200\K{} and \logg{} 5.0.
For our other targets with {BD} companions at and below the {PHOENIX-ACES} temperature range, we would need observed \snr{} >800 to detect the individual spectral lines of the companion.
With the \snr{} increasing with \(\sqrt{N}\) this would require the observational time for each target to be increased by a factor of \(\sim\)~10--64.

Our non-detection of binary companions with low flux ratios is consistent with other works.
For example~\citet{nemravova_xtauri_2016} performed extensive spectral analysis of a quadruple-star system \object{$\xi$ Tauri} using 227 spectra in 3 different wavelength bands.
Of the four stars in the system they were unable to detect the spectral component of the one which had a luminosity ratio below 1\%.

\begin{table}
    \small
    \caption{Contribution of synthetic lines within 2110--2160\nm{} of synthetic {PHOENIX-ACES} spectra to a binary model.
        \FtwoFone{} is the continuum flux ratio between a spectrum with the given \Teff{} and \logg{} and a Sun-like spectrum with \Teff{}=5800, \logg{} = 4.5 (right most column).
        \emph{No. lines} is the number of spectral lines deeper than 5\% from the continuum of the individual spectra while \emph{avg. depth} is the mean depth of those lines.
        \emph{Cont. depth} is the average contribution, or depth, of these lines in the combined spectrum of a binary with a Sun-like spectrum.
        The \snr{} is signal-to-noise level required to have Gaussian noise \(\sigma\) =1/\snr{} equal to the \emph{cont. depth} level in the binary model.
        All synthetic spectra used here have \feh{}=0.0}
    \begin{tabular}{*7c}
        \toprule
        \Teff{} (K)  & \multicolumn{2}{c}{2300} & \multicolumn{2}{c}{3200} & 5800 (\Fone{})\\
        \logg{} & 5.0 & 4.5  & 5.0 & 4.5 & 4.5 \\
        \midrule
        \FtwoFone{} & 0.006 & 0.019 & 0.029  & 0.091 & 1.000 \\  
        % {>2\%}  & no. lines & 470 &  463 & 414  & 444 & 111 \\
        % & avg. depth & 0.20 & 0.23 & 0.10  & 0.12 & 0.04 \\
        % & cont. depth \tablefootmark{a} & 0.0012 &  0.0043 & 0.0028 &  0.0100 &  0.0333\tablefootmark{b} \\ 
        % \midrule
        no. lines & 464 & 463 & 365  & 413 & 31 \\
        avg. depth & 0.2  & 0.23 & 0.11 & 0.12 & 0.10 \\
        cont. depth \tablefootmark{a} & 0.0012 & 0.0043 & 0.0031 & 0.0100 &  0.0833\tablefootmark{b} \\
        \snr{}  & 833 & 232 & 323  & 100 & 12 \\
        %  \snr{}\(\rm _N\)  & 39 & 11 & 17  & 5 & 2 \\
        \bottomrule
    \end{tabular}
    \tablefoot{
        \tablefoottext{a}{avg. depth \(\times~\ftwo{} / (\fone{} + \ftwo{})\), where \Fone{} is the component in the far right column.}
        \tablefoottext{b}{avg. depth \(\times~\fone{} / (\fone{} + \ftwo{})\), where \Ftwo{} is for the companion with \Teff{}=3200, \logg{}=4.5.}
    }
    \label{tab:line_contributions}
\end{table}

\subsubsection{\(\chi^2\) asymmetry} 
\label{subsubsec:chi2_assymetry}
In \cref{fig:injection_shape} we showed that the shape of the recovered \textchisquared{} becomes asymmetric when dealing with companion temperatures below around 3800\K{}.
A visual inspection of the spectra reveals the likely cause.
In \cref{fig:comp_spectra} we show the corresponding spectra between 2111--2165\nm{}.
As the temperature decreases the strongest lines become less prominent, disappearing progressively among the other many small lines that appear at lower temperatures.
Hence there are no strong companion lines to easily distinguish one temperature from another.
In the flatter part of the \textchisquared{} curves several low temperature companions are equally well fitted to the simulation/observation.

\cref{fig:injection-recovery,fig:injection_shape} show different recovered temperatures but both agree above 3800\K{}.
A higher companion temperature is recovered between 2800 and 3800\K{}, where as in \cref{fig:injection_shape} a lower temperature is recovered.
This is probably due to a combination of the noise added, and the asymmetries of the \textchisquared{} lines.
\cref{fig:injection-recovery} uses the noise level from the observed spectrum while \cref{fig:injection_shape} has a \snr{} of 300.
This large asymmetry can also explain the jump observed in the synthetic recovery temperature around 2700\K{} in \cref{fig:injection-recovery}.

The asymmetry also causes an asymmetry in the \textchisquared{} error bars which can be seen in the bottom panel of \cref{fig:injection_shape}.
For instance the recovered value and 1-\(\sigma\) error bars on the 3000\K{} injected companion is \(2800 ^{+20}_{-100}\), with an asymmetric error bar skewed towards lower temperatures.

The bump observed at 5100\K{} in the \textchisquared{} curves is due to a discontinuity in the {PHOENIX-ACES} modelling.
The "reference wavelength defining the mean optical depth grid" is changed at 5000\K{}~\citep[][Sect. 2.3]{husser_new_2013}.
Care needs to be taken if trying to detect a companion near this temperature.

\begin{figure}
    \centering
    %\includegraphics[width=\hsize]{images/final/companion_spectra.pdf}
    \caption{{PHOENIX-ACES} spectra for temperatures between 2500 and 5000\K{}, corresponding to to the same lines in \cref{fig:injection_shape}.
The flux units are the native units of the {PHOENIX-ACES} spectrum, (\(\rm erg\,s^{-1}\,cm^{2}\,cm^{-1}\)), and have not been scaled by the stellar radii.
All spectra have a \(\rm \logg{}=5.0\) and \(\rm \feh{}=0.0\).
The vertical dotted lines indicate the edges of the CRIRES detectors.}
    \label{fig:comp_spectra}
\end{figure}

\subsubsection{Component {RV} separation}
\label{subsubsec:rv_seperation}
Another factor which could contribute to an unsuccessful detection is the {RV} separation between the host and companion, \Rvtwo{}.
Estimates for our observations are given in the last column of \cref{tab:observations}.
If \Rvtwo{} is small compared to the line width, then all the same lines of both components will be blended.
This is indeed the case for {HD~4747}, {HD~211847}, and {HD~202206} with expected \(|\rvtwo{}| < 2\)\kmps{}.
This may have contributed to the lack of recovery with both components of the binary model trying to fit to the same features.
This may even cause some correlation between the parameters of the two components.
The {RV} separation of the two components changes with orbital phase.
Having multiple spectra of the same target distributed in phase may allow the {RV} of the spectral components to be better recovered~\citep [e.g. ][]{czekala_disentangling_2017, sablowski_spectral_2016}.


\subsubsection {Wavelength range}
\label{subsubsec:wavelenght_range_limitation}
The wavelength choice for the spectra analysed here, observed with the intention to apply the spectral differential technique, was selected due to the location of the K-band telluric absorption window.
This wavelength range, with a narrow wavelength range \(\sim50\)\nm{} set by the CRIRES instrument.
This wavelength range is likely not the best choice for the proposed study.

Changing the wavelength coverage to regions with lines sensitive to stellar parameters for both stars and {BD}s, as well as using a larger wavelength range that will be achieved by CRIRES+, may help to improve the recovery results of the companion recovery technique presented here.
We note that if the wavelength range is increased by taking separate observations at different wavelengths, not covered by a single exposure, then changes in the {RV} of both components between the different wavelength observations may need to be accounted for.


\subsubsection{The {BT-Settl} models}
\label{subsubsec:bt-settl}
We note that the {PHOENIX-ACES} models are not the only spectral libraries available with the other notable library considered for this work is the {BT-Settl} models, ~\citep{allard_model_2010,allard_btsettl_2013,baraffe_new_2015}.
The included modelling of dust and cloud formation, as well as hydro-dynamical modelling atmospheric mixing/settling for atmospheres with \Teff{} below \(\sim2600\)\K{}, make the {BT-Settl} models valid across the regime from stars to {BD}s as cool as 400\K{}.
As the {BT-Settl} models are suitable to model the atmospheres of the brown dwarfs they would be useful for the companion recovery technique developed here.
However, as shown in \cref{subsection:results-hd211847, subsection:injection-recovery}, we were unable to successfully recover the 155\Mjup{} (\Teff{}\(\sim3200\)\K{}) low mass star companion of {HD~211847} and derived a temperature upper limit for our methodology of around 3800\K{}.
These are both well above the 2300\K{} cutoff of the {PHOENIX-ACES} models and for the onset of dust- and cloud-formation phenomena, at 2600\K{}.

\cref{fig:hd211847-models} shows again the minimum \textchisquared{} solution for detector 1 of the second {HD~211847} observation, this time including the {BT-Settl} solution with the same parameters.
Although the {PHOENIX-ACES} and {BT-Settl} models differ slightly they both have a large spectral mismatch to the observations.
As such, we did not use the {BT-Settl} models for the simulation and results above as we did not see any special advantage in using them.

The ease of access to find, download, and use {PHOENIX-ACES} spectral library, available in the fits file format, compared to older {BT-Settl} libraries is another reason for the current favoured use of the {PHOENIX-ACES} library.

Although the newer generations synthetic spectral models are improving and match the overall spectral energy distribution reasonably well there are still regions in the \emph{H}- and \emph{K}-band where there is room for improvement~\citet{rajpurohit_spectral_2016}.
The spectral mismatch in the region studied here is still too large for spectral recovery of companion brown dwarfs.
In the \nir{} we have compounding problem: the model input physics of sub-stellar temperatures and chemistry combined with the general difficulty of the \nir{}.

\begin{figure}
    \centering
    %\includegraphics[width=\hsize]{images/final/HD211847_ACES_BTSettl.pdf}
    \caption{Detector 1 spectrum for {HD~211847} (blue) alongside the {PHOENIX-ACES} (orange dash-dot) and {BT-Settl} (green dashed) synthetic spectra for the host star only, with parameters \Teff{}=5700\K{}, \logg{}=4.5 and \feh{}=0.0.
        Both synthetic models have been normalized and convolved to \(\rm R=50\,000\).
        There is a 0.05 off-set between each spectrum}
    \label{fig:hd211847-models}
\end{figure}


\subsubsection{Impact of \logg{}}
\label{subsubsec:logg}
Logg, a measure of surface gravity, is related to evolutionary state and the size of the star with smaller \logg{} values usually indicating larger radii stars.
This parameter has a large impact on the radius and flux ratio of the binary models.
In the {PHOENIX-ACES} models a decrease in \logg{} from 5.0 to 4.5 increases the models effective radius by \(\sim\)1.75 in the temperature range investigated here.
This change in radius alone roughly triples (\(1.75^2\)) the absolute flux of the synthetic spectrum, neglecting any changes to the shape of the actual spectrum.
Therefore, there are large jumps in the model flux ratios if the \logg{} is allowed to vary, with lower \logg{} values for the companion being favoured as the increased flux ratio decreases the mismatch of the host component to the observations.
This large impact of \logg{} on the spectral library absolute flux is one reason for keeping the \logg{} of each component fixed in the \textchisquared{} results presented in \cref{sec:results}.

\subsubsection{Interpolation}
\label{subsubsec:interpolation}
It is common to interpolate between the synthetic spectral grids to fit and derive parameters in between the grid points~\citep[e.g.][]{nemravova_xtauri_2016, passegger_fundamental_2016}.
Instead of interpolation~\cite{czekala_constructing_2015} use a spectral emulator to use Principal Component Analysis to create eigenspectra for the synthetic library and Gaussian processes to derive a probability distribution function of possible interpolation spectra to account for uncertainties in the interpolation required for high signal-to-noise spectra.

However, we did not incorporate any interpolation into the companion recovery at this stage.
This could be something to be added in the future to refine the recovered parameters, and to help the transition between the grid \logg{} values.
Codes are readily available to perform spectral interpolation which could be utilized for this, two of them are \emph{pyterpol}\footnote{https://github.com/chrysante87/pyterpol}~\citet{nemravova_xtauri_2016} and \emph{Starfish}\footnote{https://github.com/iancze/Starfish}~\cite{czekala_constructing_2015}.


\subsection{Future implementation}
\label{subsec:future}
\subsubsection{High resolution instrumentation}
\label{subsubsec:highres}
The future of high resolution near- and mid- IR spectrographs is looking bright, with many new ground- and space-based instruments currently being developed.
Notable examples include CARMENES (550--1710\nm{},  R=82\,000) which is now operational~\citep{quirrenbach_carmenes_2014}, while SPIRou (980--2350\nm{},  R=73\,500)~\cite{artigau_spirou_2014} and NIRPS (970--1810\nm{}, R=100\,000)~\cite{bouchy_nearinfrared_2017} are still being assembled and installed.
The eagerly awaited {JWST} \textbf{cite} will also be launched soon\footnote{Recently pushed to around May 2020} providing observations in both the \nir{} (600--5300\nm{}, R=2700) and mid-IR (4900--28\,800\nm{}, R$\sim$1550--3250) regions without contamination from our atmosphere.

The upgrade of CRIRES to CRIRES+~\citep{dorn_crires_2016} will increase the wavelength coverage of a single shot capture by at least a factor of 3--5.
This larger wavelength span would be extremely beneficial for the \textchisquared{} performance of the spectral recovery method, increasing the number of useful lines and spectral features to be fitted with the models.

On the modelling side, there are continual improvements in atmospheric modelling and their associated synthetic spectral libraries: as seen with the evolution of the {BT-Settl} models~\cite{allard_btsettl_2013}.
With additional physics and improved line lists and solar abundances~\citep [e.g.][]{asplund_chemical_2009,caffau_solar_2011}, the synthetic libraries are reaching a better agreement with \nir{} observations.
An improved agreement between the \nir{} observations and synthetic spectra will be crucial to improve the performance of the spectral recovery technique presented here.

Although not successful with the CRIRES data used here, the instrumental stage is set to attempt these techniques presented here using the next-generation of high resolution spectrographs.
The lessons learned in this analysis need to be taken into account in order to achieve the best chance of a successful detection.

\subsubsection{Differential scheduling challenges}
\label{subsubsec:differential scedualing}
This work has revealed that more care needs to be taken in planning the observations for the spectral differential analysis of faint companions in the future.
Paying attention in particular to the {FWHM} of the lines in the region (governed by resolution and wavelength); the estimated companion \(\Delta RV\); the previous observations from different observing periods; and keeping consistent detector settings.

The original goal for the observations was to obtain two different and ``clearly separated radial-velocities'' for the secondary companion.
However, the program was assigned a low-priority (C, in {ESO} grading) and, possibly due to operational reasons, the original time requirements necessary to secure well separated {RV}s for the companion spectra could not be met.
This meant that all observations were insufficiently separated to extract a differential spectra for the companion.

The long orbital periods of these targets is also a contributing factor to the insufficient separations.
Most of the targets observed here have orbital periods much longer than an observing semester (183 days).
An optimal pair of observations (achieved at the extrema) would need to have been obtained from separate observing periods (between 2 months and 19 years apart).
In some cases, even observations taken at the beginning and end of the semester would not be sufficient to achieve companion separation (depending on the phase).
Requiring separate observing periods to even achieve the minimum \(\Delta \rm RV\) larger than the line {FWHM}.
At the time it was impossible to ask for time over several semesters in a regular proposal.

Our study demonstrates the importance of proposals for projects that need to be extended over several semesters or years.
In the {ESO} context, this corresponds to ``Monitoring proposals''~\citep[e.g.][pg. 18]{eso_eso_2017}.
Observations of the targets explored here, with long orbital periods in particular, would benefit from the ability for multi-period proposals and newer scheduling systems which allow for tighter scheduling constraints, such as a companion {RV} separation.

For future observations we suggest that the known orbital solution of the companion be used to estimate the companions' {RV} curve during the observing period, with the companion \(M_2\sin{i}\) providing an {RV} upper-limit.
Knowing the instrumental wavelength and resolution, a constraint can then be set to avoid taking observations when the companion spectra are insufficiently separated, or \(\Delta RV\) < {FWHM}.
This constraint can be set using the absolute and relative \emph{time-link} constraints available in {ESO}'s Phase 2 Proposal Preparation (P2PP) tool.
Additionally, analysing the known orbital solution before-hand, to determine {RV} constraints will also help identify the best time to observe, if observations from separate periods will be required or, if an optimally separated companion differential is even feasible.

\subsection{Other techniques}
We note that there are many other disentangling techniques to separate mixed spectra of binary systems,~\citep[e.g.][]{hadrava_disentangling_2009}.
These require more than two observations, with  \(n+1\) observations used to set up a system of linear equations to solve for \(n\) spectral components~\citep[e.g.][]{simon_disentangling_1994,czekala_disentangling_2017, sablowski_spectral_2016}.
These methods are ideal for many well spaced observations.
For example the ideal situation for the {SVD} method of~\citet{sablowski_spectral_2016} is homogeneous samples of at least half the period, to identify the moving spectral components.
The few and insufficiently separated observations we analyse here are not suitable to apply any of these advanced techniques and are beyond what we have attempted here.


\section{Conclusions}
\label{sec:conclusions}
This work aimed at pushing down the detection limit of faint companions, using high resolution near-infrared spectra.
Two different methods were explored with many limitations uncovered.

The objective of the observations acquired in this program was to a apply the differential technique.
For the differential technique the observations need to be sufficiently separated, such that the {RV} of the companion is greater than the {FWHM} to avoid spectral cancellation of the companion.
Unfortunately, due to operational reasons, this condition was not met.
As such we employed an alternative method, that we termed the spectral recovery; this method is in principle fully equivalent, but ended up revealing different difficulties.

For the spectral recovery the host-companion {RV} separation should also ideally be greater then the {FWHM} to avoid blended lines.
The spectral mismatch between models and reality in the \nir{} negatively affects the performance of the synthetic recovery technique on the observed spectra.
With all these effects we are unsuccessful in the detection of the \nir{} spectra of {BD} companions,  with the mass upper-limits set at \(\rm 600~M_{Jup}\) from the synthetic recovery technique.

This work highlights many of the difficulties when dealing with the spectral recovery of \nir{} spectra.
The obstacles to overcome are the data reduction of \nir{} CMOS detectors, that are not yet at the level of visible CCDs, along with a precise telluric correction and wavelength calibration (two interrelated aspects, as thoroughly discussed).
Another important aspect is the mismatch between \nir{} high-resolution spectra and the observed spectra.
In spite of the continuous effort of the modelling community, our work, along with several cited contemporary ones, shows that this mismatch is still one of the main factors preventing us from perform spectral recovery in the \nir{}.
This work highlights that this is a compound problem for Brown Dwarfs, for which the spectral models are worse informed due to lack of observations at high-resolution.

Other than the improvement of the spectral models, the observing community can increase their odds of success by paying attention to the scheduling of observations and the wavelength domains to explore.
Our work shows that observing in the areas of lower telluric absorption, as is frequently done, is not a guarantee of success due to the scarcity of deep lines in cold objects.
Moreover, due to the mismatch between models and observations, the ability to obtain a first spectra before settling on a wavelength range, or changing settings on the fly, is extremely useful for the success of these campaigns.

We hope that this work can act as a guide for the planning of future observations of targets with faint {BD} and planetary companions with the upcoming generation of high resolution spectrographs in the near- and mid- infrared such as CRIRES+ and {JWST} observations.














\citet{hoeijmakers_search_2015}
{Context.
    The spectral signature of an exoplanet can be separated from the spectrum of its host star using high-resolution spectroscopy.
    During these observations, the radial component of the planet's orbital velocity changes, resulting in a significant Doppler shift that allows its spectral features to be extracted.\\
    Aims: In this work, we aim to detect\ce{TiO} in the optical transmission spectrum of {HD~209458}b.
    Gaseous\ce{TiO} has been suggested as the cause of the thermal inversion layer invoked to explain the dayside spectrum of this planet.\\
    Methods: We used archival data from the 8.2 m Subaru telescope taken with the High Dispersion Spectrograph of a transit of {HD~209458}b in 2002.
    We created model transmission spectra that include absorption by\ce{TiO}, and cross-correlated them with the residual spectral data after removal of the dominating stellar absorption features.
    We subsequently co-added the correlation signal in time, taking the change in Doppler shift due to the orbit of the planet into account.\\
    Results: We detect no significant cross-correlation signal due to\ce{TiO}, though artificial injection of our template spectra into the data indicates a sensitivity down to a volume-mixing ratio of \textasciitilde{}10\textsuperscript{-10}.
    However, cross-correlating the template spectra with a {HARPS} spectrum of Barnard's star yields only a weak wavelength-dependent correlation, even though Barnard's star is an M4V dwarf that exhibits clear \ce{TiO} absorption.
    We infer that the\ce{TiO} line list poorly matches the real positions of\ce{TiO} lines at spectral resolutions of \textasciitilde{}100 000.
    Similar line lists are also used in the {PHOENIX} and Kurucz stellar atmosphere suites and we show that their synthetic M-dwarf spectra also correlate poorly with the {HARPS} spectra of Barnard's star and five other M dwarfs.
    We conclude that the lack of an accurate\ce{TiO} line list is currently critically hampering this high-resolution retrieval technique.},




\subsection{Junk from paper about BT-settl}

The {PHOENIX-ACES} models also provide dimensions of effective radius of the star, in the header.
This is required for the method presented here as we need to scale the spectra by their respective surface area when combining together.


The effective radius of the modelled star are provided in the fits header and required for the calculation of the flux ratio.


As we do not recover suitable results for HD\,211847 which is supposed to have a temperature around 3400\K{} we suspect that it is not possible for lower temperature either so don't try to extend to the {BT-Settl} models.


\textbf{
    {PHOENIX-ACES} defines grid as \Teff{}, \logg{} and Mass which can then be used to determine Radii.
    See~\citep{husser_new_2013} section 2.3.1 Mass.}




\subsection{Note about a target - discussion of results}
Another example is {HD~162020}, which has the lowest orbital period (8.4 days) of our targets.
It should have been possible to obtain an optimal pair of observations in one semester, but the second observation was taken immediately following the first.
This means that the \(\rm \delta {RV} = 0.363\kmps{}\)between the two observations, is comparable to the \(\Delta {RV}\)that occurs during each individual observation.
With tighter scheduling restrictions this target could have been observed with the optimal {RV} separation at the extrema of \(\Delta {RV}=2 K_{2}\).
Whether the flux ratio of \(7e^{-6}\) for this target and the interaction of different spectral lines would have made it possible to recover companion mass is a separate issue.


\subsubsection{Wavelength range}
The wavelength choice for the spectra analysed here, observed with the intention to apply the spectral differential technique, was selected due to the location of the \emph{K}-band telluric absorption window.
This wavelength range, with a narrow wavelength range \(\sim50\)\nm{} set by the {CRIRES} instrument.
This wavelength range is likely not the best choice for the proposed study.
\todo{finish this line} may contribute to the poor results from the companion recovery technique.

For instance~\citet{passegger_fundamental_2016} used four different spectral regions for the precise parameter determination of M-dwarfs.
Specific lines from the different wavelength regions are affected differently by the model parameters: \Teff{}, \logg{}, and \feh{}; and are used to break degeneracies in the {PHOENIX-ACES} parameter space.

Changing the wavelength coverage to regions with lines sensitive to stellar parameters for both stars and {BD}s, as well as using a larger wavelength range that will be achieved by {CRIRES+}, may help to improve the recovery results of the companion recovery technique presented here.
We note that if the wavelength range is increased by taking separate observations at different wavelengths, not covered by a single exposure, then changes in the {RV} of both components between the different wavelength observations may need to be accounted for.



\subsection{Comparison to other methods}
kolbl 2014, passenger 2016, 2018

Contrast our result to other works.
More phases, longer integration time, higher {\snr{}}.



kobl 2015 also have difficult separating spectra with a {RV} separation below 10 km/s, or when the spectra of the spectra and companion have small separation.\todo{}


\textbf{
    CHECK out LOCKWOOD 2014 - maximum likelihood with todcor 1e-4 flux ratio double lined spectra}

