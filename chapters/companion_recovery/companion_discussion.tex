%!TEX root = ../thesis.tex



% DISCUSSION



\citet{hoeijmakers_search_2015}
{Context.
    The spectral signature of an exoplanet can be separated from the spectrum of its host star using high-resolution spectroscopy.
    During these observations, the radial component of the planet's orbital velocity changes, resulting in a significant Doppler shift that allows its spectral features to be extracted.\\
    Aims: In this work, we aim to detect\ce{TiO} in the optical transmission spectrum of HD 209458b.
    Gaseous\ce{TiO} has been suggested as the cause of the thermal inversion layer invoked to explain the dayside spectrum of this planet.\\
    Methods: We used archival data from the 8.2 m Subaru telescope taken with the High Dispersion Spectrograph of a transit of HD 209458b in 2002.
    We created model transmission spectra that include absorption by\ce{TiO}, and cross-correlated them with the residual spectral data after removal of the dominating stellar absorption features.
    We subsequently co-added the correlation signal in time, taking the change in Doppler shift due to the orbit of the planet into account.\\
    Results: We detect no significant cross-correlation signal due to\ce{TiO}, though artificial injection of our template spectra into the data indicates a sensitivity down to a volume-mixing ratio of \textasciitilde{}10\textsuperscript{-10}.
    However, cross-correlating the template spectra with a {HARPS} spectrum of Barnard's star yields only a weak wavelength-dependent correlation, even though Barnard's star is an M4V dwarf that exhibits clear \ce{TiO} absorption.
    We infer that the\ce{TiO} line list poorly matches the real positions of\ce{TiO} lines at spectral resolutions of \textasciitilde{}100 000.
    Similar line lists are also used in the {PHOENIX} and Kurucz stellar atmosphere suites and we show that their synthetic M-dwarf spectra also correlate poorly with the {HARPS} spectra of Barnard's star and five other M dwarfs.
    We conclude that the lack of an accurate\ce{TiO} line list is currently critically hampering this high-resolution retrieval technique.},




\subsection{Junk from paper about BT-settl}

The {PHOENIX-ACES} models also provide dimensions of effective radius of the star, in the header.
This is required for the method presented here as we need to scale the spectra by their respective surface area when combining together.


The effective radius of the modelled star are provided in the fits header and required for the calculation of the flux ratio.


As we do not recover suitable results for HD\,211847 which is supposed to have a temperature around 3400\K{} we suspect that it is not possible for lower temperature either so don't try to extend to the {BT-Settl} models.


\textbf{
    {PHOENIX-ACES} defines grid as \Teff{}, \logg{} and Mass which can then be used to determine Radii.
    See~\citep{husser_new_2013} section 2.3.1 Mass.}




\subsection{Note about a target - discussion of results}
Another example is \object{HD 162020}, which has the lowest orbital period (8.4 days) of our targets.
It should have been possible to obtain an optimal pair of observations in one semester, but the second observation was taken immediately following the first.
This means that the \(\rm \delta {RV} = 0.363\kmps{}\)between the two observations, is comparable to the \(\Delta {RV}\)that occurs during each individual observation.
With tighter scheduling restrictions this target could have been observed with the optimal {RV} separation at the extrema of \(\Delta {RV}=2 K_{2}\).
Whether the flux ratio of \(7e^{-6}\) for this target and the interaction of different spectral lines would have made it possible to recover companion mass is a separate issue.


\subsubsection{Wavelength range}
The wavelength choice for the spectra analysed here, observed with the intention to apply the spectral differential technique, was selected due to the location of the \emph{K}-band telluric absorption window.
This wavelength range, with a narrow wavelength range \(\sim50\)\nm{} set by the {CRIRES} instrument.
This wavelength range is likely not the best choice for the proposed study.
\todo{finish this line} may contribute to the poor results from the companion recovery technique.

For instance~\citet{passegger_fundamental_2016} used four different spectral regions for the precise parameter determination of M-dwarfs.
Specific lines from the different wavelength regions are affected differently by the model parameters: \Teff{}, \logg{}, and \feh{}; and are used to break degeneracies in the {PHOENIX-ACES} parameter space.

Changing the wavelength coverage to regions with lines sensitive to stellar parameters for both stars and BDs, as well as using a larger wavelength range that will be achieved by {CRIRES+}, may help to improve the recovery results of the companion recovery technique presented here.
We note that if the wavelength range is increased by taking separate observations at different wavelengths, not covered by a single exposure, then changes in the {RV} of both components between the different wavelength observations may need to be accounted for.



\subsection{Comparison to other methods}
kolbl 2014, passenger 2016, 2018

Contrast our result to other works.
More phases, longer integration time, higher {\snr{}}.



kobl 2015 also have difficult separating spectra with a {RV} separation below 10 km/s, or when the spectra of the spectra and companion have small separation.\todo{}


\textbf{
    CHECK out LOCKWOOD 2014 - maximum likelihood with todcor 1e-4 flux ratio double lined spectra}

