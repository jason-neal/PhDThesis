%!TEX root = ../../thesis.tex

\section{Detecting atmospheres}
To help characterize an exoplanet, a detection of its atmosphere can provide useful information. After the detection of exoplanets and the measurement of their bulk properties, detecting their atmospheres is the next step. The detection of planetary atmosphere is difficult due to the low planet-to-star flux ratio. This requires high precision instrumentation to detect. For example the planet-to-star flux ratio in the optical is $\approx 10^{-4}$ for a hot Jupiter with a 3 day orbit, in which the main component is reflected star light. In the infrared the thermal emission of the planet dominates and the flux ratio rises to $\approx 10^{-3}$. These flux ratios requires observations with signal-to-noise ratios of $10^4$ and $10^3$ in the optical and infrared respectively to achieve a planetary signal at the same level as the noise level. Only just at the capabilities of the current generation of technology, and with very long observation cost.

Several photometric and high-resolution spectroscopic techniques are showing promising results; detailed in the following sections.


\subsection{Occultation and phase variations}
Secondary transit and phase variations are an extension of the transit method, requiring higher precision to detect the reflection and thermal emission of the exoplanet. The observed light curve is analysed considering it has two components, not only light from the star but also light from the planet, albeit at a much lower flux level.
To help visualize and discuss the components of exoplanet atmospheres \fref{fig:transits_and_occultations} is provided showing a transiting planet in orbit around a star, in which the planet also passes behind the star causing an occultation. The planet is shown at several positions of the orbit indicating the proportion of day side and night side observed. Below the star and planet is a diagram showing the changing flux variation (solid black line) over time, following the orbit. If the orbital alignment is such that the planet will pass behind the star it will cause an occultation of the planet. At this point the only light received is from the star alone, creating a baseline stellar measurement. While during the primary transit there is also a small thermal emission contribution from the night side of the planet, as well as it partially blocking the star.

\begin{figure}
    \centering
    \includegraphics[width=0.6\linewidth]{./figures/introduction/circular_diagram.png}
    \caption{Illustration of the flux contribution from a star and planet in a transiting exoplanet system throughout its orbit. Credit~\citet{winn_transits_2010}.}
    \label{fig:transits_and_occultations}
\end{figure}

Throughout the orbit of the planet there is a variation in the planetary flux due to the alternating day/night side of the planet observed. .
There are multiple components of the planetary flux, reflection and emission, that can be analysed with multi-band phase curves \citep[e.g.][]{knutson_characterizing_2009, esteves_optical_2013}. Optical phase curves will mostly show the reflected light from the day side of the planet, allowing modelling the atmospheric albedo (fraction of light reflected by the atmosphere), and can provide details on the atmospheric scattering~\citep{madhusudhan_analytic_2012} and aerosol composition~\citep{oreshenko_optical_2016} through the optical phase function (day/night fraction). Thermal emission of the planet will provide stronger modulation of infrared phase curves and can provide insights into the atmospheres thermal
structure and heat circulation~\citep{ goodman_thermodynamics_2009, koll_temperature_2016}.

An example of phase variations in the infrared spectra of {WASP-43b} obtained with the Hubble Space telescope is given in \fref{fig:stevensonphasecurve2014} (left). The large amplitude of phase variation between the day and night side indicates that the night side is much cooler and there is an inefficient heat circularity from the day to night side. A planet with an efficient day/night heat distribution mechanism would quickly equalize and have smaller phase variation. One key observable from \fref{stevensonphasecurve2014} is that the peak of the phase variation is offset from the location of the secondary transit. The hottest part of the atmosphere does not correspond to the sub-stellar point i.e., the point of the planet's surface closest to the star. This is also observed in surface temperature mapping of the hot Jupiter HD\,189733b obtained with {Spitzer Space Telescope} \citep{knutson_map_2007} shown in the right of \fref{fig:stevensonphasecurve2014}. Simulations of atmospheric circulation models find that this offset is caused by super-rotating equatorial jets which move the location of hottest point of the planet~\citep[e.g.][and references therein]{heng_atmospheric_2015}. 

\begin{figure}
    \centering
    \includegraphics[width=0.5\linewidth]{figures/introduction/stevenson_phasecurve2014.pdf}
     \includegraphics[width=0.4\linewidth]{figures/introduction/knutson_2007_temperature_map_HD_189733b.pdf}
    \caption{Left: Band integrated phase variation of {WASP-43b} from the HST - \cite{stevenson_thermal_2014}. The primary transit is inset top right.
    The peak of brightness occurs before the secondary transit.
    Right: Global temperature map of the hot Jupiter HD\,189733b obtained with {Spitzer Space Telescope}\citep{knutson_map_2007}. The hottest point is offset from the sub-stellar point with the day side and night side temperatures around 930\K{} and 650\K{} respectively.}
    \label{fig:stevensonphasecurve2014}
\end{figure}

The point of occultation, at which the planet is completely blocked by the star enables a baseline measurements for the star to be obtained without the planet. The depth of the occultation, is a direct measurement of the planet-to-star ratio between the star and the planet a \textbf{{cite secondary eclipse of reflectance, and secondary spectroscopy examples}}. Spectra obtained during the occultation will have no planetary signal and can be used remove the stellar component from spectra obtained at other phases to obtain the planetary spectrum.


The depth of the occultation is a measure of the flux from the day side of the planet which can indicate the atmospheric reflection and thermal emission of the planet's atmosphere.


\todo{get a thermal profile image/phase of hot spot?.}



\subsection{Transmission spectroscopy}
When a transiting planet crosses in-front of the host star it blocks out light from the star. However, a small portion light passes through the atmosphere of the planet as shown in \fref{fig:transmissionspectroscopy}. The light that passes through the exoplanet atmosphere is partially absorbed, and is faintly imprinted with absorption lines.
\begin{figure}
    \centering
    \includegraphics[width=0.7\linewidth]{figures/introduction/transmission_spectroscopy}
    \caption{Diagram of transmission spectroscopy imprinting the atmosphere of the exoplanet. Sourced from \href{http://www.sc.eso.org/~esedagha/research.html}{http://www.sc.eso.org/~esedagha/research.html}}
    \label{fig:transmissionspectroscopy}
\end{figure}

The height of the exoplanet atmosphere at which point all light is scattered or absorbed is dependant on wavelength. This appears as a wavelength dependent planetary radius observed during transit \citep[e.g.][]{knutson_using_2007}.


The transmission spectra observed with space space observatories and ground based high resolution spectrographs have been able to detect several elements and molecules in the atmosphere. For example \ce{Na}~\citep{charbonneau_detection_2001} \ce{H20}~\cite{Tinette 2007, brogi_carbon_2014} \ce{CO}~\citep{snellen_mass_2018}, \ce{CH4}~\citep{Redfield2008, wyttenbach 2015}, \ce{Fe} and \ce{Ti}~\citep{hoeijmakers_atomic_2018}.
The presence of clouds in the atmosphere have also been detected, as they mask the atmospheric constituents as they produce wavelength-independent fluxes ~\citep[e.g.][]{barman 2011, line 2016,sing 2016}.


\subsection{High resolution spectroscopy}
High resolution spectrocopy   
high res from big collecting  telescopes and large spectrographs. ability ot avoid or detect atmosphere absorption

Some key results from high res spectroscopy

Detection of winds

Snellen Brogi, de Kok

CRIRES 

Cross correlation mle \citet{piskorz_evidence_2016}


For non-transiting planets ... piskorz  with combiend stellar spectrum and planetary models.


\citet{rodler_weighing_2012} uses differential of a series of phases to construct a stellar mask for the host to subtract from all measurements. (non-transiting)


If the planetary signal is able to be spatial resolved, then a spectrum of the planet can be obtained without contamination.