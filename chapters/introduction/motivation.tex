%!TEX root = ../../thesis.tex

% Motivation of this work
\section{Motivation}

As shown in this introduction there is a vast field of exoplanet research, aiming to detect planetary atmospheres. The purpose of this thesis was to develop methodologies and tools to detect planetary atmospheres in \nir{} spectra.

For this we began with the nir spectra of stars with brown dwarf companions. These would have larger flu ratios than planets so would be easier to detect but be a steping stone toward exoplanet atmospheres. Detecting the spectra of brown dwarfs companions would also constrain their masses, providing useful information regarding the lack of brown dwarf companions.

\todo{take from proposal}

During the thesis there were a lack of \nir{} instruments to obtain new observations  explore new 
The motivation of this work was to detect the atmosphere of extra solar planets using \nir{} spectra.

This work touches on the RV precision available in the \nir spectra of M-dwarf stars. Following new

Throughout this work an opportunity to analyse the RV precision of spectra become available. Focus was shifted to update the code base to prepare for the up coming CARMENES spectra which was delayed. 