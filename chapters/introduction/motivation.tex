%!TEX root = ../../thesis.tex
% Motivation of this work
\section{Motivation for this thesis}
\label{sec:thesis_motivation}

As shown here there is a vast field of exoplanet research, with one of the current challenges being the detection of planetary atmospheres.
The purpose of this thesis was to develop methodologies and tools to extract the minute signals of planetary spectra from \nir{} spectra.
With access to \nir{} spectra of stars with suspected Brown Dwarf companions\footnote{Only the minimum mass \Mtwosini known} the higher temperature and relatively larger size of BDs compared to giant-planets makes the development of spectral recovery techniques for BD companions a logical step towards the spectroscopic detection of planetary atmospheres.
For one, the spectrum of the BD companion will have a higher flux ratio than an exoplanet, and hence should be easier to detect.
Secondly, being able to recover the spectra of these BD companions would help to constrain their mass and differentiate them from low-mass stars, helping to complete the puzzle regarding BD companions and their formation.

The original plan was to use this as a stepping stone to request observational time and apply the techniques on giant exoplanets in the \nir{} with the state-of-the-art detectors that are \emph{still} in development.
In \cref{cha:concepts} the fundamental concepts of \nir{} spectroscopy are presented, the radial velocity method, and present the synthetic models used in this work.
The process of reducing \nir{} CRIRES spectra, with a comparison between two different reduction software is given in \cref{cha:reduction}, followed by the post reduction calibration and atmosphere correction techniques required.

\cref{cha:direct_recovery} presents spectral disentanglement techniques, focusing on the application of a differential subtraction technique to the \nir{} spectra of BD companions, as well as analysing the phase of the observations obtained.
A second technique is developed in \cref{cha:model_comparison} which attempts to fit the observations with a model comprised of two synthetic spectral components.

Towards a slightly different goal, \cref{cha:nir_content} contains work computing the fundamental RV precision of {M-dwarf} in the \nir{}.
These are the best candidates for detecting small mass planets in their habitable zone, and are a focus for upcoming \nir{} RV spectrographs.
A focus was shifted towards updating the tools to compute the RV precision to prepare for the release of the CARMENES \nir{} spectral library.
