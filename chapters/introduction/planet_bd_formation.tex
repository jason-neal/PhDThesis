%!TEX root = ../../thesis.tex

\section{Distribtion of Exoplanets}

Exoplanetary detections have challenged the theoretical formation models with their variety and distribution of sizes, locations For instance, the discovery of the hot-Jupiter class (large mass planets on close in orbits) challenged the accepted planet formation theories at the time~\citep[.e.g][]{pollack_formation_1996} in which our Solar System was thought to be typical with small rocky planets close to the Sun and large giant planets further away.

The characterization of exoplanets with the detection of exoplanetary atmospheres allows for the constraints of exoplanetary composition and formation mechanisms.

For rocky planets models there are infinite combinations of Mass-radius. 
Models of the mass and radius for Earth like rocky planets for different compositions are given in \fref{fig:mass_radius_relation_composition}. A couple of exoplanets are shown along with the small Solar System planets. {Kepler-10\,b} and {CoRot-7\,b} seem to have similar composition to Earth/Venus of mostly mantle, while {Kepler-10\,c} is less dense with more water content. The composition varies from solid iron core (red) through a combination of silicates (rock forming minerals) out to 100\% water (blue).


For planets with atmospheres, especially gas giants on close in orbits, the mass radius relation is more complicated with inflation due to atmospheric heating.



There are two main formation mechanisms , core accretion and disk instability.
The presence of all the hot Jupiter is due to Migration in which the planets trade angular momentum with the planet forming disk to change position.

As the mass of the companion increases
Brown dwarfs

Several authors have found there are correlations with stellar metallicity, but less so with rocky planets.


- Small densities - mercury like  density~\citet{dittmann_temperate_2017, santerne_earthsized_2018, ment_second_2018} rocky super-earths\todo{move elsewhere}\\


\subsection{Exoplanet distribution} 4 different groups


\begin{figure}
    \centering
    \includegraphics[width=0.\linewidth]{./figures/introduction/exoplanetEU_a_mass.pdf}
    \caption{Distance mass diagram.
        The symbols indicate the location of the solar system planets, $\mercury$-Mercury, $\venus$-Venus, $\earth$-Earth, $\mars$-Mars, $\jupiter$-Jupiter, $\saturn$-Saturn, $\uranus$-Uranus, $\neptune$-Neptune.
        Data from \href{http://ww.exoplanet.eu}{exoplanet.eu} October 2018}
    \label{fig:pltoverlayadd}
\end{figure}


Explore what these method have found with exoplanet populations.


\begin{figure}[t]
    \centering
    \includegraphics[width=0.4\linewidth]{./figures/introduction/Mass_radius_relation-compostion_Brugger_2017.pdf}\\
    \caption{Mass-Radius relationship for (super) Earth-like planets with composition contours.
        Adapted from~\citet{brugger_constraints_2017}}
    \label{fig:mass_radius_relation_composition}
\end{figure}


The mass radius relationship for giant planets and beyond is more complicated. For one -heating...

The mass radius diagram from small mass planets and moon out to low-mass stars from ~\citet{chen_probabilistic_2016} is given in \fref{fig:mass_radius_relation}.
There are 4 different regoins in which the slow of the M-R diagram is different, potential indicating different formation features with the locations of shle indicating differnt transistion boundraries. \textbf{reread paper} 

\begin{figure}[t]
    \centering
    \includegraphics[width=0.9\linewidth]{./figures/introduction/mass_radius_relation.pdf}  \\
    \caption{Left: Mass-Radius relationship from planets to stars~\citet{chen_probabilistic_2016}.}
    \label{fig:mass_radius_relation}
\end{figure}


\citep{santos_observational_2017} Santos et al 2017 \todo{read and quote}  Observational evidence for two distinct giant planet populations


double peak histogram from an {RV} paper?? Faria 2018?


did they form from the molecular cloud when the star was forming or from the remnants of the disk after the star formed like exoplanets....?


Theory of migration in the disk through  angular moment transfer (cite).




\subsection{Brown Dwarfs}
Brown dwarfs (BDs) are sub-stellar objects unable to achieve hydrogen fusion, with masses around \(13-80~\textrm{M}_{jup} \)~\citep{chabrier_theory_2000}, bridging the gap between low-mass stars and giant planets.
Without sustained fusion, brown-dwarfs cool down over time with an age-dependent cooling rate.
Therefore, there is an inherent degeneracy between the mass, age and luminosity of a given BD~\citep{burrows_nongray_1997}.
This degeneracy may be resolved by the observation of several parameters, for instance when a BD is in a binary system with a main sequence host star, using both the host stars age and the masses derived from the dynamical motion.

A paucity of BD companions exists in short period orbits around Sun-like stars (\(\lesssim5 \)\AU), compared to stellar or planetary companions, termed the \emph{brown dwarf desert}~\citep{halbwachs_exploring_2000, zucker_analysis_2001, sahlmann_search_2011}.
As the number of known BDs orbiting solar type stars is low, the characterization of benchmark BDs in the brown dwarf desert~\citep[e.g.][]{crepp_trends_2016} is beneficial in understanding this sub-stellar population and to help constrain formation and evolution theories~\citep{whitworth_formation_2007}.
The BD desert also provides a greater challenge as it reduces the amount of good BD candidates to study.

BDs in binary systems, unlike free-floating BDs, allow for the determination of their masses, when complemented with radial velocity ({RV}) and astrometry measurements.
The {RV} technique provides the mass lower-limit (\mtwosini{}) of binary and planetary companions, while complementary astrometry measurements can often provide mass upper-limits~\citep[e.g.][]{sahlmann_search_2011}.
Measuring or tightening the constraints of BD masses improves the understanding of mass dependence on BD formation processes.
For instance, there is growing evidence that the larger giant planets and BD companions do not follow the well known metallicity-giant planet correlation seen in main-sequence stars with planets~\citep[e.g.][]{santos_spectroscopic_2004,santos_observational_2017, maldonado_searching_2017}.
Photometry along with stellar evolution models~\citep[e.g.][]{baraffe_evolutionary_2003,allard_btsettl_2013} can also be used to estimate the mass of BD companions~\citep[e.g.][]{moutou_eccentricity_2017} if there is sufficient orbital separation, and a precise determination of the age~\citep{soderblom_ages_2010}.

Recently, there has been a renewed interest in BD candidates triggered by exoplanetary searches.
While several works found similar properties on the two populations, like a similar density~\citep{hatzes_definition_2015}, others found intriguing differences.
One of the most recent is the different host metallicity of the Brown Dwarf and giant planet populations~\citep{santos_observational_2017, schlaufman_evidence_2018}, a very strong hint of different formation mechanisms.

Spectral observations of binary systems contain the spectra of both bodies, in proportion to their flux ratio, and Doppler shifted relative to each other due to their orbital motion.
One technique to recover the spectra of the companion is secondary reconstruction through a differential spectrum~\citep{ferluga_separating_1997}.
Spectra from different phases are shifted to the rest frame of the host star and subtracted to mutually cancel out the spectrum of the host star allowing the faint companion spectra to become visible.
Advances in high-resolution and near-infrared (\nir{}) capabilities should enable this technique to be applied to BDs and planet companions, in which smaller {RV} shifts can be resolved and the contrast ratio of the smaller companion is improved.

Observing in the \nir{} is specifically desirable for the cooler sub-stellar and giant planet companions as their thermal emission is stronger in the infrared compared to the optical.
This improves the contrast ratio between the host star and companion, providing favourable conditions for their detection and spectral separation.
CRIRES, a high resolution \nir{} spectrograph, has made many prominent advances in recent years with the detection of atmospheric constituents, such as \(\textrm{CO} \) and \(\textrm{H}_{2}\textrm{O} \), atmospheric winds and thermal profiles, rotation and orbital motion, for both transiting and non-transiting planets~\citep[e.g.][]{snellen_orbital_2010, brogi_signature_2012, rodler_weighing_2012, dekok_detection_2013, brogi_carbon_2014, snellen_fast_2014, piskorz_evidence_2016, brogi_rotation_2016, birkby_discovery_2017}.


{\rd{} The search and detection of faint secondary spectra is not only relevant to planetary atmospheres.
    \citet{kolbl_detection_2015} developed a method to detect the presence of optical secondary spectra down to a flux ratio of 1\% in the hosts of \emph{Kepler} transit candidates.
    The presence of which can cause ambiguities in the system configuration, and increase the uncertainty of the measured planet radius.
    The characterization of the false positive probability rate for Kepler has been found to be as high as \(\sim\)35\%~\citet{santerne_sophie_2012}.}

In this paper we apply two different techniques on FGK stars with BD companions with the aim to spectroscopically detect their companions.
In \sref{sec:data} we present the observations and reduction process as well as the spectral models used in this work.
In \sref{sec:specdiff} we explain the differential spectral technique and its applicability to these observations while in \sref{sec:results} we apply companion recovery using a \textchisquared{} approach.
In \sref{sec:discussion} we discuss our results and in \sref{sec:conclusions} we present our conclusions.


\textbf{
    brown dwarf dessert explore \citet{ranc_moa2007blg197_2015}}
