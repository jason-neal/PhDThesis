%!TEX root = ../../thesis.tex

\section{Distribtion of Exoplanets}

Exoplanetary detections have challenged the theoretical formation models with their variety and distribution of sizes, locations For instance, the discovery of the hot-Jupiter class (large mass planets on close in orbits) challenged the accepted planet formation theories at the time~\citep[.e.g][]{pollack_formation_1996} in which our Solar System was thought to be typical with small rocky planets close to the Sun and large giant planets further away.

The characterization of exoplanets with the detection of exoplanetary atmospheres allows for the constraints of exoplanetary composition and formation mechanisms.

For rocky planets models there are infinite combinations of Mass-radius. 
Models of the mass and radius for Earth like rocky planets for different compositions are given in \fref{fig:mass_radius_relation_composition}. A couple of exoplanets are shown along with the small Solar System planets. {Kepler-10\,b} and {CoRot-7\,b} seem to have similar composition to Earth/Venus of mostly mantle, while {Kepler-10\,c} is less dense with more water content. The composition varies from solid iron core (red) through a combination of silicates (rock forming minerals) out to 100\% water (blue).


For planets with atmospheres, especially gas giants on close in orbits, the mass radius relation is more complicated with inflation due to atmospheric heating.



There are two main formation mechanisms , core accretion and disk instability.
The presence of all the hot Jupiter is due to Migration in which the planets trade angular momentum with the planet forming disk to change position.

As the mass of the companion increases
Brown dwarfs

Several authors have found there are correlations with stellar metallicity, but less so with rocky planets.


- Small densities - mercury like  density~\citet{dittmann_temperate_2017, santerne_earthsized_2018, ment_second_2018} rocky super-earths\todo{move elsewhere}\\


\subsection{Exoplanet distribution} 4 different groups


\begin{figure}
    \centering
    \includegraphics[width=0.\linewidth]{./figures/introduction/exoplanetEU_a_mass.pdf}
    \caption{Distance mass diagram.
        The symbols indicate the location of the solar system planets, $\mercury$-Mercury, $\venus$-Venus, $\earth$-Earth, $\mars$-Mars, $\jupiter$-Jupiter, $\saturn$-Saturn, $\uranus$-Uranus, $\neptune$-Neptune.
        Data from \href{http://ww.exoplanet.eu}{exoplanet.eu} October 2018}
    \label{fig:pltoverlayadd}
\end{figure}


Explore what these method have found with exoplanet populations.


\begin{figure}[t]
    \centering
    \includegraphics[width=0.4\linewidth]{./figures/introduction/Mass_radius_relation-compostion_Brugger_2017.pdf}\\
    \caption{Mass-Radius relationship for (super) Earth-like planets with composition contours.
        Adapted from~\citet{brugger_constraints_2017}}
    \label{fig:mass_radius_relation_composition}
\end{figure}


The mass radius relationship for giant planets and beyond is more complicated. For one -heating...

The mass radius diagram from small mass planets and moon out to low-mass stars from ~\citet{chen_probabilistic_2016} is given in \fref{fig:mass_radius_relation}.
There are 4 different regoins in which the slow of the M-R diagram is different, potential indicating different formation features with the locations of shle indicating differnt transistion boundraries. \textbf{reread paper} 

\begin{figure}[t]
    \centering
    \includegraphics[width=0.9\linewidth]{./figures/introduction/mass_radius_relation.pdf}  \\
    \caption{Left: Mass-Radius relationship from planets to stars~\citet{chen_probabilistic_2016}.}
    \label{fig:mass_radius_relation}
\end{figure}


\citep{santos_observational_2017} Santos et al 2017 \todo{read and quote}  Observational evidence for two distinct giant planet populations


double peak histogram from an {RV} paper?? Faria 2018?


did they form from the molecular cloud when the star was forming or from the remnants of the disk after the star formed like exoplanets....?


Theory of migration in the disk through  angular moment transfer (cite).

