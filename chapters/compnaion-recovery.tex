%!TEX root = ../thesis.tex
%4-Snellen/Brogi analysis
% ability to detect singal??

\chapter{Chapter4}  % Main chapter title

\label{cha:companion-recovery}

%----------------------------------------------------------------------------------------
%	SECTION 1
%----------------------------------------------------------------------------------------

\section{Main Section 1}



%-----------------------------------
%	SUBSECTION 1
%-----------------------------------
\subsection{Subsection 1}


%-----------------------------------
%	SUBSECTION 2
%-----------------------------------

\subsection{Subsection 2}

%----------------------------------------------------------------------------------------
%	SECTION 2
%----------------------------------------------------------------------------------------

\section{Main Section 2}




\subsubsection {Wavelength range}
The wavelength choice for the spectra analyzed here, observed with the intention to apply the spectral differential technique, was selected due to the location of the K-band telluric absorption window. This wavelength range, with a narrow wavelength range \(\sim50\)~nm set by the CRIRES instrument. This wavelength range is likely not the best choice for the proposed study. may contribute to the poor results from the companion recovery technique. 

For instance \citet{passegger_fundamental_2016} used four different spectral regions for the precise parameter determination of M-dwarfs. Specific lines from the different wavelength regions are affected differently by the model parameters: \(T_{\textrm{eff}}\), logg, and [Fe/H]; and are used to break degeneracies in the PHOENIX-ACES parameter space. 

Changing the wavelength coverage to regions with lines sensitive to stellar parameters for both stars and BDs, as well as using a larger wavelength range that will be achieved by CRIRES+, may help to improve the recovery results of the companion recovery technique presented here. We note that if the wavelength range is increased by taking separate observations at different wavelengths, not covered by a single exposure, then changes in the RV of both components between the different wavelength observations may need to be accounted for. 
