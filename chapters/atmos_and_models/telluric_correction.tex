%!TEX root = ../../thesis.tex


\section{Telluric correction}
\label{sec:telluric_correction}







\subsection{Telluric models}

Utilizing telluric models has been shown to be better than the standard star method.
\subsubsection{TAPAS}
\label{subsubsec:TAPAS}

\subsection{Tapas models}
\todo{ADAPT THis section to explain the models more generally.
    Move the usage back to Reduction section}
\label{subsec:tapas_models}
For the wavelength calibration and telluric correction methods we use telluric line models.
These have been show to provide as good or better telluric correction compared to the telluric standard method \reference{telluric model correction methods original}and~\citep{ulmer-moll_telluric_2018}.

We utilized the {TAPAS} (Transmissions of the AtmosPhere for AStronomical data) web-service\footnote{\href{http://www.pole-ether.fr/tapas/}{http://www.pole-ether.fr/tapas/}}~\citep{bertaux_tapas_2014} to obtain atmospheric transmission models for each observation. {TAPAS} uses the standard line-by-line radiative transfer model code LBLRTM~\citep{clough_linebyline_1995} along with the 2008 {HITRAN} spectroscopic database~\citep{rothman_hitran_2009} and {ARLETTY} atmospheric profiles derived using meteorological measurements from the {ETHER} data center\footnote{\href{http://www.pole-ether.fr}{http://www.pole-ether.fr}} to create telluric line models.

The {ARLETTY} atmospheric profiles have a 6 hour resolution, so there may be a slight difference between the actual profile at the time of observation.

We use the mid-observation time to retrieve transmission models for each observation, with the {ARLETTY} atmospheric profiles\footnote{Nearest of the 6 hourly profiles} and vacuum wavelengths selected.
The telluric models were retrieved without any barycentric correction to keep the telluric lines at a radial velocity of zero with respect to the instrument.

{TAPAS} allows for the choice of atmospheric constituents included in the model spectra.
We obtained one model with all available species present, convolved to a resolution of \(\rm R=50\,000\), and another two models without an instrumental profile convolution applied.
For these two extra models, one contained only the transmission spectra of \ce{H2O} while the other contained all other constituents except \ce{H2O}.
This was to explore a known issue with the depth of \ce{H2O} absorption lines in the {TAPAS}~\citet{bertaux_tapas_2014}. \cref{subsec:telluric_correction}.


\todo{Look at} -> synthesizing telluric spectra \nir{} for {CRIRES}~\cite{seifahrt_synthesising_2010}

Using {TAPAS} is contrasted alongside Molecfit and Telfit in~\cite{ulmer-moll_telluric_2018}.
We conclude that \ldots




\#\#\#\#

\subsubsection{Telluric correction}
\label{subsec:telluric_correction}
The Earth's atmosphere is a spectral filter for all ground-based astronomical observations, imprinting the absorption profile of the atmosphere onto the spectrum observed.
To accurately recover the spectra of the observed target the removal of the absorption lines introduced by Earth's atmosphere is extremely important.
The motion, changing composition and \todo{See solenes paper for matereial here}.
A number of methods are available to correct for the telluric lines, telluric reference (cite), models (tapas), modelling/fitting (molecfit tellfit).
The effectiveness between these three methods has recently been performed in~\cite{ulmer-moll_telluric_2018}. \todo{Expand this section\ldots{}}. \todo{Should this be more in the introduction?}.

In this work we use telluric models without fitting, using the {TAPAS} spectra.

These observations were first taken in an atmospheric window of the \emph{K}-band in order to reduce the absorption introduced by the atmosphere~\citep{barnes_hd_2008}.
\missingfigure{The telluric spectrum around 2\um{} showing the 2.1\um{} window of low telluric absorption} To correct for the remaining telluric line contamination the spectra were divided by the {TAPAS}\citep{bertaux_tapas_2014} atmospheric transmission models for each observation.
Synthetic telluric models were used to avoid the observing overhead necessary to perform telluric standard star exposures~\citep{vacca_method_2003}, and they have been demonstrated to be superior in the quality of the correction relative to the telluric standard approach~\citep[e.g.][]{cotton_atmospheric_2014}.

Before the correction, the depth of the telluric lines were re-scaled to match the airmass of the observation using the relation \(\rm T = T^{\beta}\), where \(\rm T\) is the telluric spectrum and \(\beta\) is the airmass ratio between the observation and model.
This changed the depth of most absorption lines to match the observations, but does not correctly scale the deeper \ce{H2O} lines.
The scaled telluric model is interpolated to the wavelengths of the observed spectrum and then used to correct the observed spectra through division, leaving behind a telluric corrected spectra.
An example of a telluric corrected spectra is shown in the middle panel of \cref{fig:spectral_example}, with the light blue shading indicating where the deeper telluric lines were.

We attempted the technique suggested by~\citet{bertaux_tapas_2014} to address the poor \ce{H2O} airmass scaling, to fit a scaling factor to the  \ce{H2O} absorption lines before convolution to the instrument resolution.
This was achieved by first dividing the spectrum by a telluric model with only non-\ce{H2O} constituents, convolved to the observed resolution, and scaled by the airmass to remove the non-\ce{H2O} lines.
Then a model with only \ce{H2O} lines at full resolution was scaled by a factor \(\textrm{T}^{x}\), convolved to \(\rm R=50\,000\) and compared to the observed spectra.
The factor \(x\) was fitted to find the best scaling factor for the \ce{H2O} lines.

We found that for a few spectra in our sample this method corrected the deeper telluric lines well, but in many cases we found that the fitted scaling factor was affected by the presence of blended stellar lines (attempting to fit those also).
It was also strongly influenced by the deepest  \ce{H2O} telluric lines present.
We find that the telluric correction of the deep \ce{H2O} lines could be improved with this technique, but at the cost of worsening the correction of the many smaller \ce{H2O} lines.
Since the smaller \ce{H2O} lines covered more of the spectrum in this region than the larger lines the separate \ce{H2O} scaling was not continued.
One possible solution for this would be to perform a piece-wise telluric correction, performing this step only for the deeper \ce{H2O} lines, or by using one of the other tools that fits the telluric model to the observations.
This technique could also benefit from a larger wavelength span that would enable blended lines to be ignored while having sufficient deep \ce{H2O} lines to fit the scaling factor correctly.
This small experiment shows that a simple scaling is not enough to correct for the absorption in an effective way, for this case.

\unfinished{Add telluric spectra for \nir{} band? the plot from Molecfit?}

\unfinished{Still uneven line coverage on all detectors in this small range}

\unfinished{\ce{H2O} corrections} example of good and bad fitting\ldots


\subsection{Tapas models}
\todo{ADAPT this section to the usage of the models.}
\label{subsec:tapas_models_usage}
For the wavelength calibration and telluric correction methods we use telluric line models.
These have been show to provide as good or better telluric correction compared to the telluric standard method \reference{telluric model correction methods original}and~\citep{ulmer-moll_telluric_2018}.

We utilized the {TAPAS} (Transmissions of the {AtmosPhere} for {AStronomical} data) web-service\footnote{\href{http://www.pole-ether.fr/tapas/}{http://www.pole-ether.fr/tapas/}}~\citep{bertaux_tapas_2014} to obtain atmospheric transmission models for each observation. {TAPAS} uses the standard line-by-line radiative transfer model code {LBLRTM}~\citep{clough_linebyline_1995} along with the 2008 {HITRAN} spectroscopic database~\citep{rothman_hitran_2009} and {ARLETTY} atmospheric profiles derived using meteorological measurements from the {ETHER} data centre\footnote{\href{http://www.pole-ether.fr}{http://www.pole-ether.fr}} to create telluric line models.

The {ARLETTY} atmospheric profiles have a 6 hour resolution, so there may be a slight difference between the actual profile at the time of observation.

We use the mid-observation time to retrieve transmission models for each observation, with the {ARLETTY} atmospheric profiles\footnote{Nearest of the 6 hourly profiles} and vacuum wavelengths selected.
The telluric models were retrieved without any barycentric correction to keep the telluric lines at a radial velocity of zero with respect to the instrument.

{TAPAS} allows for the choice of atmospheric constituents included in the model spectra.
We obtained one model with all available species present, convolved to a resolution of \(\rm R=50\,000\), and another two models without an instrumental profile convolution applied.
For these two extra models, one contained only the transmission spectra of \ce{H2O} while the other contained all other constituents except \ce{H2O}.
This was to explore a known issue with the depth of \ce{H2O} absorption lines in the {TAPAS}~\citet{bertaux_tapas_2014}. \Cref{subsec:telluric_correction}.


\subsection{Issues with {TAPAS}}
There are a number of issues we encountered when using the {TAPAS} web-service, mainly due to interaction with the website.
Often their service was down for weeks at a time without any warning or notification.
With this you would waste time filling out the web form and attempt to submit it but would receive no response and no email with a link to the data.
There was a higher success rate of successful response between different web-browsers.
A number of bug reports were submitted to the owners of the webpage without any acknowledgement.

The web-page is useful for quickly obtaining a small number of spectra but can be tedious for many.
In our case we were requesting 3 telluric spectra, with varying molecular contributions, for each of our 17 observations.
There is an ability to request multiple spectra at a time but we found this would not function if trying to request more than four spectra at once.

A script\footnote{Available at \href{https://github.com/jason-neal/equanimous-octo-tribble/blob/master/octotribble/Tapas/}{https://github.com/jason-neal/equanimous-octo-tribble/blob/master/octotribble/Tapas/}} was created to automatically generate the data necessary to fill out a {TAPAS} request for each {CRIRES} observations.
The script scanned the {CRIRES} header for information such as the mid-time of observations, target coordinates, slit width (defines {CRIRES}'s instrumental resolution) etc.\ and populated the {XML} request form provided by {TAPAS}.
The script output is copied and pasted into the web-browser for submission.

Trial and error was needed to understand all the {XML} form entries, such as the molecules requested and the atmospheric model to use ({ARLETTY}) and achieve a valid {TAPAS} request.
The real issue was with the {TAPAS} {{ID}} number.
Each {TAPAS} request has an {ID} number (which is provided with the email response).
This {ID} number needs to be correctly set in the {XML} form before submission.
This number increments by 1 with each submission but its initial value is unknown unless you made the last request.
If you submit the {XML} request with the incorrect {ID} number you will get a response with the correct {ID} number, but the failed request.
You increment this {ID} number by 1 and hopefully make a valid request.
Unfortunately if someone else has made a {TAPAS} request after yours then the {ID} number will again be invalid.
It is unknown if multiple transmission spectra could have been requested at the same time with the {XML} form.

There is another issue with a one hour time offset between the requested and the time returned by {TAPAS}.
For instance if the requested time was for an observation at 0200h {UTC} then the transmission spectrum returned by {TAPAS} is for 0100h {UTC}.
This changes the position of the target, the airmass and potentially the {ARLETTY} model used (6 hour time steps), affecting the strength of the telluric lines.
It is tedious to remember to offset your input time by one hour to obtain the correct time, and slightly more work when you also have to adjust the date when going backwards past 0000h.
When submitting the {XML} script the time that is returned is the time requested.
Attempts were made to bring this issue to the attention of the {TAPAS} team in 2016 but as of August 2018 this issue is still present.

These issues need to be considered when requesting {TAPAS} spectra, adding unnecessary difficultly to the relatively simple process.


\subsubsection{Telluric masking}
The telluric spectra from {TAPAS} can not only be used for correcting individual spectra but are also easily used to create a wavelength mask telluric lines.
For instance~\citet{figueira_radial_2016} and~\citet{artigau_optical_2018} use {TAPAS} spectra to mask out atmospheric lines deeper than 2\% for computing the photon noise limited radial velocity precision.
Masking with the {TAPAS} model is similarly performed in \cref{cha:nir_content} when we extend the analysis of \citet{figueira_radial_2016}.
The telluric model used for this is an average of 52 {TAPAS} spectra (one per week in 2014), simulated at La Silla Observatory at an airmass of 1.2 (\(z = 33.5^{o}\)).
This is to incorporate long-term variations of absorption over the year.
Masking is applied by defining a cut-off line depth, typically 2\%, at which to mask out any deeper telluric lines.

