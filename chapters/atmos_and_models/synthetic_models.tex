%!TEX root = ../../thesis.tex

\section{Synthetic Stellar models}:

In the work we make extensive use of the {PHOENIX-ACES} models with some experimentation with the {BT-Settl} models.
A collection of several theoretical stellar spectral libraries can be found at Spanish Virtual Observatory \href{http://svo2.cab.inta-csic.es/theory/newov/index.php}{Theoretical Spectra Web Server}.



\subsection{PHOENIX-ACES}

Available \todo{FINISH}


\subsection{Synthetic {PHOENIX-ACES} models}
\label{subsec:spec_models}
\todo{this section is from \cref{cha:model_comparison}}
We use the {PHOENIX-ACES}~\citep{husser_new_2013} synthetic spectra library as our reference for the spectral comparison.
It uses the most recent version (16) of the {PHOENIX} code and is suitable for the spectra of cool stars.
The full parameter grid space of the {PHOENIX-ACES} spectra is given in \cref{tab:phoenix} although we only explore models constrained by the targets explored here.

%!TEX root = ../thesis.tex

\begin{table}
    \centering 
    \caption{Full parameter space of the PHOENIX-ACES spectral grid.}
    \begin{tabular}{lr@{ -- }lc}    % Seperate columns with --
        \toprule
        & \multicolumn{2}{c}{Range}       & Step size\\
        %\midrule
        \midrule
        \ \(T_{\textrm{eff}}\) [K] &  2300 & 7000  & 100 \\
        &  7000 & 12000 & 200 \\ 
        \  logg     &  0.0 & +6.0   & 0.5 \\ 
        \ [Fe/H]   &  -4.0 & $-$2.0  & 1.0 \\    % Strange spacing of [ ] in table so added \ to all rows
        &  -2.0 & +1.0  & 0.5 \\  
        \  \(\alpha\)/Fe &  -0.2 & +1.2  & 0.2 \\
        \bottomrule
    \end{tabular}
    \label{tab:phoenix}
\end{table}

The spectral model libraries were accessed using the useful ``grid tools'' interface provided in the \emph{Starfish}\footnote{\url{https://github.com/iancze/Starfish}} Python package~\citep{czekala_constructing_2015}, which made it efficient to load in the spectra when needed.

We multiply the synthetic spectra by the wavelength to convert it into photon counts, ignoring multiplicative constants, as done in~\citet{figueira_radial_2016}\footnote{Synthetic models provide the spectral energy distribution (\(\rm erg\,s^{-1}\,cm^{2}\,cm^{-1}\)).}.
The spectra were convolved with a Gaussian kernel to match the resolution of the observations (\(\rm R=50\,000\)).
Due to the distributive property of convolution it is efficient to apply it once to each spectra first, before the spectral pairs are combined.

The {PHOENIX-ACES} models include dust in equilibrium with the gas phase while ignoring dust opacity and does not include any mixing/settling which is important for cooler {BD} atmospheres.
They set a minimum library \(\teff{}=2\,300\)\K{} to avoid the temperatures at which the modelling of clouds is necessary.
This unfortunately limits the use of this library for this technique to the largest mass companions in our sample.
For example a \(\teff{}=2\,300\)\K{} corresponds to a {BD} with \(\textrm{M}\sim84\)\Mjup{} at 5\Gyr{} from the~\citet{baraffe_evolutionary_2003} evolutionary models.

There are other models that extend below 2\,300\K{} such as the {BT-Settl} models~\citep{allard_btsettl_2013, baraffe_new_2015}.
These are discussed in \cref{subsec:btsettl}.




\subsection{BT-Settl}
\label{subsec:btsettl}
Available \todo{FINISH}

Harder to work with.

Other models which have not been used here.
Kruz models a directory of different model can be found \ldots{}?

BT-Settl have less but are suitable for BDs.
We did not extend below the {PHOENIX-ACES} lower temperature of 2300\K{}


The newest btsettl models (cfist 2015) are combined with evolutionary models~\citep{barrafe_evolutionary_2015}