%!TEX root = ../thesis.tex
%2 - Tools for spectral analysis

\chapter{Advanced Concepts}
\label{cha:concepts}
To put things too detailed for the introduction?


Radial Velocity:
Picture of orbital motion.
Concepts of RV motion-
Masses,
Orbits, (period, mass, distance impact.)
Equation
What the Equations mean.

\subsection{RV calculation}
\unfinished{Move location / adjust from paper}
To determine the RV offset to apply to the simulated spectra above we calculated the RV of the companion using the published orbital parameters of the host, transforming the RV semi-amplitude of the host \(K_{1}\) into the companion semi-amplitude \(K_{2}\) using the mass ratio,
\begin{equation}
\label{eqn:mass_ratio}
q = \textrm{M}_{2} / \textrm{M}_{1} = \textrm{K}_{1} / \textrm{K}_{2}
\end{equation}

We note that for the targets in which only the minimum mass (\(\textrm{M}\sin{i}\)) is known, this equation will indicate the maximum \(K_2\) value for the companion. The estimated \(K_2\) value for each companion is given in Table~\ref{tab:flux_table} while the estimated RV for each observation is calculated in Table~\ref{tab:observations}.

If a direct measurement of the companion RV or \(\Delta RV\) from the differential spectra analysis was achieved, then this relation would have been used to calculate the companion mass.

The error on estimated RV values is calculated using the general error propagation formula~\citep{ku_notes_1966} with the RV equation and the errors on the orbital parameters. For a function, \(f\), with errors on the inputs \(\delta x\), \(\delta y\) etc, it follows:
\begin{align}
f &= f(x, y, z, \ldots)\\
\delta f &= \sqrt{{\left(\frac{\partial f}{\partial x} \delta x\right)}^2 +  {\left(\frac{\partial f}{\partial y} \delta y\right)}^2 + {\left(\frac{\partial f}{\partial z} \delta z\right)}^2 + \ldots}
\end{align}


\section{Synthetic models}:

\subsection{PHEONIX-ACES}

Available . . . \todo{FINISH}

\subsection{BT-SETTL}

Available .. . \todo{FINISH}

Harder to work with .

Other models which have not been used here. Kruz models a directory of different model can be found ...?


\subsection{Telluric models}

Utilizing telluric models has been shown to be better than the standard star method.
\subsubsection{TAPAS}

\subsection{Tapas models}
\todo{ADAPT THis section to explain the models more generally. Move the usage back to Reduction section}
\label{subsec:tapas_models}
For the wavelength calibration and telluric correction methods we use telluric line models. These have been show to provide as good or better telluric correction compared to the telluric standard method \reference{telluric model correction methods original}and~\citep{ulmer-moll_telluric_2018}.

We utilized the TAPAS (Transmissions of the AtmosPhere for AStronomical data) web-service\footnote{\url{http://www.pole-ether.fr/tapas/}}~\citep{bertaux_tapas_2014} to obtain atmospheric transmission models for each observation. TAPAS uses the standard line-by-line radiative transfer model code LBLRTM~\citep{clough_linebyline_1995} along with the 2008 HITRAN spectroscopic database~\citep{rothman_hitran_2009} and Arletty atmospheric profiles derived using meteorological measurements from the ETHER data center\footnote{\url{http://www.pole-ether.fr}} to create telluric line models.

The Arletty atmospheric profiles have a 6 hour resolution, so there may be a slight difference between the actual profile at the time of observation.

We use the mid-observation time to retrieve transmission models for each observation, with the Arletty atmospheric profiles\footnote{Nearest of the 6 hourly profiles} and vacuum wavelengths selected. The telluric models were retrieved without any barycentric correction to keep the telluric lines at a radial velocity of zero with respect to the instrument.

TAPAS allows for the choice of atmospheric constituents included in the model spectra. We obtained one model with all available species present, convolved to a resolution of \(\rm R=50\,000\), and another two models without an instrumental profile convolution applied. For these two extra models, one contained only the transmission spectra of \(\rm H_{2}O\)while the other contained all other constituents except \(\rm H_{2}O\). This was to explore a known issue with the depth of \(\rm H_{2}O\)absorption lines in the TAPAS~\citet{bertaux_tapas_2014}. \sref{subsec:telluric_correction}.


\todo{Look at} -> synthesizing telluric spectra \nir{} for CRIRES~\cite{seifahrt_synthesising_2010}

Using TAPAS is contrasted alongside Molecfit and Telfit in~\cite{ulmer-moll_telluric_2018}. We conclude that \ldots



\subsubsection{Vacuum wavelengths}

Astronomical light is refracted as it enters Earths atmosphere. The index of refraction, n, is the factor at which the velocity, \(v\), and wavelength \(\lambda\) of electromagnetic radiation changes from their values in a vacuum.
\[ \lambda = \lambda_{vac} / n, \vspace{1em} v = v_{vac}/n\]

The index of refraction of air is very close to 1, but has a complex wavelength dependence which also depends on atmospheric composition (e.g. \(\rm CO_2, H_2O\)), temperature and pressure. 
There are several empirical formula derived to convert between air and vacuum wavelengths \citep[e.g.][]{Edlen 1953, Peck and Reeder 1972, Ciddor 1996} however their validity in the near-infrared extends only to 1.7\si{\micro\meter} \citep{Ciddor1996}. Beyond 1.7\si{\micro\meter} there are theoretical models of the refractive index of air from for example 1.3 to 24 \si{\micro\meter}~\citet{mathar_refractive_2007}.  Laboratory measurements are performed at an atmospheric standard, for instance  Standard air: dry air at 15 \(^o\)C, 101.325 kPa and with 450 ppm \(CO_2\) as done in \citep{Ciddor1996} .

In this work we extensively use spectral wavelengths specified in a vacuum. The first is because the empirical relations are not valid beyond 1.7\ \si{\micro\meter}, while the CRIRES spectra we investigate are observed at 2.1\si{\micro\meter} and the CRIRES instrument manual  provides all measurements in vacuum wavelengths.

The TAPAS synthetic transmission models  \citet{bertaux_tapas_2014} we use for telluric correction
The PHOENIX-ACES library of synthetic we use provides spectra in vacuum wavelengths by default.and the

The synthetic telluric models can provide 

Their are two different 3 different wavelength wavelength regimes common for spectra. As the wave number $1/\lambda$, vacuum or air. It is possible to download the telluric spectra in any of these three units. but we just use the vacuum wavelengths and calibrate against those.

For this we remain in vacuum wavelengths to not incur any conversion errors when transforming to air wavelengths.

correction comes from refractive index of air which changes with pressure and temperature of air.




Citations:

Edlen 1953, J. Opt. Soc. Am 43 no. 5\\
Peck and Reeder 1972, J. Opt. Soc. 62 no. 8\\
Ciddor 1996, Applied Optics 35 no. 9
Morten 1991



There ar but no extensive experimental formulation above 1.7 \si{\micro\meter} \citet{ciddor}.

In this work we only deal with vacuum wavelengths. For one we the wavelengths specified in the CRIRES manual are all provided in vacuum wavelengths.
