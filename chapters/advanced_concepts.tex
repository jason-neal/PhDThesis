%!TEX root = ../thesis.tex
% 2 - Advanced Concepts

\chapter{Advanced Concepts}
\label{cha:concepts}
To put things too detailed for the introduction?


Radial Velocity:
Picture of orbital motion.
Concepts of {RV} motion-
Masses,
Orbits, (period, mass, distance impact.)
Equation
What the Equations mean.

\subsection{{RV} calculation}
\unfinished{Move location / adjust from paper}
We use the Keplerian orbit {RV} equation to estimate the {RV} of the host and companions at the time of each observation, \(t\):
\begin{equation}
\label{eq:rv_equation}
{RV} = K [\cos{(\nu(t) + \omega)} + e\cos{(\omega)}]
\end{equation}
Here, \emph{K} is the \emph{semi-major amplitude}, \(\nu\) is the \emph{true anomaly}, \(e\) is the orbital \emph{eccentricity}, and \(omega\) is the \emph{argument of periastron}. The true anomaly is not only a function time, \(t\), but also the orbital period \(P\),  the \emph{time of periastron passage}, \(T_0\), and eccentricity. The literature parameters for each target are provided in Table~\ref{tab:orbitparams}.
To determine the {RV} of the companion we transformed the {RV} semi-amplitude of the host \(K_{1} \) into the semi-amplitude for the companion \(K_{2} \) using the mass ratio,
\begin{equation}
\label{eqn:mass_ratio}
q = \textrm{M}_{2} / \textrm{M}_{1} = \textrm{K}_{1} / \textrm{K}_{2}
\end{equation}

We note that for the targets in which only the minimum mass (\(\textrm{M}\sin{i} \)) is known, this equation will indicate the maximum \(K_2\) value for the companion. The estimated \(K_2\) for each companion is provided in Table~\ref{tab:estimatedparameters} while the {RV} for both components at the time of each observation is provided in Table~\ref{tab:observations}.

The error on estimated {RV} values, shown in Fig.~\ref{fig:HD211847_result_contours} is calculated by applying the general error propagation formula~\citep{ku_notes_1966} and using the  errors on the orbital parameters. For a function, \(f\), with errors on the inputs \(\delta x\), \(\delta y\) etc., it follows:
\begin{align}
f &= f(x, y, z, \ldots)\\
\delta f &= \sqrt{{\left( \frac{\partial f}{\partial x} \delta x\right)}^2 + {\left(\frac{\partial f}{\partial y} \delta y\right)}^2 + {\left(\frac{\partial f}{\partial z} \delta z\right)}^2 + \ldots}
\end{align}




\section{JUNK}


% from draft of paper 18/7/2017
\subsection{Companion K}  Maybe something can go above from this ...
\label{sec:companion_RV}
\emph{These sections might be unnecessary}\\

To calculate the {RV} of the companion at the time of each observation. For a two-body system the {RV} semi-amplitude of the companion \(\textrm{K}_{2}\) can be determined from the orbital host-companion mass ratio $q = \textrm{M}_{1}/\textrm{M}_{2} = \textrm{K}_{2}/\textrm{K}_{1}$.
Note, that for the targets in which only the minimum mass (msini) is known, this will give the maximum {RV} semi-amplitude of the companion.
This relation was used to estimate \(\textrm{K}_2\) for the companion from the minimum mass or mass we have for each companion. These values along with the other orbital parameters of the system were use to calculate the {RV} of the companions for each observation. These values are provided with the observations in \tref{tab:observations}.


\section{Synthetic models}:

In the work we make extensive use of the {PHOENIX-ACES} models with some experimentation with the {BT-Settl} models.
A collection of several theoretical stellar spectral libraries can be found at Spanish Virtual Observatory \href{http://svo2.cab.inta-csic.es/theory/newov/index.php}{Theoretical Spectra Web Server}.

\subsection{PHOENIX-ACES}

Available \todo{FINISH}


\subsection{BT-Settl}

Available  \todo{FINISH}

Harder to work with.

Other models which have not been used here. Kruz models a directory of different model can be found \ldots{}?

\subsection{Telluric models}

Utilizing telluric models has been shown to be better than the standard star method.
\subsubsection{TAPAS}

\subsection{Tapas models}
\todo{ADAPT THis section to explain the models more generally. Move the usage back to Reduction section}
\label{subsec:tapas_models}
For the wavelength calibration and telluric correction methods we use telluric line models. These have been show to provide as good or better telluric correction compared to the telluric standard method \reference{telluric model correction methods original}and~\citep{ulmer-moll_telluric_2018}.

We utilized the {TAPAS} (Transmissions of the AtmosPhere for AStronomical data) web-service\footnote{\url{http://www.pole-ether.fr/tapas/}}~\citep{bertaux_tapas_2014} to obtain atmospheric transmission models for each observation. {TAPAS} uses the standard line-by-line radiative transfer model code LBLRTM~\citep{clough_linebyline_1995} along with the 2008 {HITRAN} spectroscopic database~\citep{rothman_hitran_2009} and {ARLETTY} atmospheric profiles derived using meteorological measurements from the {ETHER} data center\footnote{\url{http://www.pole-ether.fr}} to create telluric line models.

The {ARLETTY} atmospheric profiles have a 6 hour resolution, so there may be a slight difference between the actual profile at the time of observation.

We use the mid-observation time to retrieve transmission models for each observation, with the {ARLETTY} atmospheric profiles\footnote{Nearest of the 6 hourly profiles} and vacuum wavelengths selected. The telluric models were retrieved without any barycentric correction to keep the telluric lines at a radial velocity of zero with respect to the instrument.

{TAPAS} allows for the choice of atmospheric constituents included in the model spectra. We obtained one model with all available species present, convolved to a resolution of \(\rm R=50\,000\), and another two models without an instrumental profile convolution applied. For these two extra models, one contained only the transmission spectra of \ce{H2O}while the other contained all other constituents except \ce{H2O}. This was to explore a known issue with the depth of \ce{H2O}absorption lines in the TAPAS~\citet{bertaux_tapas_2014}. \sref{subsec:telluric_correction}.


\todo{Look at} -> synthesizing telluric spectra \nir{} for {CRIRES}~\cite{seifahrt_synthesising_2010}

Using {TAPAS} is contrasted alongside Molecfit and Telfit in~\cite{ulmer-moll_telluric_2018}. We conclude that \ldots



\subsubsection{Vacuum wavelengths}
\todo{a good reference that mentions the uv and optical uses could be good?}
Astronomical light is refracted as it enters Earths atmosphere, as it changes medium from the vacuum of space and to air. The index of refraction, \(n\), is the factor at which the velocity, \(v\), and wavelength \(\lambda\) of electromagnetic radiation changes from their values in a vacuum.
\begin{equation}
n  = \frac{\lambda_{vac}}{\lambda_{air}}, \frac{v_{vac}}{v_{air}}
\end{equation}

The index of refraction of air is very close to 1, but has a complex wavelength dependence which also depends on atmospheric composition (e.g., \ce{CO2}, \ce{H2O}), temperature and pressure.
There are several empirical formula for the index of refraction of air \citep[e.g.,][]{edlen_dispersion_1953, peck_dispersion_1972, ciddor_refractive_1996}. As an example the refractive index given by \citet{ciddor_refractive_1996}, based on optical and infrared measurements and valid up to 1.7\si{\micro\meter}, is:

\begin{equation}
    n -1 = \frac{0.05792105}{238.0185 - \lambda^{-2}} + \frac{0.00167917}{57.362 - \lambda^{-2}}.
\end{equation}

This is measured at standard air, dry air at 15\si{\degreeCelsius}, 101.325\ \si{\kilo\pascal} and with 450\,ppm \ce{CO2}.
Beyond 1.7\ \si{\micro\meter} there are only theoretical models of the refractive index of air, due to lack of experimental data. For example \citet{mathar_refractive_2007} provides theoretical values between 1.3 to 24\ \si{\micro\meter} for common air pressure and temperature of ground-based observatories.

In this work we extensively use spectral wavelengths specified in a vacuum. The first is because the empirical relations are not valid beyond 1.7\ \si{\micro\meter}, while the {CRIRES} spectra we investigate are observed around 2.1\ \si{\micro\meter} and the {CRIRES} instrument manual provides all measurements in vacuum wavelengths.

The {TAPAS} synthetic transmission models~\citet{bertaux_tapas_2014} we use for telluric correction are available for download in vacuum wavelengths or air wavelengths\footnote{The wavenumber, \(1/\lambda\), is also available.}. The {PHOENIX-ACES} library of synthetic spectra we use also provides spectra in vacuum wavelengths.
Since both are available in vacuum wavelengths and the {TAPAS} models are used for calibration, vacuum wavelength exclusively, With this there is no errors introduced from conversions between air and vacuum wavelengths.
