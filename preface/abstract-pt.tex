
\begin{abstract-pt}
    Com o vasto n\'{u}mero de exoplanetas descobertos nas \'{u}ltimas d\'{e}cadas, este campo avan\c{c}ou na detec\c{c}\~{a}o das suas atmosferas, em particular no infravermelho pr\'{o}ximo (\nir{}), onde a razão entre o fluxo do planeta e da estrela \'{e} favor\'{a}vel \`{a} detec\c{c}\~{a}o da emiss\~{a}o exoplanet\'{a}ria.
    O conte\'{u}do deste trabalho foca-se na utiliza\c{}\~{a}o de espectroscopia de alta resolu\c{c}\~{a}o no infravermelho pr\'{o}ximo para separar os espectros combinados de estrelas {FGK} com possíveis an\~{a}s castanhas companheiras.
    Este trabalho tem dois objectivos principais. O primeiro consiste em desenvolver t\'{e}cnicas de separa\c{c}\~{a}o espectral no infravernelho pr\'{o}ximo para companheiros de maior massa, com a inten\c{c}\~{a}o de avan\c{c}ar para atmosferas planet\'{a}rias. O segundo consiste em restringir a massa das an\~{a}s castanhas companheiras detectadas.
    Duas t\'{e}cnicas diferentes s\~{a}o exploradas para analisar observa\c{c}\~{o}es obtidas com o espectro\'{o}grafo {CRIRES}. A primeira consiste num m\'{e}todo de subtra\c{c}\~{a}o diferencial entre duas observa\c{c}\~{o}es independentes. A segunda num ajuste \textchisquared{} das observa\c{c}\~{o}es a um modelo bin\'{a}rio composto por espectros sint\'{e}ticos.
    Nenhuma das t\'{e}cnicas propostas teve sucesso na recuperara\c{c}\~{a}o de qualquer tipo de informa\c{c}\~{ã}o \'{u}til sobre as an\~{a}s castanhas companheiras. Esta falta de resultados deve-se principalmente a quest\~{o}es observacionais, \`{a} diminuta raz\~{a}o entre o fluxo da estrela e companheira e discrep\^{a}ncias relativamente aos modelos sint\'{e}ticos.
    O f\'{o}co do trabalho foi ent\~{a}o desviado para o estudo da precis\~{a}o na recuperação da velocidade radial de estrelas an\~{a}s do tipo M a partir dos seus espectros no infravermelho pr\'{o}ximo, um dos objectivos dos novos instrumentos planeados para observar nessa banda do espectro electromagn\'{e}tico.
    Com esse objectivo em vista, foram melhoradas as aplica\c{c}\~{o}es desenvolvidas para estimar a precis\~{a}o na velocidade radial obtida pelos calculadores de tempo de exposi\c{c}\~{a}o de dois novos \nir{} espectr\'{o}grafos: {NIRPS} e {SPIRou}.
    Finalmente, uma an\'{a}lise preliminar \'{e} realizada sobre a precis\~{a}o da velocidade radial obtida a partir do espectr\'{o}grafo {CARMENES}, comparando modelos sint\'{e}ticos com os espectros reais de estrelas an\~{a}s do tipo M.

    No Cap\'{\i}tulo 1 apresentamos e introduzimos a detec\c{c}\~{a}o de exoplanetas e suas atmosferas, juntamente com algumas propriedades de exoplanetas.
    Nos Cap\'{\i}tulos 2 e 3 fornecemos descri\c{c}\~{o}es adicionais da detec\c{c}\~{a}o de exoplaneta atrav\'{e}s do m\'{e}todo de velocidade radial e conceitos b\'{a}sicos de espectroscopia no infravermelho pr\'{o}ximo.
    No Cap\'{\i}tulo 4 apresentamos o impacto da atmosfera da Terra nas observa\c{c}\~{o}es e nos modelos usados para corrigi-la, assim como apresentamos em detalhe as bibliotecas estelares sint\'{e}ticas usadas.
    As etapas de redu\c{c}\~{a}o de dados aplicadas aos espectros de infravermelho pr\'{o}ximo s\~{a}o descritas no Cap\'{\i}tulo 5, seguidas pelas etapas de calibra\c{c}\~{a}o de comprimento de onda p\'{o}s-redu\c{c}\~{a}o e corre\c{c}\~{a}o tel\'{u}rica.
    A t\'{e}cnica de subtra\c{c}\~{a}o diferencial \'{e} apresentada em Cap\'{\i}tulo 6, identificando a separa\c{c}\~{a}o insuficiente entre observa\c{c}\~{o}es.
    No Cap\'{\i}tulo 7 apresentamos o m\'{e}todo \textchisquared{} com modelos sint\'{e}ticos bin\'{a}rios seguidos de uma discuss\~{a}o sobre os resultados observados.
    Finalmente o conte\'{u}do de informa\c{c}\~{a}o presente no infravermelho pr\'{o}ximo e a precis\~{a}o na mediç\~{a} da velocidade radial a aprtir de espectros no infravermelho pr\'{o}ximo de estrelas an\~{a}s do tipo M com o {CARMENES} s\~{a}o investigados no Cap\'{\i}tulo 8, assim com o
    c\'{a}lculo da precis\~{a}o te\'{o}rica dos espectros estelares para a \'{u}ltima gera\c{c}\~{a}o de espectr\'{o}grafos de infravermelho pr\'{o}ximo para detectar planetas em redor de estrelas an\~{a}s do tipo M.
\end{abstract-pt}

