%!TEX root = ../thesis.tex
\begin{abstract-pt}
    Com um vasto n\'{u}mero de exoplanetas descobertos nas \'{u}ltimas d\'{e}cadas, o campo avan\c{c}ou na detec\c{c}\~{a}o de suas atmosferas, particularmente no infravermelho pr\'{o}ximo (\nir{}), onde as raz\~{o}es de fluxo s\~{a}o favor\'{a}veis para a detec\c{c}\~{a}o de emiss\~{o}es exoplanet\'{a}rias.
    O conte\'{u}do deste trabalho concentra-se especificamente em espectros \nir{} de alta resolu\c{c}\~{a}o com o objetivo de separar os espectros misturados de estrelas {FGK} com suspeitos companheiros de An\~{a}o Castanho.
    Isto tem dois prop\'{o}sitos principais, desenvolver \nir{} t\'{e}cnicas de separa\c{c}\~{a}o espectral em companheiros maiores com a inten\c{c}\~{a}o de avan\c{c}ar para atmosferas planet\'{a}rias, e para restringir a massa dos companheiros an\~{o}es marrons no processo.
    Duas t\'{e}cnicas diferentes s\~{a}o exploradas para analisar as observa\c{c}\~{o}es {CRIRES} dispon\'{\i}veis: um m\'{e}todo de subtra\c{c}\~{a}o diferencial entre duas observa\c{c}\~{o}es separadas, e um ajuste \textchisquared{} das observa\c{c}\~{o}es para um modelo bin\'{a}rio composto de espectros sint\'{e}ticos.
    Ambas as t\'{e}cnicas n\~{a}o tiveram sucesso em recuperar informa\c{c}\~{o}es \'{u}teis sobre os companheiros Brown Dwarf, principalmente devido a quest\~{o}es observacionais, as pequenas raz\~{o}es de fluxo dos companheiros e discrep\^{a}ncias para modelos sint\'{e}ticos.
    Permanecendo no \nir{}, o esfor\c{c}o foi desviado para ampliar o entendimento da precis\~{a}o da velocidade radial dos espectros {M-dwarf}, um foco de nova instrumenta\c{c}\~{a}o \nir{}.
    O software para calcular a precis\~{a}o da velocidade radial \'{e} melhorado para aplicar as estimativas de precis\~{a}o de velocidade radial para as calculadoras de tempo de exposi\c{c}\~{a}o de dois novos \nir{} espectr\'{o}grafos, {NIRPS} e {SPIRou}.
    Finalmente, uma an\'{a}lise preliminar \'{e} realizada sobre a precis\~{a}o da velocidade radial obtida a partir do espectr\'{o}grafo {CARMENES}, comparando modelos sint\'{e}ticos com os espectros {M-dwarf} observados.

     Cap\'{\i}tulo 1 apresenta e introduz a detec\c{c}\~{a}o de exoplanetas e suas atmosferas, juntamente com algumas propriedades exoplanetas.
     Cap\'{\i}tulos 2 e 3 fornece descri\c{c}\~{o}es adicionais de detec\c{c}\~{a}o de exoplaneta atrav\'{e}s do m\'{e}todo de velocidade radial e conceitos b\'{a}sicos de \nir{} espectroscopia respectivamente.
     Cap\'{\i}tulo 4 apresenta o efeito da atmosfera da Terra nas observa\c{c}\~{o}es e nos modelos usados para corrigi-la, bem como detalha as bibliotecas estelares sint\'{e}ticas usadas.
     As etapas de redu\c{c}\~{a}o de dados aplicadas aos espectros de \nir{} s\~{a}o dadas em Cap\'{\i}tulo 5 seguidas pelas etapas de calibra\c{c}\~{a}o de comprimento de onda p\'{o}s-redu\c{c}\~{a}o e corre\c{c}\~{a}o tel\'{u}rica.
     A t\'{e}cnica de subtra\c{c}\~{a}o diferencial \'{e} apresentada em Cap\'{\i}tulo 6, identificando a separa\c{c}\~{a}o insuficiente entre observa\c{c}\~{o}es.
     Cap\'{\i}tulo 7 apresenta o m\'{e}todo \textchisquared{} com modelos sint\'{e}ticos bin\'{a}rios seguidos de uma discuss\~{a}o sobre os resultados observados.
     Finalmente o \nir{} conte\'{u}do de informa\c{c}\~{a}o e a precis\~{a}o da velocidade radial dos espectros {M-dwarf} {CARMENES} s\~{a}o investigados em Cap\'{\i}tulo 8.
     Calculando a precis\~{a}o te\'{o}rica dos espectros estelares para a \'{u}ltima gera\c{c}\~{a}o de espectr\'{o}grafos de infravermelho pr\'{o}ximo para detectar planetas ao redor de estrelas {M-dwarf}.
\end{abstract-pt}
