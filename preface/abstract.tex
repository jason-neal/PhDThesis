%!TEX root = ../thesis.tex

\begin{abstract}
    \label{preface:abstract}
    With a vast number of exoplanets discovered in the last decades, the field has pushed towards detecting their atmospheres, particularly in the near-infrared (\nir{}) where flux ratios are favourable for detecting exoplanetary emission.
    The contents of this work focuses specifically on high resolution \nir{} spectra with the aim to separate the blended spectra of {FGK} stars with suspected Brown Dwarf companions.
    This has two main purposes, to develop \nir{} spectral separation techniques on larger companions with the intent to move towards planetary atmospheres, and to constrain the mass of the Brown Dwarf companions in the process.
    Two different techniques are explored to analyse the available CRIRES observations: a differential subtraction method between two separate observations, and a \textchisquared{} fitting of the observations to a binary model comprised of synthetic spectra.
    Both techniques were unsuccessful in recovering useful information about the Brown Dwarf companions mainly due to observational issues, the small flux ratios of the companions and discrepancies to synthetic models.

    Remaining in the \nir{}, effort was diverted to extended the understanding of radial velocity precision of {M-dwarf} spectra, a focus of new \nir{} instrumentation.
    Software to calculate the radial velocity precision is improved to apply to provide the radial velocity precision estimates for the exposure time calculators of two new \nir{} spectrographs, {NIRPS} and {SPIRou}.
    Finally a preliminary analysis is performed on the radial velocity precision attainable from the {CARMENES} spectrograph, comparing synthetic models to observed {M-dwarf} spectra.

    \cref{cha:introduction} presents and introduction to detecting exoplanet and their atmospheres along with some exoplanet properties.
    \cref{,cha:rv_concepts,cha:nir_spectroscopy} give further descriptions of detecting exoplanet via the radial velocity method and basic concepts of \nir{} spectroscopy respectively.
    \cref{cha:atmospheres_and_models} presents the affect of Earth's atmosphere on observations and the models used to correct it, as well as detailing the synthetic stellar libraries used.
    The data reduction steps applied to \nir{} spectra are given in \cref{cha:reduction} followed by the post reduction wavelength calibration and telluric correction steps.
    The differential subtraction technique is presented in \cref{cha:direct_recovery}, identifying the insufficient separation between observations.
    \cref{cha:model_comparison} presents the \textchisquared{} method with binary synthetic models followed by a discussion on the results observed.
    Finally the \nir{} information content and radial velocity precision of {M-dwarf} {CARMENES} spectra are investigated in \cref{cha:nir_content}.
    Calculating the theoretical precision of stellar spectra for the latest generation of near-infrared spectrographs to detect planets around {M-dwarf} stars.

\end{abstract}
