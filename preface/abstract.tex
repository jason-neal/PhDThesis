%!TEX root = ../thesis.tex
\begin{abstract}
    The contents of this work focus on detecting exoplanets in the nIR.
    
    It starts of with reduction of nir spectra from CRIRES. 
    
    In chapters 4 we  approach a differential subtraction technique to separate the spectra from the faint companion. Contrasting the result to other recent detections. 
   In chapter 4 a second technique is developed in which the BD companions are attempted to be recovered from the $\nearrow$-infrared spectra by fitting to a complex model of two synthetic stellar spectra. Finding the technique unsuitable for the data on hand. 
   
   Focusing on BD as they should have a stronger flux ratio than planets. 
    
    In chapter 6 we change focus towards RV precision of M-dwarf starts. Calculating th theoretical precision of stellar spectra t. Useful for the next-generation  of near-infrared spectrographs to detect planets around M-dwarf stars. 
    
   Some thought on future are also provided.
    
\end{abstract}
