%!TEX root = ../thesis.tex
\section{Attended Conferences and Schools}\label{appsec:conferences}

\Other[Talk] {Conference}
{Towards Other Earths II:\@The star-planet connection}% Title
% Type: Seminar/Talk/Poster
{Porto, Portugal}% Location
{2014, Sept.\,15--19}% Date
{http://www.astro.up.pt/investigacao/conferencias/toe2014/}% Website
{0}% Points
{The study of extrasolar planets is one of the most active areas of research of modern astronomy.
The number of discoveries attests for the importance of a topic that reaches out and captivates the imagination of scientists and public alike.
This conference aims at reviewing the state of the art of star-planet connection, with some focus on the detection and characterization of Earth like planets orbiting other stars.
We propose to debate how the field of extrasolar planets will evolve in respect to this and how it will face the challenges of the upcoming years.}% Description

%%%%%%%%%%%%%%%%%%%%%%%%%%%%%%%%%%%%%%%%%%%%%%%%%%%%%%%%%%%%%%%%%%%%%%%%%%%%%%%%%%%%%%%%%%%%%%
\Other[Talk] {Conference}{I Vietri Advanced School on Exoplanetary Science}% Title
% Type: Seminar/Talk/Poster
{Vietri sul Mare (Salerno), Italy}% Location
{2015, May\,25--29}% Date
{http://www.iiassvietri.it/en/ases2015.html}% Website
{0}% Points
{The School was aimed to provide a comprehensive, state-of-the-art picture of a variety of relevant aspects of the fast-developing, highly interdisciplinary field of extrasolar planets research.
The Lecture topics of the School were focused on exoplanet detection with the Radial Velocity, Photometric Transits, Gravitational Microlensing, and Direct Imaging techniques.
The Lectures were be delivered by four senior researchers to an audience of graduate students, Ph.D students and young post-docs.}% Description

%%%%%%%%%%%%%%%%%%%%%%%%%%%%%%%%%%%%%%%%%%%%%%%%%%%%%%%%%%%%%%%%%%%%%%%%%%%%%%%%%%%%%%%%%%%%%%
\Other[Talk] {Conference}{IVth Azores International Advanced School in Space Sciences}% Title
% Type: Seminar/Talk/Poster
{Horta, Faial, Azores Islands, Portugal}% Location
{2016, July 17--27}% Date
{http://www.iastro.pt/research/conferences/faial2016/}% Website
{0}% Points
{This International Advanced School addresses the topics at the forefront of scientific research being conducted in the fields of stellar physics and exoplanetary science.

The School covers two scientific topics that share many synergies and resources: Asteroseismology and Exoplanets.
Therefore, the program aims at building opportunities for cooperation and sharing of methods that will benefit both communities.
This cooperation has experienced great success in the context of past space missions such as CoRoT and Kepler.
Upcoming photometry and astrometry from space, as well as complementary data from ground-based networks, will continue to foster this cooperation.
Observations of bright stars and clusters in the ecliptic plane are being made by the re-purposed K2 mission, and NASA's TESS and ESA's CHEOPS missions will soon start obtaining similar data over the entire sky.
ESA's PLATO mission will then build upon these successes by providing photometric light curves on a wealth of stars.
Ground-based spectroscopy from the Stellar Observations Network Group (SONG) will complement the satellite data for the brightest stars in the sky, as will also be the case with the new generation of high-precision spectrographs being developed for the ESO, like the Echelle SPectrograph for Rocky Exoplanets and Stable Spectroscopic Observations (ESPRESSO).}% Description

%%%%%%%%%%%%%%%%%%%%%%%%%%%%%%%%%%%%%%%%%%%%%%%%%%%%%%%%%%%%%%%%%%%%%%%%%%%%%%%%%%%%%%%%%%%%%%
\Other[Talk] {Conference} {XXVI Encontro Nacional de Astronomia e Astrof\'{\i}sica}% Title
% Type: Seminar/Talk/Poster
{Aveiro, Portugal}% Location
{2016, Sept.\,08--09}% Date
{http://gravitation.web.ua.pt/enaa2016/index2a62.html?q=node/3}% Website
{0}% Points
{The main goal of this national meeting is to present a general overview of the international state of the art research done in Astronomy and Astrophysics.
    ENAA also allows for researchers to get together and discuss scientific ideas and policies, strengthen the collaboration between national researchers and integrate young researchers in the community through the presentation of their work.}


% Need two spaces at end of communication files
