%!TEX root = ../thesis.tex
\section{Talks and Seminars}\label{appsec:talks}

%%%%%%%%%%%%%%%%%%%%%%%%%%%%%%%%%%%%%%%%%%%%%%%%%%%%%%%%%%%%%%%%%%%%%%%%%%%%%%%%%%%%%%%%%%%%%%
\Communication{Talk}
{Towards exoplanetary atmospheres: new data reduction techniques for the \nir{}.} % Title
{2016, Sept.\,09}% Date
{XXVI Encontro Nacional de Astronomia e Astrof\'{\i}sica (ENAA), Aveiro, Portugal}% Location
{http://gravitation.web.ua.pt/enaa2016/index2a62.html?q=node/3}% Website
{0}% Points
{Exoplanetary atmospheres are one of the forefronts of exoplanet science.
The measurement of exoplanet atmospheres can help break degeneracies in the mass-radius relationship for solid exoplanets allowing proper derivation of the planets properties, and may provide important clues to the origin and evolution of planets.
The technological breakthrough of high-resolution spectrographs ($\Delta \lambda / \lambda = 100,000$) allow for the separation and wavelength tracking of individual molecular spectral features.
This allows for the separation of the stationary telluric absorption lines from the moving planet-star spectral lines.
We focus our initial investigation on extracting the atmospheric signal of brown dwarfs, which are interesting objects between giant planets and small stars.
Brown dwarf targets are good candidates to begin our investigation as their larger mass induces larger radial velocity shifts to the atmospheric lines and their larger radius (and surface area) produce a stronger signal compared to smaller planets.
In this talk I will discuss the ongoing work of my PhD towards extracting the signal of brown dwarf atmospheres from high-resolution \nir{}spectra.
I will discuss the methodology and techniques developed towards extracting the spectral lines of the atmospheres and show some of our current results.}% Description


% Need two spaces at end of communication files
