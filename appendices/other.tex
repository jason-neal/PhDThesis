%!TEX root = ../thesis.tex
\section{Other}\label{appsec:other}

\Other[Talk] {Teaching}
{Monitor at the 11\textsuperscript{th}, 12\textsuperscript{th}, 13\textsuperscript{th} Physics Summer School}
{Faculty of Science, University of Porto, Portugal}% Location
{2015--2017}% Date
{http://e-fisica.fc.up.pt/edicoes/}% Website
{}% Points
{Instructing and monitoring 4--5 secondary school students though activities related to planet detection and characterization.
One week per year.\\
  \emph{Na sua Edi\c{c}\~{a}o a Escola desafia os participantes com o programa mais radical de sempre com projectos envolvendo lasers, supercondutores, nanotecnologias, f\'{\i}sica espacial, fibras \'{o}pticas, biosensores e astrof\'{\i}sica.
      Projectos que abrangem as tr\^{e}s grandes áreas de investiga\c{c}\~{a}o do DFA, F\'{\i}sica, Astronomia e Engenharia F\'{\i}sica.
      Assim, se tens apet\^{e}ncia pelo conhecimento e pelo processo de investiga\c{c}\~{a}o que permite inquirir a Natureza e testar hip\'{o}teses acerca das causas das coisas, ent\~{a}o deixa-te arrastar pelo encantamento da experimenta\c{c}\~{a}o e do entendimento das ideias mais profundas e abrangentes.
      O DFA garante uma viagem através do Universo, do microcosmo ao macrocosmo, inserindo-te num projecto que te dilatar\'{a} a imagina\c{c}\~{a}o.
Tens coragem? Aparece!}}% Description


\Other[Other] {Observing}
{HARPS-N@Telescopio Nazionale Galileo observing run}
{La Palma, Spain}% Location
{2017 Feb.\,6--8}% Date
{http://www.tng.iac.es/instruments/harps/}% Website
{}% Points
{Observing runs are an essential part of the training of any astronomer.
    During my PhD I had the opportunity to observe for 3 nights on {HARPS-N}.
    These were for the \emph{TROY} project, with which some IA-Porto researchers were collaborating.
    These observations were to fill in the phase curve of previously detected low-mass exoplanets to further constrain their masses.
    These observations were used in~\citet{lillo-box_troy_2018}.
}% Description


\Other[Other] {Observing}
{ESPRESSO@VLT}
{Paranal, Chile} % Location
{2019 Jan.\,4--10} % Date
{https://www.eso.org/sci/facilities/paranal/instruments/espresso.html} % Website
{} % Points
{An observing run with {ESPRESSO} at Paranal was completed between 4--10 January 2019.
This observed the transit of WASP-121 as part of the {ESPRESSO} {GTO} program in P102 as well as observing several transit follow up targets and targets for Radial Velocity blind searches.}
