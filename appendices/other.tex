%!TEX root = ../thesis.tex
\section{Other}	\label{app_sec:other}

\Other[Talk]	{Teaching}
{Monitor at the 11\textsuperscript{th}Physics Summer School}
{Faculty of Science, University of Porto, Portugal}% Location
{2015 Sept.}% Date
{http://e-fisica.fc.up.pt/edicoes/11a-edicao}% Website
{}% Points
{Na sua 11a Edição a Escola desafia os participantes com o programa mais radical de sempre com projectos envolvendo lasers, supercondutores, nanotecnologias, física espacial, fibras ópticas, biosensores e astrofísica. Projectos que abrangem as três grandes áreas de investigação do DFA, Física, Astronomia e Engenharia Física. Assim, se tens apetência pelo conhecimento e pelo processo de investigação que permite inquirir a Natureza e testar hipóteses acerca das causas das coisas, então deixa-te arrastar pelo encantamento da experimentação e do entendimento das ideias mais profundas e abrangentes. O DFA garante uma viagem através do Universo, do microcosmo ao macrocosmo, inserindo-te num projecto que te dilatará a imaginação. Tens coragem? Aparece!}% Description


\Other[Talk]	{Teaching}
{Monitor at the 12\textsuperscript{th}Physics Summer School}
{Faculty of Science, University of Porto, Portugal}% Location
{2016 Sept.}% Date
{http://e-fisica.fc.up.pt/edicoes/12a-edicao}% Website
{}% Points
{Na sua 12a Edição a Escola desafia os participantes com o programa mais radical de sempre com projectos envolvendo lasers, supercondutores, nanotecnologias, física espacial, fibras ópticas, biosensores e astrofísica. Projectos que abrangem as três grandes áreas de investigação do DFA, Física, Astronomia e Engenharia Física. Assim, se tens apetência pelo conhecimento e pelo processo de investigação que permite inquirir a Natureza e testar hipóteses acerca das causas das coisas, então deixa-te arrastar pelo encantamento da experimentação e do entendimento das ideias mais profundas e abrangentes. O DFA garante uma viagem através do Universo, do microcosmo ao macrocosmo, inserindo-te num projecto que te dilatará a imaginação. Tens coragem? Aparece!}% Description


\Other[Talk]	{Teaching}
{Monitor at the 13\textsuperscript{th}Physics Summer School}
{Faculty of Science, University of Porto, Portugal}% Location
{2017 Sept.}% Date
{http://e-fisica.fc.up.pt/edicoes/13a-edicao}% Website
{}% Points
{Na sua 13a Edição a Escola desafia os participantes com o programa mais radical de sempre com projectos envolvendo lasers, supercondutores, nanotecnologias, física espacial, fibras ópticas, biosensores e astrofísica. Projectos que abrangem as três grandes áreas de investigação do DFA, Física, Astronomia e Engenharia Física. Assim, se tens apetência pelo conhecimento e pelo processo de investigação que permite inquirir a Natureza e testar hipóteses acerca das causas das coisas, então deixa-te arrastar pelo encantamento da experimentação e do entendimento das ideias mais profundas e abrangentes. O DFA garante uma viagem através do Universo, do microcosmo ao macrocosmo, inserindo-te num projecto que te dilatará a imaginação. Tens coragem? Aparece!}% Description


\Other[Other]	{Observing}
{HARPS-N@Telescopio Nazionale Galileo observing run}
{La Palma, Spain}% Location
{2017 Feb. 6--8}% Date
{N/A}% Website
{}% Points
{Observing runs are an essential part of the training of any astronomer.
    During my PhD I had the opportunity to observe for 3 nights on HARPS-N. These were for the \emph{TROY} project, which some IA-Porto researchers were collaborating.  These observations were to fill in the phase curve of previously detected low-mass exoplanets to further constrain their masses. These observations were used in a recent publication~\citet{lillo-box_troy_2018}.

    {\red{} I hope to preform observations in Chile in early 2019 also.}
}% Description
