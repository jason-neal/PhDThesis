%!TEX root = ../thesis.tex

\chapter{Vacuum wavelengths}
Astronomical light is refracted as it enters Earth's atmosphere, as the medium changes from the vacuum of space and to air.
The index of refraction, \(n\), is the factor at which the velocity, \(v\), and wavelength \(\lambda\) of electromagnetic radiation changes from their values in a vacuum.
\begin{equation}
n  = \frac{\lambda_{vac}}{\lambda_{air}}, \frac{v_{vac}}{v_{air}}
\end{equation}

The index of refraction of air is very close to 1, but has a complex wavelength dependence which also depends on atmospheric composition (e.g.\ \ce{CO2}, \ce{H2O}), temperature and pressure.
There are several empirical formula for the index of refraction of air~\citep[e.g.][]{edlen_dispersion_1953, peck_dispersion_1972, ciddor_refractive_1996}.
As an example the refractive index given by~\citet{ciddor_refractive_1996}, based on optical and infrared measurements and valid up to 1.7\um{}, is:

\begin{equation}
    n -1 = \frac{0.05792105}{238.0185 - \lambda^{-2}} + \frac{0.00167917}{57.362 - \lambda^{-2}}.
\end{equation}

This is measured at standard, dry air at 15\si{\degreeCelsius}, 101.325\,\si{\kilo\pascal} and with 450\,ppm \ce{CO2}.
Beyond 1.7\um{} there are only theoretical models of the refractive index of air, due to lack of experimental data.
For example~\citet{mathar_refractive_2007} provides theoretical values between 1.3--24\um{} for common air pressure and temperature of ground-based observatories.

In this work we extensively use spectral wavelengths specified in a vacuum.
The first is because the empirical relations are not valid beyond 1.7\um{}, while the {CRIRES} spectra investigated are observed around 2.1\um{} and the {CRIRES} instrument manual provides all measurements in vacuum wavelengths.

The {TAPAS} synthetic transmission models~\citet{bertaux_tapas_2014} used for telluric correction are available for download in vacuum wavelengths or air wavelengths\footnote{The wavenumber, \(1/\lambda\), is also available.}.
The {PHOENIX-ACES} library of synthetic spectra used also provide spectra in vacuum wavelengths.
Remaining exclusively in vacuum wavelengths means that no errors are introduced from the conversion between air an vacuum wavelengths in the \nir{}.
As such, the vacuum {TAPAS} models are obtained and the observations calibrated using vacuum wavelengths.
