%!TEX root = ../../thesis.tex
% Appendix Template

\chapter{RV Precision Tables} % Main appendix title

\label{app:nir_prec_amendment}

The updated relative RV precision results attainable from \nir{} spectra are presented in the following tables. \tref{tab:rv_aces_btsettl} shows the precision results for the same M-dwarfs analysed in \citep{figueira_radial_2016}.
That is stellar temperatures 3900, 3500, 2800, 2600K corresponding to spectral types M0, M3, M6, M9 respectively. \logg{}=4.5 and \feh{}=0.0. The rotation applied are $\vsini{}$=1, 5, 10\kmps{} and instrumental profiles with R = 60k, 80k, 100k.

Columns 2-4 contain the RV precision calculated using {PHOENIX-ACES} spectra, as done in \citet{figueira_radial_2016}. These values differ from two effects. There is small difference in Conditions 1 and 3 from the change in numerical differentiation implemented (see \sref{sec:numerical_gradient}). The values for Condition 2 however, are completely different due to the implementation error in the telluric masking discovered (see \sref{subsec:condition_two_bug}). 

In columns 5-7 are the same RV precision calculation but using the {BT-Settl} spectral library instead (with same spectral parameters), a recent addition in \emph{eniric}.


These can be created with \emph{eniric} with the following shell incantation (after installation and configuration).


\begin{lstlisting}
    phoenix_precision.py -t 3900 3500 2800 2600 \   # Temperature
                         -l 4.5 -m 0.5 \            # Logg and Metalicity
                         -r 60000 80000 100000 \    # Resolutions
                         -v 1.0 5.0 10.0 \          # Rotational velocities
                         -b Z Y J H K \             # Wavelength bands
                         --snr 100 \                # Relative SNR
                         --ref_band J               # SNR reference band
\end{lstlisting}


\emph{eniric} was also used to calculate {RV} precision for the {NIRPS} and {SPIRou}. 

For {SPIRou} the requested precisions were provided with the SNR relative to the centre of each individual band. The values are provided in \tref{tab:spirou_precisions} and can be generated with the following code, note the change in.


This has the effect of Z Y band precisions being \textbf{lower} and H and K band precisions being\textbf{ higher}.\todo{check direction and magnitude in table} compared to \tref{tab:}


\begin{lstlisting}
    phoenix_precision.py -t 3900 3500 2800 2600 -l 4.5, -m 0.5 \
        -r 60000 80000 100000 -v 1.0 5.0 10.0 -b Z Y J H K     \
        --snr 100 --ref_band self
\end{lstlisting}


For {NIRPS} RV precisions with an instrumental resolution of 75\,000 was requested to match the NIRPS instrument, and provided relative to the \emph{J} and \emph{H}-bands. The results for the NIRPS precision relative to the \emph{J}-band are given in \tref{tab:nirps_precisions}, and can be reproduced with the following code.

\begin{lstlisting}
    phoenix_precision.py -t 3900 3500 2800 2600 -l 4.5, -m 0.5  \ 
        -r 60000 75000 80000 100000 -v 1.0 5.0 10.0             \
        -b Z Y J H K --snr 100 --ref_band H                     \
\end{lstlisting}


%!TEX root = ../../thesis.tex

% Manually create table instead of load from file using \DTLsetseparator.

\begin{longtable}{crrrrrr}
      \caption[RV precisions for the {PHOENIX-ACES} and {BT-Settl} spectral libraries.]{RV precisions for the {PHOENIX-ACES} and {BT-Settl} synthetic spectral libraries.
          The {PHOENIX-ACES} values given here are the updated version of Table\,A.1 of~\citet{figueira_radial_2016}.}\\
      \hline\hline
      & \multicolumn{3}{c}{PHOENIX-ACES} & \multicolumn{3}{c}{BT-SETTL}\\
       & \(\sigma_{RV}\)& \(\sigma_{RV}\) & \(\sigma_{RV}\) & \(\sigma_{RV}\) & \(\sigma_{RV}\) & \(\sigma_{RV}\)\\
      Simulation & Cond.\,1 & Cond.\,2 & Cond.\,3 & Cond.\,1 & Cond.\,2 & Cond.\,3\\
      (SpTp - Band - $v.\sin{i}$ - R) & [m/s]& [m/s] & [m/s] & [m/s] & [m/s] & [m/s]\\
      \hline
      \endfirsthead
      \caption[]{continued.}\\
      \hline\hline
      & \multicolumn{3}{c}{PHOENIX-ACES} & \multicolumn{3}{c}{BT-SETTL}\\
       & \(\sigma_{RV}\)& \(\sigma_{RV}\) & \(\sigma_{RV}\) & \(\sigma_{RV}\) & \(\sigma_{RV}\) & \(\sigma_{RV}\)\\
      Simulation  & Cond.\,1 & Cond.\,2 & Cond.\,3 & Cond.\,1 & Cond.\,2 & Cond.\,3\\
      \hline
      \endhead
      \hline
      \endfoot

      3900-Z-1.0-60k   &   9.7 &  15.5 &  10.0 &   9.4 &  15.3 &   9.7 \\
      3900-Z-1.0-80k   &   6.4 &  10.4 &   6.6 &   6.6 &  10.7 &   6.8 \\
      3900-Z-1.0-100k  &   4.7 &   7.8 &   4.9 &   5.2 &   8.5 &   5.3 \\
      3900-Z-5.0-60k   &  14.2 &  22.6 &  14.6 &  13.0 &  21.0 &  13.4 \\
      3900-Z-5.0-80k   &  10.9 &  17.6 &  11.3 &  10.1 &  16.4 &  10.5 \\
      3900-Z-5.0-100k  &   9.2 &  14.8 &   9.5 &   8.6 &  13.9 &   8.9 \\
      3900-Z-10.0-60k  &  24.5 &  38.6 &  25.3 &  21.8 &  35.1 &  22.4 \\
      3900-Z-10.0-80k  &  20.3 &  32.2 &  21.0 &  18.1 &  29.2 &  18.7 \\
      3900-Z-10.0-100k &  17.8 &  28.2 &  18.3 &  15.9 &  25.6 &  16.3 \\
      3900-Y-1.0-60k   &   9.6 &  11.5 &   9.8 &  12.5 &  15.0 &  12.7 \\
      3900-Y-1.0-80k   &   6.0 &   7.1 &   6.0 &   8.3 &  10.0 &   8.4 \\
      3900-Y-1.0-100k  &   4.2 &   5.1 &   4.3 &   6.3 &   7.5 &   6.3 \\
      3900-Y-5.0-60k   &  15.5 &  18.4 &  15.7 &  18.6 &  22.3 &  18.9 \\
      3900-Y-5.0-80k   &  11.6 &  13.8 &  11.8 &  14.3 &  17.0 &  14.4 \\
      3900-Y-5.0-100k  &   9.7 &  11.5 &   9.8 &  11.9 &  14.3 &  12.1 \\
      3900-Y-10.0-60k  &  30.8 &  36.8 &  31.2 &  34.8 &  41.6 &  35.2 \\
      3900-Y-10.0-80k  &  25.2 &  30.1 &  25.5 &  28.6 &  34.3 &  29.0 \\
      3900-Y-10.0-100k &  21.8 &  26.0 &  22.1 &  24.9 &  29.9 &  25.2 \\
      3900-J-1.0-60k   &  15.7 &  41.7 &  16.6 &  16.2 &  45.6 &  17.1 \\
      3900-J-1.0-80k   &  10.5 &  26.9 &  11.0 &  11.5 &  31.4 &  12.2 \\
      3900-J-1.0-100k  &   7.9 &  19.6 &   8.3 &   9.2 &  24.4 &   9.7 \\
      3900-J-5.0-60k   &  22.7 &  63.6 &  24.0 &  21.8 &  65.6 &  23.1 \\
      3900-J-5.0-80k   &  17.5 &  48.1 &  18.5 &  17.1 &  50.2 &  18.2 \\
      3900-J-5.0-100k  &  14.8 &  40.3 &  15.6 &  14.6 &  42.1 &  15.4 \\
      3900-J-10.0-60k  &  38.6 & 122.9 &  41.0 &  35.4 & 122.3 &  37.7 \\
      3900-J-10.0-80k  &  32.0 & 100.7 &  34.0 &  29.5 & 100.5 &  31.4 \\
      3900-J-10.0-100k &  28.0 &  87.6 &  29.8 &  25.9 &  87.5 &  27.6 \\
      3900-H-1.0-60k   &   7.2 &  11.4 &   7.4 &   7.6 &  11.9 &   7.8 \\
      3900-H-1.0-80k   &   5.0 &   8.0 &   5.1 &   5.4 &   8.4 &   5.5 \\
      3900-H-1.0-100k  &   4.0 &   6.3 &   4.0 &   4.2 &   6.6 &   4.3 \\
      3900-H-5.0-60k   &  10.1 &  15.9 &  10.3 &  10.8 &  16.8 &  11.0 \\
      3900-H-5.0-80k   &   7.9 &  12.4 &   8.0 &   8.3 &  13.1 &   8.5 \\
      3900-H-5.0-100k  &   6.6 &  10.5 &   6.8 &   7.0 &  11.0 &   7.2 \\
      3900-H-10.0-60k  &  17.5 &  27.3 &  17.9 &  18.8 &  29.4 &  19.2 \\
      3900-H-10.0-80k  &  14.5 &  22.7 &  14.9 &  15.6 &  24.4 &  16.0 \\
      3900-H-10.0-100k &  12.7 &  19.9 &  13.0 &  13.6 &  21.3 &  14.0 \\
      3900-K-1.0-60k   &  14.5 &  63.7 &  15.5 &  13.7 &  63.2 &  14.6 \\
      3900-K-1.0-80k   &   9.7 &  43.6 &  10.4 &   9.6 &  45.2 &  10.3 \\
      3900-K-1.0-100k  &   7.4 &  33.6 &   8.0 &   7.6 &  36.1 &   8.1 \\
      3900-K-5.0-60k   &  21.7 &  90.3 &  23.2 &  19.4 &  85.4 &  20.8 \\
      3900-K-5.0-80k   &  16.6 &  70.0 &  17.7 &  15.0 &  67.0 &  16.1 \\
      3900-K-5.0-100k  &  13.9 &  59.2 &  14.8 &  12.6 &  56.9 &  13.5 \\
      3900-K-10.0-60k  &  39.4 & 155.8 &  42.1 &  34.2 & 142.8 &  36.6 \\
      3900-K-10.0-80k  &  32.6 & 128.9 &  34.9 &  28.4 & 118.5 &  30.4 \\
      3900-K-10.0-100k &  28.5 & 112.5 &  30.5 &  24.8 & 104.0 &  26.6 \\
      3500-Z-1.0-60k   &   8.4 &  13.9 &   8.7 &   8.4 &  14.0 &   8.6 \\
      3500-Z-1.0-80k   &   5.2 &   8.8 &   5.4 &   5.6 &   9.4 &   5.7 \\
      3500-Z-1.0-100k  &   3.7 &   6.3 &   3.8 &   4.2 &   7.1 &   4.3 \\
      3500-Z-5.0-60k   &  13.2 &  21.3 &  13.6 &  12.3 &  20.1 &  12.7 \\
      3500-Z-5.0-80k   &  10.0 &  16.4 &  10.3 &   9.4 &  15.5 &   9.7 \\
      3500-Z-5.0-100k  &   8.3 &  13.6 &   8.6 &   7.9 &  13.0 &   8.2 \\
      3500-Z-10.0-60k  &  24.6 &  39.1 &  25.4 &  21.9 &  35.1 &  22.5 \\
      3500-Z-10.0-80k  &  20.3 &  32.4 &  20.9 &  18.1 &  29.1 &  18.6 \\
      3500-Z-10.0-100k &  17.6 &  28.2 &  18.2 &  15.8 &  25.4 &  16.3 \\
      3500-Y-1.0-60k   &   8.5 &  10.1 &   8.6 &  11.3 &  13.3 &  11.4 \\
      3500-Y-1.0-80k   &   5.2 &   6.2 &   5.2 &   7.4 &   8.7 &   7.4 \\
      3500-Y-1.0-100k  &   3.6 &   4.3 &   3.7 &   5.5 &   6.5 &   5.5 \\
      3500-Y-5.0-60k   &  13.9 &  16.5 &  14.1 &  17.1 &  20.2 &  17.3 \\
      3500-Y-5.0-80k   &  10.4 &  12.3 &  10.5 &  13.0 &  15.3 &  13.1 \\
      3500-Y-5.0-100k  &   8.6 &  10.2 &   8.7 &  10.9 &  12.8 &  11.0 \\
      3500-Y-10.0-60k  &  28.2 &  33.5 &  28.5 &  32.4 &  38.4 &  32.8 \\
      3500-Y-10.0-80k  &  23.0 &  27.3 &  23.3 &  26.7 &  31.5 &  27.0 \\
      3500-Y-10.0-100k &  19.9 &  23.6 &  20.1 &  23.2 &  27.4 &  23.5 \\
      3500-J-1.0-60k   &  15.1 &  38.4 &  15.9 &  16.7 &  47.9 &  17.8 \\
      3500-J-1.0-80k   &   9.8 &  24.0 &  10.3 &  11.8 &  32.6 &  12.5 \\
      3500-J-1.0-100k  &   7.1 &  17.1 &   7.5 &   9.3 &  25.1 &   9.9 \\
      3500-J-5.0-60k   &  22.5 &  60.2 &  23.8 &  22.9 &  69.2 &  24.4 \\
      3500-J-5.0-80k   &  17.3 &  45.4 &  18.3 &  17.9 &  52.9 &  19.0 \\
      3500-J-5.0-100k  &  14.5 &  37.8 &  15.3 &  15.2 &  44.4 &  16.2 \\
      3500-J-10.0-60k  &  39.9 & 117.8 &  42.4 &  37.9 & 129.4 &  40.5 \\
      3500-J-10.0-80k  &  33.0 &  96.4 &  35.1 &  31.5 & 106.4 &  33.7 \\
      3500-J-10.0-100k &  28.8 &  83.7 &  30.6 &  27.6 &  92.6 &  29.5 \\
      3500-H-1.0-60k   &   7.7 &  12.4 &   7.9 &   7.7 &  12.4 &   7.9 \\
      3500-H-1.0-80k   &   5.4 &   8.8 &   5.6 &   5.4 &   8.8 &   5.5 \\
      3500-H-1.0-100k  &   4.3 &   6.9 &   4.4 &   4.3 &   6.9 &   4.4 \\
      3500-H-5.0-60k   &  10.7 &  17.1 &  10.9 &  10.9 &  17.5 &  11.2 \\
      3500-H-5.0-80k   &   8.3 &  13.3 &   8.5 &   8.4 &  13.5 &   8.6 \\
      3500-H-5.0-100k  &   7.1 &  11.3 &   7.2 &   7.1 &  11.4 &   7.3 \\
      3500-H-10.0-60k  &  18.1 &  28.7 &  18.5 &  19.5 &  31.4 &  20.0 \\
      3500-H-10.0-80k  &  15.0 &  23.9 &  15.4 &  16.1 &  26.0 &  16.6 \\
      3500-H-10.0-100k &  13.1 &  20.9 &  13.5 &  14.1 &  22.7 &  14.5 \\
      3500-K-1.0-60k   &  13.5 &  49.0 &  14.4 &  12.1 &  43.8 &  13.0 \\
      3500-K-1.0-80k   &   9.0 &  32.7 &   9.6 &   8.4 &  30.1 &   8.9 \\
      3500-K-1.0-100k  &   6.8 &  24.8 &   7.3 &   6.5 &  23.3 &   7.0 \\
      3500-K-5.0-60k   &  20.4 &  71.7 &  21.7 &  17.6 &  62.6 &  18.8 \\
      3500-K-5.0-80k   &  15.5 &  55.2 &  16.5 &  13.5 &  48.4 &  14.4 \\
      3500-K-5.0-100k  &  13.0 &  46.4 &  13.9 &  11.4 &  40.8 &  12.1 \\
      3500-K-10.0-60k  &  37.5 & 126.6 &  39.9 &  31.8 & 109.6 &  34.0 \\
      3500-K-10.0-80k  &  30.9 & 104.6 &  33.0 &  26.3 &  90.7 &  28.1 \\
      3500-K-10.0-100k &  27.0 &  91.3 &  28.8 &  23.0 &  79.3 &  24.6 \\
      2800-Z-1.0-60k   &   4.4 &   8.4 &   4.5 &   4.0 &   7.6 &   4.2 \\
      2800-Z-1.0-80k   &   2.6 &   5.1 &   2.7 &   2.6 &   4.8 &   2.7 \\
      2800-Z-1.0-100k  &   1.8 &   3.6 &   1.9 &   1.9 &   3.5 &   1.9 \\
      2800-Z-5.0-60k   &   7.2 &  13.7 &   7.5 &   6.3 &  11.7 &   6.6 \\
      2800-Z-5.0-80k   &   5.4 &  10.3 &   5.6 &   4.8 &   8.8 &   4.9 \\
      2800-Z-5.0-100k  &   4.4 &   8.5 &   4.6 &   4.0 &   7.3 &   4.1 \\
      2800-Z-10.0-60k  &  14.6 &  27.3 &  15.2 &  12.6 &  22.8 &  13.0 \\
      2800-Z-10.0-80k  &  11.9 &  22.4 &  12.4 &  10.3 &  18.6 &  10.6 \\
      2800-Z-10.0-100k &  10.3 &  19.3 &  10.7 &   8.9 &  16.2 &   9.2 \\
      2800-Y-1.0-60k   &   5.8 &   7.4 &   5.9 &  11.7 &  14.0 &  11.9 \\
      2800-Y-1.0-80k   &   3.7 &   4.8 &   3.8 &   7.6 &   9.1 &   7.7 \\
      2800-Y-1.0-100k  &   2.7 &   3.5 &   2.7 &   5.6 &   6.7 &   5.7 \\
      2800-Y-5.0-60k   &   8.7 &  10.7 &   8.8 &  17.7 &  21.3 &  17.9 \\
      2800-Y-5.0-80k   &   6.7 &   8.3 &   6.8 &  13.5 &  16.2 &  13.7 \\
      2800-Y-5.0-100k  &   5.6 &   7.0 &   5.7 &  11.3 &  13.6 &  11.4 \\
      2800-Y-10.0-60k  &  14.7 &  17.5 &  14.9 &  32.5 &  39.0 &  32.9 \\
      2800-Y-10.0-80k  &  12.2 &  14.6 &  12.4 &  26.8 &  32.1 &  27.1 \\
      2800-Y-10.0-100k &  10.7 &  12.8 &  10.8 &  23.3 &  28.0 &  23.6 \\
      2800-J-1.0-60k   &   8.8 &  23.2 &   9.3 &  11.0 &  31.7 &  11.8 \\
      2800-J-1.0-80k   &   5.5 &  14.4 &   5.8 &   7.6 &  21.5 &   8.1 \\
      2800-J-1.0-100k  &   4.0 &  10.2 &   4.2 &   5.8 &  16.3 &   6.2 \\
      2800-J-5.0-60k   &  13.5 &  35.9 &  14.3 &  15.6 &  45.3 &  16.7 \\
      2800-J-5.0-80k   &  10.3 &  27.3 &  10.9 &  12.1 &  35.1 &  13.0 \\
      2800-J-5.0-100k  &   8.6 &  22.8 &   9.1 &  10.2 &  29.6 &  10.9 \\
      2800-J-10.0-60k  &  24.3 &  63.8 &  25.7 &  26.7 &  78.5 &  28.5 \\
      2800-J-10.0-80k  &  20.1 &  52.9 &  21.2 &  22.2 &  65.1 &  23.7 \\
      2800-J-10.0-100k &  17.5 &  46.1 &  18.5 &  19.4 &  57.0 &  20.7 \\
      2800-H-1.0-60k   &   6.6 &  12.8 &   6.9 &   5.7 &  11.1 &   5.9 \\
      2800-H-1.0-80k   &   4.4 &   8.5 &   4.6 &   3.9 &   7.5 &   4.0 \\
      2800-H-1.0-100k  &   3.3 &   6.4 &   3.5 &   3.0 &   5.8 &   3.1 \\
      2800-H-5.0-60k   &   9.8 &  18.8 &  10.2 &   8.4 &  16.4 &   8.7 \\
      2800-H-5.0-80k   &   7.5 &  14.4 &   7.8 &   6.5 &  12.6 &   6.7 \\
      2800-H-5.0-100k  &   6.3 &  12.1 &   6.5 &   5.4 &  10.5 &   5.6 \\
      2800-H-10.0-60k  &  17.8 &  34.7 &  18.4 &  15.6 &  31.0 &  16.2 \\
      2800-H-10.0-80k  &  14.7 &  28.7 &  15.2 &  12.9 &  25.5 &  13.3 \\
      2800-H-10.0-100k &  12.8 &  25.0 &  13.3 &  11.2 &  22.2 &  11.6 \\
      2800-K-1.0-60k   &   8.2 &  27.0 &   8.7 &   7.1 &  22.9 &   7.5 \\
      2800-K-1.0-80k   &   5.4 &  17.9 &   5.7 &   4.8 &  15.6 &   5.1 \\
      2800-K-1.0-100k  &   4.0 &  13.5 &   4.3 &   3.7 &  12.0 &   4.0 \\
      2800-K-5.0-60k   &  12.4 &  39.7 &  13.2 &  10.3 &  33.3 &  11.0 \\
      2800-K-5.0-80k   &   9.4 &  30.5 &  10.0 &   7.9 &  25.6 &   8.4 \\
      2800-K-5.0-100k  &   7.9 &  25.6 &   8.4 &   6.6 &  21.5 &   7.1 \\
      2800-K-10.0-60k  &  23.0 &  70.7 &  24.6 &  19.1 &  59.6 &  20.4 \\
      2800-K-10.0-80k  &  19.0 &  58.5 &  20.3 &  15.8 &  49.3 &  16.8 \\
      2800-K-10.0-100k &  16.6 &  51.1 &  17.7 &  13.8 &  43.0 &  14.7 \\
      2600-Z-1.0-60k   &   3.7 &   7.7 &   3.9 &   3.4 &   6.5 &   3.6 \\
      2600-Z-1.0-80k   &   2.2 &   4.6 &   2.3 &   2.2 &   4.2 &   2.3 \\
      2600-Z-1.0-100k  &   1.5 &   3.2 &   1.6 &   1.6 &   3.0 &   1.7 \\
      2600-Z-5.0-60k   &   6.1 &  12.5 &   6.4 &   5.4 &  10.0 &   5.6 \\
      2600-Z-5.0-80k   &   4.6 &   9.4 &   4.8 &   4.0 &   7.6 &   4.2 \\
      2600-Z-5.0-100k  &   3.8 &   7.8 &   3.9 &   3.4 &   6.3 &   3.5 \\
      2600-Z-10.0-60k  &  12.3 &  24.9 &  12.9 &  10.6 &  19.6 &  11.0 \\
      2600-Z-10.0-80k  &  10.1 &  20.4 &  10.5 &   8.7 &  16.0 &   9.0 \\
      2600-Z-10.0-100k &   8.7 &  17.6 &   9.1 &   7.5 &  13.9 &   7.8 \\
      2600-Y-1.0-60k   &   4.8 &   6.3 &   4.9 &   7.7 &   9.2 &   7.8 \\
      2600-Y-1.0-80k   &   3.0 &   4.0 &   3.1 &   5.0 &   5.9 &   5.0 \\
      2600-Y-1.0-100k  &   2.1 &   2.9 &   2.2 &   3.6 &   4.4 &   3.7 \\
      2600-Y-5.0-60k   &   7.2 &   9.1 &   7.3 &  11.7 &  14.2 &  11.9 \\
      2600-Y-5.0-80k   &   5.6 &   7.0 &   5.6 &   8.9 &  10.8 &   9.0 \\
      2600-Y-5.0-100k  &   4.7 &   5.9 &   4.7 &   7.4 &   9.0 &   7.5 \\
      2600-Y-10.0-60k  &  12.1 &  14.6 &  12.2 &  22.0 &  26.3 &  22.2 \\
      2600-Y-10.0-80k  &  10.1 &  12.2 &  10.2 &  18.0 &  21.6 &  18.3 \\
      2600-Y-10.0-100k &   8.8 &  10.7 &   8.9 &  15.7 &  18.8 &  15.9 \\
      2600-J-1.0-60k   &   6.4 &  17.1 &   6.8 &   8.1 &  24.0 &   8.6 \\
      2600-J-1.0-80k   &   4.0 &  10.5 &   4.2 &   5.4 &  16.0 &   5.8 \\
      2600-J-1.0-100k  &   2.8 &   7.4 &   3.0 &   4.1 &  12.0 &   4.4 \\
      2600-J-5.0-60k   &  10.0 &  26.9 &  10.6 &  11.7 &  35.0 &  12.5 \\
      2600-J-5.0-80k   &   7.6 &  20.4 &   8.0 &   9.0 &  26.9 &   9.6 \\
      2600-J-5.0-100k  &   6.3 &  17.0 &   6.7 &   7.6 &  22.6 &   8.1 \\
      2600-J-10.0-60k  &  18.1 &  47.2 &  19.1 &  20.8 &  60.7 &  22.2 \\
      2600-J-10.0-80k  &  15.0 &  39.2 &  15.8 &  17.2 &  50.4 &  18.3 \\
      2600-J-10.0-100k &  13.0 &  34.2 &  13.8 &  15.0 &  44.1 &  16.0 \\
      2600-H-1.0-60k   &   5.2 &  10.2 &   5.4 &   4.6 &   9.1 &   4.8 \\
      2600-H-1.0-80k   &   3.4 &   6.7 &   3.5 &   3.1 &   6.2 &   3.2 \\
      2600-H-1.0-100k  &   2.6 &   5.1 &   2.7 &   2.4 &   4.7 &   2.5 \\
      2600-H-5.0-60k   &   7.8 &  15.4 &   8.0 &   6.8 &  13.5 &   7.0 \\
      2600-H-5.0-80k   &   5.9 &  11.7 &   6.2 &   5.2 &  10.3 &   5.4 \\
      2600-H-5.0-100k  &   5.0 &   9.8 &   5.2 &   4.4 &   8.6 &   4.5 \\
      2600-H-10.0-60k  &  14.2 &  28.9 &  14.7 &  12.6 &  25.5 &  13.0 \\
      2600-H-10.0-80k  &  11.7 &  23.8 &  12.1 &  10.3 &  21.0 &  10.7 \\
      2600-H-10.0-100k &  10.2 &  20.7 &  10.6 &   9.0 &  18.3 &   9.4 \\
      2600-K-1.0-60k   &   6.2 &  20.2 &   6.6 &   5.7 &  18.8 &   6.1 \\
      2600-K-1.0-80k   &   4.0 &  13.2 &   4.3 &   3.9 &  12.9 &   4.2 \\
      2600-K-1.0-100k  &   3.0 &   9.9 &   3.2 &   3.0 &   9.9 &   3.2 \\
      2600-K-5.0-60k   &   9.4 &  30.1 &  10.0 &   8.3 &  27.2 &   8.9 \\
      2600-K-5.0-80k   &   7.1 &  23.0 &   7.6 &   6.4 &  20.9 &   6.8 \\
      2600-K-5.0-100k  &   6.0 &  19.3 &   6.4 &   5.3 &  17.6 &   5.7 \\
      2600-K-10.0-60k  &  17.4 &  54.0 &  18.6 &  15.2 &  48.5 &  16.2 \\
      2600-K-10.0-80k  &  14.4 &  44.7 &  15.3 &  12.5 &  40.1 &  13.4 \\
      2600-K-10.0-100k &  12.5 &  39.1 &  13.4 &  10.9 &  35.0 & 11.7
    \label{tab:rv_aces_btsettl}
\end{longtable}


\clearpage{}


%
%\DTLsetseparator{,}
%\DTLloadrawdb{precisions_aces_btsettl}{data/precision_results/combined_aces_btsettl_precsion.csv} % relative to root
%
%%\longtab[2]{
%\begin{longtable}{ccccccc}
%
%    \caption[RV precisions for the {PHOENIX-ACES} and {BT-Settl} spectral libraries.]{RV precisions from the {PHOENIX-ACES} and {BT-Settl} synthetic spectral libraries.
%        The {PHOENIX-ACES} values are the updated values from Table\,A.1 of~\citet{figueira_radial_2016}.}
%    \\
%    \hline\hline
%    & \multicolumn{3}{c}{PHOENIX-ACES} & \multicolumn{3}{c}{BT-SETTL}\\
%    Simulation & \(\sigma_{RV}\)(Cond.\,1) & Cond.\,2 & \(\sigma_{RV}\)(Cond.\,3) \\
%    (SpTp - Band - $v.\sin{i}$ - R) & [m/s]& [m/s] & [m/s]\\
%    \hline
%    \endfirsthead
%    \caption[]{continued.}\\
%    \hline\hline
%    & \multicolumn{3}{c}{PHOENIX-ACES} & \multicolumn{3}{c}{BT-SETTL}\\
%    Simulation & \(\sigma_{RV}\)(Cond.\,1) & \(\sigma_{RV}\)(Cond.\,2) & \(\sigma_{RV}\)(Cond.\,3) \\
%    \hline
%    \endhead
%    \hline
%    \endfoot
%    \DTLforeach*{precisions_aces_btsettl}{
%        \dbtemp=temp,\dbband=band,\dbvsini=vsini,\dbres=resolution,
%        \acone=cond1_aces,\actwo=cond2_aces,\acthree=cond3_aces,\btone=cond1_btsettl,\bttwo=cond2_btsettl,\btthree=cond3_btsettl}{%  \DTLiffirstrow{}{\\}  if first row don't add a new line at start.
%        \DTLiffirstrow{} {\\}\dbtemp-\dbband-\dbvisi-\dbres & \acone & \actwo & \acthree& \btone & \bttwo & \btthreeS
%    }
%    \label{tab:rv_aces_btsettl}
%\end{longtable}


%!TEX root = ../../thesis.tex
% NIRPS Table

% Manually create table instead of load from file using \DTLsetseparator.
%\clearpage{}
\begin{longtable}{crrr}

    \caption[RV precisions calculated for the {NIRPS} {ETC}.]{RV precisions calculated for the {NIRPS} {ETC}.
        All temperatures spanning 2500--4000\K{} at a resolution of 75\,000, 100\,000 only.}\\
    \hline\hline
    Simulation & \(\sigma_{RV}\)(Cond.\,1) & \(\sigma_{RV}\)(Cond.\,2) & \(\sigma_{RV}\)(Cond.\,3)\\
    (\Teff{}-Band-$v.\sin{i}$-R) & [m/s] & [m/s] & [m/s] \\
    \hline
    \endfirsthead
    \caption[]{continued.}\\
    \hline\hline
    Simulation  & \(\sigma_{RV}\)(Cond.\,1) & \(\sigma_{RV}\)(Cond.\,2) & \(\sigma_{RV}\)(Cond.\,3) \\
    \hline
    \endhead
    \hline
    \endfoot

    4000-Z-1.0-75k   &    6.9 &   11.2 &    7.1 \\
    4000-Z-1.0-100k  &    4.7 &    7.7 &    4.9 \\
    4000-Z-5.0-75k   &   11.2 &   18.2 &   11.6 \\
    4000-Z-5.0-100k  &    9.0 &   14.5 &    9.3 \\
    4000-Z-10.0-75k  &   20.5 &   32.6 &   21.1 \\
    4000-Z-10.0-100k &   17.2 &   27.4 &   17.8 \\
    4000-Y-1.0-75k   &    6.8 &    8.1 &    6.8 \\
    4000-Y-1.0-100k  &    4.4 &    5.2 &    4.4 \\
    4000-Y-5.0-75k   &   12.5 &   14.9 &   12.6 \\
    4000-Y-5.0-100k  &    9.8 &   11.7 &    9.9 \\
    4000-Y-10.0-75k  &   26.4 &   31.6 &   26.8 \\
    4000-Y-10.0-100k &   21.9 &   26.2 &   22.2 \\
    4000-J-1.0-75k   &   11.3 &   29.3 &   11.9 \\
    4000-J-1.0-100k  &    7.8 &   19.7 &    8.2 \\
    4000-J-5.0-75k   &   18.1 &   50.1 &   19.1 \\
    4000-J-5.0-100k  &   14.5 &   39.7 &   15.3 \\
    4000-J-10.0-75k  &   32.3 &  102.4 &   34.3 \\
    4000-J-10.0-100k &   27.2 &   85.5 &   28.9 \\
    4000-H-1.0-75k   &    5.4 &    8.5 &    5.5 \\
    4000-H-1.0-100k  &    3.9 &    6.2 &    4.0 \\
    4000-H-5.0-75k   &    8.2 &   13.0 &    8.4 \\
    4000-H-5.0-100k  &    6.6 &   10.4 &    6.7 \\
    4000-H-10.0-75k  &   15.0 &   23.5 &   15.4 \\
    4000-H-10.0-100k &   12.6 &   19.7 &   12.9 \\
    4000-K-1.0-75k   &   10.4 &   47.3 &   11.1 \\
    4000-K-1.0-100k  &    7.3 &   33.7 &    7.8 \\
    4000-K-5.0-75k   &   17.2 &   73.5 &   18.4 \\
    4000-K-5.0-100k  &   13.7 &   59.0 &   14.6 \\
    4000-K-10.0-75k  &   33.4 &  132.8 &   35.7 \\
    4000-K-10.0-100k &   28.0 &  111.8 &   29.9 \\
    3900-Z-1.0-75k   &    7.0 &   11.4 &    7.2 \\
    3900-Z-1.0-100k  &    4.7 &    7.8 &    4.9 \\
    3900-Z-5.0-75k   &   11.5 &   18.6 &   11.9 \\
    3900-Z-5.0-100k  &    9.2 &   14.8 &    9.5 \\
    3900-Z-10.0-75k  &   21.2 &   33.6 &   21.8 \\
    3900-Z-10.0-100k &   17.8 &   28.2 &   18.3 \\
    3900-Y-1.0-75k   &    6.6 &    7.9 &    6.7 \\
    3900-Y-1.0-100k  &    4.2 &    5.1 &    4.3 \\
    3900-Y-5.0-75k   &   12.3 &   14.7 &   12.5 \\
    3900-Y-5.0-100k  &    9.7 &   11.5 &    9.8 \\
    3900-Y-10.0-75k  &   26.3 &   31.4 &   26.6 \\
    3900-Y-10.0-100k &   21.8 &   26.0 &   22.1 \\
    3900-J-1.0-75k   &   11.5 &   29.5 &   12.1 \\
    3900-J-1.0-100k  &    7.9 &   19.6 &    8.3 \\
    3900-J-5.0-75k   &   18.5 &   50.9 &   19.6 \\
    3900-J-5.0-100k  &   14.8 &   40.3 &   15.6 \\
    3900-J-10.0-75k  &   33.4 &  104.9 &   35.4 \\
    3900-J-10.0-100k &   28.0 &   87.6 &   29.8 \\
    3900-H-1.0-75k   &    5.4 &    8.6 &    5.5 \\
    3900-H-1.0-100k  &    4.0 &    6.3 &    4.0 \\
    3900-H-5.0-75k   &    8.3 &   13.0 &    8.5 \\
    3900-H-5.0-100k  &    6.6 &   10.5 &    6.8 \\
    3900-H-10.0-75k  &   15.1 &   23.6 &   15.5 \\
    3900-H-10.0-100k &   12.7 &   19.9 &   13.0 \\
    3900-K-1.0-75k   &   10.6 &   47.3 &   11.3 \\
    3900-K-1.0-100k  &    7.4 &   33.6 &    8.0 \\
    3900-K-5.0-75k   &   17.5 &   73.8 &   18.7 \\
    3900-K-5.0-100k  &   13.9 &   59.2 &   14.8 \\
    3900-K-10.0-75k  &   34.0 &  133.7 &   36.3 \\
    3900-K-10.0-100k &   28.5 &  112.5 &   30.5 \\
    3800-Z-1.0-75k   &    6.9 &   11.3 &    7.2 \\
    3800-Z-1.0-100k  &    4.7 &    7.7 &    4.8 \\
    3800-Z-5.0-75k   &   11.7 &   18.7 &   12.0 \\
    3800-Z-5.0-100k  &    9.3 &   14.9 &    9.6 \\
    3800-Z-10.0-75k  &   21.7 &   34.2 &   22.4 \\
    3800-Z-10.0-100k &   18.2 &   28.7 &   18.8 \\
    3800-Y-1.0-75k   &    6.5 &    7.7 &    6.5 \\
    3800-Y-1.0-100k  &    4.1 &    4.9 &    4.2 \\
    3800-Y-5.0-75k   &   12.1 &   14.4 &   12.3 \\
    3800-Y-5.0-100k  &    9.5 &   11.3 &    9.6 \\
    3800-Y-10.0-75k  &   26.0 &   31.0 &   26.3 \\
    3800-Y-10.0-100k &   21.5 &   25.7 &   21.8 \\
    3800-J-1.0-75k   &   11.5 &   29.3 &   12.1 \\
    3800-J-1.0-100k  &    7.8 &   19.4 &    8.2 \\
    3800-J-5.0-75k   &   18.7 &   51.2 &   19.8 \\
    3800-J-5.0-100k  &   14.9 &   40.5 &   15.8 \\
    3800-J-10.0-75k  &   34.1 &  106.2 &   36.2 \\
    3800-J-10.0-100k &   28.7 &   88.6 &   30.4 \\
    3800-H-1.0-75k   &    5.5 &    8.8 &    5.7 \\
    3800-H-1.0-100k  &    4.1 &    6.4 &    4.1 \\
    3800-H-5.0-75k   &    8.4 &   13.2 &    8.6 \\
    3800-H-5.0-100k  &    6.7 &   10.6 &    6.9 \\
    3800-H-10.0-75k  &   15.2 &   23.7 &   15.6 \\
    3800-H-10.0-100k &   12.8 &   20.0 &   13.1 \\
    3800-K-1.0-75k   &   10.7 &   46.0 &   11.4 \\
    3800-K-1.0-100k  &    7.5 &   32.5 &    8.0 \\
    3800-K-5.0-75k   &   17.7 &   72.6 &   18.9 \\
    3800-K-5.0-100k  &   14.0 &   58.2 &   15.0 \\
    3800-K-10.0-75k  &   34.4 &  132.6 &   36.7 \\
    3800-K-10.0-100k &   28.8 &  111.5 &   30.8 \\
    3700-Z-1.0-75k   &    6.8 &   11.1 &    7.0 \\
    3700-Z-1.0-100k  &    4.5 &    7.4 &    4.6 \\
    3700-Z-5.0-75k   &   11.6 &   18.7 &   12.0 \\
    3700-Z-5.0-100k  &    9.2 &   14.8 &    9.5 \\
    3700-Z-10.0-75k  &   22.0 &   34.6 &   22.7 \\
    3700-Z-10.0-100k &   18.4 &   29.1 &   19.0 \\
    3700-Y-1.0-75k   &    6.3 &    7.5 &    6.3 \\
    3700-Y-1.0-100k  &    4.0 &    4.8 &    4.0 \\
    3700-Y-5.0-75k   &   11.9 &   14.1 &   12.0 \\
    3700-Y-5.0-100k  &    9.3 &   11.0 &    9.4 \\
    3700-Y-10.0-75k  &   25.6 &   30.4 &   25.9 \\
    3700-Y-10.0-100k &   21.2 &   25.2 &   21.4 \\
    3700-J-1.0-75k   &   11.4 &   28.8 &   12.0 \\
    3700-J-1.0-100k  &    7.7 &   18.9 &    8.1 \\
    3700-J-5.0-75k   &   18.8 &   50.9 &   19.9 \\
    3700-J-5.0-100k  &   15.0 &   40.2 &   15.8 \\
    3700-J-10.0-75k  &   34.7 &  106.0 &   36.8 \\
    3700-J-10.0-100k &   29.1 &   88.4 &   30.9 \\
    3700-H-1.0-75k   &    5.7 &    9.0 &    5.8 \\
    3700-H-1.0-100k  &    4.2 &    6.6 &    4.3 \\
    3700-H-5.0-75k   &    8.5 &   13.4 &    8.7 \\
    3700-H-5.0-100k  &    6.8 &   10.8 &    7.0 \\
    3700-H-10.0-75k  &   15.3 &   24.0 &   15.7 \\
    3700-H-10.0-100k &   12.9 &   20.2 &   13.2 \\
    3700-K-1.0-75k   &   10.6 &   43.7 &   11.4 \\
    3700-K-1.0-100k  &    7.5 &   30.7 &    8.0 \\
    3700-K-5.0-75k   &   17.7 &   70.1 &   18.9 \\
    3700-K-5.0-100k  &   14.0 &   56.0 &   15.0 \\
    3700-K-10.0-75k  &   34.5 &  128.8 &   36.8 \\
    3700-K-10.0-100k &   28.9 &  108.3 &   30.9 \\
    3600-Z-1.0-75k   &    6.3 &   10.5 &    6.6 \\
    3600-Z-1.0-100k  &    4.1 &    6.9 &    4.3 \\
    3600-Z-5.0-75k   &   11.2 &   18.2 &   11.6 \\
    3600-Z-5.0-100k  &    8.9 &   14.4 &    9.2 \\
    3600-Z-10.0-75k  &   21.8 &   34.5 &   22.5 \\
    3600-Z-10.0-100k &   18.2 &   28.9 &   18.8 \\
    3600-Y-1.0-75k   &    6.0 &    7.2 &    6.1 \\
    3600-Y-1.0-100k  &    3.8 &    4.6 &    3.9 \\
    3600-Y-5.0-75k   &   11.5 &   13.6 &   11.6 \\
    3600-Y-5.0-100k  &    9.0 &   10.6 &    9.1 \\
    3600-Y-10.0-75k  &   24.9 &   29.6 &   25.2 \\
    3600-Y-10.0-100k &   20.6 &   24.5 &   20.9 \\
    3600-J-1.0-75k   &   11.1 &   27.9 &   11.7 \\
    3600-J-1.0-100k  &    7.5 &   18.1 &    7.8 \\
    3600-J-5.0-75k   &   18.7 &   49.8 &   19.7 \\
    3600-J-5.0-100k  &   14.8 &   39.3 &   15.7 \\
    3600-J-10.0-75k  &   34.7 &  104.0 &   36.9 \\
    3600-J-10.0-100k &   29.1 &   86.7 &   30.9 \\
    3600-H-1.0-75k   &    5.8 &    9.2 &    5.9 \\
    3600-H-1.0-100k  &    4.3 &    6.8 &    4.3 \\
    3600-H-5.0-75k   &    8.6 &   13.7 &    8.8 \\
    3600-H-5.0-100k  &    6.9 &   11.0 &    7.1 \\
    3600-H-10.0-75k  &   15.4 &   24.3 &   15.8 \\
    3600-H-10.0-100k &   13.0 &   20.5 &   13.3 \\
    3600-K-1.0-75k   &   10.3 &   39.8 &   11.0 \\
    3600-K-1.0-100k  &    7.2 &   27.8 &    7.7 \\
    3600-K-5.0-75k   &   17.2 &   64.7 &   18.4 \\
    3600-K-5.0-100k  &   13.7 &   51.6 &   14.6 \\
    3600-K-10.0-75k  &   33.7 &  120.0 &   36.0 \\
    3600-K-10.0-100k &   28.2 &  100.8 &   30.2 \\
    3500-Z-1.0-75k   &    5.8 &    9.7 &    6.0 \\
    3500-Z-1.0-100k  &    3.7 &    6.3 &    3.8 \\
    3500-Z-5.0-75k   &   10.6 &   17.3 &   10.9 \\
    3500-Z-5.0-100k  &    8.3 &   13.6 &    8.6 \\
    3500-Z-10.0-75k  &   21.1 &   33.8 &   21.8 \\
    3500-Z-10.0-100k &   17.6 &   28.2 &   18.2 \\
    3500-Y-1.0-75k   &    5.8 &    6.9 &    5.8 \\
    3500-Y-1.0-100k  &    3.6 &    4.3 &    3.7 \\
    3500-Y-5.0-75k   &   11.0 &   13.1 &   11.2 \\
    3500-Y-5.0-100k  &    8.6 &   10.2 &    8.7 \\
    3500-Y-10.0-75k  &   24.0 &   28.5 &   24.3 \\
    3500-Y-10.0-100k &   19.9 &   23.6 &   20.1 \\
    3500-J-1.0-75k   &   10.7 &   26.6 &   11.3 \\
    3500-J-1.0-100k  &    7.1 &   17.1 &    7.5 \\
    3500-J-5.0-75k   &   18.3 &   48.1 &   19.3 \\
    3500-J-5.0-100k  &   14.5 &   37.8 &   15.3 \\
    3500-J-10.0-75k  &   34.4 &  100.5 &   36.5 \\
    3500-J-10.0-100k &   28.8 &   83.7 &   30.6 \\
    3500-H-1.0-75k   &    5.8 &    9.4 &    6.0 \\
    3500-H-1.0-100k  &    4.3 &    6.9 &    4.4 \\
    3500-H-5.0-75k   &    8.8 &   14.0 &    9.0 \\
    3500-H-5.0-100k  &    7.1 &   11.3 &    7.2 \\
    3500-H-10.0-75k  &   15.6 &   24.8 &   16.0 \\
    3500-H-10.0-100k &   13.1 &   20.9 &   13.5 \\
    3500-K-1.0-75k   &    9.8 &   35.7 &   10.5 \\
    3500-K-1.0-100k  &    6.8 &   24.8 &    7.3 \\
    3500-K-5.0-75k   &   16.4 &   58.2 &   17.5 \\
    3500-K-5.0-100k  &   13.0 &   46.4 &   13.9 \\
    3500-K-10.0-75k  &   32.2 &  108.7 &   34.4 \\
    3500-K-10.0-100k &   27.0 &   91.3 &   28.8 \\
    3400-Z-1.0-75k   &    5.3 &    8.9 &    5.5 \\
    3400-Z-1.0-100k  &    3.3 &    5.7 &    3.5 \\
    3400-Z-5.0-75k   &    9.8 &   16.3 &   10.2 \\
    3400-Z-5.0-100k  &    7.7 &   12.8 &    8.0 \\
    3400-Z-10.0-75k  &   20.2 &   32.7 &   20.8 \\
    3400-Z-10.0-100k &   16.8 &   27.2 &   17.3 \\
    3400-Y-1.0-75k   &    5.5 &    6.6 &    5.6 \\
    3400-Y-1.0-100k  &    3.5 &    4.2 &    3.5 \\
    3400-Y-5.0-75k   &   10.5 &   12.4 &   10.6 \\
    3400-Y-5.0-100k  &    8.2 &    9.7 &    8.3 \\
    3400-Y-10.0-75k  &   22.3 &   26.4 &   22.5 \\
    3400-Y-10.0-100k &   18.5 &   21.9 &   18.7 \\
    3400-J-1.0-75k   &   10.2 &   25.0 &   10.8 \\
    3400-J-1.0-100k  &    6.7 &   15.9 &    7.1 \\
    3400-J-5.0-75k   &   17.6 &   45.6 &   18.6 \\
    3400-J-5.0-100k  &   14.0 &   35.9 &   14.7 \\
    3400-J-10.0-75k  &   33.6 &   95.5 &   35.7 \\
    3400-J-10.0-100k &   28.1 &   79.4 &   29.9 \\
    3400-H-1.0-75k   &    5.9 &    9.7 &    6.0 \\
    3400-H-1.0-100k  &    4.3 &    7.1 &    4.4 \\
    3400-H-5.0-75k   &    8.9 &   14.5 &    9.2 \\
    3400-H-5.0-100k  &    7.2 &   11.7 &    7.4 \\
    3400-H-10.0-75k  &   16.0 &   25.8 &   16.4 \\
    3400-H-10.0-100k &   13.4 &   21.7 &   13.8 \\
    3400-K-1.0-75k   &    9.2 &   31.8 &    9.8 \\
    3400-K-1.0-100k  &    6.4 &   22.2 &    6.8 \\
    3400-K-5.0-75k   &   15.4 &   52.0 &   16.4 \\
    3400-K-5.0-100k  &   12.2 &   41.4 &   13.0 \\
    3400-K-10.0-75k  &   30.5 &   97.5 &   32.4 \\
    3400-K-10.0-100k &   25.5 &   81.8 &   27.2 \\
    3300-Z-1.0-75k   &    4.8 &    8.2 &    5.0 \\
    3300-Z-1.0-100k  &    3.0 &    5.2 &    3.1 \\
    3300-Z-5.0-75k   &    9.1 &   15.3 &    9.4 \\
    3300-Z-5.0-100k  &    7.1 &   11.9 &    7.3 \\
    3300-Z-10.0-75k  &   19.0 &   31.3 &   19.7 \\
    3300-Z-10.0-100k &   15.8 &   26.0 &   16.4 \\
    3300-Y-1.0-75k   &    5.8 &    7.1 &    5.9 \\
    3300-Y-1.0-100k  &    3.8 &    4.6 &    3.8 \\
    3300-Y-5.0-75k   &   10.3 &   12.2 &   10.4 \\
    3300-Y-5.0-100k  &    8.1 &    9.7 &    8.2 \\
    3300-Y-10.0-75k  &   19.9 &   23.5 &   20.1 \\
    3300-Y-10.0-100k &   16.6 &   19.6 &   16.8 \\
    3300-J-1.0-75k   &    9.5 &   23.0 &   10.1 \\
    3300-J-1.0-100k  &    6.2 &   14.6 &    6.6 \\
    3300-J-5.0-75k   &   16.7 &   42.5 &   17.7 \\
    3300-J-5.0-100k  &   13.2 &   33.3 &   13.9 \\
    3300-J-10.0-75k  &   32.3 &   89.0 &   34.3 \\
    3300-J-10.0-100k &   27.0 &   74.0 &   28.6 \\
    3300-H-1.0-75k   &    5.9 &   10.0 &    6.1 \\
    3300-H-1.0-100k  &    4.3 &    7.2 &    4.4 \\
    3300-H-5.0-75k   &    9.1 &   15.2 &    9.4 \\
    3300-H-5.0-100k  &    7.3 &   12.2 &    7.5 \\
    3300-H-10.0-75k  &   16.5 &   27.3 &   16.9 \\
    3300-H-10.0-100k &   13.9 &   22.9 &   14.2 \\
    3300-K-1.0-75k   &    8.7 &   28.9 &    9.2 \\
    3300-K-1.0-100k  &    6.0 &   20.2 &    6.4 \\
    3300-K-5.0-75k   &   14.6 &   47.2 &   15.5 \\
    3300-K-5.0-100k  &   11.5 &   37.6 &   12.3 \\
    3300-K-10.0-75k  &   28.8 &   88.8 &   30.7 \\
    3300-K-10.0-100k &   24.1 &   74.6 &   25.7 \\
    3200-Z-1.0-75k   &    4.3 &    7.5 &    4.5 \\
    3200-Z-1.0-100k  &    2.7 &    4.7 &    2.8 \\
    3200-Z-5.0-75k   &    8.3 &   14.2 &    8.6 \\
    3200-Z-5.0-100k  &    6.5 &   11.1 &    6.7 \\
    3200-Z-10.0-75k  &   17.8 &   29.8 &   18.4 \\
    3200-Z-10.0-100k &   14.7 &   24.7 &   15.3 \\
    3200-Y-1.0-75k   &    5.5 &    6.7 &    5.6 \\
    3200-Y-1.0-100k  &    3.6 &    4.4 &    3.6 \\
    3200-Y-5.0-75k   &    9.7 &   11.6 &    9.8 \\
    3200-Y-5.0-100k  &    7.6 &    9.2 &    7.7 \\
    3200-Y-10.0-75k  &   18.4 &   21.7 &   18.6 \\
    3200-Y-10.0-100k &   15.4 &   18.2 &   15.6 \\
    3200-J-1.0-75k   &    8.7 &   20.6 &    9.2 \\
    3200-J-1.0-100k  &    5.6 &   13.0 &    5.9 \\
    3200-J-5.0-75k   &   15.5 &   38.5 &   16.3 \\
    3200-J-5.0-100k  &   12.2 &   30.2 &   12.9 \\
    3200-J-10.0-75k  &   30.3 &   80.8 &   32.1 \\
    3200-J-10.0-100k &   25.3 &   67.2 &   26.8 \\
    3200-H-1.0-75k   &    5.9 &   10.3 &    6.1 \\
    3200-H-1.0-100k  &    4.2 &    7.4 &    4.3 \\
    3200-H-5.0-75k   &    9.3 &   15.9 &    9.6 \\
    3200-H-5.0-100k  &    7.4 &   12.7 &    7.7 \\
    3200-H-10.0-75k  &   17.0 &   29.0 &   17.5 \\
    3200-H-10.0-100k &   14.3 &   24.4 &   14.7 \\
    3200-K-1.0-75k   &    8.2 &   26.9 &    8.7 \\
    3200-K-1.0-100k  &    5.7 &   18.8 &    6.1 \\
    3200-K-5.0-75k   &   13.8 &   43.9 &   14.7 \\
    3200-K-5.0-100k  &   10.9 &   34.9 &   11.6 \\
    3200-K-10.0-75k  &   27.4 &   82.7 &   29.2 \\
    3200-K-10.0-100k &   22.9 &   69.4 &   24.4 \\
    3100-Z-1.0-75k   &    4.0 &    7.3 &    4.2 \\
    3100-Z-1.0-100k  &    2.5 &    4.6 &    2.6 \\
    3100-Z-5.0-75k   &    7.8 &   13.8 &    8.1 \\
    3100-Z-5.0-100k  &    6.1 &   10.8 &    6.3 \\
    3100-Z-10.0-75k  &   16.7 &   29.0 &   17.3 \\
    3100-Z-10.0-100k &   13.9 &   24.1 &   14.4 \\
    3100-Y-1.0-75k   &    5.5 &    6.8 &    5.6 \\
    3100-Y-1.0-100k  &    3.6 &    4.5 &    3.7 \\
    3100-Y-5.0-75k   &    9.4 &   11.3 &    9.5 \\
    3100-Y-5.0-100k  &    7.5 &    9.0 &    7.5 \\
    3100-Y-10.0-75k  &   17.3 &   20.5 &   17.5 \\
    3100-Y-10.0-100k &   14.5 &   17.2 &   14.7 \\
    3100-J-1.0-75k   &    7.7 &   17.9 &    8.1 \\
    3100-J-1.0-100k  &    5.0 &   11.2 &    5.2 \\
    3100-J-5.0-75k   &   13.9 &   34.0 &   14.7 \\
    3100-J-5.0-100k  &   11.0 &   26.6 &   11.6 \\
    3100-J-10.0-75k  &   27.7 &   71.5 &   29.3 \\
    3100-J-10.0-100k &   23.1 &   59.4 &   24.4 \\
    3100-H-1.0-75k   &    5.8 &   10.4 &    6.0 \\
    3100-H-1.0-100k  &    4.1 &    7.4 &    4.3 \\
    3100-H-5.0-75k   &    9.3 &   16.4 &    9.6 \\
    3100-H-5.0-100k  &    7.4 &   13.1 &    7.7 \\
    3100-H-10.0-75k  &   17.3 &   30.6 &   17.9 \\
    3100-H-10.0-100k &   14.5 &   25.6 &   15.0 \\
    3100-K-1.0-75k   &    7.7 &   25.2 &    8.2 \\
    3100-K-1.0-100k  &    5.4 &   17.6 &    5.7 \\
    3100-K-5.0-75k   &   13.1 &   41.3 &   13.9 \\
    3100-K-5.0-100k  &   10.3 &   32.8 &   11.0 \\
    3100-K-10.0-75k  &   26.0 &   77.9 &   27.7 \\
    3100-K-10.0-100k &   21.7 &   65.4 &   23.1 \\
    3000-Z-1.0-75k   &    3.6 &    6.8 &    3.8 \\
    3000-Z-1.0-100k  &    2.3 &    4.2 &    2.4 \\
    3000-Z-5.0-75k   &    7.1 &   12.9 &    7.4 \\
    3000-Z-5.0-100k  &    5.5 &   10.0 &    5.7 \\
    3000-Z-10.0-75k  &   15.3 &   27.1 &   15.9 \\
    3000-Z-10.0-100k &   12.7 &   22.5 &   13.2 \\
    3000-Y-1.0-75k   &    5.0 &    6.3 &    5.1 \\
    3000-Y-1.0-100k  &    3.3 &    4.2 &    3.4 \\
    3000-Y-5.0-75k   &    8.6 &   10.4 &    8.7 \\
    3000-Y-5.0-100k  &    6.8 &    8.3 &    6.9 \\
    3000-Y-10.0-75k  &   15.7 &   18.6 &   15.9 \\
    3000-Y-10.0-100k &   13.2 &   15.7 &   13.3 \\
    3000-J-1.0-75k   &    7.9 &   20.4 &    8.3 \\
    3000-J-1.0-100k  &    5.1 &   13.1 &    5.4 \\
    3000-J-5.0-75k   &   13.9 &   37.0 &   14.7 \\
    3000-J-5.0-100k  &   11.0 &   29.1 &   11.6 \\
    3000-J-10.0-75k  &   26.7 &   72.6 &   28.3 \\
    3000-J-10.0-100k &   22.3 &   60.8 &   23.7 \\
    3000-H-1.0-75k   &    5.6 &   10.3 &    5.8 \\
    3000-H-1.0-100k  &    3.9 &    7.3 &    4.1 \\
    3000-H-5.0-75k   &    9.1 &   16.5 &    9.4 \\
    3000-H-5.0-100k  &    7.2 &   13.2 &    7.5 \\
    3000-H-10.0-75k  &   17.2 &   31.4 &   17.8 \\
    3000-H-10.0-100k &   14.4 &   26.3 &   14.9 \\
    3000-K-1.0-75k   &    7.2 &   23.5 &    7.7 \\
    3000-K-1.0-100k  &    5.0 &   16.4 &    5.3 \\
    3000-K-5.0-75k   &   12.2 &   38.6 &   13.0 \\
    3000-K-5.0-100k  &    9.6 &   30.7 &   10.2 \\
    3000-K-10.0-75k  &   24.2 &   73.0 &   25.8 \\
    3000-K-10.0-100k &   20.3 &   61.2 &   21.6 \\
    2900-Z-1.0-75k   &    3.3 &    6.2 &    3.4 \\
    2900-Z-1.0-100k  &    2.0 &    3.9 &    2.1 \\
    2900-Z-5.0-75k   &    6.4 &   11.8 &    6.6 \\
    2900-Z-5.0-100k  &    4.9 &    9.2 &    5.1 \\
    2900-Z-10.0-75k  &   13.8 &   25.1 &   14.4 \\
    2900-Z-10.0-100k &   11.4 &   20.8 &   11.9 \\
    2900-Y-1.0-75k   &    4.6 &    5.8 &    4.6 \\
    2900-Y-1.0-100k  &    3.0 &    3.9 &    3.0 \\
    2900-Y-5.0-75k   &    7.8 &    9.6 &    7.9 \\
    2900-Y-5.0-100k  &    6.2 &    7.6 &    6.3 \\
    2900-Y-10.0-75k  &   14.2 &   16.8 &   14.3 \\
    2900-Y-10.0-100k &   11.9 &   14.2 &   12.0 \\
    2900-J-1.0-75k   &    7.1 &   18.6 &    7.5 \\
    2900-J-1.0-100k  &    4.6 &   12.0 &    4.9 \\
    2900-J-5.0-75k   &   12.5 &   33.5 &   13.3 \\
    2900-J-5.0-100k  &    9.9 &   26.4 &   10.5 \\
    2900-J-10.0-75k  &   24.0 &   64.3 &   25.5 \\
    2900-J-10.0-100k &   20.1 &   53.9 &   21.3 \\
    2900-H-1.0-75k   &    5.3 &    9.9 &    5.4 \\
    2900-H-1.0-100k  &    3.7 &    6.9 &    3.8 \\
    2900-H-5.0-75k   &    8.6 &   16.1 &    8.9 \\
    2900-H-5.0-100k  &    6.9 &   12.8 &    7.1 \\
    2900-H-10.0-75k  &   16.5 &   31.2 &   17.1 \\
    2900-H-10.0-100k &   13.8 &   26.1 &   14.3 \\
    2900-K-1.0-75k   &    6.6 &   21.6 &    7.0 \\
    2900-K-1.0-100k  &    4.5 &   15.0 &    4.8 \\
    2900-K-5.0-75k   &   11.1 &   35.5 &   11.8 \\
    2900-K-5.0-100k  &    8.8 &   28.3 &    9.4 \\
    2900-K-10.0-75k  &   22.1 &   67.3 &   23.6 \\
    2900-K-10.0-100k &   18.5 &   56.5 &   19.7 \\
    2800-Z-1.0-75k   &    2.9 &    5.7 &    3.0 \\
    2800-Z-1.0-100k  &    1.8 &    3.6 &    1.9 \\
    2800-Z-5.0-75k   &    5.7 &   10.9 &    5.9 \\
    2800-Z-5.0-100k  &    4.4 &    8.5 &    4.6 \\
    2800-Z-10.0-75k  &   12.4 &   23.3 &   13.0 \\
    2800-Z-10.0-100k &   10.3 &   19.3 &   10.7 \\
    2800-Y-1.0-75k   &    4.1 &    5.3 &    4.2 \\
    2800-Y-1.0-100k  &    2.7 &    3.5 &    2.7 \\
    2800-Y-5.0-75k   &    7.1 &    8.7 &    7.2 \\
    2800-Y-5.0-100k  &    5.6 &    7.0 &    5.7 \\
    2800-Y-10.0-75k  &   12.7 &   15.2 &   12.9 \\
    2800-Y-10.0-100k &   10.7 &   12.8 &   10.8 \\
    2800-J-1.0-75k   &    6.1 &   16.0 &    6.5 \\
    2800-J-1.0-100k  &    4.0 &   10.2 &    4.2 \\
    2800-J-5.0-75k   &   10.9 &   28.9 &   11.5 \\
    2800-J-5.0-100k  &    8.6 &   22.8 &    9.1 \\
    2800-J-10.0-75k  &   20.9 &   55.0 &   22.1 \\
    2800-J-10.0-100k &   17.5 &   46.1 &   18.5 \\
    2800-H-1.0-75k   &    4.8 &    9.3 &    5.0 \\
    2800-H-1.0-100k  &    3.3 &    6.4 &    3.5 \\
    2800-H-5.0-75k   &    8.0 &   15.3 &    8.2 \\
    2800-H-5.0-100k  &    6.3 &   12.1 &    6.5 \\
    2800-H-10.0-75k  &   15.3 &   29.9 &   15.9 \\
    2800-H-10.0-100k &   12.8 &   25.0 &   13.3 \\
    2800-K-1.0-75k   &    5.9 &   19.5 &    6.3 \\
    2800-K-1.0-100k  &    4.0 &   13.5 &    4.3 \\
    2800-K-5.0-75k   &   10.0 &   32.2 &   10.6 \\
    2800-K-5.0-100k  &    7.9 &   25.6 &    8.4 \\
    2800-K-10.0-75k  &   19.8 &   60.9 &   21.1 \\
    2800-K-10.0-100k &   16.6 &   51.1 &   17.7 \\
    2700-Z-1.0-75k   &    2.6 &    5.3 &    2.7 \\
    2700-Z-1.0-100k  &    1.6 &    3.3 &    1.7 \\
    2700-Z-5.0-75k   &    5.2 &   10.2 &    5.4 \\
    2700-Z-5.0-100k  &    4.0 &    7.9 &    4.2 \\
    2700-Z-10.0-75k  &   11.3 &   21.9 &   11.8 \\
    2700-Z-10.0-100k &    9.3 &   18.1 &    9.7 \\
    2700-Y-1.0-75k   &    3.7 &    4.8 &    3.7 \\
    2700-Y-1.0-100k  &    2.4 &    3.2 &    2.4 \\
    2700-Y-5.0-75k   &    6.4 &    8.0 &    6.5 \\
    2700-Y-5.0-100k  &    5.1 &    6.4 &    5.2 \\
    2700-Y-10.0-75k  &   11.5 &   13.8 &   11.6 \\
    2700-Y-10.0-100k &    9.7 &   11.6 &    9.8 \\
    2700-J-1.0-75k   &    5.2 &   13.6 &    5.5 \\
    2700-J-1.0-100k  &    3.4 &    8.7 &    3.5 \\
    2700-J-5.0-75k   &    9.3 &   24.9 &    9.9 \\
    2700-J-5.0-100k  &    7.4 &   19.6 &    7.8 \\
    2700-J-10.0-75k  &   18.0 &   47.1 &   19.1 \\
    2700-J-10.0-100k &   15.1 &   39.5 &   16.0 \\
    2700-H-1.0-75k   &    4.3 &    8.3 &    4.4 \\
    2700-H-1.0-100k  &    3.0 &    5.8 &    3.1 \\
    2700-H-5.0-75k   &    7.1 &   13.9 &    7.4 \\
    2700-H-5.0-100k  &    5.7 &   11.0 &    5.9 \\
    2700-H-10.0-75k  &   13.8 &   27.6 &   14.3 \\
    2700-H-10.0-100k &   11.5 &   23.1 &   12.0 \\
    2700-K-1.0-75k   &    5.2 &   17.2 &    5.5 \\
    2700-K-1.0-100k  &    3.5 &   11.9 &    3.8 \\
    2700-K-5.0-75k   &    8.8 &   28.5 &    9.4 \\
    2700-K-5.0-100k  &    6.9 &   22.7 &    7.4 \\
    2700-K-10.0-75k  &   17.4 &   54.0 &   18.6 \\
    2700-K-10.0-100k &   14.6 &   45.3 &   15.5 \\
    2600-Z-1.0-75k   &    2.5 &    5.2 &    2.6 \\
    2600-Z-1.0-100k  &    1.5 &    3.2 &    1.6 \\
    2600-Z-5.0-75k   &    4.8 &    9.9 &    5.1 \\
    2600-Z-5.0-100k  &    3.8 &    7.8 &    3.9 \\
    2600-Z-10.0-75k  &   10.5 &   21.3 &   11.0 \\
    2600-Z-10.0-100k &    8.7 &   17.6 &    9.1 \\
    2600-Y-1.0-75k   &    3.3 &    4.4 &    3.4 \\
    2600-Y-1.0-100k  &    2.1 &    2.9 &    2.2 \\
    2600-Y-5.0-75k   &    5.9 &    7.4 &    6.0 \\
    2600-Y-5.0-100k  &    4.7 &    5.9 &    4.7 \\
    2600-Y-10.0-75k  &   10.5 &   12.7 &   10.6 \\
    2600-Y-10.0-100k &    8.8 &   10.7 &    8.9 \\
    2600-J-1.0-75k   &    4.4 &   11.7 &    4.7 \\
    2600-J-1.0-100k  &    2.8 &    7.4 &    3.0 \\
    2600-J-5.0-75k   &    8.0 &   21.6 &    8.5 \\
    2600-J-5.0-100k  &    6.3 &   17.0 &    6.7 \\
    2600-J-10.0-75k  &   15.6 &   40.8 &   16.5 \\
    2600-J-10.0-100k &   13.0 &   34.2 &   13.8 \\
    2600-H-1.0-75k   &    3.7 &    7.4 &    3.9 \\
    2600-H-1.0-100k  &    2.6 &    5.1 &    2.7 \\
    2600-H-5.0-75k   &    6.3 &   12.4 &    6.5 \\
    2600-H-5.0-100k  &    5.0 &    9.8 &    5.2 \\
    2600-H-10.0-75k  &   12.2 &   24.8 &   12.7 \\
    2600-H-10.0-100k &   10.2 &   20.7 &   10.6 \\
    2600-K-1.0-75k   &    4.4 &   14.4 &    4.7 \\
    2600-K-1.0-100k  &    3.0 &    9.9 &    3.2 \\
    2600-K-5.0-75k   &    7.6 &   24.3 &    8.1 \\
    2600-K-5.0-100k  &    6.0 &   19.3 &    6.4 \\
    2600-K-10.0-75k  &   15.0 &   46.5 &   16.0 \\
    2600-K-10.0-100k &   12.5 &   39.1 &   13.4 \\
    2500-Z-1.0-75k   &    2.5 &    5.5 &    2.6 \\
    2500-Z-1.0-100k  &    1.6 &    3.5 &    1.6 \\
    2500-Z-5.0-75k   &    4.8 &   10.4 &    5.0 \\
    2500-Z-5.0-100k  &    3.7 &    8.2 &    3.9 \\
    2500-Z-10.0-75k  &   10.3 &   22.0 &   10.9 \\
    2500-Z-10.0-100k &    8.6 &   18.3 &    9.0 \\
    2500-Y-1.0-75k   &    3.1 &    4.2 &    3.1 \\
    2500-Y-1.0-100k  &    2.0 &    2.7 &    2.0 \\
    2500-Y-5.0-75k   &    5.5 &    7.1 &    5.6 \\
    2500-Y-5.0-100k  &    4.4 &    5.7 &    4.4 \\
    2500-Y-10.0-75k  &    9.9 &   12.2 &   10.0 \\
    2500-Y-10.0-100k &    8.4 &   10.3 &    8.5 \\
    2500-J-1.0-75k   &    3.7 &   10.0 &    3.9 \\
    2500-J-1.0-100k  &    2.4 &    6.3 &    2.5 \\
    2500-J-5.0-75k   &    6.8 &   18.7 &    7.3 \\
    2500-J-5.0-100k  &    5.4 &   14.7 &    5.7 \\
    2500-J-10.0-75k  &   13.5 &   35.7 &   14.3 \\
    2500-J-10.0-100k &   11.3 &   29.9 &   11.9 \\
    2500-H-1.0-75k   &    3.3 &    6.5 &    3.4 \\
    2500-H-1.0-100k  &    2.3 &    4.4 &    2.4 \\
    2500-H-5.0-75k   &    5.5 &   11.0 &    5.7 \\
    2500-H-5.0-100k  &    4.4 &    8.7 &    4.5 \\
    2500-H-10.0-75k  &   10.7 &   21.9 &   11.1 \\
    2500-H-10.0-100k &    9.0 &   18.3 &    9.3 \\
    2500-K-1.0-75k   &    3.8 &   11.8 &    4.0 \\
    2500-K-1.0-100k  &    2.6 &    8.0 &    2.7 \\
    2500-K-5.0-75k   &    6.5 &   20.5 &    6.9 \\
    2500-K-5.0-100k  &    5.1 &   16.2 &    5.5 \\
    2500-K-10.0-75k  &   12.8 &   39.8 &   13.7 \\
    2500-K-10.0-100k &   10.8 &   33.4 &    11.5
    \label{tab:nirps_precisions}
\end{longtable}


%!TEX root = ../../thesis.tex
% SPIRou Table

% Manually create table instead of load from file using \DTLsetseparator.
%\clearpage{}
\begin{longtable}{crrr}
    \caption{RV precisions calculated for the {SPIRou} {ETC}.
        The same {PHOENIX-ACES} parameter combinations as \cref{tab:rv_aces_btsettl} but with the {SNR=100} relative to the centre of each band individually.}\\
    \hline\hline
    Simulation & \(\sigma_{RV}\)(Cond.\,1) & \(\sigma_{RV}\)(Cond.\,2) & \(\sigma_{RV}\)(Cond.\,3)\\
    ($\teff$-Band-$v.\sin{i}$-R) & [m/s] & [m/s] & [m/s] \\
    \hline
    \endfirsthead
    \caption{continued.}\\
    \hline\hline
    Simulation  & \(\sigma_{RV}\)(Cond.\,1) & \(\sigma_{RV}\)(Cond.\,2) & \(\sigma_{RV}\)(Cond.\,3) \\
    \hline
    \endhead
    \hline
    \endfoot

    3900-Z-1.0-60k   &    9.9 &   16.0 &   10.3 \\
    3900-Z-1.0-80k   &    6.6 &   10.7 &    6.8 \\
    3900-Z-1.0-100k  &    4.9 &    8.1 &    5.1 \\
    3900-Z-5.0-60k   &   14.5 &   23.2 &   15.0 \\
    3900-Z-5.0-80k   &   11.2 &   18.0 &   11.6 \\
    3900-Z-5.0-100k  &    9.4 &   15.2 &    9.7 \\
    3900-Z-10.0-60k  &   25.0 &   39.4 &   25.8 \\
    3900-Z-10.0-80k  &   20.7 &   32.9 &   21.4 \\
    3900-Z-10.0-100k &   18.1 &   28.8 &   18.7 \\
    3900-Y-1.0-60k   &   10.0 &   11.9 &   10.1 \\
    3900-Y-1.0-80k   &    6.2 &    7.4 &    6.3 \\
    3900-Y-1.0-100k  &    4.4 &    5.3 &    4.5 \\
    3900-Y-5.0-60k   &   15.9 &   19.0 &   16.1 \\
    3900-Y-5.0-80k   &   12.0 &   14.3 &   12.1 \\
    3900-Y-5.0-100k  &   10.0 &   11.8 &   10.1 \\
    3900-Y-10.0-60k  &   31.6 &   37.8 &   32.0 \\
    3900-Y-10.0-80k  &   25.8 &   30.8 &   26.1 \\
    3900-Y-10.0-100k &   22.4 &   26.7 &   22.6 \\
    3900-H-1.0-60k   &    7.3 &   11.5 &    7.4 \\
    3900-H-1.0-80k   &    5.1 &    8.1 &    5.2 \\
    3900-H-1.0-100k  &    4.0 &    6.4 &    4.1 \\
    3900-H-5.0-60k   &   10.2 &   16.0 &   10.4 \\
    3900-H-5.0-80k   &    7.9 &   12.5 &    8.1 \\
    3900-H-5.0-100k  &    6.7 &   10.5 &    6.8 \\
    3900-H-10.0-60k  &   17.5 &   27.3 &   17.9 \\
    3900-H-10.0-80k  &   14.5 &   22.7 &   14.8 \\
    3900-H-10.0-100k &   12.7 &   19.8 &   13.0 \\
    3900-K-1.0-60k   &   10.5 &   46.3 &   11.3 \\
    3900-K-1.0-80k   &    7.1 &   31.8 &    7.6 \\
    3900-K-1.0-100k  &    5.4 &   24.5 &    5.8 \\
    3900-K-5.0-60k   &   15.7 &   65.4 &   16.8 \\
    3900-K-5.0-80k   &   12.0 &   50.8 &   12.8 \\
    3900-K-5.0-100k  &   10.1 &   42.9 &   10.8 \\
    3900-K-10.0-60k  &   28.5 &  112.6 &   30.4 \\
    3900-K-10.0-80k  &   23.6 &   93.1 &   25.2 \\
    3900-K-10.0-100k &   20.6 &   81.2 &   22.0 \\
    3500-Z-1.0-60k   &    8.3 &   13.7 &    8.6 \\
    3500-Z-1.0-80k   &    5.2 &    8.7 &    5.4 \\
    3500-Z-1.0-100k  &    3.7 &    6.3 &    3.8 \\
    3500-Z-5.0-60k   &   12.9 &   20.9 &   13.3 \\
    3500-Z-5.0-80k   &    9.8 &   16.0 &   10.1 \\
    3500-Z-5.0-100k  &    8.2 &   13.4 &    8.5 \\
    3500-Z-10.0-60k  &   24.0 &   38.1 &   24.8 \\
    3500-Z-10.0-80k  &   19.7 &   31.5 &   20.4 \\
    3500-Z-10.0-100k &   17.2 &   27.5 &   17.7 \\
    3500-Y-1.0-60k   &    8.7 &   10.3 &    8.8 \\
    3500-Y-1.0-80k   &    5.3 &    6.3 &    5.4 \\
    3500-Y-1.0-100k  &    3.7 &    4.5 &    3.8 \\
    3500-Y-5.0-60k   &   14.1 &   16.7 &   14.3 \\
    3500-Y-5.0-80k   &   10.6 &   12.5 &   10.7 \\
    3500-Y-5.0-100k  &    8.8 &   10.4 &    8.9 \\
    3500-Y-10.0-60k  &   28.5 &   33.9 &   28.8 \\
    3500-Y-10.0-80k  &   23.2 &   27.6 &   23.5 \\
    3500-Y-10.0-100k &   20.1 &   23.9 &   20.3 \\
    3500-H-1.0-60k   &    7.6 &   12.2 &    7.8 \\
    3500-H-1.0-80k   &    5.4 &    8.7 &    5.5 \\
    3500-H-1.0-100k  &    4.2 &    6.9 &    4.3 \\
    3500-H-5.0-60k   &   10.5 &   16.7 &   10.7 \\
    3500-H-5.0-80k   &    8.2 &   13.1 &    8.4 \\
    3500-H-5.0-100k  &    6.9 &   11.1 &    7.1 \\
    3500-H-10.0-60k  &   17.6 &   28.0 &   18.1 \\
    3500-H-10.0-80k  &   14.6 &   23.3 &   15.0 \\
    3500-H-10.0-100k &   12.8 &   20.4 &   13.1 \\
    3500-K-1.0-60k   &    9.9 &   35.8 &   10.6 \\
    3500-K-1.0-80k   &    6.6 &   24.0 &    7.0 \\
    3500-K-1.0-100k  &    5.0 &   18.2 &    5.3 \\
    3500-K-5.0-60k   &   14.8 &   52.3 &   15.8 \\
    3500-K-5.0-80k   &   11.3 &   40.2 &   12.1 \\
    3500-K-5.0-100k  &    9.5 &   33.8 &   10.1 \\
    3500-K-10.0-60k  &   27.2 &   92.1 &   29.1 \\
    3500-K-10.0-80k  &   22.5 &   76.0 &   24.0 \\
    3500-K-10.0-100k &   19.6 &   66.3 &   20.9 \\
    2800-Z-1.0-60k   &    3.8 &    7.3 &    3.9 \\
    2800-Z-1.0-80k   &    2.3 &    4.5 &    2.4 \\
    2800-Z-1.0-100k  &    1.6 &    3.2 &    1.7 \\
    2800-Z-5.0-60k   &    6.1 &   11.6 &    6.4 \\
    2800-Z-5.0-80k   &    4.6 &    8.8 &    4.8 \\
    2800-Z-5.0-100k  &    3.8 &    7.3 &    4.0 \\
    2800-Z-10.0-60k  &   12.2 &   22.8 &   12.7 \\
    2800-Z-10.0-80k  &    9.9 &   18.7 &   10.4 \\
    2800-Z-10.0-100k &    8.6 &   16.2 &    9.0 \\
    2800-Y-1.0-60k   &    5.6 &    7.1 &    5.7 \\
    2800-Y-1.0-80k   &    3.6 &    4.7 &    3.7 \\
    2800-Y-1.0-100k  &    2.6 &    3.4 &    2.7 \\
    2800-Y-5.0-60k   &    8.3 &   10.1 &    8.4 \\
    2800-Y-5.0-80k   &    6.4 &    7.9 &    6.5 \\
    2800-Y-5.0-100k  &    5.4 &    6.7 &    5.5 \\
    2800-Y-10.0-60k  &   13.8 &   16.4 &   13.9 \\
    2800-Y-10.0-80k  &   11.4 &   13.7 &   11.6 \\
    2800-Y-10.0-100k &   10.0 &   12.0 &   10.1 \\
    2800-H-1.0-60k   &    6.5 &   12.6 &    6.8 \\
    2800-H-1.0-80k   &    4.4 &    8.5 &    4.6 \\
    2800-H-1.0-100k  &    3.3 &    6.4 &    3.5 \\
    2800-H-5.0-60k   &    9.6 &   18.3 &    9.9 \\
    2800-H-5.0-80k   &    7.3 &   14.1 &    7.6 \\
    2800-H-5.0-100k  &    6.2 &   11.8 &    6.4 \\
    2800-H-10.0-60k  &   17.2 &   33.5 &   17.8 \\
    2800-H-10.0-80k  &   14.2 &   27.6 &   14.7 \\
    2800-H-10.0-100k &   12.4 &   24.1 &   12.8 \\
    2800-K-1.0-60k   &    6.3 &   20.8 &    6.7 \\
    2800-K-1.0-80k   &    4.2 &   13.9 &    4.5 \\
    2800-K-1.0-100k  &    3.1 &   10.5 &    3.4 \\
    2800-K-5.0-60k   &    9.5 &   30.4 &   10.1 \\
    2800-K-5.0-80k   &    7.2 &   23.3 &    7.7 \\
    2800-K-5.0-100k  &    6.0 &   19.6 &    6.4 \\
    2800-K-10.0-60k  &   17.5 &   53.7 &   18.7 \\
    2800-K-10.0-80k  &   14.4 &   44.4 &   15.4 \\
    2800-K-10.0-100k &   12.6 &   38.8 &   13.4 \\
    2600-Z-1.0-60k   &    3.0 &    6.2 &    3.1 \\
    2600-Z-1.0-80k   &    1.8 &    3.8 &    1.9 \\
    2600-Z-1.0-100k  &    1.3 &    2.7 &    1.4 \\
    2600-Z-5.0-60k   &    4.8 &    9.8 &    5.0 \\
    2600-Z-5.0-80k   &    3.6 &    7.4 &    3.8 \\
    2600-Z-5.0-100k  &    3.0 &    6.2 &    3.2 \\
    2600-Z-10.0-60k  &    9.5 &   19.2 &    9.9 \\
    2600-Z-10.0-80k  &    7.7 &   15.7 &    8.1 \\
    2600-Z-10.0-100k &    6.7 &   13.6 &    7.1 \\
    2600-Y-1.0-60k   &    4.4 &    5.8 &    4.5 \\
    2600-Y-1.0-80k   &    2.8 &    3.8 &    2.9 \\
    2600-Y-1.0-100k  &    2.0 &    2.7 &    2.1 \\
    2600-Y-5.0-60k   &    6.5 &    8.2 &    6.6 \\
    2600-Y-5.0-80k   &    5.1 &    6.4 &    5.1 \\
    2600-Y-5.0-100k  &    4.3 &    5.4 &    4.3 \\
    2600-Y-10.0-60k  &   10.7 &   12.9 &   10.8 \\
    2600-Y-10.0-80k  &    8.9 &   10.8 &    9.0 \\
    2600-Y-10.0-100k &    7.8 &    9.5 &    7.9 \\
    2600-H-1.0-60k   &    5.2 &   10.3 &    5.4 \\
    2600-H-1.0-80k   &    3.5 &    6.9 &    3.6 \\
    2600-H-1.0-100k  &    2.7 &    5.2 &    2.8 \\
    2600-H-5.0-60k   &    7.7 &   15.3 &    8.0 \\
    2600-H-5.0-80k   &    5.9 &   11.7 &    6.1 \\
    2600-H-5.0-100k  &    5.0 &    9.8 &    5.2 \\
    2600-H-10.0-60k  &   13.9 &   28.4 &   14.5 \\
    2600-H-10.0-80k  &   11.5 &   23.3 &   11.9 \\
    2600-H-10.0-100k &   10.0 &   20.4 &   10.4 \\
    2600-K-1.0-60k   &    5.0 &   16.2 &    5.3 \\
    2600-K-1.0-80k   &    3.3 &   10.7 &    3.5 \\
    2600-K-1.0-100k  &    2.5 &    8.0 &    2.6 \\
    2600-K-5.0-60k   &    7.4 &   23.9 &    7.9 \\
    2600-K-5.0-80k   &    5.7 &   18.3 &    6.1 \\
    2600-K-5.0-100k  &    4.8 &   15.4 &    5.1 \\
    2600-K-10.0-60k  &   13.7 &   42.5 &   14.6 \\
    2600-K-10.0-80k  &   11.3 &   35.1 &   12.1 \\
    2600-K-10.0-100k &    9.9 &   30.7 &   10.5 \\
    \label{tab:spirou_precisions}
\end{longtable}



%%\longtab[2]{
%\begin{longtable}{cccc}
%    2
%    \caption{Correction to results of the simulations described in Table\,A.1 of~\citet{figueira_radiasl_016}.}\label{TableData}
%    \\
%    \hline\hline
%    Simulation & \(\sigma_{RV}\)(Cond.\,1) & \(\sigma_{RV}\)(Cond.\,2) & \(\sigma_{RV}\)(Cond.\,3) \\
%    (SpTp - Band - $v.\sin{i}$ - R) & [m/s]& [m/s] & [m/s]\\
%    \hline
%    \endfirsthead
%    \caption{continued.}\\
%    \hline\hline
%    Simulation & \(\sigma_{RV}\)(Cond.\,1) & \(\sigma_{RV}\)(Cond.\,2) & \(\sigma_{RV}\)(Cond.\,3) \\
%    \hline
%    \endhead
%    \hline
%    \endfoot
%    \ldots{}&\ldots{}.&\ldots{}&precision\_results\_2017\_ref\_band-J\_snr-100.0.dat\\
%    %\DTLforeach*{precisions}{\sim=id, \condone=prec1, \condtwo=prec2, \condthree=prec3}{%  \DTLiffirstrow{}{\\}  if first row don't add a new line at start.
%    %\DTLiffirstrow{}{\\}\sim & \condone & \condtwo & \condthree
%    %}
%
%\end{longtable}
%%}

