%%%%%%%%%%%%%%%%%%%%%%%%%%%%%%%%%%%%%%%%%%%%%%%%%%
% Basic setup. Most papers should leave these options alone.
\documentclass[fleqn,usenatbib]{mnras}

% MNRAS is set in Times font. If you don't have this installed (most LaTeX
% installations will be fine) or prefer the old Computer Modern fonts, comment
% out the following line
\usepackage{newtxtext,newtxmath}
% Depending on your LaTeX fonts installation, you might get better results with one of these:
%\usepackage{mathptmx}
%\usepackage{txfonts}

% Use vector fonts, so it zooms properly in on-screen viewing software
% Don't change these lines unless you know what you are doing
\usepackage[T1]{fontenc}
\usepackage{ae,aecompl}


%%%%% AUTHORS - PLACE YOUR OWN PACKAGES HERE %%%%%

% Only include extra packages if you really need them. Common packages are:
\usepackage{graphicx}	% Including figure files
\usepackage{amsmath}	% Advanced maths commands
%\usepackage{amssymb}	% Extra maths symbols
\usepackage[para, flushleft]{threeparttable}
\usepackage{hyperref}
\usepackage{natbib}
\usepackage{booktabs}
\usepackage{color}
\usepackage{fix-cm}
\usepackage{caption}   % Caption is required to avoid errors.
%%%%%%%%%%%%%%%%%%%%%%%%%%%%%%%%%%%%%%%%%%%%%%%%%%

%%%%% AUTHORS - PLACE YOUR OWN COMMANDS HERE %%%%%

% Please keep new commands to a minimum, and use \newcommand not \def to avoid
% overwriting existing commands. Example:
%\newcommand{\pcm}{\,cm$^{-2}$}	% per cm-squared
\newcommand{\kmps}{\,\si{\kilo\meter\per\second}} % kilometers per second
\newcommand{\cmps}{\,\si{\centi\meter\per\second}} % centimeters per second
\newcommand{\mps}{\,\si{\meter\per\second}} % meters per second
\newcommand{\nm}{\,\si{\nano\meter}} % nanometer
\newcommand{\um}{\,\si{\micro\meter}} % micrometer
\newcommand{\K}{\,\si{\kelvin}} % kelvin
\newcommand{\Gyr}{\,\si{\giga{yr}}} % Gigayear


%% word shortcuts
\newcommand{\thar}{Th-Ar}
\newcommand{\nir}{nIR}
\newcommand{\logg}{logg}
\newcommand{\feh}{$\textrm{[Fe/H]}$}
\newcommand{\teff}{\(T_\mathrm{eff}\)}
\newcommand*\bl{\color{blue}}
\newcommand*\rd{\color{red}}
%%%%%%%%%%%%%%%%%%%%%%%%%%%%%%%%%%%%%%%%%%%%%%%%%%

%%%%%%%%%%%%%%%%%%% TITLE PAGE %%%%%%%%%%%%%%%%%%%

% Title of the paper, and the short title which is used in the headers.
% Keep the title short and informative.
%\title[Short title, max. 45 characters]{MNRAS \LaTeXe\ template -- title goes here}
\title[Towards the nIR detection of BD companions]{Towards the near-infrared detection of brown dwarf companions: Exploring methods to detect low mass stellar companions from blended spectra \thanks{Based on observations collected at the European Organisation for Astronomical Research in the Southern Hemisphere under ESO programme 089.C-0977(A)}}
%\subtitle{Exploring methods to detect low mass stellar companions from blended spectra}
% The list of authors, and the short list which is used in the headers.
% If you need two or more lines of authors, add an extra line using \newauthor

\author[J. J. Neal et al.]{
J. J. Neal$^{1,2}$\thanks{E-mail: jason.neal@astro.up.pt}
P. Figueira$^{3,1}$
N. C. Santos$^{1,2}$
C. Melo$^{3}$
\\
% List of institutions
$^{1}$Instituto de Astrof\'{\i}sica e Ci\^encias do Espa\c{c}o, CAUP, Universidade do Porto, Rua das Estrelas, PT4150-762 Porto, Portuga\\l
$^{2}$Departamento de F\'{\i}sica e Astronomia, Faculdade de Ci\^{e}ncias, Universidade do Porto, Portugal\\
$^{3}$European Southern Observatory, Alonso de C\'{o}rdova 3107, Vitacura, Casilla 19001, Santiago 19, Chile\\
}

% These dates will be filled out by the publisher
\date{Accepted XXX. Received YYY; in original form ZZZ}

% Enter the current year, for the copyright statements etc.
\pubyear{2018}

% Don't change these lines
\begin{document}
\label{firstpage}
\pagerange{\pageref{firstpage}--\pageref{lastpage}}
\maketitle

% Abstract of the paper
\begin{abstract}
{\rd{} In this paper we attempt to directly detect the near-infrared spectrum of candidate brown dwarf (BD) companions around FGK stars to assert or discard their stellar nature. We explore two different methods to probe the faint spectra of BD companions. The first technique involves the direct subtraction of two observations shifted to mutually cancel out the spectra of the host star. The second technique applies a \(\chi^{2}\) fit to the individual observations of a synthetic binary model comprised of two PHOENIX-ACES spectra. The observed spectra are wavelength calibrated and corrected for the atmospheric absorption with the aid of synthetic telluric models. The direct subtraction method failed to detect the signature of the companions due to poor observational constraints (the observations were insufficiently separated in radial velocity). The \(\chi^{2}\) technique developed here is able to fit the component for the host star but unable to successfully detect the faint spectra of even the largest companion in our sample. We explore how the companion recovery fitting performs on simulated observations and discuss reasons for the non-detection observed. From the injection-recovery analysis, this technique in its current form, is insufficient to recover a companion below 3800\K{} which corresponds to an upper mass limit for the companions around 600~\Mjup{}. This work highlights the challenges in the spectral detection of faint companions. We explore the limitations of the direct subtraction method in the case of small {RV} separation, and companion detection with synthetic models at low companion/host flux ratios below 1\%, made difficult by the discrepancy between observed and synthetic spectra around  2100\nm{}.}
\end{abstract}

\begin{keywords}
brown dwarfs -- binaries: spectroscopic -- infrared: stars -- techniques: spectroscopic -- methods: miscellaneous
\end{keywords}




\section{Motivation and target selection}
\label{sec:motivation}
The work of~\citet{sahlmann_search_2011} identified several candidate brown dwarf companions of FGK stars with \(\rm M_2 \sin{i}\) values > 10~\Mjup{}. Seven candidates {\rd{} from~\citet{sahlmann_search_2011}, which were visible in Period 89,  were selected for  further} observation in order to identify their stellar nature. The target host stars are presented in Table~\ref{tab:starparams} with their stellar parameters, while the companion orbital parameters are provided in Table~\ref{tab:orbitparams}.

We note that some of these orbital parameters have been refined in the literature since observations took place. For example three candidates have had their masses refined in recent works. The companion to {HD 211847} was determined to be a low mass star with an \(\rm M_2=155~\Mjup{}\)~\citep{moutou_eccentricity_2017}, while the companion to {HD 4747} was found to have a mass of \(\rm M_2=60~\Mjup{}\)~\citep{crepp_trends_2016}. The two companions of {HD 202206} (B and c) were found to have masses of \(\rm M_B =93.6~\Mjup{}\) and \(\rm M_c = 17.9~\Mjup{}\), respectively, classifying {HD 202206}c as a ``circumbinary brown dwarf''~\citep{benedict_hd_2017}. These targets with recently refined masses create good benchmarks for us to compare any results of the techniques developed in this paper, and show that these masses do span the BD -- low mass star range. All target companions except {HD 162020} (P=8.4 days) are in (very) long period orbits (P=0.7--38 years) with masses (or \(M\sin{i}\)) greater than 10~\Mjup{}.

\textit{K}-band spectra were obtained with the goal to achieve a direct detection of the companion spectra through the application of a spectral-differential approach. Doing so would enable a further constraint to be placed on their masses. The \textit{K}-band is used to achieve a high contrast relative to the host star, detected in the extreme V-K colour indexes (>7.8).

% Table of stellar parameters
%\input{tables/stellarparameters}

% Table of orbit parameters
%\input{tables/orbitalparameters}

%--------------------------------------------------------------------
\section{Data and data reduction}
\label{sec:data}
In this section we explain the observations and the data reduction process used. We also discuss the models used to correct for atmospheric absorption.


\subsection{CRIRES data}
\label{subsec:CRIRES}
Observations were performed with the CRIRES instrument~\citep{kaeufl_crires_2004} configured to observe a narrow wavelength domain of the \textit{K}-band between 2120--2160\nm{} using the {Ks} and the {Hx5e-2} filters. The slit width of \(0.4\sec \) resulted in an instrumental resolving power of \(\rm R=50\,000\), with no adaptive optics to ensure that the slit was entirely covered by each target. This prevents strong slit illumination variations that could change the shape of spectral lines.

The observations were performed in service mode during period 89 with run ID.~089.C-0977(A) between April and August 2012. An observation is composed of 8 individual spectra each with an integration time of 180 seconds, observed in a ABBAABBA nod cycle pattern to obtain a high (>180) signal-to-noise when combined.

{\rd{} The list of observations obtained with CRIRES are provided in Table~\ref{tab:observations}.  The {SNR} is calculated with the formula \(\textrm{SNR} = \frac{\mu}{\sigma}\) where \(\mu\) and \(\sigma\) are the mean and standard deviation in the continuum of detector 2 between 2130 and 2134 (see Fig.~\ref{fig:spectral_example}).
The estimated {RV} values for the host and companion at the time of each observation are calculated using Equation~\ref{eq:rv_equation} using the best known orbital parameters and the companion masses from Tables~\ref{tab:starparams} and~\ref{tab:orbitparams}. For hosts with multiple companions the {RV} value is for the largest companion only, i.e.\ {HD 202206}B and {HD 168443}c. The {RV} difference between the host and the companion \(rv_2\) = (\(RV_2 - RV_1\)) is parameter used in the binary model from Sect.~\ref{subsubsec:binary-model}. }

%%!TEX root = ../../nir_companions.tex
\todo{rotate table?}
% Table of observations
\begin{table*}
    \small
    \centering
    \begin{threeparttable}[b]

        \caption{Details about the each {CRIRES} observation. The number of artefacts removed in \sref{subsubsec:reductionartefacts} as well as the \snr{} of the combined spectra is provided. The last three columns are the calculated {RV} of both host and largest companion, from the orbital solution, as well as the {RV} difference between the two components.}
        %\begin{tabular}{l c c c c cl cl r@{.}l r@{.}l r@{.}l}
        \begin{tabular}{l c c c c c c | r@{.}l r@{.}l r@{.}l}
            \toprule
            Object & Obs.& Start date  & Filter & Airmass  & Artefacts & \snr{} & \multicolumn{2}{c}{\(RV_1\)} & \multicolumn{2}{c}{\(RV_2\)} & \multicolumn{2}{c}{\(rv_2\)}  \\  % & \(Date \)
            &  \#   & (yyyy-mm-dd hh:mm:ss)  &  & (at start) & {/ 32} & & \multicolumn{2}{c}{\kmps{}} & \multicolumn{2}{c}{\kmps{}} & \multicolumn{2}{c}{\kmps{}}\\ % data ref    % & (JD\(^{\star} \))
            \midrule
            {HD 4747}   & 1 & 2012-07-06 07:36:06 & Ks            & 1.25     & 7 & 340 & $-$0 & 219 & $-$0  & 154 & 0&065 \\ %-1   & 2456\,114.81674
            {HD 162020} & 1 & 2012-07-04 06:23:22 & Ks      & 1.30  & 2 & 127 & $-$28  & 760 & 50 & 785\tnote{a}  & 79&545\tnote{a} \\ %-1   & 2456\,112.76624
            {HD 162020} & 2 & 2012-07-04 06:57:48 & Ks      & 1.44   & 2 & 128 & $-$28  & 717 & 48 & 440\tnote{a} & 77&157\tnote{a} \\ %-2   & 2456\,112.79015
            {HD 167665} & 1 & 2012-07-28 05:00:53 & Hx5e-2  & 1.24  & 7 & 371 & 7      & 581 & 18 & 024\tnote{a} & 10&443\tnote{a} \\ %-1a  & 2456\,136.70895
            {HD 167665} & 2 & 2012-07-28 05:37:27 & Hx5e-2  & 1.39   & 4 & 374 & 7      & 581 & 18 & 025\tnote{a}  & 10&444\tnote{a} \\ %-1b  & 2456\,136.73434
            {HD 167665} & 3 & 2012-08-05 02:54:03 & Hx5e-2  & 1.04   & 4 & 358 & 7      & 575 & 18 & 163\tnote{a} & 10&588\tnote{a} \\ %-2   & 2456\,144.62087
            {HD 168443} & 1 & 2012-08-05 04:29:32 & Ks      & 1.31  & 2& 192 & $-$0   & 121 & 50 & 932\tnote{a,b}  & 51&053\tnote{a,b} \\ %-1   & 2456\,144.68718
            {HD 168443} & 2 & 2012-08-05 04:58:50 & Ks      & 1.47  & 4 & 190 & $-$0   & 121 & 51 & 189 \tnote{a,b} & 51&310\tnote{a,b} \\ %-2   & 2456\,144.70753
            {HD 202206} & 1 & 2012-07-12 06:54:44 & Ks      & 1.01  & 3& 189 & 14   & 843 & 12 & 992\tnote{b}  & -1&851 \\ %-1   & 2456\,120.78801
            {HD 202206} & 2 & 2012-07-13 05:41:40 & J          & 1.01    & 3 & 209 & 14   & 837 & 13 & 065\tnote{b}  & -1&772 \\ %-2   & 2456\,121.73727
            {HD 202206} & 3 & 2012-07-11 08:29:55 & Ks      & 1.15 & 4& 180 & 14   & 849 & 12 & 920\tnote{b}  & -1&929 \\ %-3   & 2456\,119.85411
            {HD 211847} & 1 & 2012-07-06 07:02:57 & Ks      & 1.07  & 4& 272 & 6     & 613 & 7   & 171 & 0& 558\\ %-1   & 2456\,114.79372
            {HD 211847} & 2 & 2012-07-13 06:54:37 & Ks      & 1.05  & 5& 283 & 6     & 614 & 7   & 167 & 0&553 \\ %-2   & 2456\,121.78793
            {HD 30501}  & 1 & 2012-04-07 00:08:29 & Hx5e-2   & 1.60   & 3& 217 & 22   &  372 & 36 & 377 & 14&005 \\ %-1   & 2456\,024.50590
            {HD 30501}  & 2 & 2012-08-01 09:17:30 & Hx5e-2 & 1.42  & 10& 212 & 22   & 505 & 35  & 120 & 12&615 \\ %-2a  & 2456\,140.88716
            {HD 30501}  & 3 & 2012-08-02 08:47:30 & Hx5e-2   & 1.53   & 8& 237 & 22   & 507 &  35 & 102 & 12&595 \\ %-3   & 2456\,141.86633
            {HD 30501}  & 4 & 2012-08-06 09:42:07 & Ks       & 1.28   & 7& 235& 22   & 514 & 35 & 031 & 12&517 \\ %-2b  & 2456\,145.90426
            \bottomrule
            & & & &
        \end{tabular}\label{tab:observations}
        \begin{tablenotes}
            \item  [a]{Maximum {RV} given \mtwosini{} only.}
            \item  [b]{Largest mass companion only.}
        \end{tablenotes}
    \end{threeparttable}
\end{table*}




\subsubsection{Data reduction}
\label{subsubsec:reduction}
\begin{figure}
%    \includegraphics[width=\hsize]{images/Bad_pixel_replacement.pdf} % chktex 8
\caption{Example of an artefact in the optimally extracted spectra. The top panel contains the 8 normalized nod spectra after optimal extraction, while the middle panel shows the rectangular extraction for the exact same spectra. A vertical offset is included between each spectra. A single large spike in the seventh nod (pink) near pixel 230 creates a wide and noticeable artefact in the optimal extraction. The bottom panel shows the difference between a combined spectrum using the optimal nods only and a combined spectrum in which the seventh nod is replaced with its rectangular counterpart. The nod spectra are in observation order from top to bottom.}
\label{fig:nod_artefacts}
\end{figure}

The observations were reduced using a custom reduction pipeline~\citep{figueira_radial_2010}. Written in IRAF's CL\footnote{IRAF is distributed by the National Optical Astronomy Observatories, which are operated by the Association of Universities for Research in Astronomy, {Inc.}, under cooperative agreement with the National Science Foundation.}~\citep{tody_iraf_1993} it provides for automated dark and non-linearity corrections (using the non-linearity coefficients provided by ESO), as well as the flagging and replacement of bad pixels. The images were corrected from sensitivity variations by dividing by a flat-field corrected from the blaze function effect. The nodding pairs were mutually subtracted and the order tracing was fitted by {\rd{} linear or cubic splines selected for each detector}.

There are two types of extraction commonly used. The \emph{rectangular} extraction performs a rectangular aperture sum in the spatial direction while the \emph{optimal} extraction~\citep{horne_optimal_1986} also includes variance weighting to reduce the impact of the noise and deviant pixels on the total flux measurement. {\rd{} The extracted nod-cycle spectra are normalized by dividing each by a polynomial fitted to the continuum.}

At this point one would normally combine all the optimally extracted nod spectra together to improve the signal-to.noise. {\rd{} However, we discovered extended artefacts in the optimally reduced nod spectra. We were not able to identify why these artefacts are large and extended but they occur near the presence of a large cosmic rays or bad pixel spike seen in the rectangular extraction. These artefacts were not observed in previous works using this pipeline in the \textit{H}-band, so there may be a wavelength dependent affect. The variance weighting procedure during the optimal extraction is somehow being heavily affected by these spikes.} An example of this can be seen in Fig.~\ref{fig:nod_artefacts} where a spike in the rectangular extraction corresponds to the large extended features in the optimal extraction. The amplitude of these artefacts created flux deviations in the combined optimally extracted spectra up to \(\sim\)0.5\%.
Therefore, we took measures to remove these artefacts before combining the nod spectra as we are trying to recover companion spectra with expected flux ratios \(\rm {F_2}/{F_1} < 1\% \). {\rd{} Parameters of the pipeline were adjusted to try and remove these artefacts, such as the \(\sigma\) rejection limits and the aperture width with limited success. In a small number of instances allowing the aperture width to be automatically adjusted removed the artefacts.}

All nod spectra for each observation were visually inspected together, as shown in  Fig.~\ref{fig:nod_artefacts}, and any spectra containing artefacts were marked. The optimally extracted nods (top panel)  that were identified were replaced with their rectangularly extracted counter-parts (middle panel). An iterative 4-\(\sigma \) rejection algorithm\footnote{Found at \url{https://github.com/jason-neal/nod_combination}} was applied to the replacement rectangular extractions to remove the erroneous pixels that created the artefacts. The \(\sigma\) value for each pixel was calculated as the standard deviation of the nearest 2 pixels on either side of all 8 nod spectra. The rejected pixels were replaced using linear interpolation along the dispersion axis. {\rd{} Out of the 544 nod spectra from individual detectors, 79 (14\%) contained artefacts and were replaced using this technique.}

For the remainder of the paper we use combined spectra constructed by averaging the 8 nod-cycle spectra together, where some of the optimally extracted spectra have been replaced using the above method. The last panel of Fig.~\ref{fig:nod_artefacts} shows that the difference between the combined optimally extracted spectra and the \emph{combined extraction with replacements}.

The continuum normalization and nod combining steps were also performed using IRAF while the following post reduction procedures and analysis all utilize \emph{Python}. This pipeline was chosen over the ESO CRIRES pipeline because it seemed relatively simple to use, being semi-automated, and appeared to have less bad pixel/cosmic ray artefacts in the extracted spectra. In hindsight this was not the case, with the removal of the extended artefacts that appeared.

% One possible explanation for the artefacts present is an instrumental effect, such as instrument glow. This is known to affect CRIRES and included in the data reduction cookbook~\citep{smoker_very_2012}. These artefacts in the \textit{K}-band spectra were not observed in previous works in the \textit{H}-band using this pipeline and as such may have a wavelength dependent affect, as the higher wavelength nIR is more susceptible to thermal instrument glow.


\
\subsubsection{Telluric correction}
\label{subsec:telluric_correction}
Ground based observations require the removal of the absorption lines introduced by Earth's atmosphere. These observations were first taken in an atmospheric window of the \textit{K}-band in order to reduce the absorption introduced by the atmosphere~\citep{barnes_hd_2008}. To correct for the remaining telluric line contamination the spectra were divided by the TAPAS\citep{bertaux_tapas_2014} atmospheric transmission models for each observation. Synthetic telluric models were used to avoid the observing overhead necessary to perform telluric standard star exposures~\citep{vacca_method_2003}, and they have been demonstrated to be superior in the quality of the correction relative to the telluric standard approach~\citep[e.g.][]{cotton_atmospheric_2014}.

Before the correction, the depth of the telluric lines were re-scaled to match the airmass of the observation using the relation \(\rm T = T^{\beta} \), where \(\rm T\) is the telluric spectrum and \(\beta \) is the airmass ratio between the observation and model. This changed the depth of most absorption lines to match the observations, but does not correctly scale the deeper \(\rm H_{2}O \) lines. The scaled telluric model is interpolated to the wavelengths of the observed spectrum and then used to correct the observed spectra through division, leaving behind a telluric corrected spectra. An example of a telluric corrected spectra is shown in the middle panel of Fig.~\ref{fig:spectral_example}, with the light blue shading indicating where the deeper telluric lines were.

We attempted the technique suggested by~\citet{bertaux_tapas_2014} to address the poor \(\rm H_{2}O \) airmass scaling, to fit a scaling factor to the \(\rm {H}_{2}O \) absorption lines before convolution to the instrument resolution. This was achieved by first dividing the spectrum by a telluric model with only non-\(\rm H_{2}O \) constituents, convolved to the observed resolution, and scaled by the airmass to remove the non-\(\rm H_{2}O \) lines. Then a model with only \(\rm H_{2}O \) lines at full resolution was scaled by a factor \(\textrm{T}^{x} \), convolved to \(\rm R=50\,000 \) and compared to the observed spectra. The factor \(x \) was fitted to find the best scaling factor for the \(\rm H_{2}O \) lines.

We found that for a few spectra in our sample this method corrected the deeper telluric lines well, but in many cases we found that the fitted scaling factor was affected by the presence of blended stellar lines (attempting to fit those also). It was also strongly influenced by the deepest \(H_{2}O\) telluric lines present. We find that the telluric correction of the deep \(\rm H_{2}O \) lines could be improved with this technique, but, at the cost of worsening the correction of the many smaller \(\rm H_{2}O \) lines. Since the smaller \(\rm H_{2}O \) lines covered more of the spectrum in this region than the larger lines we chose not to continue with this separate \(\rm H_{2}O \) scaling. One possible solution for this would be to perform a piece-wise telluric correction, performing this step only for the deeper \( \rm H_{2}O\) lines, or by using one of the other tools that fits the telluric model to the observations. This technique could also benefit from a larger wavelength span that would enable blended lines to be ignored while having sufficient deep \(\rm H_{2}O\) lines to fit the scaling factor correctly. This small experiment shows that a simple scaling is not enough to correct for the absorption in an effective way, for this case.

After the telluric correction is performed, the spectra are corrected for Earth's barycentric {RV} using the \emph{helcor} PyAstronomy\footnote{https://pyastronomy.readthedocs.io} function ported from the REDUCE IDL package (See~\citet[][]{piskunov_new_2002}).


\subsection{Tapas models}
\label{subsec:tapas_models}
We used telluric line models to wavelength calibrate the reduced spectra in Sect.~\ref{subsec:wave_cal} and to correct for the atmospheric absorption in Sect.~\ref{subsec:telluric_correction}. We utilized the {TAPAS} (Transmissions of the AtmosPhere for AStronomical data) web-service\footnote{\url{http://www.pole-ether.fr/tapas/}}~\citep{bertaux_tapas_2014} to obtain atmospheric transmission models for each observation. {TAPAS} uses the standard line-by-line radiation transfer model code LBLRTM~\citep{clough_linebyline_1995} along with the 2008 HITRAN spectroscopic database~\citep{rothman_hitran_2009} and Arletty atmospheric profiles derived using meteorological measurements from the ETHER data center\footnote{\url{http://www.pole-ether.fr}}, which has a 6 hour resolution in atmospheric profiles.
We use the mid-observation time to retrieve transmission models for each observation, with the Arletty atmospheric profiles\footnote{Nearest of the 6 hourly profiles} and vacuum wavelengths. The telluric models were retrieved without any barycentric correction to keep the telluric lines at a {RV} of zero with respect to the instrument. We obtained one model with all provided species present, convolved to a resolution of \(\rm R=50\,000 \), and another two models without an instrumental profile convolution applied. For these two extra models, one contained only the transmission spectra of \(\rm H_{2}O \) while the other was for all other constituents except \(\rm H_{2}O \). This was to explore a known issue~\citep{bertaux_tapas_2014} with the depth of \(\rm H_{2}O \) absorption lines in Sect.~\ref{subsec:telluric_correction}.


\subsection{Wavelength masking}
We apply several wavelength masks to remove wavelength regions from which we cannot extract information.
Firstly, regions near the edges of each detector where the wavelength solution is extrapolated outside of the calibrating telluric lines are removed, reducing the effective size of each detector by about \(10\%\) or \(\sim\)100~pixels.

Secondly, we mask out any remaining artefacts present in the spectra and the centers of deep telluric lines where telluric correction was not corrected properly, sometimes resulting in ``emission-like'' peaks in the corrected spectrum. These factors combined result in masking out around a further 10\% of the observed spectra.

In Sect.~\ref{sec:results} we also apply a further wavelength restriction to mask out regions where there is a large mismatch between the observed spectrum and the closest synthetic spectra to the host. This significantly restricts the wavelength span utilized for that purpose to around only 43\% and the masked regions are visible in Fig.~\ref{fig:visualinspection-hd2118471}.


\section{Differential subtraction}
\label{sec:spec_diff}

The observations were gathered having in mind the application of a differential subtraction method~\citep[e.g.][]{ferluga_separating_1997, kostogryz_spectral_2013} to detect the spectra of the faint BD companions. In short, we Doppler shift each observation to the rest frame of the host star and then mutually subtract the spectra from pairs of observations to cancel out the spectra of the brighter host star. The residuals from this method should contain two copies of the faint companion, subtracted from each other with a radial velocity offset \(\Delta {RV}\) between them. This {RV} offset, if detectable, would allow the mass ratio to be determined and hence the companion mass.

Due to the poorly separated observation times relative to the long orbital periods, this method was revealed to be inappropriate for these observations as the {RV} separation of the companion spectra between observations (\(\le 2.3\)\kmps{}) is significantly smaller than the FWHM (full width half maximum) of individual spectral lines (\(\sim\)6\kmps{} at \(\rm R=50\,000\)).  The small separation of the companion causes the lines of the companion to also mutually cancel, severely reducing the residual signal to well below the available noise level. The requirement of well separated RVs for the companion spectra was clearly stated in the original proposal but was not satisfied when the observations were obtained\footnote{see Sect.~\ref{subsubsec:differential-schedualing} for more details}. {\rd{} The very large orbital periods of some of the targets would not produce a sufficient {RV} signal during one semester. This was a possible oversight during the proposal stage.} The largest estimated companion \(\Delta {RV}\) separation between the observations of each target is provided in the seventh column of Table~\ref{tab:estimatedparameters}. Radial velocity constraints are also valid for other studies such as the detection of reflected light from exoplanets~\cite{martins_evidence_2015}.

Despite these problems, we still tried to apply the method to our data. Given the negative result, however, we decided to move this section to Appendix~\ref{appendix:A1}, where we describe the method and some exploratory results that may be useful for future studies.

%%%%%%%%%%%%%%%%%%%%%%%%%%%%%%%%%%%%%%%%%%%%%%%%%%%%%%%%%%%%%
%%%%%%%%%%%%%%%%%%%%%%%%%%%%%%%%%%%%%%%%%%%%%%%%%%%%%%%%%%%%%


\section{Discussion}
\label{sec:discussion}
The spectral differential and the synthetic recovery methods attempted here were both unsuccessful in a detection of the BD companion spectra.  The upper mass limits of \(600^{+20}_{-40} \) we set for these companions is very high, roughly six times higher then the BD mass limit \(\sim\)80--90~\(\rm \Mjup{}\).
We discuss potential reasons and solutions for these poor results below, list the lessons learned in this exploratory study {\rd{} of this dataset}, and provide some guidance for any future attempts with these methods.


\subsection{Synthetic recovery limitations}
\label{subsec:limitations}
In this section we discuss some of the limitations from this synthetic binary method and some options to overcome some of these.

\subsubsection{Mismatch in synthetic models}
\label{subsubsec:mismatch}
We believe the spectral mismatch between the observation and synthetic spectra is the main cause of the unsuccessful companion detection with the $\chi^2$ method with several strong lines in the model not observed in the spectra. This impacts the recovery in two ways; the spectral mismatch causes the \(\chi^2\) values to be large in general, but also causes the companion temperature to be pushed to higher temperatures, up to the constraints allowed by the grid.

In our examples the logg and metallicity of the synthetic models are held fixed, leaving only temperature to vary. The temperature impacts the synthetic spectral models in two main ways: the flux level of the continuum; and the number and strength of the absorption lines. In the binary model the contributions from the individual components is scaled by the flux ratio. If the temperature of the companion increases then the flux and radius of the companion increases. The contribution of the companion to the binary model increases and the flux ratio \(\rm F_1/F_2\) decreases. This effectively makes the lines of the spectrum of the host component relatively smaller in the normalized binary model spectrum. Due to the large initial mismatch of synthetic spectral lines of the host, a decrease in relative strength of the host lines decreases the \(\chi^2\) value, and is a better ``match'' to the observation. This causes the recovered temperature of the companion to be much higher than expected. If the companion temperature grid is allowed to extend it will recover a companion with a temperature >2000\K{} above the expected temperature. The \(\chi^2\) approach is dominated by reducing the mismatch in the spectrum of the host rather than actually detecting the spectra of the companion. When preforming the simulations  in Sect.~\ref{subsec:simulated_binaries} there is no spectral mismatch between the simulation and the models, hence they do recover the correct host spectra and get closer values for the companion.

{\rd{} To specifically check that the mismatch was not due to errors in our reduction we obtained the final spectral atlas of {10 LEO} from the CRIRES-POP archive, fully reduced and telluric corrected~\citep{nicholls_crirespop_2017}. Comparing the spectrum of {10 LEO} to a PHOENIX-ACES model with the corresponding parameters \(\teffsub{1}=4800\)\K{}, logg=3.0, \feh{}=0.0 and convolved to R=90\,000,  similar line mismatches are observed. For 10 LEO, there is more spectral lines that agree with the model, but in the region of our observations between 2110--2160\nm{} there are deep lines in the model that are not observed, lines that have significantly different strengths, lines observed that are not modelled whilst other lines `appear' to be offset in wavelength, at the same positions observed in the work presented here. This indicates that the mismatch is not specific to our reduction only and that there is still room for improvement in synthetic models around 2100\nm{}.}

{\rd{} Other works have also indicated regions or specific lines in which synthetic models did not reproduce all of the spectral features seen in stellar spectra~\citep[e.g.][]{delburgo_physical_2009, bonnefoy_library_2014, passegger_spectroscopic_2016, rajpurohit_spectral_2016}.}


\subsubsection{Line contribution of faint companions}
\label{subsubsec:line_contributions}
We calculate the line depths of the synthetic companion spectra to determine the {SNR} levels required to detect the lines of the binary companions.
One thing easy to overlook when attempting to detect the binary companion at low flux ratios is the actual contribution of the spectral lines of the companion.
The flux ratio of the continuum for our most promising target is \(\rm F_2/F_1\)\(\sim\)1\% with the other targets having an expected flux ratio 0.5\%, and some well below. The spectral lines of the individual components which are the features we are trying to detect with the binary model, have depths on average around 10--20\% of their respective continua,  at-least between 2110--2160\nm{}. In effect, the companion line features have a depth \(\ll\) 1\% relative to the continuum of the combined spectra.

In Table~\ref{tab:line_contributions} we calculate some properties of the spectral lines in the PHOENIX-ACES library between 2110--2160\nm{}. We count the number of spectral lines (\emph{no.\ lines}) deeper than 5\%, and take the average depth (\emph{avg.\ depth}) of these lines. The contribution \emph{cont.\ depth} of the companion lines to a combined spectrum accounts for the flux ratio between the two components. Here we use a Sun-like host with \(\teffsub{1}=5800\)\K{}. This simplified combination neglects the continuum shapes of both spectral components and uses the average flux ratios for this wavelength range. The PHOENIX-ACES spectra in the temperature range of 2500--5000\K{} shown in Fig.~\ref{fig:comp_spectra} can be used to get a visual indication of the line density and depth measured here.

There are more lines >5\% deep for the lower temperature spectra, with 360--460 lines in this wavelength range, to be compared with the 31 deep absorption lines found in a Sun-like spectrum. The average line depth of these lines is also larger than the Sun-like spectrum, around twice as deep. However, when combined, the contribution of the companion lines is 1--2 orders of magnitude smaller than the hosts lines due to the low continuum flux ratios.

For example, with the synthetic model for the companion of {HD 211847}, the average contributions of lines >5\% become only 0.3\% deep in a binary with the Sun-like spectrum. For a companion with a temperature of 2300\K{} (the lower PHOENIX-ACES temperature limit) the deepest lines contribute lines around 0.1\%.

%{\bl We use the contributed line depth values to calculate the {SNR} level required to have Gaussian noise of the same height and the observed {SNR} required to achieve equivalent contribution from all N lines of the spectrum, \(\rm \textrm{SNR}_N = \textrm{SNR} /\sqrt{N}\). This is for the synthetic spectra which have many more lines than the observed spectra in this wavelength range. }

The {SNR} of the observed spectra is between 150--350, which is below the {SNR} of 323 needed for the detection of the low-mass star companion of {HD 211847} with temperature 3200\K{} and logg 5.0. For our other targets with BD companions at and below the PHOENIX-ACES temperature range, we would need observed {SNR} >800 to detect the individual spectral lines of the companion. With the {SNR} increasing with \(\sqrt{N}\) this would require the observational time for each target to be increased by a factor of \(\sim\)10--64.
{\rd{} This is in line with the recent detection of the spectrum of a non-transiting giant planet by~\cite{piskorz_evidence_2016} which utilized nIR spectra with {SNR} > 2000, from 1--3 hours of observation each.}

Our non-detection of binary companions with low flux ratios is consistent with results from other works. For example~\citet{nemravova_xtauri_2016} performed extensive spectral analysis of a quadruple-star system {\(\xi\) Tauri} using 227 spectra in 3 different wavelength bands. Of the four stars in the system they were unable to detect the spectral component of the one which had a luminosity ratio below 1\%. {\rd{} The secondary detection in optical spectra using spectral matching of KOI was also only able to reach flux ratios of 1\%~\citet{kolbl_detection_2015}.}


%\input{tables/linedepths.tex}

\subsubsection{\(\chi^2\) asymmetry}
\label{subsubsec:chi2_assymetry}
In Fig.~\ref{fig:injection_shape} we showed that the shape of the recovered \(\chi^2\) becomes asymmetric when dealing with companion temperatures below around 3800\K{}. A visual inspection of the spectra reveals the likely cause. In Fig.~\ref{fig:comp_spectra} we show the corresponding spectra between 2111--2165\nm{}. As the temperature decreases the strongest lines become less prominent, disappearing progressively among the other many small lines that appear at lower temperatures. Hence there are no strong companion lines to easily distinguish one temperature from another. In the flatter part of the \(\chi^2\) curves several low temperature companions are equally well fitted to the simulation/observation.

Figures~\ref{fig:injection-recovery} and~\ref{fig:injection_shape} show different recovered temperatures but both agree above 3800\K{}. A higher companion temperature is recovered between 2800 and 3800\K{}, where as in Fig.~\ref{fig:injection_shape} a lower temperature is recovered. This is probably due to a combination of the noise added, and the asymmetries of the \(\chi^2\) lines. Figure~\ref{fig:injection-recovery} uses the noise level from the observed spectrum while Fig.~\ref{fig:injection_shape} has a {SNR} of 300.
This large asymmetry can also explain the jump observed in the synthetic recovery temperature around 2700\K{} in Fig.~\ref{fig:injection-recovery}.

The asymmetry also causes an asymmetry in the \(\chi^2\) error bars which can be seen in the bottom panel of Fig.~\ref{fig:injection_shape}. For instance the recovered value and 1-\(\sigma\) error bars on the 3000\K{} injected companion is \(2800 ^{+20}_{-100} \), with an asymmetric error bar skewed towards lower temperatures.

The bump observed at 5100\K{} in the \(\chi^2\) curves is due to a discontinuity in the PHOENIX-ACES modelling. The ``reference wavelength defining the mean optical depth grid'' is changed at 5000\K{}~\citep[][Sect. 2.3]{husser_new_2013}. Care needs to be taken if trying to detect a companion near this temperature.

\begin{figure}
\centering
%    \includegraphics[width=0.95\hsize]{images/companion_spectra.pdf}
\caption{PHOENIX-ACES spectra for temperatures between 2500 and 5000\K{}, corresponding to to the same lines in Fig.~\ref{fig:injection_shape}. The flux units are the native units of the PHOENIX-ACES spectrum, (\(\rm erg\,s^{-1}\,cm^{2}\,cm^{-1}\)), and have not been scaled by the stellar radii. All spectra have a \(\rm logg=5.0\) and \(\rm \feh{}=0.0\). The vertical dotted lines indicate the edges of the CRIRES detectors.}
\label{fig:comp_spectra}
\end{figure}

\subsubsection{Component {RV} separation}
\label{subsubsec:rv_seperation}
Another factor which could contribute to an unsuccessful detection is the {RV} separation between the host and companion, \(rv_2\). Estimates for our observations are given in the last column of Table~\ref{tab:observations}. If \({rv}_2\) is small compared to the line width, then all the same lines of both components will be blended. This is indeed the case for {HD 4747}, {HD 211847}, and {HD 202206} with expected \(|{rv}_2| < 2\)\kmps{}, {\rd{} due to poor observational planning}. This may have contributed to the lack of recovery with both components of the binary model attempting to fit the same features. This may even cause correlation between the parameters of the two components. The {RV} separation of the two components changes with orbital phase. Having multiple spectra of the same target distributed in phase may allow the {RV} of the spectral components to be better recovered~\citep[e.g.][]{czekala_disentangling_2017, sablowski_spectral_2016}.
{\rd{} Similarly~\cite{kolbl_detection_2015} were unable to detect companion stars within 10\kmps{} of the host using optical spectra}.

\subsubsection{Wavelength range}
\label{subsubsec:wavelenght_range_limitation}
{\rd{} The wavelength range for these observations was chosen specifically due to the location of the \textit{K}-band telluric absorption window. This was to reduce telluric contamination present in the spectra intended for the spectral differential technique. The wavelength range is also very narrow (\(\sim\)50\nm{}) and was set by the CRIRES instrument. The small number and inconsistent distribution telluric lines made the wavelength calibration method using the telluric lines difficult in some regions (specifically detectors 2 and 3). For the $\chi^2$ fitting of faint companions this narrow wavelength region is likely not the best choice given the small number of stellar lines and unique spectral features of the companion. For example~\citet{passegger_fundamental_2016} use 4 different wavelength regions with lines from different species to fit PHOENIX-ACES models to M-dwarf stars while other studies aiming to detect planetary companions choose wavelength regions which contain strong planetary absorption features such as the absorption of CO and H\(_2\)O near 2.3 \(\mu\)m~\citep[e.g.][]{dekok_detection_2013, brogi_carbon_2014}.
Applying the binary fitting to a different wavelength region with lines more sensitive to stellar parameters for both stars and BDs, as well as using a larger wavelength range (i.e.\ provided by the cross-dispersion on CRIRES+~\citep{dorn_crires_2016}), should improve the recovery results of the technique presented here. We note that if the wavelength range is increased by taking separate observations at different wavelengths, not covered by a single exposure, then changes in the {RV} of both components between the different wavelength observations will need to be accounted for.}


\subsubsection{The {BT-Settl} models}
\label{subsubsec:BT-Settl}
We note that the PHOENIX-ACES models are not the only spectral libraries available with the other notable library considered for this work is the {BT-Settl} models~\citep{allard_model_2010,allard_btsettl_2013,baraffe_new_2015}. The included modelling of dust and cloud formation, as well as hydro-dynamical modelling atmospheric mixing/settling for atmospheres with \teff{} below \(\sim\)2600\K{}, make the {BT-Settl} models valid across the regime from stars to BDs as cool as 400\K{}. As the {BT-Settl} models are suitable to model the atmospheres of the brown dwarfs they would be useful for the companion recovery technique developed here. However, as shown in Sect.~\ref{subsection:results-hd211847} and~\ref{subsection:injection-recovery}, we were unable to successfully recover the 155 \(\rm M_J\) (\(T_{\textrm{eff}}\sim\)3200\K{}) low mass star companion of {HD 211847} and derived a temperature upper limit for our methodology of around 3800\K{}. These are both well above the 2300\K{} cut-off of the PHOENIX-ACES models and for the onset of dust- and cloud-formation phenomena, at 2600 K..

Fig.~\ref{fig:hd211847-models} shows again the minimum \(\chi^2\) solution for detector 1 of the second {HD 211847} observation, this time including the {BT-Settl} solution with the same parameters. Although the PHOENIX-ACES and {BT-Settl} models differ slightly they both have a large spectral mismatch to the observations. As such, we did not use the {BT-Settl} models for the simulation and results above as we did not see any special advantage in using them.

The ease of access to find, download, and use PHOENIX-ACES spectral library, available in the fits file format, compared to older {BT-Settl} libraries is another reason for the current favoured use of the PHOENIX-ACES library.

Although the newer generations synthetic spectral models are improving and match the overall spectral energy distribution reasonably well there are still regions in the H and \emph{K}-band where there is room for improvement~\citet{rajpurohit_spectral_2016}. The spectral mismatch in the region studied here is still too large for spectral recovery of companion brown dwarfs. In the nIR we have compounding problem: the model input physics of sub-stellar temperatures and chemistry combined with the general difficulty of the nIR.

\begin{figure}
\centering
%    \includegraphics[width=0.95\hsize]{images/HD211847_ACES_BTSettl.pdf}
\caption{Detector 1 spectrum for {HD 211847} (blue) alongside the PHOENIX-ACES (orange dash-dot) and {BT-Settl} (green dashed) synthetic spectra for the host star only, with parameters \(T_{\textrm{eff}}=5700\)\K{}, \(\rm logg=4.5\) and \(\rm \feh{}=0.0\). Both synthetic models have been normalized and convolved to \(\rm R=50\,000\). There is a 0.05 off-set between each spectrum}
\label{fig:hd211847-models}
\end{figure}


\subsubsection{Impact of logg}
\label{subsubsec:logg}
Logg, a measure of surface gravity, is related to evolutionary state and the size of the star with smaller logg values usually indicating larger radii stars. This parameter has a large impact on the radius and flux ratio of the binary models. In the PHOENIX-ACES models a decrease in logg from 5.0 to 4.5 increases the models effective radius by \(\sim\)1.75 in the temperature range investigated here. This change in radius alone roughly triples (\(1.75^2\)) the absolute flux of the synthetic spectrum, neglecting any changes to the shape of the actual spectrum. Therefore, there are large jumps in the model flux ratios if the logg is allowed to vary, with lower logg values for the companion being favoured as the increased flux ratio decreases the mismatch of the host component to the observations. This large impact of logg on the spectral library absolute flux is one reason for keeping the logg of each component fixed in the \(\chi^2\) results presented in Sect.~\ref{sec:results}.

\subsubsection{Interpolation}
\label{subsubsec:interpolation}
It is common to interpolate between the synthetic spectral grids to fit and derive parameters in between the grid points~\citep[e.g.][]{nemravova_xtauri_2016, passegger_fundamental_2016}. Instead of interpolation~\cite{czekala_constructing_2015} use a spectral emulator to use Principal Component Analysis to create eigenspectra for the synthetic library and Gaussian processes to derive a probability distribution function of possible interpolation spectra to account for uncertainties in the interpolation required for high signal-to-noise spectra.

However, we did not incorporate any interpolation into the companion recovery at this stage. This could be something to be added in the future to refine the recovered parameters, and to help the transition between the grid logg values. Codes are readily available to perform spectral interpolation which could be utilized for this, two of them are \emph{pyterpol}\footnote{https://github.com/chrysante87/pyterpol}~\citet{nemravova_xtauri_2016} and \emph{Starfish}\footnote{https://github.com/iancze/Starfish}~\cite{czekala_constructing_2015}.


\subsection{Future implementation}
\label{subsec:future}
\subsubsection{High resolution instrumentation}
\label{subsubsec:highres}
The future of high resolution near- and mid- IR spectrographs is looking bright, with many new ground- and space-based instruments currently being developed. Notable examples include {CARMENES} (550--1710\nm{}, R=82\,000) which is now operational~\citep{quirrenbach_carmenes_2014}, while SPIRou (980--2350\nm{}, R=73\,500)~\cite{artigau_spirou_2014} and NIRPS (970--1810\nm{}, R=100\,000)~\cite{bouchy_nearinfrared_2017} are still being assembled and installed. The eagerly awaited JWST \textbf{cite} will also be launched soon\footnote{Recently pushed to around May 2020} providing observations in both the near-IR (600--5300\nm{}, R=2700) and mid-IR (4900--28\,800\nm{}, R\(\sim\)1550--3250) regions without contamination from our atmosphere.

The upgrade of CRIRES to CRIRES+~\citep{dorn_crires_2016} will increase the wavelength coverage of a single shot capture by at least a factor of 3--5. This larger wavelength span would be extremely beneficial for the \(\chi^2\) performance of the spectral recovery method, increasing the number of useful lines and spectral features to be fitted with the models.

On the modelling side, there are continual improvements in atmospheric modelling and their associated synthetic spectral libraries: as seen with the evolution of the {BT-Settl} models~\cite{allard_btsettl_2013}. With additional physics and improved line lists and solar abundances~\citep[e.g.][]{asplund_chemical_2009,caffau_solar_2011}, the synthetic libraries are reaching a better agreement with nIR observations. An improved agreement between the nIR observations and synthetic spectra will be crucial to improve the performance of the spectral recovery technique presented here.

Although not successful with the CRIRES data used here, the instrumental stage is set to attempt these techniques presented here using the next-generation of high resolution spectrographs. The lessons learned in this analysis need to be taken into account in order to achieve the best chance of a successful detection.


\subsection{Other techniques}
We note that there are many other disentangling techniques to separate mixed spectra of binary systems,~\citep[e.g.][]{hadrava_disentangling_2009}. These require more than two observations, with  \(n+1 \) observations used to set up a system of linear equations to solve for \(n \) spectral components~\citep[e.g.][]{simon_disentangling_1994,czekala_disentangling_2017, sablowski_spectral_2016}. These methods are ideal for many well spaced observations. For example the ideal situation for the SVD method of~\citet{sablowski_spectral_2016} is homogeneous samples of at least half the period, to identify the moving spectral components.
{\rd{} Recently the cross-correlation and maximum-likelihood techniques~\citep[e.g.][]{lockwood_nearir_2014, piskorz_evidence_2016} have been successful in detecting the faint companion spectra of giant planets using several spectra with very-high {SNR} (>2000) obtained with longer observational time (1--3 hours) each taken across the full orbital range.} The few, short and insufficiently separated observations we analyse here are not suitable to apply these advanced techniques and are beyond what we have attempted here.

{\rd{}  The recent work of~\citet{piskorz_evidence_2016} use a PCA technique to correct for the telluric spectra by applying it to several (number not specified) AB nodding pairs over their 60--180 minute integration time. We are uncertain if this technique would work as effectively on our observations due to our shorter integration time (24 minutes) would have less telluric variation present across the 8 nod spectra. A recent comparison of three telluric correction methods, \emph{Molecfit} and {TelFit} and {TAPAS} to the standard star method by~\cite[][in prep.]{ulmer-moll_telluric_2018} found that all three synthetic models preform better at correcting for atmospheric H\(_2\)O compared to the standard star method with \emph{Molecfit}, being a more complete tool, preforming slightly better over {TAPAS}.}




% Don't change these lines
\bsp{}	% typesetting comment
\label{lastpage}
\end{document}

% End of mnras_template.tex
